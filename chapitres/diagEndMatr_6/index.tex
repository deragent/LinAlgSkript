\chapter{Diagonalisation des endomorphismes et des matrices}

%
\subsection{Réduction des endomorphismes}
%
\paragraph{Rappels} Si $E$ est un espace vectoriel, on rappelle qu'un \underline{endomorphisme de $E$} est une application linéaire de $E$ dans $E$.

\paragraph{Définition} On appelle réduction des endomorphismes (d'un espavec vectoriel de dimension finie) la démarche consistant à chercher une base de l'espace dans laquelle la matrice de l'endomorphisme à une form relativement simple. La diagonalisation des endomorphisme, lorsqu'elle est possible, s'inscrit dans cette démarche.

%
%
\section{Matrice d'une application linéaire et changement de bases}
%
%

\paragraph{Théorème} Soient $E$ et $F$ deux espaces vectoriels de dimension finie, $f: E \rightarrow F$ une application linéaire, $\cE$ et $\cE'$ deux bases de $E$, et $\cF$ et $\cF'$ deux bases de $F$. On note $M = [f]_{\cF \cE}$ la matrice de $f$ dans les bases $\cE$ et $\cF$ et $M' = [f]_{\cF' \cE'}$ la matrice de $f$ dans les bases $\cE'$ et $\cF'$. On note également $P = P_{\cF \cF'}$ la matrice de passage de $\cF$ à $\cF'$ et $Q = P_{\cE \cE'}$ la matrice de passage de $\cE$ à $\cE'$. On a alors 
\begin{eqnarray*}
  M' &=& P^{-1} M Q \\
  \left[f\right]_{\cF' \cE'} &=& P_{\cF \cF'}^{-1} [f]_{\cF \cE} P_{\cE \cE'}
\end{eqnarray*}

\paragraph{Démonstration} Pour tout $\vec{v} \in E$, on a:
$$[f(\vec{v})]_{\cF} = M [\vec{v}]_{\cE}$$
On a $[\vec{v}]_{\cE} = Q [\vec{v}]_{\cE'}$ et $[f(\vec{v})]_{\cF} = P [f(\vec{v})]_{\cF'}$ Il vient $P [f(\vec{v})]_{\cF'} = M Q [\vec{v}]_{\cE'}$, d'où $[f(\vec{v})]_{\cF'} = (P^{-1} M Q ) [\vec{v}]_{\cE'}.$ il s'ensuit que $M' = P^{-1} M Q$.

\paragraph{Example} Considérons l'application linéaire
\begin{eqnarray*}
  f: \R^3 &\rightarrow& \R^2 \\
  (x, y, z) &\mapsto& f(x, y, z) = (x + y, x - y + 2z)
\end{eqnarray*}
Soient $\cC_3$ la base canonique de $\R^3$ et $\cC_2$ celle de $\R^2$. Soient $\cE = ((1, 0, 0), (1, 1, 0), (1, 1, 1))$ une base de $\R^3$ et $\cF = ((1, 1), (1, 2))$ une base de $\R^2$. \\
Calculons la matrice de $f$ dans les bases $\cE$ et $\cF$. On a
$$\left[f\right]_{\cC_2 \cC_3} = 
\begin{pmatrix}
  1 & 1  & 0 \\
  1 & -1 & 2
\end{pmatrix}$$
On a également $P = P_{\cC_2 \cF} = \begin{pmatrix} 1 & 1 \\ 1 & 2 \end{pmatrix}$ et $Q = P_{\cC_3 \cE} = \begin{pmatrix} 1 & 1 & 1 \\ 0 & 1 & 1 \\ 0 & 0 & 1 \end{pmatrix}$. Calculons $P^{-1}$. On a $det(P) = 2 - 1 = 1$, $com(P) = \begin{pmatrix} 2 & -1 \\ -1 & 1 \end{pmatrix}$ d'où
$$P^{-1} = \frac{1}{det(P)} com(P)^{T} = \begin{pmatrix} 2 & -1 \\ -1 & 1 \end{pmatrix}$$
On obtien alors
\begin{eqnarray*}
  \left[f\right]_{\cF \cE} &=& P^{-1}  [f]_{\cC_2 \cC_3} Q \\
    &=& \begin{pmatrix} 2 & -1 \\ -1 & 1 \end{pmatrix} \begin{pmatrix} 1 & 1 & 0 \\ 1 & -1 & 2 \end{pmatrix} \begin{pmatrix} 1 & 1 & 1 \\ 0 & 1 & 1 \\ 0 & 0 & 1 \end{pmatrix} \\
    &=& \begin{pmatrix} 2 & -1 \\ -1 & 1 \end{pmatrix} \begin{pmatrix} 1 & 2 & 2 \\ 1 & 0 & 2 \end{pmatrix} \\
    &=& \begin{pmatrix} 1 & 4 & 2 \\ 0 & -2 & 0 \end{pmatrix}
\end{eqnarray*}

%
%
\section{Valeurs propres, vecteurs propres d'un endomorphisme}
%
%
%
\subsection{Valeur propre}
%
\paragraph{Définition} Soient $E$ un espace vectoriel et $f: E \rightarrow E$ un endomorphisme de $E$. On dit qu'un réel $\lambda$ est une valeur propre de $f$ s'il existe $\vec{v} \in E$, $\vec{v} \neq \vec{0}$, tel que 
$$f(\vec{v}) = \lambda\vec{v}$$

%
\subsection{Vecteur propre}
%
\paragraph{Définition} Soit $\lambda$ une valeur propre de $f$. On appelle vecteur propre de $f$ pour $\lambda$ tout vecteur $\vec{v}$ de $E$ tel que 
$$f(\vec{v}) = \lambda \vec{v}$$

%
\subsection{Sous-espace propre}
%
\paragraph{Définition} L'ensemble des vecteurs de $f$ associés à $\lambda$, noté généralement $E_\lambda$, s'appelle le sous-espace propre de $f$ associé à $\lambda$. On a
\begin{eqnarray*}
  E_{\lambda} &=& \{\vec{v} \in E \vert f(\vec{v} = \lambda\vec{v} \} \\
    &=& \{\vec{v} \in E \vert (f-\lambda id_E)(\vec{v}) = \vec{0} \} \\
    &=& Ker(f-id_E)
\end{eqnarray*}
Par suite, $E_\lambda$ est un sous-espace vectoriel de $E$.

\paragraph{Examples}
\begin{enumerate}
  \item Soit 
    \begin{eqnarray*}
      f: \R^2 &\rightarrow& \R^2 \\
      (x, y) &\mapsto& f(x, y) = (2x - 4 y, x -3 y)
    \end{eqnarray*}
    Alors, on a 
    $$f(1, 1) = (-2, -2) = -2 \cdot (1, 1)$$
    Par suite, $-2$ est valeur propre de $f$, et $(1, 1)$ est un vecteur propre de $f$ associé à $-2$. Déterminons 
    \begin{eqnarray*}
      E_{-2} &=& Ker(f-(-2)\cdot id_{\R^2} ) \\
        &=& Ker(f + 2 \cdot id_{\R^2})
    \end{eqnarray*}
    On est conduit à résoudre le sytème linéaire.
    $$\left(\begin{pmatrix}
        2 & -4 \\
        1 & -3
      \end{pmatrix} + 2 \cdot
      \begin{pmatrix}
        1 & 0 \\
        0 & 1
      \end{pmatrix} \right) X = 0$$
    c'est-à-dire $\begin{pmatrix} 4 & -4 \\ 1 & -1 \end{pmatrix} X = 0$ Il vient
    $$\begin{pmatrix} 4 & -4 & \vert & 0 \\ 1 & 1 & \vert & 0 \end{pmatrix} \rightsquigarrow \begin{pmatrix} 1 & 0 & \vert & 0 \\ 0 & 0 & \vert & 0 \end{pmatrix}$$
    Donc 
    \begin{eqnarray*}
      Ker(f + 2 \cdot id_E) &=& \{(t, t) \in \R^2 \vert \R \} \\
        &=& \{t \cdot (1, 1) \in R^2 \vert t \in \R \} \\
        &=& Vect((1, 1))
    \end{eqnarray*}
    
  \item Soit 
    \begin{eqnarray*}
      f: \M_{3 \times 3}(\R) &\rightarrow& M_{3 \times 3}(\R) \\
      A &\mapsto& f(A) = A^{T}
    \end{eqnarray*}
    On sait que $f$ est un endomorphisme de $M_{3 \times 3}(\R)$. Alors $1$ est valeur propre de $f$ et
    $$E_{1} = \{A \in M_{3 \times 3}(\R) \vert A^{T} = A \}$$
    est le sous-espace vectoriel de $M_{3 \times 3}(\R)$ consittué des matrices symétriques. \\
    De même $-1$ est valeur propre de $f$, et
    $$E_{-1} = \{A \in M_{3\times 3}(\R) \vert A^{T} = -A\}$$
    est le sous-espace vectoriel de $M_{3\times 3}(\R)$ constitué des matrices antisymétriques.
    
\end{enumerate}

\paragraph{Théorème} Soit $f: E \rightarrow E$ un endomorphisme de $E$. On suppose que $E$ est de dimension finie. Soit $\lambda \in \R$. \\
Alors $\lambda$ est valeur propre de $f \Leftrightarrow f- \lambda \cdot id_E$ n'est pas bijectif.
      
\paragraph{Démonstration} $\lambda$ est valeur propre de $f$
\begin{enumerate}[$\Leftrightarrow$]
  \item il existe $\vec{v} \in E$, $\vec{v} \neq \vec{0}$, $(f - \lambda \cdot id_E)(\vec{v}) = \vec{0}$ 
  \item $Ker(f-\lambda\cdot id_E) \neq \{\vec{0}\}$
  \item $f-\lambda \cdot id_E$ n'est pas injectif 
  \item $f-\lambda \cdot id_E$ n'est pas bijectif.
\end{enumerate}

%
\subsection{Polynôme caractéristique}
%
\paragraph{Définition} Soient $f: E \rightarrow E$ un endomorphisme ($E$ est de dimension finie), \cE une base $E$, et $M$ la matrice de $f$ dans les bases $\cE$ et $\cE$\footnote{On dit simplement "dans la base $\cE$".} \\
On appelle polynôme caractéristique de $f$ le polynôme
$$P(x) = (-1)^{n} det(M - X I_n)$$

\paragraph{Remarque importante} Le polynôme $P$ est indépendant de la base choisie. \\
En effet, si $\cE'$ est une autre base de $E$, et si $P = P_{\cE \cE'}$ est la matrice de passe de $\cE$ à $\cE'$, alors:
\begin{itemize}
  \item la matrice de $f$ dans la base $\cE'$ est $M' = P^{-1} M P$
  \item 
    \begin{eqnarray*}
      (-1)^{n} det(M' - X I_n) &=& (-1)^{n} det(P^{-1} M P - X I_n) \\
        &=& (-1)^{n} det(P^{-1} (M - X I_n) P) \\
        &=& (-1)^{n} det(P)^{-1} det(M - X I_n) det(P) \\
        &=& (-1)^{n} det(M - X I_n) \\
    \end{eqnarray*}
\end{itemize}

\paragraph{Définition} $\lambda$ valeur de propre de $f$ $\Leftrightarrow$ $P(\lambda) = 0$
\paragraph{Définition} $\lambda$ f est diagonalisable s'il existe une base de $E$ formée de vecteurs propres de $f$
