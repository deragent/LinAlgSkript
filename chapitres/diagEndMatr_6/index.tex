\chapter{Diagonalisation des endomorphismes et des matrices}

%
\subsection{Réduction des endomorphismes}
%
\paragraph{Rappels} Si $E$ est un espace vectoriel, on rappelle qu'un \underline{endomorphisme de $E$} est une application linéaire de $E$ dans $E$.

\paragraph{Définition} On appelle réduction des endomorphismes (d'un espace vectoriel de dimension finie) la démarche consistant à chercher une base de l'espace dans laquelle la matrice de l'endomorphisme à une form relativement simple. La diagonalisation des endomorphisme, lorsqu'elle est possible, s'inscrit dans cette démarche.

%
%
\section{Matrice d'une application linéaire et changement de bases}
%
%

\paragraph{Théorème} Soient $E$ et $F$ deux espaces vectoriels de dimension finie, $f: E \rightarrow F$ une application linéaire, $\cE$ et $\cE'$ deux bases de $E$, et $\cF$ et $\cF'$ deux bases de $F$. On note $M = [f]_{\cF \cE}$ la matrice de $f$ dans les bases $\cE$ et $\cF$ et $M' = [f]_{\cF' \cE'}$ la matrice de $f$ dans les bases $\cE'$ et $\cF'$. On note également $P = P_{\cF \cF'}$ la matrice de passage de $\cF$ à $\cF'$ et $Q = P_{\cE \cE'}$ la matrice de passage de $\cE$ à $\cE'$. On a alors 
\begin{eqnarray*}
  M' &=& P^{-1} M Q \\
  \left[f\right]_{\cF' \cE'} &=& (P_{\cF \cF'})^{-1} [f]_{\cF \cE} P_{\cE \cE'}
\end{eqnarray*}

\demo{Pour tout $\vec{v} \in E$, on a:
  $$[f(\vec{v})]_{\cF} = M [\vec{v}]_{\cE}$$
  On a $[\vec{v}]_{\cE} = Q [\vec{v}]_{\cE'}$ et $[f(\vec{v})]_{\cF} = P [f(\vec{v})]_{\cF'}$ Il vient 
  $$P [f(\vec{v})]_{\cF'} = M Q [\vec{v}]_{\cE'}$$
  d'où 
  $$[f(\vec{v})]_{\cF'} = (P^{-1} M Q ) [\vec{v}]_{\cE'}$$
  il s'ensuit que $M' = P^{-1} M Q$.
}

\paragraph{Exemple} Considérons l'application linéaire
\begin{eqnarray*}
  f: \R^3 &\rightarrow& \R^2 \\
  (x, y, z) &\mapsto& f(x, y, z) = (x + y, x - y + 2z)
\end{eqnarray*}
Soient $\cC_3$ la base canonique de $\R^3$ et $\cC_2$ celle de $\R^2$. Soient $\cE = ((1, 0, 0), (1, 1, 0), (1, 1, 1))$ une base de $\R^3$ et $\cF = ((1, 1), (1, 2))$ une base de $\R^2$. \\
Calculons la matrice de $f$ dans les bases $\cE$ et $\cF$. On a
$$\left[f\right]_{\cC_2 \cC_3} = 
\begin{pmatrix}
  1 & 1  & 0 \\
  1 & -1 & 2
\end{pmatrix}$$
On a également $P = P_{\cC_2 \cF} = \begin{pmatrix} 1 & 1 \\ 1 & 2 \end{pmatrix}$ et $Q = P_{\cC_3 \cE} = \begin{pmatrix} 1 & 1 & 1 \\ 0 & 1 & 1 \\ 0 & 0 & 1 \end{pmatrix}$. Calculons $P^{-1}$. On a $det(P) = 2 - 1 = 1$, $com(P) = \begin{pmatrix} 2 & -1 \\ -1 & 1 \end{pmatrix}$ d'où
$$P^{-1} = \frac{1}{det(P)} com(P)^{T} = \begin{pmatrix} 2 & -1 \\ -1 & 1 \end{pmatrix}$$
On obtien alors
\begin{eqnarray*}
  \left[f\right]_{\cF \cE} &=& P^{-1}  [f]_{\cC_2 \cC_3} Q \\
    &=& \begin{pmatrix} 2 & -1 \\ -1 & 1 \end{pmatrix} \begin{pmatrix} 1 & 1 & 0 \\ 1 & -1 & 2 \end{pmatrix} \begin{pmatrix} 1 & 1 & 1 \\ 0 & 1 & 1 \\ 0 & 0 & 1 \end{pmatrix} \\
    &=& \begin{pmatrix} 2 & -1 \\ -1 & 1 \end{pmatrix} \begin{pmatrix} 1 & 2 & 2 \\ 1 & 0 & 2 \end{pmatrix} \\
    &=& \begin{pmatrix} 1 & 4 & 2 \\ 0 & -2 & 0 \end{pmatrix}
\end{eqnarray*}

%
%
\section{Valeurs propres, vecteurs propres d'un endomorphisme}
%
%
%
\subsection{Valeur propre}
%
\paragraph{Définition} Soient $E$ un espace vectoriel et $f: E \rightarrow E$ un endomorphisme de $E$. On dit qu'un réel $\lambda$ est une valeur propre de $f$ s'il existe $\vec{v} \in E$, $\vec{v} \neq \vec{0}$, tel que 
$$f(\vec{v}) = \lambda\vec{v}$$

%
\subsection{Vecteur propre}
%
\paragraph{Définition} Soit $\lambda$ une valeur propre de $f$. On appelle vecteur propre de $f$ pour $\lambda$ tout vecteur $\vec{v}$ de $E$ tel que 
$$f(\vec{v}) = \lambda \vec{v}$$

%
\subsection{Sous-espace propre}
\label{sec:se_prop}
%
\paragraph{Définition} L'ensemble des vecteurs de $f$ associés à $\lambda$, noté généralement $E_\lambda$, s'appelle le sous-espace propre de $f$ associé à $\lambda$. On a
\begin{eqnarray*}
  E_{\lambda} &=& \{\vec{v} \in E ~ \vert ~ f(\vec{v}) = \lambda\vec{v} \} \\
    &=& \{\vec{v} \in E ~ \vert ~ (f-\lambda id_E)(\vec{v}) = \vec{0} \} \\
    &=& Ker(f - \lambda id_E)
\end{eqnarray*}
Par suite, $E_\lambda$ est un sous-espace vectoriel de $E$.

\paragraph{Exemple}
\begin{enumerate}
  \item Soit 
    \begin{eqnarray*}
      f: \R^2 &\rightarrow& \R^2 \\
      (x, y) &\mapsto& f(x, y) = (2x - 4 y, x -3 y)
    \end{eqnarray*}
    Alors, on a 
    $$f(1, 1) = (-2, -2) = -2 \cdot (1, 1)$$
    Par suite, $-2$ est valeur propre de $f$, et $(1, 1)$ est un vecteur propre de $f$ associé à $-2$. Déterminons 
    \begin{eqnarray*}
      E_{-2} &=& Ker(f-(-2)\cdot id_{\R^2} ) \\
        &=& Ker(f + 2 \cdot id_{\R^2})
    \end{eqnarray*}
    On est conduit à résoudre le sytème linéaire.
    $$\left(\begin{pmatrix}
        2 & -4 \\
        1 & -3
      \end{pmatrix} + 2 \cdot
      \begin{pmatrix}
        1 & 0 \\
        0 & 1
      \end{pmatrix} \right) X = 0$$
    c'est-à-dire $\begin{pmatrix} 4 & -4 \\ 1 & -1 \end{pmatrix} X = 0$ Il vient
    $$\left( \begin{array}{cc|c} 4 & -4 & 0 \\ 1 & -1 & 0 \end{array} \right) 
      \rightsquigarrow 
      \left( \begin{array}{cc|c} 1 & -1 & 0 \\ 0 & 0 & 0 \end{array} \right) $$
    Donc 
    \begin{eqnarray*}
      Ker(f + 2 \cdot id_E) &=& \{(t, t) \in \R^2 ~ \vert ~ \R \} \\
        &=& \{t \cdot (1, 1) \in \R^2 ~ \vert ~ t \in \R \} \\
        &=& Vect((1, 1))
    \end{eqnarray*}
    
  \item Soit 
    \begin{eqnarray*}
      f: M_{3 \times 3}(\R) &\rightarrow& M_{3 \times 3}(\R) \\
      A &\mapsto& f(A) = A^{T}
    \end{eqnarray*}
    On sait que $f$ est un endomorphisme de $M_{3 \times 3}(\R)$. Alors $1$ est valeur propre de $f$ et
    $$E_{1} = \{A \in M_{3 \times 3}(\R) ~ \vert ~ A^{T} = A \}$$
    est le sous-espace vectoriel de $M_{3 \times 3}(\R)$ consittué des matrices symétriques. \\
    De même $-1$ est valeur propre de $f$, et
    $$E_{-1} = \{A \in M_{3\times 3}(\R) ~ \vert ~ A^{T} = -A\}$$
    est le sous-espace vectoriel de $M_{3\times 3}(\R)$ constitué des matrices antisymétriques.
    
\end{enumerate}

\paragraph{Théorème} Soit $f: E \rightarrow E$ un endomorphisme de $E$. On suppose que $E$ est de dimension finie. Soit $\lambda \in \R$. \\
Alors 
$$\lambda \text{ est valeur propre de } f \Leftrightarrow f- \lambda \cdot id_E \text{ n'est pas bijectif}$$
      
\demo{$\lambda$ est valeur propre de $f$
  \begin{itemize}
    \item[$\Leftrightarrow$] il existe $\vec{v} \in E$, $\vec{v} \neq \vec{0}$, $(f - \lambda \cdot id_E)(\vec{v}) = \vec{0}$ 
    \item[$\Leftrightarrow$] $Ker(f-\lambda\cdot id_E) \neq \{\vec{0}\}$
    \item[$\Leftrightarrow$] $f-\lambda \cdot id_E$ n'est pas injectif 
    \item[$\Leftrightarrow$] $f-\lambda \cdot id_E$ n'est pas bijectif.
  \end{itemize}
}

%
\subsection{Polynôme caractéristique d'un endomorphisme}
%
\paragraph{Définition} Soient $f: E \rightarrow E$ un endomorphisme ($E$ est de dimension finie $n$), $\cE$ une base $E$, et $M$ la matrice de $f$ dans les bases $\cE$ et $\cE$.\footnote{On dit simplement "dans la base $\cE$".} \\
On appelle polynôme caractéristique de $f$ le polynôme
$$P(x) = (-1)^{n} det(M - X I_n)$$

\paragraph{Remarque importante} Le polynôme $P$ est indépendant de la base choisie. \\
En effet, si $\cE'$ est une autre base de $E$, et si $P = P_{\cE \cE'}$ est la matrice de passe de $\cE$ à $\cE'$, alors:
\begin{itemize}
  \item la matrice de $f$ dans la base $\cE'$ est $M' = P^{-1} M P$
  \item 
    \begin{eqnarray*}
      (-1)^{n} det(M' - X I_n) &=& (-1)^{n} det(P^{-1} M P - X I_n) \\
        &=& (-1)^{n} det(P^{-1} (M - X I_n) P) \\
        &=& (-1)^{n} det(P^{-1}) det(M - X I_n) det(P) \\
        &=& (-1)^{n} det(M - X I_n) \\
    \end{eqnarray*}
\end{itemize}

\paragraph{Exemple} Soit 
\begin{eqnarray*}
  f: \R^2 &\rightarrow& \R^2 \\
  (x, y) &\mapsto& (x + 3y, -x + 2y)
\end{eqnarray*}
La matrice de $f$ dans la base canonique de $\R^2$ est 
$$\begin{pmatrix} 1 & 3 \\ -1 & 2 \end{pmatrix}$$
Le polynôme caracteristique de $f$ est
\begin{eqnarray*}
  P(X) &=& (-1)^2 det(M-X I_2) \\
    &=& \begin{vmatrix} 1-X & 3 \\ -1 & 2-X \end{vmatrix} \\
    &=& (1-X)(2-X) + 3 \\
    &=& X^2 -3X + 5 
\end{eqnarray*}

\paragraph{Corrolaire} De façon générale le polynôme caracteristique $P(X)$ de $f$ est unitaire\footnote{Le coefficient du terme de plus haut degré de $P$ est égal à $1$.} de degré $n = dim(E)$. En effet,
\begin{eqnarray*}
  P(X) &=& (-1)^n det(M-X I_n) \\
   &=& (-1)^n \sum_{\sigma \in S_n} p_{\sigma}(M- X I_n)
\end{eqnarray*}
La plus grand epuissance de $X$ dasn $P(X)$ est obtenue dans
$$p_{id}(M - X I_n) = (a_{11} - X) \cdot \ldots \cdot (a_{nn} - X) = (-1)^n X^n \cdot \ldots$$
Finalement, le monôme de plus haut degré dans $P(X)$ est 
$$(-1)^n \big( (-1)^n X^n \big) = X^n$$

%
\subsection{Polynôme caractéristique d'une matrice}
%
\paragraph{Définition} Par extension si $A$ est une matrice carrée de taille $n\times n$, on définit le polynôme caractéristique de $A$ comme étant
$$P(X) = (-1)^n det(A -X I_n)$$

\paragraph{Théorème} Soit $f: E \rightarrow E$ un endomorphisme. Alors un réel $\lambda$ est valeur propre de $f$ si et seulement si $P(\lambda) = 0$, où $P$ est le polynôme caractéristique de $f$. Autrement dit, les valeurs propres de $f$ sont exactément les racines de $P$.

\demo{Soit $\lambda \in \R$. $\lambda$ est valeur propre de $f$
  \begin{itemize}
    \item[$\Leftrightarrow$] $f - \lambda id_E$ n'est pas bijectif 
    \item[$\Leftrightarrow$] la matrice\footnote{$M$ désigne la matrice de $f$ dans une base de $E$.} $M- \lambda id_E$ de $f- \lambda id_E$ n'est pas inversible
    \item[$\Leftrightarrow$] $det(M - \lambda I_n) = 0$
    \item[$\Leftrightarrow$] $P(\lambda) = 0$.
  \end{itemize}
}

\paragraph{Exemple} 
\begin{eqnarray*}
  f: \R^2 &\rightarrow& \R^2 \\
  (x, y) &\mapsto& (2x + y, 3x+ y)
\end{eqnarray*}
Calculons les valeurs propres de $f$.
\begin{eqnarray*}
  M = [f]_{\cC \cC} = \begin{pmatrix} 2 & 1 \\ 3 & -1 \end{pmatrix}
\end{eqnarray*}
Le polynôme caractéristique de $f$ est
\begin{eqnarray*}
  P(X) = \begin{vmatrix} 2-X & 1 \\ 3 & -1-X \end{vmatrix} &=& (2 -X)(-1 -X) -3 \\
    &=& X^2 - X - 5 \\
    &=& \left( X-\frac{1}{2} \right)^2 - \frac{21}{4} \\
    &=& \left(X - \frac{1 + \sqrt{21}}{2} \right) \left(X - \frac{1 - \sqrt{21}}{2} \right)
\end{eqnarray*}
Les racines de $P$ sont $\frac{1 + \sqrt{21}}{2}$ et $\frac{1 - \sqrt{21}}{2}$. \\
Le valeur propres de $f$ sont donc $\frac{1 + \sqrt{21}}{2}$ et $\frac{1 - \sqrt{21}}{2}$.

%
%
\section{Endomorphismes et matrices diagonalisables}
%
%

%
\subsection{Endomorphisme diagonalisable}
%
\paragraph{Définition} Soit $f$ un endormorphisme d'un expace vectoriel $D$ de dimension finie $n$. On dit que $f$ est diagonalisable s'il existe une base de $E$ formée de vecteurs propres de $f$.

\paragraph{Interprétation matricielle de cette définition} Si $f$ est diagonalisable, consiérons une base de $E$ 
$$\cB = (\vec{v}_1, \ldots, \vec{v}_n)$$
formée de vecteurs propres de $f$.  Pour tout $i$ disons que $\vec{v}_i$ est vecteur propre de $f$ pour la valeur propre $\lambda_i$. La matrice de $f$ dans cette base est
\begin{eqnarray*}
  \begin{pmatrix} \\ [f(\vec{v}_1)]_{\cB} & \ldots & [f(\vec{v}_n)]_{\cB} \\ \\ \end{pmatrix} &=& 
    \begin{pmatrix} \\ [\lambda \vec{v}_1]_{\cB} & \ldots & [\lambda \vec{v}_n]_{\cB} \\ \\ \end{pmatrix} \\
    &=& \begin{pmatrix} 
      \lambda_1 & 0 & \ldots & 0 \\
      0 & \lambda_2 & \ldots & 0 \\
      \vdots & \vdots & \ddots & \vdots \\
      0 & 0 & \ldots & \lambda_n \\
    \end{pmatrix}
\end{eqnarray*}
Dans la base $\cB$, la matrice de $f$ est diagonale. \\
Si $\cC$ est une base de $E$ dans laquelle on considère $f$ initalement, si $M$ est la matrice de $f$ dans $\cC$ et si $P = P_{\cC \cB}$ est la matrice de passage de $\cC$ à $\cB$, alors la matrice de $f$ dans la base $\cB$ 
$$D = P^{-1} M P$$
C'est donc une  matrce diagonale. De plus les coefficients diagonaux de cette matrice sont les valeurs propres de $f$, éventuellement répétées.

\paragraph{Exemple}
\begin{eqnarray*}
  f: \R^2 &\rightarrow& \R^2 \\
  (x, y) &\mapsto& (x + y, 2y)
\end{eqnarray*}
Déterminons si $f$ est diagonalisable ou non.
\begin{enumerate}[(1)]
  \item Formons la matrice de $f$ dans la base canonique de $\R^2$
    $$M = \begin{pmatrix} 1 & 1 \\ 0 & 2 \end{pmatrix}$$
  
  \item Calculons le polynôme caractéristique de $f$
    $$P(x) = (-1)^2 det(M - X I_2) = \begin{vmatrix} 1-X & 1 \\ 0 & 2-X \end{vmatrix} = (1-X)(2-X)$$
    
  \item Déterminons les valeur propres de $f$ en cherchant les racines de $P$. Les racines de $P$ sont $1$ et $2$ donc les valeurs propres de $f$ sont $1$ et $2$.
  
  \item Pour chaque valeur propre $\lambda$ de $f$, déterminons le sous-espace propre associé à $E_\lambda$.
    \begin{itemize}
      \item pour $\lambda = 1$, 
        $$E_1 = Ker(f - id_{\R^2})$$
        Il s'agit de résoudre le système linéaire
        $$(M - I_2) X = 0$$
        c'est-à-dire $\begin{pmatrix} 0 & 1 \\ 0 & 1 \end{pmatrix} X = 0$
        $$\left( \begin{array}{cc|c} 0 & 1 & 0 \\ 0 & 1 & 0 \end{array} \right) \rightsquigarrow
          \left( \begin{array}{cc|c} 0 & 1 & 0 \\ 0 & 0 & 0 \end{array} \right) $$
        L'ensemble de solutions de ce système est
        $$\cL_1 = \left\{(t, 0) \in \R^2 ~ \vert ~ t \in \R \right\}$$
        donc 
        \begin{eqnarray*}
          E_1 &=& \left\{(t, 0) \in \R^2 ~ \vert ~ t \in \R \right\} \\
            &=& \left\{t(1, 0) \in \R^2 ~ \vert ~ t \in \R \right\} \\
            &=& Vect\big((1, 0)\big)
        \end{eqnarray*}
        
      \item pour $\lambda = 2$,
        $$E_2 = Ker(f - 2 id_{\R^2})$$
        Il s'agit de résoudre le système linéaire
        $$\begin{pmatrix} -1 & 1 \\ 0 & 0 \end{pmatrix} X = 0$$
        $$\left( \begin{array}{cc|c} -1 & 1 & 0 \\ 0 & 0 & 0 \end{array} \right) \rightsquigarrow
          \left( \begin{array}{cc|c} 1 & -1 & 0 \\ 0 & 0 & 0 \end{array} \right)$$
        L'ensemble de solutions de ce système est
        $$\cL_2 = \left\{(u, u) \in \R^2 ~ \vert ~ u \in \R \right\}$$
        donc 
        \begin{eqnarray*}
          E_2 &=& \left\{(u, u) \in \R^2 ~ \vert ~ u \in \R \right\} \\
            &=& \left\{u(1, 1) \in \R^2 ~ \vert ~ u \in \R \right\} \\
            &=& Vect\big((1, 1)\big)
        \end{eqnarray*}
    \end{itemize}
  
  \item Examinons si l'on peut trouver une base de $\R^2$ fomrée de vecteurs propres de $f$ posons
    $$\vec{v}_1 = (1, 0) ~ ~ \vec{v}_2 = (1, 1)$$
    $\vec{v}_1$ et $\vec{v}_2$ sont vecteurs propres de $f$ associés aux valeurs propres $1$ et $2$ respectivement.
    $$\begin{vmatrix} 1 & 1 \\ 0 & 1 \end{vmatrix} = 1 \neq 0$$
    donc $(\vec{v}_1, \vec{v}_2)$ est une base de $\R^2$. Par conséquent, $f$ est diagonalisable.
    
  \item Déterminons une matrice inversible $P$ de taille $2 \times 2$ et une matrice diagonale $D$ de taille $2 \times 2$ telles que 
    $$D = P^{-1} M P$$
    On pose $P = P_{\cC \cB}$ matrice de passage de la base canonique $\cC$ à la base $\cB = (\vec{v}_2, \vec{v}_2)$.
    $$P = \begin{pmatrix} 1 & 1 \\ 0 & 1 \end{pmatrix}$$
    Inversons $P$; $det(P) = 1$ et $com(P) = \begin{pmatrix} 1 & 0 \\ -1 & 1 \end{pmatrix}$, donc
    $$P^{-1} = \frac{1}{det(P)} com(P)^{T} = \begin{pmatrix} 1 & -1 \\ 0 & 1 \end{pmatrix}$$
    Calculons
    \begin{eqnarray*}
      P^{-1} M P &=& \begin{pmatrix} 1 & -1 \\ 0 & 1 \end{pmatrix} \begin{pmatrix} 1 & 1 \\ 0 & 2 \end{pmatrix} \begin{pmatrix} 1 & 1 \\ 0 & 1 \end{pmatrix} \\
        &=& \begin{pmatrix} 1 & -1 \\ 0 & 1 \end{pmatrix} \begin{pmatrix} 1 & 2 \\ 0 & 2 \end{pmatrix} \\
        &=& \begin{pmatrix} 1 & 0 \\ 0 & 2 \end{pmatrix} 
    \end{eqnarray*}
    on pose $D = \begin{pmatrix} 1 & 0 \\ 0 & 2 \end{pmatrix}$.
\end{enumerate}

\paragraph{Définition} Par extension, si $A$ est une matrice carrée de taille $n \times n$, on dit que $A$ est diagonalisable s'il ensuit une matrice inversible $P$ de taille $n \times n$ et une matrice diagonale $D$ de taille $n \times n$ tells que
$$D = P^{-1} A P$$

\paragraph{Théorème} Soit $f: E \rightarrow E$ un endomorphisme. Alors $f$ est diagonalisable si et seulement si
$$dim(E_{\lambda_1}) + \ldots + dim(E_{\lambda_r}) = dim(E)$$
où $\lambda_1, \ldots, \lambda_r$ sont les valeurs propres de $f$ deux à deux distinctes. On obtient alors une base de $E$ formée de vecteurs propres de $f$ en concatenant des bases des $E_{\lambda_i}$, c'est-à-dire si
\begin{itemize}
  \item $\cB_1$ est une base de $E_{\lambda_1}$
  \item $\ldots$
  \item $\cB_r$ est une base de $E_{\lambda_r}$
\end{itemize}
alors $\cB_1 \cup \ldots \cup \cB_r$ est une base de $E$.
