\chapter{Le déterminant}

%
%
\section{Groupes symétriques}
%
%
%
\subsection{Permutation}
%
\paragraph{Définition} Soit $E$ un ensemble. On appelle permutation de $E$ un bijection  de $E$ dans $D$. On note $Sym(e)$ l'ensemble de permutations de $E$

\paragraph{Définition} On munit $Sym(E)$ de la composition
\begin{eqnarray*}
  \circ: Sym(E) \times Sym(E) &\rightarrow& Sym(E) \\
  (\sigma , \tau) &\mapsto& \sigma \circ \tau
\end{eqnarray*}
Par définition, pour tout $x\in E$
$$(\sigma \circ \tau)(x) = \sigma(\tau(x))$$
La composée de deux biijections étant encore une bijection, la composée de deux permutations de $E$ est encore une permutation de $E$.

%
\subsection{Groupe symétrique}
%
Alors $(Sym(E), \circ)$ est un groupe. En effet
\begin{itemize}
  \item la composition des permutations est associative
    $$\alpha \circ (\beta \circ \gamma) = (\alpha \circ \beta) \circ \gamma ~ \forall \alpha, \beta, \gamma \in Sym(E)$$
  
  \item l'application
    \begin{eqnarray*}
      id_E: E &\rightarrow& E \\
      x &\mapsto& id_E(x) = x 
    \end{eqnarray*}
    est une permutation de $E$ qui est élément neuter pour $circ$
    
  \item si $\sigma \in Sym(E)$, alors la permutation inverse de $\sigma$ est la bijection réciproque $\sigma^{-1}: E \rightarrow E$ de $\sigma$
\end{itemize}
On appelle ce groupe le groupe symétrique de $E$.

\paragraph{Remarque} Dans la suite nous nous concentrons sur le cas où $E = \{1; 2; \ldots; n\}$ pour un certain entier naturel $n\geq 1$. On note alors
$$S_n = Sym(\{1; 2; \ldots; n\})$$
Si $\sigma \in S_n$, on pourra écrire
$$\sigma = \begin{pmatrix} 1 & 2 & \ldots & n \\ \sigma(1) & \sigma(2) & \ldots & \sigma(n) \end{pmatrix}$$

%
\subsection{Support}
%
\paragraph{Définition} Pour tout $\sigma \in S_n$, on appelle support de $\sigma$ et on note $supp(\sigma)$ l'ensemble
$$supp(\sigma) = \{i \in \{1; 2; \ldots; n\} \vert \sigma(i) \neq i\}$$

%
\subsection{Cycle}
%
\paragraph{Définition} Soit $n \geq 1$ un entier naturel et soit $k \in \{1; 2; \ldots; n\}$. On dit qu'une permutation $\sigma \in S_n$ est un cycle de longeur $k$ ou $k$-cycle, s'il existe $i_1, i_2, \ldots, i_k \in  \{1; 2; \ldots; n\}$ tels que
$$\sigma(i_1) = i_2 \rightarrow \sigma(i_2) = i_3 \rightarrow \ldots \rightarrow \sigma(i_{k-1}) = i_k \rightarrow \sigma(i_k) = i_1$$
et $\sigma(i) = i$ si $i \notin \{i_1; i_2; \ldots; i_k\}$

\paragraph{Remarque} L'ensemble $\{i_1; i_2; \ldots; i_k\}$ est le support du $k$-cycle.

\paragraph{Notation} Soit $\sigma$ un $k$-cycle, on note $\sigma = (i_1 ~ i_2 ~ \ldots ~ i_k)$


%
\subsection{Transposition}
%
\paragraph{Définition} Les 2-cycles de $S_n$ s'appellent les transpositions de $S_n$

\paragraph{Proposition} Soit $m \geq 2$ un entier naturel. Alors toute permutation de $S_n$ peut sécrire comme composée de transpositions.

\paragraph{Démonstration}  On raisonne par récurence sur le cardinal\footnote{Le cardinal d'un ensemble fini est le nombre de ses éléments.} de $supp(\sigma)$.
\subparagraph{Annonce de la récurence} Soit $\sigma \in S_n$ telle que $supp(\sigma)$ soit de cardinal égal à $0$. Donc $supp(\sigma)$ est vide (on note $supp(\sigma) = \emptyset$). Autrement dit, $\sigma$ fixe tous les éléments de $\{1; 2; \ldots; n\}$. Par conséquent, $\sigma = id$. \\
On peut écrire $\sigma = id = (1 ~ 2)\circ(1 ~ 2)$, ce qui est une écriture de $\sigma$ comme composée de transpositions.

\subparagraph{Hypothése de récurence} Soit $k \in \{1; 2; \ldots; n\}$. On suppose que pour tout $\tau \in S_n$, telle que $card(supp(\tau)) \leq k-1$, $\tau$ s'écrit comme composée de transposition.

\subparagraph{Pas de récurence} Soit $\sigma \in S_n$ telle que le cardinal de $supp(\sigma)$ soit égal à $k$.
$$supp(\sigma) = \{i_1; i_2; \ldots; i_k\} \text{ avec } 1 \leq i_1 \leq i_2 \leq \ldots \leq i_k \leq n$$
On forme la permutation $\tau = (i_k ~ \sigma(i_k)) \circ \sigma$. La permutation $\tau$ a un support de cardinal $< k$. En effet, on verifie que $supp(\tau)$ est contenue dans $supp(\sigma)$. Or on a
\paragraph{[TO CHECK]}
$$\tau(i_k) = (i_k ~ \sigma(i_k)) \circ \sigma) (i_k) = (i_k ~ \sigma(i_k))(\sigma(i_k)) = i_k$$
Donc $supp(\tau)$ est contenue dans $\{i_1; i_2; \ldots; i_k-1\}$, donc est de cardinal $\leq k-1 < k$. Par hypothèse de récurence, $\tau$ s'écrit comme composée de transpositions
$$\tau = \tau_1 \circ \tau_2 \circ \ldots \circ \tau_n$$
On en déduit que 
\begin{eqnarray*}
  \sigma &=& (i_k ~ \sigma(i_k))^{-1} \circ \tau \\
    &=& (i_k ~ \sigma(i_k)) \circ \tau_1 \circ \tau_2 \circ \ldots \circ \tau_n
\end{eqnarray*}
\\
Par principe de récurence, la propriété est vraie pour toute permutation de $S_n$.

\paragraph{Proposition} Soit $n \geq 1$ un entier naturel. Alors $S_n$ est de cardinal $n! = 1 \cdot 2 \codt \ldots \cdot n$.

\paragraph{Démonstration} Pour construire une permutation $\sigma$ de $S_n$,
\begin{itemize}
  \item on choisit l'image $\sigma(1)$ de $1$ pour $\sigma$, il y a $n$ choix possibles.
  \item puis on choisit $\sigma(2)$ qui est l'image de $2$ par $\sigma$, il y a $n-1$ choix possile dans $\{1; 2; \ldots; n\} \backslash \{\sigma(1)\}$.
  \item puis on choisit $\sigma(3)$; il y a $n-1$ choix possibles dans $\{1; 2; \ldots; n\} \backslash \{\sigma(1); \sigma(2)\}$.
  \item ainsi de suite
\end{itemize}
Par conséquent, le nombre de permutations que l'on peut construire est 
$$n \cdot (n-1) \cdot \ldots \cdot 2 \cdot 1 = n!$$

\paragraph{Example} Décrivons $S_2$ et $S_3$
\begin{itemize}
  \item $S_2$ est de cardinal $2! = 2$. On a 
    $$S_2 = \{id; (1 ~ 2)\}$$
  \item $S_3$ est de cardinal $3! = 6$. On a
    $$S_3 = \{id; (1 ~ 2); (1 ~ 3); (2 ~ 3); (1 ~ 2 ~ 3); (1 ~ 3 ~ 2)\}$$
\end{itemize}

%
\subsection{Inversion de paire}
%
\paragraph{Définition} Soient $i, j \in \{1; 2; \ldots; n\}$, $i < j$, et $\sigma \in S_n$. On dit que $\sigma$ présente une inversion en la paire $(i, j)$ si 
$$\sigma(i) > \sigma(j)$$

%
\subsection{Permutation paire}
%
\paragraph{Définition} Soit $\sigma \in S_n$. On dit que $\sigma$ est une permutation paire si elle présente un nombre paire d'inversions, et une permutation impaire sinon.

%
\subsection{Signature}
%
\paragraph{Définition} On appelle signature de $\sigma$, et on note $\epsilon(\sigma)$ le nombre
$$\epsilon(\sigma) \left\{ \begin{array}{lr} 1 & \text{si } \sigma \text{ est une permutation paire} \\ -1 & \text{si } \sigma \text{ est une permutation impaire} \end{array}$$
Autrement dit, si $n_{\sigma}$ est le nombre d'inversions de $\sigma$ on a
$$\epsilon(\sigma) = (-1)^{n_{\sigma}} \in \{-1; 1\}$$

\paragraph{Example}
$$\sigma = \begin{pmatrix} 1 & 2 & 3 & 4 & 5 \\ 3 & 1 & 5 & 2 & 4 \end{pmatrix} \in S_5$$
Paires d'inversions
$$(1, 2), (1, 4), (3, 4), (3, 5)$$
$\sigma$ presente $4$ inversion, $\sigma$ est donc paire et $\epsilon(\sigma) = 1$

\paragraph{Théorème} Soient $\sigma, \tau \in S_n$. Alors
$$\epsilon(\sigma \circ \tau) = \epsilon(\sigma) \cdot \epsilon(\tau)$$

%
\subsection{Produit élémentaire}
%
\paragraph{Définition} Soit $A = (a_{ij})$ une matrice carrée de taille $n\times n$. Soit $\sigma in S_n$. On appelle produit élémentaire associé à $\sigma$ le nombre
$$a_{\sigma(1) 1} \cdot a_{\sigma(2) 2} \cdot \ldots \cdot a_{\sigma(n) n} = \prod_{i=1}^{n} a_{\sigma(i) i}$$
%
\subsection{Produit signé}
%
\paragraph{Définition} On appelle produit signé associé à $\sigma$ le nombre 
$$p_{\sigma}(A) = \epsilon(\sigma) \cdot a_{\sigma(1) 1} \cdot a_{\sigma(2) 2} \cdot \ldots \cdot a_{\sigma(n) n} = \epsilon(\sigma) \prod_{i=1}^{n} a_{\sigma(i) i}$$

%
\subsection{Déterminant}
%
\paragraph{Définition} Soit $A = (a_{ij})_{1 \leq i, j \leq n}$ une matrice carrée de taille $n \times n$. On appelle déterminant de $A$ le nombre
$$det(A) = \sum_{\sigma \in S_n} p_{\sigma}(A) = \sum_{\sigma \in S_n} \epsilon(\sigma) \prod_{i=1}^{n} a_{\sigma(i) i}$$

\paragraph{Example} Soit $A = \begin{pmatrix} 1 & 2 & 3 \\ 0 & 5 & 1 \\ -7 & 4 & -2 \end{pmatrix} \in M_{3}(\R)$.
$$S_3 = \{id; (1 ~ 2); (1 ~ 3); (2 ~ 3); (1 ~ 2 ~ 3); (1 ~ 3 ~ 2)\}$$
On a 
\begin{eqnarray*}
  \sigma = id           &\Rightarrow& p_{\sigma}(A) = 1 \cdot A_{11} \cdot A_{22} \cdot A_{33} \\
  \sigma = (1 ~ 2)      &\Rightarrow& p_{\sigma}(A) = 1 \cdot A_{21} \cdot A_{21} \cdot A_{33} \\
  \sigma = (1 ~ 3)      &\Rightarrow& p_{\sigma}(A) = 1 \cdot A_{31} \cdot A_{22} \cdot A_{13} \\
  \sigma = (2 ~ 3)      &\Rightarrow& p_{\sigma}(A) = 1 \cdot A_{11} \cdot A_{32} \cdot A_{23} \\
  \sigma = (1 ~ 2 ~ 3)  &\Rightarrow& p_{\sigma}(A) = 1 \cdot A_{21} \cdot A_{32} \cdot A_{13} \\
  \sigma = (1 ~ 3 ~ 2)  &\Rightarrow& p_{\sigma}(A) = 1 \cdot A_{31} \cdot A_{12} \cdot A_{23}
\end{eqnarray*}

\paragraph{Notation} Si $A = (a_{ij})_{1 \leq i, j\leq n}$ est une matrice carrée de taille $n\times n$ on note
$$\vert a_{ij} \vert _{1 \leq i, j \leq n} = det(A)$$

%
\subsection{Déterminant d'un système des matrices colonnes}
%
\paragraph{Définition} On peut aussi parler du döterminant d'un système de $n$ matrices colonnes $m$-lignes
$$C_1, C_2, \ldots, C_n  \in M_{n\times 1}(\R)$$
Suivant
$$C_{j} = \begin{pmatrix} c_{1j} \\ c_{2j} \\ \vdots \\ c_{nj} \end{pmatrix} \text{ pour tout } j \in \{1; \ldots ; n\}$$
on définit le déterminant du système $(C_1, C_2, \ldots, C_n)$ par
$$det(C_1, C_2, \ldots, C_n) = \sum_{\sigma \in S_n} \epsilon(\sigma) \cdot c_{\sigma(1) 1} \cdot \ldots \cdot c_{\sigma(n) n} \in \R$$
Si $C$ est la matrice de taille $n \times n$ dont les colonnes sont les $C_j$, $C = (C_1, C_2, \ldots, C_n)$ alors on a
$$det(C_1, C_2, \ldots, C_n) = det(C)$$
\paragraph{Remarque} Le point de vue adopté, matrice ou système de matrices colonnes sera clair d'après la notation et le contexte.

\paragraph{Théorème} Soit $n\geq 1$ un entier naturel. Alors l'application $n$ fois
\begin{eqnarray*}
  det: M_{n\times 1}(\R) \times M_{n\times 1}(\R) \times \ldots \times M_{n\times 1}(\R) &\rightarrow& \R \\
  (C_1, C_2, \ldots, C_n) &\mapsto& det(C_1, C_2, \ldots, C_n)
\end{eqnarray*}
posséde les propriétés suivantes
\begin{enumerate}
  \item Pour tout $j \in \{1; \ldots; n \}$, pour tous $C_1, \ldots, C_j, C_j', \ldots, C_n \in M_{n\times 1}(\R)$
    \begin{eqnarray*}
      det(C_1, \ldots, C_j + C_j', \ldots, C_n) =& det(C_1, \ldots, C_j, \ldots, C_n) \\
        &+ det(C_1, \ldots, C_j', \ldots, C_n)
    \end{eqnarray*}
    
  \item Pour tout $j \in \{1; \ldots; n\}$, pour tous $C_1, C_2, \ldots, C_n \in M_{n\times 1}(\R)$, pour tout $\lambda \in \R$
    $$det(C_1, \ldots, \lambda C_j, \ldots, C_n) = \lambda \cdot det(C_1, \ldots, C_j, \ldots, C_n)$$
    
  \item Pour tous $C_1, C_2, \ldots, C_n \in M_{n\times 1}(\R)$, pour tous $i, j \in \{1; \ldots; n\}, i \neq j$, si $C_i = C_j$, alors
    $$det(C_1, \ldots, C_i, \ldots, C_j(=C_i), \ldots, C_n) = 0$$
\end{enumerate}

\paragraph{Démonstration} 
\begin{enumerate}
  \item On a
    $$det(C_1, \ldots, C_j + C_j', \ldots, C_n) = \sum_{\sigma \in S_n} p_\sigma(C_1, \ldots, C_j + C_j' , \ldots, C_n)$$
    où
    $$ p_{\sigma}(C_1, \ldots, C_j + C_j', \ldots, C_n) = \epsilon(\sigma) \cdot c_{\sigma(1) 1} \cdot \ldots \cdot (c_{\sigma(j) j} + c_{\sigma(j) j}') \cdot \ldots \cdot c_{\sigma(n) n}$$
    $\Rightarrow$ distributivité de $\circ$ par rapport à $+$
   
  \item On a
    $$det(C_1, \ldots, \lambda C_j', \ldots, C_n) = \sum_{\sigma \in S_n} p_\sigma(C_1, \ldots, \lambda C_j' , \ldots, C_n)$$
    où
    $$ p_{\sigma}(C_1, \ldots, \lambda C_j', \ldots, C_n) = \epsilon(\sigma) \cdot c_{\sigma(1) 1} \cdot \ldots \cdot \lambda \cdot c_{\sigma(j) j}  \cdot \ldots \cdot c_{\sigma(n) n}$$

  \item Soit $\tau = (i ~ j) \in S_n$. On regroupe les permutations de $S_n$ par paires comme suit: Chaque $\sigma \in S_n$ avec $\sigma \circ \tau \in S_n$ \\
    On a: si $\sigma = \sigma \circ \tau$ on aurait
    \begin{eqnarray*}
      id = \sigma^{-1} \circ \sigma &=& \sigma^{-1} \circ (\sigma \circ \tau) \\
        &=& (\sigma^{-1} \circ \sigma) \circ \tau \\
        &=& id \circ \tau = \tau \rightarrow\text{ impossibru}
    \end{eqnarray*}
    donc $\sigma \neq \sigma \circ \tau$. On a
    \begin{eqnarray*}
      (\sigma \circ \tau) \circ \tau &=& \sigma \circ (\tau \circ \tau) \\
        &=& \sigma \circ (id) \\
        &=& \sigma 
    \end{eqnarray*}
    %
    On obtien aisni $\frac{n!}{2}$ paires de la forme $\{\sigma; \sigma  \circ \tau\}$. Dans chaque de ces paires on sélectionne une permutation $\sigma_n$ tout, on a donc sélectioné $\frac{n!}{2}$ permutations $\sigma_1, \sigma_2, \ldots, \sigma_{\frac{n!}{2}}$. On a donc
    $$S_n = \{ \sigma_1; \sigma_1 \circ \tau; \sigma_2; \sigma_2 \circ \tau; \ldots; \sigma_{\frac{n!}{2}}; \sigma_{\frac{n!}{2}} \circ \tau \}$$
    On obtient 
    \begin{eqnarray*}
      det(C_1, C_2, \ldots, C_n) &=& \sum_{\sigma \in S_n} p_{\sigma}(C_1, C_2, \ldots, C_n) \\
       &=& \sum_{i=1}^{\frac{n!}{2}} p_{\sigma_i}(C_1, C_2, \ldots, C_n) + p_{\sigma_i \circ \tau}(C_1, C_2, \ldots, C_n)
    \end{eqnarray*}
    %
    Montrons que pour tout $\sigma \in S_n$, on a
    $$p_{\sigma \circ \tau}(C_1, C_2, \ldots, C_n) = -p_{\sigma}(C_1, C_2, \ldots, C_n)$$
    Soit $\sigma \in S_n$. On a 
    $$p_{\sigma \circ \tau}(C_1, C_2, \ldots, C_n) = \epsilon(\sigma \circ \tau) \cdot a_{(\sigma \circ\tau)(1) 1} \cdot a_{(\sigma \circ\tau)(2) 2} \cdot \ldots \cdot a_{(\sigma \circ\tau)(n) n}$$
    Soit $k \in \{1; \ldots; n\}$
    \begin{itemize}
      \item si $k\notin \{i; j\}$, alors $\tau(k) = k$. Par suite 
        $$a_{(\sigma \circ \tau)(k) k} = a_{\sigma(k) k}$$
      
      \item si $k = i$, alors $\tau(k) = j$. Par suite $a_{(\sigma \circ \tau)(k) k} = a_{\sigma(j) i}$. Puisque $C_i = C_j$ on a $a_{\sigma(j) i} = a_{\sigma(j) j}$. On a donc
        $$a_{(\sigma \circ \tau)(k) k} = a_{(\sigma \circ \tau)(i) i} = a_{\sigma(j) j}$$
        
      \item si $k = j$, alors $\tau(k) = i$. Par suite $a_{(\sigma \circ \tau)(k) k} = a_{\sigma(i) j}$. Puisque $C_j = C_i$ on a $a_{\sigma(i) j} = a_{\sigma(i) i}$. On a donc
        $$a_{(\sigma \circ \tau)(k) k} = a_{(\sigma \circ \tau)(j) j} = a_{\sigma(i) i}$$
    \end{itemize}
    On a donc
    \begin{eqnarray*}
      a_{(\sigma \circ \tau)(1) 1} \cdot \ldots \cdot a_{(\sigma \circ \tau)(i) i} \cdot \ldots \cdot a_{(\sigma \circ \tau)(j) j} \cdot \ldots \cdot a_{(\sigma \circ \tau)(n) n} \\
        = a_{\sigma(1) 1} \cdot \ldots \cdot a_{\sigma(j) j} \cdot \ldots \cdot a_{\sigma(i) i} \cdot \ldots \cdot a_{\sigma(n) n} \\
        = a_{\sigma(1) 1} \cdot \ldots \cdot a_{\sigma(i) i} \cdot \ldots \cdot a_{\sigma(j) j} \cdot \ldots \cdot a_{\sigma(n) n} \\
    \end{eqnarray*}
    %
    De plus $\epsilon(\sigma \circ \tau) = \epsilon(\sigma) \cdot \epsilon(\tau)$ où on a $\epsilon(\tau) = -1$, donc
    $$\epsilon(\sigma \circ \tau) = -\epsilon(\sigma)$$
    Par conéquent 
    $$p_{\sigma \circ \tau}(C_1, C_2, \ldots, C_n) = -p_{\sigma}(C_1, C_2, \ldots, C_n)$$
    Ainsi pour tout $i\in \{1; \ldots; \frac{n!}{2}\}$
    \begin{eqnarray*}
      p_{\sigma_i \circ \tau}(C_1, C_2, \ldots, C_n) &=& -p_{\sigma_i}(C_1, C_2, \ldots, C_n) \\
      p_{\sigma_i \circ \tau}(C_1, C_2, \ldots, C_n) &+& p_{\sigma_i}(C_1, C_2, \ldots, C_n) ~=~ 0
    \end{eqnarray*}
    Il s'ensuit que $det(C_1, \ldots, C_i, \ldots, C_j(=C_i), \ldots, C_n) = 0$
\end{enumerate}

\paragraph{Corollaire} Soient $i, j \in \{1; \ldots; n\}$, $i\neq j$ et $C_1, C_2, \ldots, C_n$ $n$ matrices colonnes £ $n$ lignes. Alors
$$det(C_1, \ldots, C_i, \ldots, C_j, \ldots, C_n) = -det(C_1, \ldots, C_j, \ldots, C_i, \ldots, C_n)$$

\paragraph{Démonstration} On a
\begin{eqnarray*}
  0 =& det(C_1, \ldots, C_i+C_j, \ldots, C_i+C_j, \ldots, C_n) \\
    =& det(C_1, \ldots, C_i, \ldots, C_i+C_j, \ldots, C_n) \\
    &+ det(C_1, \ldots, C_j, \ldots, C_i+C_j, \ldots, C_n) \\
    =& det(C_1, \ldots, C_i, \ldots, C_i, \ldots, C_n) \\
    &+ det(C_1, \ldots, C_i, \ldots, C_j, \ldots, C_n) \\
    &+ det(C_1, \ldots, C_j, \ldots, C_i, \ldots, C_n) \\
    &+ det(C_1, \ldots, C_j, \ldots, C_j, \ldots, C_n) \\
    =& 0 \\
    &+ det(C_1, \ldots, C_i, \ldots, C_j, \ldots, C_n) \\
    &+ det(C_1, \ldots, C_j, \ldots, C_i, \ldots, C_n) \\
    &+ 0 \\
    =& det(C_1, \ldots, C_i, \ldots, C_j, \ldots, C_n) + det(C_1, \ldots, C_j, \ldots, C_i, \ldots, C_n) = 0
\end{eqnarray*}
On dit alors que l'application 
$$det: M_{n \times 1}(\R) \times \ldots \times M_{n\times 1}(\R) \rightarrow \R$$
est antisymétrique.

\paragraph{Théorème} Soit $A =(a_{ij})$ une matrice carrée de taille $n\times n$. Alors
$$det(A^{T}) = det(A)$$

\paragraph{Démonstration} On observe que l'application
\begin{eqnarray*}
  S_n &\rightarrow& S_n \\
  \sigma &\mapsto& \sigma^{-1}
\end{eqnarray*}
est bijective (elle est sa propre bijection réciproque). Il vient
\begin{eqnarray*}
  det(A^{T}) &=& \sum_{\sigma \in S_n} p_{\sigma}(A^{T}) \\
    &=& \sum_{\sigma^{-1} \in S_n} p_{\sigma^{-1}}(A^{T})
\end{eqnarray*}
Soit $\sigma \in S_n$
$$p_{\sigma^{-1}}(A^{T}) = \epsilon(\sigma^{-1}) \cdot (A^{T})_{\sigma^{-1}(1) 1} \cdot \ldots \cdot (A^{T})_{\sigma^{-1}(n) n}$$
On a $\epsilon(\sigma) \epsilon(\sigma^{-1}) = \epsilon(\sigma \circ \sigma^{-1}) = \epsilon(id) = 1$. Puisque $\epsilon(\sigma) \pm 1$ et $\epsilon(\sigma^{-1}) \pm 1$, on a 
$$\epsilon(\sigma) = \epsilon(\sigma^{-1})$$
Maintenant
$$(A^{T})_{\sigma^{-1}(1) 1} \cdot \ldots \cdot (A^{T})_{\sigma^{-1}(n) n} = A_{1 \sigma^{-1}(1)} \cdot \ldots \cdot A_{n \sigma^{-1}(n)}$$
Comme $\sigma^{-1}: \{1; \ldots; n\} \rightarrow \{1; \ldots; n\}$ est bijective on a
\begin{eqnarray*}
  A_{1 \sigma^{-1}(1)} \cdot \ldots \cdot A_{n \sigma^{-1}(n)} &=& \prod_{j=1}^n A_{j \sigma^{-1}(j)} \\
    &=& \prod_{k=1}^n A_{\sigma(k) k} ~~ (\text{on a posé } k = \sigma^{-1}(j))
\end{eqnarray*}
Donc $p_{\sigma^{-1}}(A^{T}) = p_{\sigma}(A)$. Par suite
$$det(A^{T}) = \sum_{\sigma \in S_n} p_{\sigma}(A) = det(A)$$
