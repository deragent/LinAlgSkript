\chapter{Le déterminant}

%
%
\section{Groupes symétriques}
%
%
%
\subsection{Permutation}
%
\paragraph{Définition} Soit $E$ un ensemble. On appelle permutation de $E$ un bijection  de $E$ dans $D$. On note $Sym(e)$ l'ensemble de permutations de $E$

\paragraph{Définition} On munit $Sym(E)$ de la composition
\begin{eqnarray*}
  \circ: Sym(E) \times Sym(E) &\rightarrow& Sym(E) \\
  (\sigma , \tau) &\mapsto& \sigma \circ \tau
\end{eqnarray*}
Par définition, pour tout $x\in E$
$$(\sigma \circ \tau)(x) = \sigma(\tau(x))$$
La composée de deux biijections étant encore une bijection, la composée de deux permutations de $E$ est encore une permutation de $E$.

%
\subsection{Groupe symétrique}
%
Alors $(Sym(E), \circ)$ est un groupe. En effet
\begin{itemize}
  \item la composition des permutations est associative
    $$\alpha \circ (\beta \circ \gamma) = (\alpha \circ \beta) \circ \gamma ~ \forall \alpha, \beta, \gamma \in Sym(E)$$
  
  \item l'application
    \begin{eqnarray*}
      id_E: E &\rightarrow& E \\
      x &\mapsto& id_E(x) = x 
    \end{eqnarray*}
    est une permutation de $E$ qui est élément neuter pour $circ$
    
  \item si $\sigma \in Sym(E)$, alors la permutation inverse de $\sigma$ est la bijection réciproque $\sigma^{-1}: E \rightarrow E$ de $\sigma$
\end{itemize}
On appelle ce groupe le groupe symétrique de $E$.

\paragraph{Remarque} Dans la suite nous nous concentrons sur le cas où $E = \{1; 2; \ldots; n\}$ pour un certain entier naturel $n\geq 1$. On note alors
$$S_n = Sym(\{1; 2; \ldots; n\})$$
Si $\sigma \in S_n$, on pourra écrire
$$\sigma = \begin{pmatrix} 1 & 2 & \ldots & n \\ \sigma(1) & \sigma(2) & \ldots & \sigma(n) \end{pmatrix}$$

%
\subsection{Support}
%
\paragraph{Définition} Pour tout $\sigma \in S_n$, on appelle support de $\sigma$ et on note $supp(\sigma)$ l'ensemble
$$supp(\sigma) = \{i \in \{1; 2; \ldots; n\} \vert \sigma(i) \neq i\}$$

%
\subsection{Cycle}
%
\paragraph{Définition} Soit $n \geq 1$ un entier naturel et soit $k \in \{1; 2; \ldots; n\}$. On dit qu'une permutation $\sigma \in S_n$ est un cycle de longeur $k$ ou $k$-cycle, s'il existe $i_1, i_2, \ldots, i_k \in  \{1; 2; \ldots; n\}$ tels que
$$\sigma(i_1) = i_2 \rightarrow \sigma(i_2) = i_3 \rightarrow \ldots \rightarrow \sigma(i_{k-1}) = i_k \rightarrow \sigma(i_k) = i_1$$
et $\sigma(i) = i$ si $i \notin \{i_1; i_2; \ldots; i_k\}$

\paragraph{Remarque} L'ensemble $\{i_1; i_2; \ldots; i_k\}$ est le support du $k$-cycle.

\paragraph{Notation} Soit $\sigma$ un $k$-cycle, on note $\sigma = (i_1 ~ i_2 ~ \ldots ~ i_k)$


%
\subsection{Transposition}
%
\paragraph{Définition} Les 2-cycles de $S_n$ s'appellent les transpositions de $S_n$

\paragraph{Proposition} Soit $m \geq 2$ un entier naturel. Alors toute permutation de $S_n$ peut sếcrire comme composée de transpositions.

\paragraph{Démonstration}  On raisonne par récurence sur le cardinal\footnote{Le cardinal d'un ensemble fini est le nombre de ses éléments} de $supp(\sigma)$.
\subparagraph{Annonce de la récurence} Soit $\sigma \in S_n$ telle que $supp(\sigma)$ soit de cardinal égal à $0$. Donc $supp(\sigma)$ est vide (on note $supp(\sigma) = \emptyset$). Autrement, $\sigma$ fixe tous les éléments de $\{1; 2; \ldots; n\}$. Par conséquent, $\sigma = id$. \\
On peut écrire $\sigma = id = (1 ~ 2)\circ(1 ~ 2)$, ce qui est une écriture de $\sigma$ comme composée de transpositions.

\subparagraph{Hypothése de récurence} Soit $k \in \{1; 2; \ldots; n\}$. On suppose que pour tout $\tau \in S_n$, telle que $card(supp(\tau)) \leq k-1$, $\tau$ s'écrit comme composée de transposition.

\subparagraph{Pas de récurence} Soit $\sigma \in S_n$ telle que le cardinal de $supp(\sigma)$ soit égal à $k$.
$$supp(\sigma) = \{i_1; i_2; \ldots; i_k\} \text{ avec } 1 \leq i_1 \leq i_2 \leq \ldots \leq i_k \leq n$$
On forme la permutation $\tau = (i_k ~ \sigma(i_k)) \circ \sigma$. La permutation $\tau$ a un support de cardinal $< k$. En effet, on verifi que $supp(\tau)$ est contenue dans $supp(\sigma)$. Or on a
\paragraph{[TO CHECK]}
$$\tau(i_k) = (i_k ~ \sigma(i_k)) \circ \sigma) (i_k) = (i_k ~ \sigma(i_k))(\sigma(i_k)) = i_k$$
Donc $supp(\tau)$ est contenue dans $\{i_1; i_2; \ldots; i_k-1\}$, donc est de cardinal $\leq k-1 < k$. Par hypothèse de récurence, $\tau$ s'écrit comme composée de transpositions
$$\tau = \tau_1 \circ \tau_2 \circ \ldots \circ \tau_n$$
On en déduit que 
\begin{eqnarray*}
  \sigma &=& (i_k ~ \sigma(i_k))^{-1} \circ \tau \\
    &=& (i_k ~ \sigma(i_k)) \circ \tau_1 \circ \tau_2 \circ \ldots \circ \tau_n
\end{eqnarray*}
\\
Par principe de récurence, la propriété est vraie pour toute permutation de $S_n$.

\paragraph{Proposition} Soit $n \geq 1$ un entier naturel. Alors $S_n$ est de cardinal $n! = 1 \cdot 2 \codt \ldots \cdot n$.

\paragraph{Démonstration} Pour construire une permutation $\sigma$ de $S_n$,
\begin{itemize}
  \item on choisit l'image $\sigma(1)$ de $1$ pour $\sigma$, il y a $n$ choix possibles.
  \item puis on choisit $\sigma(2)$ qui est l'image de $2$ par $\sigma$, il y a $n-1$ choix possile dans $\{1; 2; \ldots; n\} \backslash \{\sigma(1)\}$.
  \item puis on choisit $\sigma(3)$; il y a $n-1$ choix possibles dans $\{1; 2; \ldots; n\} \backslash \{\sigma(1); \sigma(2)\}$.
  \item ainsi de suite
\end{itemize}
Par conséquent, le nombre de permutations que l'on peut construire est 
$$n \cdot (n-1) \cdot \ldots \cdot 2 \cdot 1 = n!$$

\paragraph{Example} Décrivons $S_2$ et $S_3$
\begin{itemize}
  \item $S_2$ est de cardinal $2! = 2$. On a 
    $$S_2 = \{id; (1 ~ 2)\}$$
  \item $S_3$ est de cardinal $3! = 6$. On a
    $$S_3 = \{id; (1 ~ 2); (1 ~ 3); (2 ~ 3); (1 ~ 2 ~ 3); (1 ~ 3 ~ 2)\}$$
\end{itemize}

%
\subsection{Inversion de paire}
%
\paragraph{Définition} Soient $i, j \in \{1; 2; \ldots; n\}$, $i < j$, $est \sigma \in S_n$. On dit que $\sigma$ présente une inversion en la paire $(i, j)$ si 
$$\sigma(i) > \sigma(j)$$

%
\subsection{Permutation paire}
%
\paragraph{Définition} Soit $\sigma \in S_n$. On dit que $\sigma$ est une permutation paire si elle présente un nombre paire d'inversions, et une permutation impaire sinon.

%
\subsection{Signature}
%
\paragraph{Définition} On appelle signature de $\sigma$, et on note $\epsilon(\sigma)$ le nombre
$$\epsilon(\sigma) \left\{ \begin{array}{lr} 1 & \text{si } \sigma \text{ est une permutation paire} \\ -1 & \text{si } \sigma \text{ est une permutation impaire} \end{array}$$
Autrement dit, si $n_{\sigma}$ est le nombre d'inversions de $\sigma$ on a
$$\epsilon(\sigma) = (-1)^{n_{\sigma}} \in \{-1; 1\}$$

\paragraph{Example}
$$\sigma = \begin{pmatrix} 1 & 2 & 3 & 4 & 5 \\ 3 & 1 & 5 & 2 & 4 \end{pmatrix} \in S_5$$
Paires d'inversions
$$(1, 2), (1, 4), (3, 4), (3, 5)$$
$\sigma$ presente $4$ inversion, $\sigma$ est donc paire et $\epsilon(\sigma) = 1$

\paragraph{Théorème} Soient $\sigma, \tau \in S_n$. Alors
$$\epsilon(\sigma \circ \tau) = \epsilon(\sigma) \cdot \epsilon(\tau)$$
