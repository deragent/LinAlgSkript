\chapter{Calcul matriciel}

%
%
\section{Matrices}
%
%

%
\subsection{Matrice}
%
\paragraph{Définition} Soient $n \geq 1$ et $m \geq 1$ deux entiers naturels. On appelle matrice de taille $n\times m$ à coefficients dans $\R$ un tableau à $n$ lignes et $m$ colonnes de la forme
$$\begin{pmatrix}
  a_{11} & a_{12} & \ldots & a_{1m} \\
  a_{21} & a_{22} & \ldots & a_{2m} \\
  \vdots & \vdots & \ddots & \vdots \\
  a_{n1} & a_{n2} & \ldots & a_{nm}
\end{pmatrix} \text{ avec } a_{ij} \in \R$$

\paragraph{Exemple}
$$\begin{pmatrix}
  2 & -5 & \sqrt{2} \\
  3 &0 & 1 \\
\end{pmatrix} \text{ est une matrice de taille $2\times 3$ à coefficients dans $\R$. Le coefficient $a_{12}$ est -5.} $$

%
\subsection{Ensemble des matrices}
%
\paragraph{Définition} On note $M_{n, m}(\R)$ l'ensemble des matrices de taille $n\times m$ à coefficients dans $\R$.

%
\subsection{Matrice carrée}
%
\paragraph{Définition} Soit $n \geq 1$ un entier naturel et $A$ une matrice de taille $n\times n$. On dit alors que $A$ est une matrice carrée. \\ 
On note $M_{n}(\R)$ l'ensemble des matrices carrées de taille $n\times n$ à coefficients dans $\R$. Les coefficients $a_{ii}$ pour $1 \leq i \leq n$ sont appellés coefficients diagonaux de $A$.

%
%
\section{Opérations sur les matrices}
%
%

%
\subsection{Somme}
%
\paragraph{Définition} Soient $m\geq 1$ et $n \geq 1$ deux entiers naturels. On peut définir la somme de deux matrices de même taille. Si $A = (a_{ij}), B = (b_{ij}) \in M_{n, m}(\R)$ on définit leur somme $S = A + B \in M_{n, m}(\R)$ par
$$s_{ij} = a_{ij} + b_{ij} ~ \forall ~ 1 \leq i \leq n, 1 \leq j \leq m$$

\paragraph{Proposition} Soient $m \geq 1$ et $n\geq 1$ des entiers naturels.
\begin{enumerate}
  \item Soient $A, B, C \in M_{n, m}(\R)$. Alors
    $$(A + B) + C = A + (B + C)$$
  \item On note $0$ la matrice nulle dans $M_{n, m}(\R)$, c'est-à-dire la matrice dont tous les coefficients sont nuls. Alors pour tout $A \in M_{n, m}(\R)$ on a
    $$A + 0 = A = 0 + A$$
  \item Soit $A \in M_{n, m}(\R)$. On note $-A$ la matrice $(-a_{ij})$ dans l'ensemble $M_{n, m}(\R)$. Alors on a
    $$(-A) + A = 0 = A + (-A)$$
\end{enumerate}
Ainsi, l'addition $+$ dans $M_{n, m}(\R)$ est associative, unifère et tout $A \in M_{n, m}(\R)$ admet une matrice opposée. Par conséquence $(M_{n, m}(\R), +)$ est un groupe.\\
De plus, comme $+$ dans $\R$ est commutative, l'addition dans $M_{n,m}(\R)$ est commutative:
$$\forall A, B \in M_{n, m}(\R), ~ A + B = B + A$$
En conclusion, $(M_{n, m}(\R), +)$ est un groupe commutatif.

\paragraph{Démonstration}
\begin{enumerate}
  \item Cela découle de l'associativié de l'addition dans $\R$.
  \item Cela découle de la propriété de l'élément $0$ dans $\R$.
  \item Cela résulte des propriétés de l'opposé d'un réel.
\end{enumerate}

%
\subsection{Produit avec un réel}
%
\paragraph{Définition} Soient $n \geq 1, m \geq 1$ deux entiers naturels, $A = (a_{ij}) \in M_{n, m}(\R)$ et $\alpha \in \R$. On  définit le produit de $A$ par $\alpha$ que l'on note $\alpha A$, par
$$\alpha A = (\alpha a_{ij}) \in M_{n, m}(\R)$$

\paragraph{Proposition} Soient $n \geq 1, m \geq 1$ des entiers naturels. On a
\begin{enumerate}
  \item $\forall \alpha \in \R, ~ A, B \in M_{n, m}(\R)$ 
    $$\alpha \cdot (A + B) = \alpha \cdot A + \alpha \cdot B$$
    
  \item $\forall \alpha, \beta \in \R, ~ A \in M_{n, m}(\R)$
    \begin{eqnarray*}
      (\alpha + \beta) \cdot A &=& \alpha \cdot A + \beta \cdot A \\
      (\alpha \cdot \beta) \cdot A &=& \alpha \cdot (\beta \cdot A)
    \end{eqnarray*}
    
  \item $\forall A \in M_{n, m}(\R)$
    $$1 \cdot A = A$$
\end{enumerate}
On peut donc dire, que $M_{n, m}(\R)$ muni de l'addition des matrices et de la multiplication par les réels $(M_{n, m}(\R), +, \cdot)$ est un $\R$-espace vectoriel.

%
\subsection{Produit matriciel}
%
\paragraph{Définition} Soient $n \geq 1, m \geq 1$ et $p \geq 1$ trois entiers naturels. On peut définir le produit $A \times B$ d'une matrice $A$ et d'une matrice $B$ dans cet ordre si le nombre de colonnes de $A$ est égal au nombre de lignes de $B$. \\
Si $A = (a_{ij}) \in M_{n, m}(\R)$ et $B = (b_{ij}) \in M_{m, p}(\R)$, alors on définit $C = A \times B \in M_{n, p}(\R)$
$$c_{ij} = \sum_{l=1}^{m} a_{il} * b_{lj} = a_{i1} \cdot b_{1j} + a_{i2} \cdot b_{2j} + \ldots + a_{im} \cdot b_{mj}$$

\paragraph{Exemple} Soient $A = 
\begin{pmatrix}
  2 & -5 & 1 \\
  0 & 4 & 7 \\
\end{pmatrix} \in M_{2,3}(\R)$ et $B = 
\begin{pmatrix}
  1 & -5 & 2 & 4 \\
  2 & 4 & -7 & 3\\
  0 & 1 & 0 & 1 \\
\end{pmatrix} \in M_{3,4}(\R)$. Le nombre de colonnes de $A$ est égal au nombre de lignes de $B$. Le produit est donc défini. 
$$AB = 
\begin{pmatrix}
  -8 & -29 & 39 & -6 \\
  8 & 23 & -28 & 19 \\
\end{pmatrix} \in M_{2,4}(\R)$$

\paragraph{Proposition} A conditions que les produits et les sommes considerés soient bien définit, on a
\begin{enumerate}
  \item $$A \times (B \times C) = (A \times B) \times C$$
  \item $$A \times (B + C) = A \times B + A \times C$$ 
    $$(B + C) \times A = B \times A + C \times A$$
\end{enumerate}

\paragraph{Attention} Deux matrices quelconques ne commutent pas nécessairement.
\paragraph{Exemple} Soient $A = \begin{pmatrix} 1 & 3 \\ 2 & 7 \end{pmatrix}$ et $B = \begin{pmatrix} -3 & 0 \\ 4 & 1 \end{pmatrix}$. Alors on a
$$A B = \begin{pmatrix} 9 & 3 \\ 22 & 7 \end{pmatrix}$$
$$B A = \begin{pmatrix} -3 & -9 \\ 6 & 19 \end{pmatrix}$$
On voit donc que $A B \neq B A$.

\paragraph{Attention}
\begin{itemize}
  \item Si $A$ et $B$ sont deux matrices on peut avoir $A \neq 0, B \neq 0$ et $AB = 0$
  \item On peut donc avoir également $AB = AC$ et $B \neq C$ pour des matrices $A, B, C$.
\end{itemize}

\paragraph{} On a, $(M_{n, m}(\R), +, \times, \cdot)$ est une $\R$-algèbre. En particulier $(M_{n, m}(\R), +, \times)$ est un anneau.

%
%
\section{Inversion des matrices}
%
%


%
\subsection{Matrice identité}
%
\paragraph{Définition} Pour tout entier $n \geq 1$ on appelle matrice identité de taille $n \times n$ la matrice carrée
    $$I_n = (a_{ij}) ~ ~ a_{ij} = \left\{ \begin{array}{lr} 1, & i=j \\ 0, & i \neq j \end{array} \right.$$

%
\subsection{Matrice inversible}
%
\paragraph{Définition} Soit $n \geq 1$ un entier naturel. Soit $A$ une matrice carrée de taille $n \times n$. On dit que $A$ est inversible s'il existe une matrice carrée $B$ de taille $n \times n$ telle que $A B = I_n = B A$. Dans ce cas on dit que $B$ est un inverse de $A$.

\paragraph{Proposition} Soit $A$ une matrice carrée de taille $n \times n$. Si $A$ est inversible, alors son inverse est unique.

\paragraph{Démonstration} On suppose que $A$ est inversible. Si $B$ et $C$ sont deux inverses pour $A$, alors on a
$$B A C = (B A) C = I_n C = C$$
on a également
$$B A C = B (A C) = B I_n = B$$
d'où $B = C$.

\paragraph{Notation} Si $A$ est inversible, on note $A^{-1}$ son inverse. On a donc
$$A A^{-1} = I_n = A^{-1} A$$

\paragraph{Exemple} La matrice $\begin{pmatrix} 2 & 1 \\ 7 & 4 \end{pmatrix}$ est inversible d'inverse $\begin{pmatrix} 4 & -1 \\ -7 & 2 \end{pmatrix}$.
$$\begin{pmatrix}2 & 1 \\ 7 & 4 \end{pmatrix} 
  \begin{pmatrix} 4 & -1 \\ -7 & 2 \end{pmatrix} 
  = \begin{pmatrix} 1 & 0 \\ 0 & 1 \end{pmatrix} = I_2$$
$$\begin{pmatrix} 4 & -1 \\ -7 & 2 \end{pmatrix} 
  \begin{pmatrix}2 & 1 \\ 7 & 4 \end{pmatrix}  
  = \begin{pmatrix} 1 & 0 \\ 0 & 1 \end{pmatrix} = I_2$$
On vérifie que $A A^{-1} = I_n = A^{-1} A$.

%
\subsection{Matrice de taille $2\times 2$}
%
\paragraph{Lemme} Soient $a, b, c, d \in \R$, $A = \begin{pmatrix} a & b \\ c & d \end{pmatrix} \in M_{2}(\R)$ et $B = \begin{pmatrix} d & -b \\ -c & a \end{pmatrix} \in M_{2}(\R)$. Alors 
$$A B = (a d - b c) I_2 = B A$$

\paragraph{Démonstration} 
$$A B 
  = \begin{pmatrix} a & b \\ c & d \end{pmatrix} \begin{pmatrix} d & -b \\ -c & a \end{pmatrix} 
  = \begin{pmatrix} ad-bc & 0 \\ 0 & ad-bc \end{pmatrix} 
  = (ad-bc) \begin{pmatrix} 1 & 0 \\ 0 & 1 \end{pmatrix} 
  = (ad-bc) I_2$$
et 
$$B A 
  = \begin{pmatrix} d & -b \\ -c & a \end{pmatrix} \begin{pmatrix} a & b \\ c & d \end{pmatrix} 
  = \begin{pmatrix} ad-bc & 0 \\ 0 & ad-bc \end{pmatrix} 
  = (ad-bc) \begin{pmatrix} 1 & 0 \\ 0 & 1 \end{pmatrix} 
  = (ad-bc) I_2$$

\paragraph{Proposition} Soit $A = \begin{pmatrix} a & b \\ c & d \end{pmatrix} \in M_{2}(\R)$. Alors $A$ est inversible si et seuelement si $a d - b c \neq 0$, et dans ce cas
$$A^{-1} = \frac{1}{a d - b c} \begin{pmatrix} d & -b \\ -c & a \end{pmatrix}$$

\paragraph{Démonstration} Cet énoncé est une équicalence constituée de deux implications:
\begin{enumerate}
  \item si $a d - b c \neq 0$, alors $A$ est inversible et $A^{-1} = \frac{1}{ad-bc} \begin{pmatrix} d & -b \\ -c & a \end{pmatrix}$
  
  \item si $a d - b c = 0$, alors $A$ n'est pas inversible.
\end{enumerate}
Démontrons les deux implications:
\begin{enumerate}
  \item On suppose que $a d - b c \neq 0$. Par suite, $a d - b c$ est inversible dans $\R$. D'après le lemme, on a
    $$\begin{pmatrix} a & b \\ c & d \end{pmatrix} \begin{pmatrix} d & -b \\ -c & a \end{pmatrix} 
      = (a d - b c) I_2 
      =  \begin{pmatrix} d & -b \\ -c & a \end{pmatrix} \begin{pmatrix} a & b \\ c & d \end{pmatrix}$$
    d'où
    $$\begin{pmatrix} a & b \\ c & d \end{pmatrix} \left( \frac{1}{a d - b c} \begin{pmatrix} d & -b \\ -c & a \end{pmatrix} \right) 
      = I_2 
      = \left( \frac{1}{a d - b c} \begin{pmatrix} d & -b \\ -c & a \end{pmatrix} \right) \begin{pmatrix} a & b \\ c & d \end{pmatrix}$$
    Par conséquent, $A$ est inversible d'inverse $A^{-1}=\frac{1}{a d - b c}\begin{pmatrix} d & -b \\ -c & a\end{pmatrix}$.
    
  \item On suppose que $a d - b c = 0$. On raisonne par l'absurde. Si $A$ est inversible, on obtient: D'après le lemme on a
    $$\begin{pmatrix} a & b \\ c & d \end{pmatrix} \begin{pmatrix} d & -b \\ -c & a \end{pmatrix} = (a d - b c) I_2 = 0$$
    d'où
    \begin{eqnarray*}
      A^{-1} \left( A \begin{pmatrix} d & -b \\ -c & a \end{pmatrix} \right) &=& A^{-1} 0 = 0 \\
        &=& A^{-1} A \begin{pmatrix} d & -b \\ -c & a \end{pmatrix} \\
      0 &=& \begin{pmatrix} d & -b \\ -c & a \end{pmatrix}
    \end{eqnarray*}
    Par conséquent $a = b = c = d = 0$, donc $A = 0$ donc $A$ n'est pas inversible, ce qui donn lien à une contradiction. Ainsi $A$ n'est pas inversible si $a d - b c = 0$.
    
\end{enumerate}

\paragraph{Proposition} Soient $A$ et $B$ des matrices inversibles de taille $n \times n$. Alors
\begin{enumerate}
  \item $A^{-1}$ est inversible d'inverse $A$
  \item $A B$ est inversible et $(A B)^{-1} = B^{-1} A^{-1}$
\end{enumerate}

\paragraph{Démonstration}
\begin{enumerate}
  \item Puisuque $A$ est inversible d'inverse $A^{-1}$, on a
    $$A A^{-1} = I_n = A^{-1} A$$
    donc $A^{-1}$ est inversible d'inverse $A$.
  \item On a
    $$(A B)(B^{-1} A^{-1}) = A (B B^{-1}) A^{-1} = A I_n A^{-1} = A A^{-1} = I_n$$
    De même, on a $(B^{-1} A^{-1}) A B = I_n$ \\
    Par conséquence, $A B$ est inversible d'inverse $(B^{-1} A^{-1})$
\end{enumerate}

\paragraph{Proposition} Plus généralement, si $A_1, A_2, \ldots, A_m$ sont des matrices inversibles on a 
$$(A_1 A_2 \ldots A_m)^{-1} = A_m^{-1} A_{m-1}^{-1} \ldots A_1^{-1}$$

%
\subsection{Groupe des matrices inversibles}
%
\paragraph{Proposition} L'ensemble des matrices inversibles de taille $n \times n$ muni de la multiplication des matrices est un group. En effet
\begin{itemize}
  \item L'assertion $2.$ de la proposition précédante assure que $\times$ restreinte à l'ensemble des matrices inversibles est une loi de composition interne.
  \item $\times$ étant associative sur $M_{n}(\R)$, elle l'est en restriction à l'ensemble des matrices inversibles.
  \item $I_n$ est élément neutre.
  \item L'assertion $1.$ de la proposition assure que l'inverse d'une matrice inversible est bien inversible.
\end{itemize}
On note $(GL_{n}(\R), \times)$ le groupe des matrices inversibles de taille $n$ ($GL$ est mis pour "groupe linéaire").

\paragraph{Exemple} Soient $A = \begin{pmatrix} 2 & 7 \\ 1 & 4 \end{pmatrix}$ et $B = \begin{pmatrix} 5 & 3 \\ 3 & 2 \end{pmatrix}$.
$A$ est inversible car $2 \cdot 4 - 1 \cdot 7 = 1 \neq 0$, et
$$A^{-1} = \frac{1}{1} \begin{pmatrix} 2 & -7 \\ -1 & 4 \end{pmatrix} = \begin{pmatrix} 2 & -7 \\ -1 & 4 \end{pmatrix}$$
$B$ est inversible car $5 \cdot 2 - 3 \cdot 3 = 1 \neq 0$, et 
$$B^{-1} = \frac{1}{1} \begin{pmatrix} 2 & -3 \\ -3 & 5 \end{pmatrix} = \begin{pmatrix} 2 & -3\\ -3 & 5 \end{pmatrix}$$
On a 
$$A B = \begin{pmatrix} 2 & 7 \\ 1 & 4 \end{pmatrix} 
  \begin{pmatrix}5&3\\3&2\end{pmatrix}
  = \begin{pmatrix} 31 & 20 \\ 17 & 11 \end{pmatrix}$$
et
$$B^{-1} A^{-1} = \begin{pmatrix} 31 & 20 \\ 17 & 11 \end{pmatrix} 
  \begin{pmatrix} 2 & -7 \\ -1 & 4 \end{pmatrix} 
  = \begin{pmatrix} 11 & -20 \\ -17 & 31 \end{pmatrix}$$
On a $(31 \cdot 11) - (17 \cdot 20) = 341 - 340 = 1 \neq 0$. Donc $A B$ est inversible 
$$(AB)^{-1} = \frac{1}{1} \begin{pmatrix} 11 & -20 \\ -17 & 31 \end{pmatrix} 
  = \begin{pmatrix} 11 & -20 \\ -17 & 31 \end{pmatrix} = B^{-1} A^{-1}$$
qui est bien égale à $B^{-1} A^{-1}$.

%
%
\section{Matrices élémentaires}
%
%

%
\subsection{Matrice élémentaire}
%
\paragraph{Définition} Soient $n \geq 1$ un entier naturel et $(i, j) \in \{1; \ldots; n\} \times \{1; \ldots; n\}$. On note $F_{i, j}$ la matrice carrée de taille $n \times n$ contenant un $1$ en $i^{\text{ème}}$ ligne et $j^{\text{ème}}$ colonne et des $0$ partout ailleurs. On dit que $F_{i, j}$ est une matrice standard.

\paragraph{Exemple} 
$$n = 3 \rightarrow F_{2,3} = 
\begin{pmatrix}
  0 & 0 & 0 \\
  0 & 0 & 1 \\
  0 & 0 & 0
\end{pmatrix}$$
$$n = 4 \rightarrow F_{4,1} = 
\begin{pmatrix}
  0 & 0 & 0 & 0 \\
  0 & 0 & 0 & 0 \\
  0 & 0 & 0 & 0 \\
  1 & 0 & 0 & 0
\end{pmatrix}$$

\paragraph{Proposition} Toute matrice carrée de taille $\times n$ peut s'exprimer à l'aide des matrices standard de taille $n\times n$
$$A = \sum_{(i, j) \in \{1; \ldots; n\}\times\{1; \ldots; n\}} a_{ij} F_{ij}$$

\paragraph{Proposition} Soit $A$ une matrice de taille $n\times m$.
\begin{enumerate}[a)]
  \item Soit $F_{ij}$ une matrice standard de taille $n\times n$. Alors $F_{ij} A$ est la matrice de taille $n \times m$ dont la $i^{\text{ème}}$ ligne est la $j^{\text{ème}}$ ligne de $A$ et dont les autres sont nulles.
  \item Soit $F_{ij}$ une matrice standard de taille $m\times m$. Alors $A F_{ij}$ est la matrice de taille $n\times m$ dont la $j^{\text{ème}}$ colonne est la $i^{\text{ème}}$ de $A$ et dont les autres colonnes sont nulles.
\end{enumerate}

\paragraph{Démonstration}
\begin{enumerate}[a)]
  \item Soit $k \in \{1, \ldots, n \}$ et $l \in \{1, \ldots, n \}$. On a
    $$(F_{ij} A)_{k l} = \sum_{r = 1}^{n} (F_{i j})_{k r} A_{r l}$$
    si $k\neq i$, alors la $k$-ième ligne de $F_{i j}$ est nulle, donc
    $$(F_{i j})_{k l}=0 \text{ pour tout } l$$
    On obtient alors
    $$(F_{i j})_{k l}=0 \text{ si } k\neq i$$
    Si $k=i$, alors 
    $$(F_{i j} A)_{i l} = \sum_{r = 1}^{n} (F_{i j})_{i r}A_{r l}$$
    si $r \neq j$, alors la $r$-ième colonne de $F_{i j}$ est nulle, donc
    $$(F_{ij})_{i r} = 0$$
    il suit donc
    $$(F_{i j} A)_{i l} = (F_{i j})_{i j} A_{j l} =A_{j l}$$
    D'après ce qui procède, si $k \neq i$, alors la $k$-ième ligne de $F_{i j}A$ est nulle. Si $k = i$, alors la $i$-ième ligne de $F_{i j} A$ est $(A_{j 1} ~ A_{j 2} ~ \ldots ~ A_{j m})$ qui est la $j$-ième ligne de $A$.

  \item Ceci se montre de façon similaire.
\end{enumerate}

\paragraph{Exemple}
\begin{itemize}
  \item Soient $A = \begin{pmatrix} 1 & -7 & 5 \\ 0 & 3 & 4 \end{pmatrix} \in M_{2\times 3}(\R)$ et $F_{1 2} = \begin{pmatrix} 0 & 1 \\ 0 & 0 \end{pmatrix} \in M_{2 \times 2}(\R)$. Alors
    $$F_{1 2} A 
      = \begin{pmatrix} 0 & 1 \\ 0 & 0 \end{pmatrix} \begin{pmatrix} 1 & -7 & 5 \\ 0 & 3 & 4 \end{pmatrix} 
      = \begin{pmatrix} 0 & 3 & 4 \\ 0 & 0 & 0 \end{pmatrix}$$

  \item Soient $A = \begin{pmatrix} 1 & 5 \\ 0 & 4 \\ -7 & 2 \end{pmatrix} \in  M_{3 \times 2}(\R)$ et $F_{3 1} = \begin{pmatrix} 0 & 0 & 0 \\ 0 & 0 & 0 \\ 1 & 0 & 0 \end{pmatrix} \in M_{3 \times 3}(\R)$. Alors
    $$F_{3 1} A 
      = \begin{pmatrix} 0 & 0 & 0 \\ 0 & 0 & 0 \\ 1 & 0 & 0 \end{pmatrix} \begin{pmatrix} 1 & 5 \\ 0 & 4 \\ -7 & 2 \end{pmatrix} 
      = \begin{pmatrix} 0 & 0 \\ 0 & 0 \\ 1 & 5 \end{pmatrix}$$
      
  \item Soient $A = \begin{pmatrix} 1 & 5 \\ 0 & 4 \\ -7 & 2 \end{pmatrix} \in  M_{3 \times 2}(\R)$  et $F_{1 2} = \begin{pmatrix} 0 & 1 \\ 0 & 0 \end{pmatrix} \in M_{2 \times 2}(\R)$. Alors
    $$A F_{1 2} 
      = \begin{pmatrix} 1 & 5 \\ 0 & 4 \\ -7 & 2 \end{pmatrix} \begin{pmatrix} 0 & 1 \\ 0 & 0 \end{pmatrix} 
      =\begin{pmatrix} 0 & 1 \\ 0 & 0 \\ 0 & -7 \end{pmatrix}$$
\end{itemize}

%
\subsection{Symbole de Kronecker}
%
\paragraph{Définition} Soit $n \geq 1$ un entier naturel. On note
$$\mS: \{1; \ldots; n\}\times\{1; \ldots; n \} \rightarrow \{0; 1\}$$
L'application est défini par
$$\delta_{ij} = \left\{\begin{array}{lr} 1 & \text{si i=j} \\ 0 & \text{sinon} \end{array} \right.$$ 

\paragraph{Corollaire} Soient $n \geq 1$ un entier naturel et
$$(i, j), (k, l) \in \{1; \ldots; n\}\times\{1; \ldots; n\}$$
alors
$$F_{ij}F_{kl} = \delta_{ik} F_{il} = 
  \left\{\begin{array}{lr} F_{il} & \text{si i=j} \\ 0 & \text{sinon} \end{array} \right.$$

\paragraph{Démonstration} D'après la proposition, précédente, $F_{ij} F_{kl}$ est la matrice don la $i^{\text{ème}}$ ligne est la $j^{\text{ème}}$ ligne de $F_{kl}$ et dont les autres lignes sont nulles. Il vient
\begin{itemize}
  \item si $j\neq k$, la $j^{\text{ème}}$ ligne de $F_{kl}$ est nulle
  \item si $j = k$ la $j^{\text{ème}} = k^{\text{ème}}$ ligne de $F_{kl}$ est $(0_{1} \ldots 0_{l-1} ~ 1_{l} ~ 0_{l+1} \ldots 0_{n})$
\end{itemize}
Par conséquent, si $j\neq k$, toutes les lignes de $F_{ij} F_{kl}$ sont nulles donc $F_{ij} F_{kl} = 0$; si $j=k$, alors 
$$F_{ij} F_{kl} = \begin{pmatrix} 
  & & & 0 & \\
  & & & \vdots & \\
  & & & 0 & \\
  0 & \ldots & 0 & 1 & 0 \\
  & & & 0 & \\
\end{pmatrix} = F_{il}$$
Ainsi $F_{ij} F_{kl} = \delta_{ik} F_{il}$.

%
\subsection{Les opérations élémentaires}
%
\paragraph{Rappel} On rappelle que l'on distingue trois types d'opérations éLémentaires que l'on peut effectuer sur les lignes d'une matrice
\begin{itemize}
  \item multiplier une ligne par un réel
  \item ajouter à une ligne un multiple quelconque d'une autre ligne
  \item permuter deux lignes
\end{itemize}
On introdiut les matrices suivantes ($n \geq 1$ entier naturel)
\begin{itemize}
  \item $c \in \R \backslash \{0 \}, i \in \{1; \ldots; n\}$
    $$E_i(c) = \left( \sum_{l=1, l\neq i}^n F_{ll} \right) +  c F_{ii} = 
      \begin{pmatrix}
        1 & & \\
        & 1 & \\
        & & \ddots & \\
        & & & c & \\
        & & & & 1
      \end{pmatrix}$$
      
  \item $c \in \R, (i, j) \in \{1;\ldots;n\}\times\{1;\ldots;n\}, i\neq j$
    $$E_{ij}(c) = I_n + c F_{ij} =
      \begin{pmatrix}
        1 & & & \\
        & 1 & & c \\
        & & \ddots & \\
        & & & 1
      \end{pmatrix}$$
      
  \item $(i, j) \in \{1; \ldots; n\}\times\{1;\ldots; n\}, i\neq j$
    $$E_{ij} = \left( \sum_{l=1, l \notin \{i; j\}}^n F_{ll} \right) + F_{ij} + F_{ji} = 
      \begin{pmatrix}
        1 & & & & & & 0 \\
        & 0 & & & & 1 & \\
        & & 1 & & 0 & & \\
        & & & \ddots & & & \\
        & & 0 & & 1 & & \\
        & 1 & & & & 0 & \\
        0 & & & & & & 1
      \end{pmatrix}$$
\end{itemize} 
Les opérations élémentaires que l'un peut effectuer sur les lignes d'une matrice peuvent s'interpréter en termes de multiplications de cette par les matrices précédantes.

\paragraph{Exemple} Pour $n = 2$ on a
$$E_1(-3) = \begin{pmatrix} -3 & 0 \\ 0 & 1 \end{pmatrix} ~~ 
  E_{1 2} = \begin{pmatrix} 0 & 1 \\ 1 &  0 \end{pmatrix} ~~ 
  E_{2 1}(5) = \begin{pmatrix} 1 & 0 \\ 5 & 1 \end{pmatrix}$$
Pour $n = 3$ on a
$$E_2(-4) = \begin{pmatrix} 1 & 0 & 0 \\ 0 & -4 & 0 \\ 0 & 0 & 1 \end{pmatrix} ~~
  E_{2 1} = \begin{pmatrix} 0 & 1 & 0 \\ 1 & 0 & 0 \\ 0 & 0 & 1 \end{pmatrix} ~~
  E_{2 3}(7) = \begin{pmatrix} 1 & 0 & 0 \\ 0 & 1 & 7 \\ 0 & 0 & 1 \end{pmatrix}$$

\paragraph{Théorème} Soit $A$ une matrice de taille $n\times m$
\begin{enumerate}[a)]
  \item Multiplier la $i^{\text{ème}}$ ligne de $A$ par un réel $c$ non nul est équvalent à multiplier $A$ à gauche par la matrice $E_i(c)$ de taille $n \times n$
  
  \item Ajouter à la $i^{\text{ème}}$ ligne de $A$ $c$-fois la ligne $j$ de $A$ où $i\neq j$ et $c\in \R$ est équivalent à multiplier $A$ à gauche par la matrice $E_{ij}(c)$.
  
  \item Permuter la $i^{\text{ème}}$ ligne et la $j^{\text{ème}}$ ligne de $A$ où $i\neq j$ est équivalent à multiplier $A$ à gauche par la matrice $E_{ij}$.
\end{enumerate}

\paragraph{Démonstration}
\begin{enumerate}[a)]
  \item $$E_i(c) A = \left( \sum_{l=1, l\neq i}^n F_{ll} \right) A +  c F_{ii} A$$
    Si $l\neq i$ on a
    $$F_{ll} A =
      \begin{pmatrix} 
        \\ l^{\text{ème}} \text{ ligne de } A \\ \\
      \end{pmatrix} 
      \begin{array}{l} 
        \leftarrow ~ 1, \ldots, l-1 \\ \leftarrow ~ l \\ \leftarrow ~ l+1, \ldots, n 
      \end{array}$$
    sinon, on a
    $$c F_{ii} A =
      \begin{pmatrix}
        \\ c\text{-fois } i^{\text{ème}} \text{ ligne de } A \\ \\
      \end{pmatrix} 
      \begin{array}{l} 
        \leftarrow ~ 1, \ldots, i-1 \\ \leftarrow ~ i \\ \leftarrow ~ i+1, \ldots, n 
      \end{array}$$
    On voit alors que $\left( \sum_{l=1, l\neq i}^n F_{ll} \right) A +  c F_{ii} A$ est la matrice cherchée.
    
  \item $$E_{ij}(c) A = (I_n + c F_{ij}) A = E_{ij}(c) = A + c F_{ij} A$$
    Où on a
      $$c F_{ij} A =
        \begin{pmatrix}
          \\ c\text{-fois } j^{\text{ème}} \text{ ligne de } A \\ \\
        \end{pmatrix} 
        \begin{array}{l} 
          \leftarrow ~ 1, \ldots, i-1 \\ \leftarrow ~ i \\ \leftarrow ~ i+1, \ldots, n 
        \end{array}$$
    Donc $A + c F_{ij} A$ est bien la matrice cherchée.
    
  \item $$E_{ij} A = \left( \sum_{l=1, l \notin {i; j}}^n F_{ll} A \right) + F_{ij} A + F_{ji} A$$
    Si $l\neq i$ et $l\neq j$ 
    $$F_{ll} A =
      \begin{pmatrix}
        \\ l^{\text{ème}} \text{ ligne de } A \\ \\
      \end{pmatrix} 
      \begin{array}{l} 
        \leftarrow ~ 1, \ldots, l-1 \\ \leftarrow ~ l \\ \leftarrow ~ l+1, \ldots, n 
      \end{array}$$
    sinon
      $$F_{ij} A =
      \begin{pmatrix}
        \\ j^{\text{ème}} \text{ ligne de } A \\ \\
      \end{pmatrix} 
      \begin{array}{l} 
        \leftarrow ~ 1, \ldots, i-1 \\ \leftarrow ~ i \\ \leftarrow ~ i+1, \ldots, n 
      \end{array}$$
      $$F_{ji} A =
      \begin{pmatrix}
        \\ i^{\text{ème}} \text{ ligne de } A \\ \\
      \end{pmatrix} 
      \begin{array}{l} 
        \leftarrow ~ 1, \ldots, j-1 \\ \leftarrow ~ j \\ \leftarrow ~ j+1, \ldots, n 
      \end{array}$$
    Ainsi $\left( \sum_{l=1, l \notin {i; j}}^{n} F_{ll} A \right) + F_{ij} A + F_{ji} A$ est la matrice cherchée.
\end{enumerate}

\paragraph{Exemple} Soit $A = \begin{pmatrix} 5 & -3 & 0 \\ 4 & 2 & 7 \end{pmatrix} \in M_{2 \times 3}(\R)$. 
\begin{itemize}
  \item On veut multiplier la première ligne par $-7$.
    $$A \xrightarrow {L_1 \leftarrow (-7)L_1} 
    \begin{pmatrix} -35 & 21 & 0 \\ 4 & 2 & 7 \end{pmatrix}$$
    On calcule
    $$E_1(-7) A 
      = \begin{pmatrix} -7 & 0 \\ 0 & 1 \end{pmatrix} \begin{pmatrix} 5 & -3 & 0 \\ 4 & 2 & 7 \end{pmatrix} 
      = \begin{pmatrix} -35 & 21 & 0 \\ 4 & 2 & 7 \end{pmatrix}$$

  \item Ajouter à la première ligne de $A$ $3$ fois la deuxième ligne.
    $$A \xrightarrow{L_1\leftarrow L_1+3L_2} 
      \begin{pmatrix} 17 & 3 & 21 \\ 4 & 2 & 7 \end{pmatrix}$$
    On calcule
    $$E_{1 2}(3) A 
      = \begin{pmatrix} 1 & 3 \\ 0 & 1 \end{pmatrix} \begin{pmatrix} 5 & -3 & 0 \\ 4 & 2 & 7 \end{pmatrix} 
      = \begin{pmatrix} 17 & 3 & 21 \\ 4 & 2 & 7 \end{pmatrix}$$
\end{itemize}

\paragraph{Théorème} Soit $n$ un entier naturel et $n \geq 1$. Alors les matrices élémentaires de taille $n \times n$ sont inversibles, et l'on a:
\begin{enumerate}
  \item $$E_i(C)^{-1} = \frac{1}{C} E_i \text{ avec } C \in \R \backslash \{ 0 \}$$
  \item $$E_{i j}(C)^{-1} = E_{i j}(-C) \text{ avec } C \in \R$$
  \item $$E_{i j}^{-1} = E_{j i}$$
\end{enumerate}

\paragraph{Démonstration}
\begin{enumerate}
  \item On a
    $$E_i(C)E_i(\frac{1}{C}) = 
      \begin{pmatrix}
        1 & 0 & \dots & & 0 \\
        0 & \ddots & & & \vdots \\
        \vdots & & c \\
        & & & \ddots & 0 \\
        0 & \dots & & 0 & 1
      \end{pmatrix} 
      \begin{pmatrix}
        1 & 0 & \dots & & 0 \\
        0 & \ddots & & & \vdots \\
        \vdots & & \frac{1}{c} \\
        & & & \ddots & 0 \\
        0 & \dots & & 0 & 1
      \end{pmatrix} = 
      \begin{pmatrix}
        1 & 0 & \dots & & 0 \\
        0 & \ddots & & & \vdots \\
        \vdots & & 1 \\
        & & & \ddots & 0 \\
       0 &\dots & & 0 & 1
      \end{pmatrix} = I_n$$
    On vérifie de même que $E_i(\frac{1}{C}) E_i(C) = I_n$.
  
  \item On a
    $$E_{i j}(C) E_{i j}(-C) = (I_n + C F_{i j}) (I_n - C F_{i j}) = I_n + C F_{i j} - C F_{i j} - C^2 F_{i j} F_{i j}$$
    puisque $i\neq j$, on a $F_{i j} F_{i j} = 0$. Donc
    $$E_{i j}(C) E_{i j}(-C) = I_n$$
    On vérifie de même que $E_{i j}(-C) E_{i j}(C) = I_n$.
  
  \item On a
    \begin{eqnarray*}
      E_{i j}E_{j i} 
        &=& \left( \sum_{r = 1, r \notin \{i, j\}}^{n} F_{r r} + F_{i j} + F_{j i}\right) \left( \sum_{s = 1, s \notin \{i, j\}}^{n} F_{s s}+F_{i j}+F_{j i} \right) \\
        &=& (I_n - F_{i i} - F_{j j} + F_{i j} + F_{j i}) (I_n - F_{i i} - F_{j j} + F_{i j} + F_{j i}) \\
        &=& I_n - F_{i i} - F_{j j} + F_{i j} + F_{j i} - F_{j j} + F_{i i} F_{i i} + F_{j j} F_{i i} - F_{i j} F_{i i} - F_{j i} F_{i i} \\
          && - F_{j j} + F_{i i} F_{j j} + F_{j j} F_{j j} - F_{i j} F_{j j} - F_{j i} F_{j j} + F_{i j} - F_{i i} F_{i j} - F_{j j} F_{i j} \\
          && + F_{i j} F_{i j} + F_{j i} F_{i j} + F_{j i} - F_{i i} F_{i i} - F_{j j} F_{j i} + F_{i j} F_{j i} + F_{j i} F_{j i}\\
        &=& I_n - F_{i i} - F_{j j} + F_{i j} + F_{j i} - F_{i i} + F_{i i} - F_{j i} - F_{j j} \\
          && + F_{j j} - F_{i j} + F_{i j} - F_{i j} + F_{j j} + F_{j i} - F_{j i} + F_{i i} \\
        &=& I_n
    \end{eqnarray*}
\end{enumerate}
Le fait que les matrices élémentaires soient inversibles confirme le fait que les opérations élémentaires soient réversibles.

%
%
\section{Écriture matricielle des systèmes linéaires}
%
%
\paragraph{Écriture} Considérons le système linéaire à $n$ équations et $m$ inconnues suivant:
$$\left\{ \begin{array}{rcl}
  a_{1 1} x_{1} + a_{1 2} x_2 + \ldots + a_{1 m} x_m & = & b_1 \\
  a_{2 1} x_{1} + a_{2 2} x_2 + \ldots + a_{2 m} x_m & = & b_2 \\
  \vdots & & \vdots \\
  a_{n 1} x_{1} + a_{n 2} x_2 + \ldots + a_{n m} x_m & = & b_n
\end{array} \right.$$
où les $a_{i j}$ et les $b_{i}$ sont des réels. Soit
$$A = \begin{pmatrix}
  a_{1 1} & a_{1 2} & \dots & a_{1 m} \\
  a_{2 1} & a_{2 2} & \dots & a_{2 m} \\
  \vdots & \vdots & \ddots & \vdots \\
  a_{n 1} & a_{n 2} & \dots & a_{n m}
\end{pmatrix}$$
la matrice du système et soit
$$B = \begin{pmatrix} b_1 \\ b_2 \\ \vdots \\ b_m \end{pmatrix} \text{ et } 
  X = \begin{pmatrix} x_1 \\ x_2 \\ \vdots \\ x_m \end{pmatrix}$$
Alors le système peut s'écrire matriciellement sous la forme
$$A X = B$$

%
\subsection{Nombres de solutions possibles pour un système linéaire}
%
\paragraph{Théorème} Soit $A X = B$ un système linéaire. Alors, on se trouve dans l'un des cas suivants:
\begin{itemize}
  \item le système a aucune solution (système incompatible) 
  \item le système a une solution unique
  \item le système a une infinité de solutions
\end{itemize}

\paragraph{Démonstrations}
\begin{itemize}
  \item Si le système est incomaptible, il n'a pas de solutions.
  \item S'il est compatible, il faut voir que dans le cas où il admet au moins deux solutions détérminés, il admet une infinité de solutions. Supposons que $t = (t_1, t_2, \ldots, t_m)$ et $t' = (t'_1, t'_2, \ldots, t'_m)$ soient deux solutions distinctes de ce système. On a donc
    $$A \begin{pmatrix} t_1 \\ \vdots \\ t_m \end{pmatrix} = B = A \begin{pmatrix} t'_1 \\ \vdots \\ t'_m \end{pmatrix}$$
    Il vient
    $$A \left( \begin{pmatrix} t_1 \\ \vdots \\ t_m \end{pmatrix} - \begin{pmatrix} t'_1 \\ \vdots \\ t'_m \end{pmatrix} \right) = 0$$
    puisque $t \neq t'$, $t - t' \neq 0$. On obtient alors pour tout $\alpha \in \R$
    $$A \left( \begin{pmatrix} t_1 \\ \vdots \\ t_m \end{pmatrix} + \alpha \left( \begin{pmatrix} t_1 \\ \vdots \\ t_m \end{pmatrix} - \begin{pmatrix} t'_1 \\ \vdots \\ t'_m \end{pmatrix} \right) \right) 
      = A \begin{pmatrix} t_1 \\ \vdots \\ t_m \end{pmatrix} + \alpha A \left( \begin{pmatrix} t_1 \\ \vdots \\ t_m \end{pmatrix} - \begin{pmatrix} t'_1 \\ \vdots \\ t'_m \end{pmatrix} \right) 
      = B + 0 = B$$
    donc $(t_1, t_2, \ldots, t_m) + \alpha ((t_1, t_2, \ldots, t_m) - (t'_1, t'_2, \ldots, t'_m))$ est une solution du système. Lorsque $\alpha$ parcourt $\R$ tout entier, on obtient une infinité de solutions.
\end{itemize}

%
%
\section{Calcul de $A^{-1}$ et $A^{-1} B$}
%
%

\paragraph{Théorème} Soit $A$ une matrice carrée de taille $n \times n$. Alors les conditions suivantes sont équivalentes
\begin{enumerate} 
  \item $A$ est inversible,
  \item Pour tout $B \in M_{n, 1}(\R)$, le système linéaire $A X = B$ admet une solutions unique.
  \item $A$ admet $I_n$ pour forme échelonné réduite.
  \item $A$ est produit de matrices élémentaires.
\end{enumerate}

\paragraph{Démonstration}
\begin{itemize}
  \item[$1. \Rightarrow 2.$] On suppose que $A$ est inversible. Soit $B \in M_{n, 1}(\R)$. On a
    $$A (A^{-1} B) = (A A^{-1}) B = I_n B = B$$
    donc $A^{-1} B \in M_{n, 1}(\R)$ est une solution du système $A X = B$. Maintenant, si $T \in M_{n, 1}(\R)$ est une solutions de ce système, on a $A T = B$, d'où\footnote{avec $T = (A^{-1} A) T$} $A^{-1}(A T) = A^{-1} B$ ce qui montre que $T$ est unique. Par conséquent, le système $A X = B$ a une unique solution.
    
  \item[$2. \Rightarrow 3.$] Supposons que pour tout $B \in M_{n, 1}(\R)$, le système $A X = B$ admet une unique solutions. Puisque ce système admet une unique solution, toutes les inconnues du système sont principale. En effet, si l'une\footnote{où plusieurs} des inconnues était sécondaire, on se trouverait dans l'un des cas suivants
    \begin{itemize}
      \item le système est incompatible et n'a pas de solution
      \item le système est compatible et a une infinité de solutions
    \end{itemize}
    Comme $A$ est carrée, cela implque que la forme réduite de $A$ est $I_n$.
    
  \item[$3. \Rightarrow 4.$] Supposons que la forme échelonnée réduite de $A$ soit $I_n$. Il existe donc des matrices élémentaires $E_1, E_2, \ldots , E_r$, telles que
    $$I_n = E_r \cdot \ldots \cdot E_2 \cdot E_1 \cdot A$$
    Il vient alors $E_1^{-1} \cdot E_2^{-1} \cdot \ldots \cdot E_r^{-1} \cdot I_n = A$ d'où
    $$A = E_1^{-1} \cdot E_2^{-1} \cdot \ldots \cdot E_r^{-1}$$
    Or les $E_i^{-1}$ sont également des matrices élémentaires. Donc $A$ est bien produit de matrices élémentaires.
    
  \item[$4. \Rightarrow 1.$] On suppose que $A$ est produit de matrices élémentaires $A = E_1^{-1} \cdot E_2^{-1} \cdot \ldots \cdot E_r^{-1}$. Comme les $E_i$ sont inversibles, il en est de même de leur produit, donc de $A$.
\end{itemize}
Ce qui précède fournit une méthode pour calcule l'inverse d'une matrice, si cet inverse existe.

%
\subsection{Calculer l'inverse d'une matrice}
%
\paragraph{Methode de calcul} Soit $A$ matrice carrée de taille $n\times n$. \\
On forme la matrice $(A ~ \vert ~ I_n)$. En utilisant les opérations élémentaires on met la matrice $E$ en forme échelonnée réduite et on éffectue parallelement les mêmes opérations élémentaires sur les lignes de $I_n$. Si $A$ sous la forme échelonnée réduite est égale à $I_n$ $A$ est inversible. La matrice inverse de $A$ est alors la partie $I_n$ où on a éffectue les opérations élémentaires. \\
$E_1, E_2, \ldots, E_l$ sont des matrices élémentaires.
\begin{eqnarray*}
  A \cdot E_1 \cdot E_2 \cdot &\ldots& \cdot E_l = I_n \\
  &\Downarrow& \\
  I_n \cdot E_1 \cdot E_2 \cdot &\ldots& \cdot E_l = A^{-1} \\
  &\Downarrow& \\
  E_1 \cdot E_2 \cdot &\ldots& \cdot E_l = A^{-1}
\end{eqnarray*}
De la même manière on peut calculer $A^{-1} B$.
$$A^{-1} B = E_1 \cdot E_2 \cdot \ldots \cdot E_l \cdot B$$
où $E_1, E_2, \ldots, E_l$ sont les opérations élémentaires dont on à besoin pour mettre $A$ sous forme échelonnée réduite égal à $I_n$.

\paragraph{Exemple} Soient $A = \begin{pmatrix} 1 & 2 & 1 \\ 4 & 0 & -1 \\ -1 & 2 & 2 \end{pmatrix}$ et $B = \begin{pmatrix} 1 & 2 & 0 & 4 \\ -3 & 1 & 2 & 0 \\ 5 & 1 & -1 & 3 \end{pmatrix}$.
On a
$$A^{-1} B 
  = \frac{1}{4} \begin{pmatrix} -2 & 2 & 2 \\ 7 & -3 & -5 \\ -8 & 4 & 8 \end{pmatrix} 
    \begin{pmatrix} 1 & 2 & 0 & 4 \\ -3 & 1 & 2 & 0 \\ 5 & 1 & -1 & 3 \end{pmatrix} 
  = \frac{1}{4} \begin{pmatrix} 2 & 0 & 2 & -2 \\ -9 & 6 & -1 & 13 \\ 20 & -4 & 0 & -8 \end{pmatrix}$$
En applicant la méthode décrite, on obtient:
\begin{eqnarray*}
  \left( \begin{array}{ccc|cccc}
    1 & 2 & 1 &    1 & 2 & 0 & 4 \\ 
    4 & 0 & -1 &   -9 & 6 & -1 & 13 \\
    -8 & 4 & 8 &   20 & -4 & 0 & -8
  \end{array}\right)
  &\xrightarrow[L_3 \leftarrow L_3 + L_1]{L_2 \leftarrow L_2 - 4 L_1}&
  \left( \begin{array}{ccc|cccc}
    1 & 2 & 1 &     1 & 2 & 0 & 4 \\ 
    0 & -8 & -5 &   -7 & -7 & 2 & -16 \\
    0 & 4 & 3 &     6 & 3 & -1 & 7
  \end{array}\right) \\
%
  \xrightarrow{L_3 \leftarrow \frac{1}{4} L_3}
  \left( \begin{array}{ccc|cccc}
    1 & 2 & 1 &             1 & 2 & 0 & 4 \\ 
    0 & -8 & -5 &           -7 & -7 & 2 & -16 \\
    0 & 1 & \frac{3}{4} &   \frac{3}{2} & \frac{3}{4} & \frac{-1}{4} & \frac{7}{4}
  \end{array}\right)
  &\xrightarrow[L_3 \leftarrow L_3 + 8 L_2]{L_2 \leftrightarrow L_3}&
  \left( \begin{array}{ccc|cccc}
    1 & 2 & 1 &             1 & 2 & 0 & 4 \\ 
    0 & 1 & \frac{3}{4} &   \frac{3}{2} & \frac{3}{4} & \frac{-1}{4} & \frac{7}{4} \\
    0 & 0 & 1 &             5 & -1 & 0 & -2
  \end{array}\right) \\
%
  \xrightarrow[L_1 \leftarrow L_1 - L_3]{L_2 \leftarrow L_2 - \frac{3}{4} L_3}
  \left( \begin{array}{ccc|cccc}
    1 & 2 & 0 &   -4 & 3 & 0 & 6 \\ 
    0 & 1 & 0 &   \frac{-9}{4} & \frac{3}{2} & \frac{-1}{4} & \frac{13}{4} \\
    0 & 0 & 1 &   5 & -1 & 0  & -2
  \end{array}\right)
  &\xrightarrow{L_1 \leftarrow L_1 - 2L_2}&
  \left( \begin{array}{ccc|cccc}
    1 & 0 & 0 &  	\frac{1}{2} & 0 & \frac{1}{2} & \frac{-1}{2} \\ 
    0 & 1 & 0 &   \frac{-9}{4} & \frac{3}{2} & \frac{-1}{4} & \frac{13}{4} \\
    0 & 0 & 1 &   5 & -1 & 0 & -2
  \end{array}\right)
\end{eqnarray*}
On a donc
$$A^{-1} B = \frac{1}{4} \begin{pmatrix} 
  2 & 0 & 2 & -2 \\ 
  -9 & 6 & -1 & 13 \\ 
  20 & -4 & 0 & -8 
\end{pmatrix}$$


%
%
\section{Matrices triangulaires, matrice diagonales}
%
%
\paragraph{} Soient $n \geq 1$ un entier naturel et $A_{i j}$ une matrice carrée de taille $n\times n$.

%
\subsection{Matrice triangulaire supérieure}
%
\paragraph{Définition} On dit que $A$ est triangulaire supérieure si les coefficients de $A$ en-dessous de la diagonale sont nuls, autrement dit si 
$$i > j \Rightarrow a_{ij} = 0$$

%
\subsection{Matrice triangulaire inférieure}
%
\paragraph{Définition} On dit que $A$ est triangulaire inférieure si tous les coefficients de $A$ au-dessus de la diagonale sont nuls, autrement dit si 
$$i < j \Rightarrow a_{ij} = 0$$

%
\subsection{Matrice diagonale}
%
\paragraph{Définition} On dit que $A$ est diagonale si les coefficients de $A$ en dehors de la diagonale sont nuls, autrement dit si 
$$i\neq j \Rightarrow a_{ij} = 0$$

\paragraph{Théorème} Soit $A$ une matrice triangulaire supérieure de taille $n\times n$. Alors $A$ est inversible si et seulement si tous ses coefficients diagonaux sont nun nuls. Si $A$ est inversible, alors $A^{-1}$ est encore une matrice triangulaire supérieure.
\\\\
On applique le même pour les matrices triangulaires inférieures et les matrices diagonaux.

\paragraph{Démonstration} Comme $A$ est triangulaire supérieure, elle est échelonnée. Alors $A$ est inversible si et seuelement si sa form échelonnée réduit est $I_n$, si et seulement si chaque coefficient diagonal de $A$ donne un coefficicent pivot, autrement di si et seuelement si chacun de ces coefficients diagonaux est non nul. \\
Montrons par récurence sur $n \geq 1$ que l'inverse d'une matrice triangulaire supérieure inversible de taille $n\times n$ est encore triangulaire supérieure.
\begin{enumerate}[(1)]
  \item Si $n=1$, alors $A$ est de forme $A = ( a )$, pour un certain $a \in \R$. Puisque $A$ est inversible, $a \neq 0$ donc $a$ est inversible dans $\R$. On a alors 
    $$(a) (a^{-1}) = I_n = (a^{-1})(a)$$
    et donc $A^{-1} = (a^{-1})$ est triangulaire supérieure.
    
  \item On suppose que l'inverse de toute matrice triangulaire supérieure inversible de taille $(n-1) \times (n-1)$ est encore triangulaire supérieure.
  
  \item Soit $A$ une matrice triangulaire supérieure inversible de taille $n\times n$. Présentons $A$ par blocs de la façon suivante
    $$A = \left( \begin{array}{c|ccc}
      a &   & L &   \\ \hline
        & * & * & * \\
      0 & 0 & T & * \\
        & 0 & 0 & * \\
    \end{array} \right)$$
    où $a \in \R$, $L$ est une matrice ligne de taille $1 \times (n-1)$ et $T$ est une matrice triangulaire supérieure de taille $(n-1)\times(n-1)$. Du fait que $A$ est inversible, tous ses coefficients diagonaux sont non nuls. Cela implique que $a \neq 0$ et que les coefficients diagonaux de $T$ sont non nuls, donc $T$ est inversible. Par l'hypothèse de récurence, l'inverse $T^{-1}$ de $T$ est une matrice triangulaire supérieure. \\
    Calculons de façon générale le produit de deux matrices triangulaires supérieure décomposées par blocs:
    $$\left( \begin{array}{c|ccc}
        a' &   & L' & ~ \\ \hline
          &   &   &  ~  \\
        0 &   & T' & ~  \\
          &   &   &  ~ \\
      \end{array} \right)
      \left( \begin{array}{c|ccc}
        a &   & L &  ~ \\ \hline
          &   &   &  ~ \\
        0 &   & T &  ~ \\
          &   &   &  ~ \\
      \end{array} \right)
      =\left( \begin{array}{c|ccc}
        a' a &   & a' L + L' T & ~ \\ \hline
          &   &   & ~ \\
        0 &   & T' T & ~ \\
          &   &   & ~ \\
      \end{array} \right)$$
    On veut que $T' T = I_{n-1}$ et que $a' L + L' T = 0$; posons $a' = a^{-1}$, $T' = T^-{1}$ et $L' = - a^{-1} L T^{-1}$. Il vient
    \begin{eqnarray*}
      \left( \begin{array}{c|ccc}
        a^{-1} &   & -a^{-1} L T^{-1} & ~ \\ \hline
          &   &   &  ~  \\
        0 &   & T^{-1} & ~  \\
          &   &   &  ~ \\
      \end{array} \right) \cdot
      \left( \begin{array}{c|ccc}
        a &   & L &  ~ \\ \hline
          &   &   &  ~ \\
        0 &   & T &  ~ \\
          &   &   &  ~ \\
      \end{array} \right) &=& 
      \left( \begin{array}{c|ccc}
        a^{-1} a &   & a^{-1} L - a^{-1} L T^{-1} T & ~ \\ \hline
          &   &   & ~ \\
        0 &   & T^{-1} T & ~ \\
          &   &   & ~ \\
      \end{array} \right)  \\
      &=& \left( \begin{array}{c|ccc}
        1 &   & 0 & ~ \\ \hline
          &   &   & ~ \\
        0 &   & I_{n-1} & ~ \\
          &   &   & ~ \\
      \end{array} \right)
      = I_n
    \end{eqnarray*}
    On a également
    \begin{eqnarray*}
      \left( \begin{array}{c|ccc}
        a &   & L &  ~ \\ \hline
          &   &   &  ~ \\
        0 &   & T &  ~ \\
          &   &   &  ~ \\
      \end{array} \right) \cdot
      \left( \begin{array}{c|ccc}
        a^{-1} &   & -a^{-1} L T^{-1} & ~ \\ \hline
          &   &   &  ~  \\
        0 &   & T^{-1} & ~  \\
          &   &   &  ~ \\
      \end{array} \right) &=& 
      \left( \begin{array}{c|ccc}
        a a^{-1} &   & - a^{-1} L T^{-1} T + a^{-1} L & ~ \\ \hline
          &   &   & ~ \\
        0 &   & T T^{-1} & ~ \\
          &   &   & ~ \\
      \end{array} \right)  \\ 
      &=& \left( \begin{array}{c|ccc}
        1 &   & 0 & ~ \\ \hline
          &   &   & ~ \\
        0 &   & I_{n-1} & ~ \\
          &   &   & ~ \\
      \end{array} \right)
      = I_n
    \end{eqnarray*}
    donc
    $$A^{-1} = 
      \left( \begin{array}{c|ccc}
        a^{-1} &   & -a^{-1} L T^{-1} & ~ \\ \hline
          &   &   &  ~  \\
        0 &   & T^{-1} & ~  \\
          &   &   &  ~ \\
      \end{array} \right)$$
    est triangulaire supérieure. Par principe de récurence, la propriété est vrai pour tout $n \geq 1$
\end{enumerate} 
    

%
%
\section{Transposition}
%
%
%
\subsection{Matrice transposée}
%
\paragraph{Définition} Soit $A$ une matrice de taille $n\times m$. On appelle transposée de $A$ la matrice de taille $m\times n$ notée $A^T$, définie par
\begin{itemize}
  \item la colonne $i$ de $A$ devient la ligne $i$ de $A^T$
  \item la ligne $i$ de $A$ deivent la colonne $i$ de $A^T$
\end{itemize}

\paragraph{Exemple}
$$\text{Si } A = \begin{pmatrix} 1 & 3 & 0 \\ 2 & -7 & 4 \end{pmatrix} 
  \text{, alors } A^{-1} = \begin{pmatrix} 1 & 2 \\ 3 & -7 \\ 0 & 4 \end{pmatrix}$$
$$\text{Si } B = \begin{pmatrix} 4 & 1 & 7 \\ 0 & -5 & 3 \\ 1 & 2 & 6 \end{pmatrix} 
  \text{, alors } B^{-1} = \begin{pmatrix} 4 & 0 & 1 \\ 1 & -5 & 2 \\ 7 & 3 & 6 \end{pmatrix}$$

\paragraph*{Proposition}
\begin{enumerate}[1)]
  \item $\forall A, B \in M_{n\times m}(\R)$ on a
    $$(A + B)^{T} = A^{T} + B^{T}$$
  \item $\forall A \in M_{n\times m}(\R), \lambda \in \R$ on a
    $$(\lambda A)^{T} = \lambda A^{T}$$
  \item $\forall A \in M_{n\times m}(\R), B \in M_{m\times n}(\R)$ on a
    $$(A \cdot B)^{T} = B^{T} \cdot A^{T}$$
  \item $\forall A \in M_{n\times m}(\R)$ on a
    $$(A^{T})^{T} = A$$
  \item $\forall A \in M_{n}(\R)$ si $A$ est inversible, alors $A^{T}$ est inversible et 
    $$(A^{-1})^{T} = (A^{T})^{-1}$$
\end{enumerate}

%
%
\section{Matrices symétriques, matrices antisymétriques}
%
%
%
\subsection{Matrice symétrique}
%
\paragraph{Définition} Soit $A$ une matrice carrée de taille $n\times n$. On dit que $A$ est symétrique si $A^{T} = A$.

\paragraph{Exemple}$$\begin{pmatrix}1&-7\\-7&1\end{pmatrix} \text{ et }\begin{pmatrix}0&3&5\\3&1&6\\5&6&2\end{pmatrix}\text{ sont des matrices symétriques.}$$

%
\subsection{Matrice antisymétrique}
%
\paragraph{Définition} Soit $A$ une matrice carrée de taille $n\times n$. On dit que $A$ est antisymétrique si $A^{T} = -A$
\paragraph{Remarque} Si $A$ est antisymétrique $a_{ij}$ pour $i=j$ sont nuls.

\paragraph{Exemple}$$\begin{pmatrix}0&1\\-1&0\end{pmatrix} \text{ et }\begin{pmatrix}0&-7&-3\\7&0&2\\3&-2&0\end{pmatrix}\text{ sont des matrices antisymétriques.}$$

\paragraph{Théorème} Soit $A$ une matrice carrée. Alors $A$ peut s'écrire comme la somme d'une matrice symétrique et d'une matrice antisymétrique.

\paragraph{Démonstration}  On pose $C = \frac{1}{2} ( A + A^{T})$ et $D = \frac{1}{2} ( A - A^{T})$ et alors on a
$$ C + D = \frac{1}{2} ( A + A^{T}) + \frac{1}{2} ( A - A^{T}) = A$$
On a 
\begin{eqnarray*} 
  C^{T} = (\frac{1}{2} ( A + A^{T}))^{T} &=& \frac{1}{2} ( A^{T} + A) \\
    &=& \frac{1}{2} ( A + A^{T}) = C
\end{eqnarray*}
donc $C$ est symétrique. On a également
\begin{eqnarray*}
  D^{T} = (\frac{1}{2} ( A - A^{T}))^{T} &=& \frac{1}{2} ( A^{T} - A) \\
    &=& -\frac{1}{2} ( A - A^{T}) = -C
\end{eqnarray*}
donc $D$ est antisymétrique. Ainsi $A = C + D$ est une composition de la forme cherché.

%
%
\section{La trace}
%
%
%
\subsection{Trace d'une matrice}
%
\paragraph{Définition} Soit $A$ une matrice carrée de taille $n\times n$. On appelle trace de $A$ le nombre réel noté $tr(A)$ défini par
$$tr(A) = a_{11} + a_{22} + \ldots + a_{nn}$$
La trace de $A$ est donc la somme des coefficients diagonaux de $A$.

\paragraph{Exemple} 
$$A = \begin{pmatrix} 1 & 4 \\ 3 & -2 \end{pmatrix} \rightarrow tr(A) = -2 + 1 = -1$$
$$B = \begin{pmatrix} 7 & 3 & 4 \\ 2 & 0 & -6 \\ -3 & 1 & -2 \end{pmatrix} \rightarrow tr(B) = 7 + 0 - 2 = 5$$

\paragraph{Proposition}
\begin{enumerate}[1)]
  \item Soient $A, B \in M_{n}(\R)$. Alors
    $$tr(A + B) = tr(A) + tr(B)$$
  \item Soient $A \in M_{n}(\R)$ et $\lambda \in \R$. Alors
    $$tr(\lambda A) = \lambda \cdot tr(A)$$
  \item Soient $A, B \in M_{n}(\R)$. Alors
    $$tr(A \cdot B) = tr(B \cdot A)$$
  \item Soient $A \in M_{n}(\R)$ . Alors
    $$tr(A^{T}) = tr(A)$$
\end{enumerate}

\paragraph{Démonstration}
\begin{enumerate}[1)]
  \item 
    \begin{eqnarray*}
      tr(A + B) = \sum_{k=1}^{n}(A + B)_{kk} &=& \sum_{k=1}^{n}(A)_{kk} + (B)_{kk} \\
        &=& \sum_{k=1}^{n}(A)_{kk} + \sum_{k=1}^{n}(B)_{kk} \\
        &=& tr(A) + tr(B)
    \end{eqnarray*}
    
  \item
    \begin{eqnarray*}
      tr(\lambda A) = \sum_{k=1}^{n}(\lambda A)_{kk} &=& \sum_{k=1}^{n} \lambda \cdot (A)_{kk} \\
        &=& \lambda \sum_{k=1}^{n}(A)_{kk} \\
        &=& \lambda \cdot tr(A)
    \end{eqnarray*}
  
  \item 
    \begin{eqnarray*}
      tr(A \cdot B) = \sum_{k=1}^{n} (A B)_{kk} &=& \sum_{k=1}^{n} \left( \sum_{l=1}^{n} A_{kl} B_{lk} \right) \\
        &=& \sum_{l=1}^{n} \left( \sum_{k=1}^{n} A_{kl} B_{lk} \right) \\
        &=& \sum_{l=1}^{n} \left( \sum_{k=1}^{n} B_{lk} A_{kl} \right) \\
        &=& \sum_{l=1}^{n} (B A)_{ll} = tr(B \cdot A)
    \end{eqnarray*}
    
  \item
    $$tr(A^{T}) = \sum_{k=1}^{n} (A^{T})_{kk} = \sum_{k=1}^{n} (A)_{kk} = tr(A)$$
\end{enumerate}
