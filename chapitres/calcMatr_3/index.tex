\chapter{Calcul matriciel}

%
%
\section{Matrices}
%
%

%
\subsection{Matrice}
%
\paragraph{Définition} Soient $n \geq 1$ et $m \geq 1$ deux entiers naturels. On appelle matrice de taille $n\times m$ à coefficients dans $\R$ un tableau à $n$ lignes et $m$ colonnes de la forme
$$\begin{pmatrix}
  a_{11} & a_{12} & \ldots & a_{1m} \\
  a_{21} & a_{22} & \ldots & a_{2m} \\
  \vdots & \vdots & \ddots & \vdots \\
  a_{n1} & a_{n2} & \ldots & a_{nm}
\end{pmatrix} \text{ avec } a_{ij} \in \R$$

%
\subsection{Ensemble des matrices}
%
\paragraph{Définition} On note $M_{n, m}(\R)$ l'ensemble des matrices de taille $n\times m$ à coefficients dans $\R$.

\paragraph{Exemple}
$\begin{pmatrix}
  2 & -5 & \sqrt{2} \\
 3 &0 & 1 \\
\end{pmatrix}$ est une matrice de taille $2\times 3$ à coefficients dans $\R$. Le coefficient $a_{12}$ est -5.

%
\subsection{Matrice carrée}
%
\paragraph{Définition} Soit $n \geq 1$ un entier naturel et $A$ une matrice de taille $n\times n$. On dit alors que $A$ est une matrice carrée. \\ 
On note $M_{n}(\R)$ l'ensemble des matrices carrées de taille $n\times n$ à coefficients dans $\R$. Les coefficients $a_{ii}$ pour $1 \leq i \leq n$ sont appellés coefficients diagonaux de $A$.

%
%
\section{Opérations sur les matrices}
%
%

%
\subsection{Somme}
%
\paragraph{Définition} Soient $m\geq 1$ et $n \geq 1$ deux entiers naturels. On peut définir la somme de deux matrices de même taille. Si $A = (a_{ij}), B = (b_{ij}) \in M_{n, m}(\R)$ on définit leur somme $S = A + B \in M_{n, m}(\R)$ par
$$s_{ij} = a_{ij} + b_{ij} ~ \forall ~ 1 \leq i \leq n, 1 \leq j \leq m$$

\paragraph{Proposition} Soient $m \geq 1$ et $n\geq 1$ des entiers naturels.
\begin{enumerate}
  \item Soient $A, B, C \in M_{n, m}(\R)$. Alors
    $$(A + B) + C = A + (B + C)$$
  \item On note $0$ la matrice nulle dans $M_{n, m}(\R)$, c'est-à-dire la matrice dont tous les coefficients sont nuls. Alors pour tout $A \in M_{n, m}(\R)$ on a
    $$A + 0 = A = 0 + A$$
  \item Soit $A \in M_{n, m}(\R)$. On note $-A$ la matrice $(-a_{ij})$ dans l'ensemble $M_{n, m}(\R)$. Alors on a
    $$(-A) + A = 0 = A + (-A)$$
\end{enumerate}
Ainsi, l'addition $+$ dans $M_{n, m}(\R)$ est associative, unifère et tout $A \in M_{n, m}(\R)$ admet une matrice opposée. Par conséquence $(M_{n, m}(\R), +)$ est un groupe.\\
De plus, comme $+$ dans $\R$ est commutative, l'addition dans $M_{n,m}(\R)$ est commutative:
$$\forall A, B \in M_{n, m}(\R), ~ A + B = B + A$$
En conclusion, $(M_{n, m}(\R)$ est un groupe commutatif.

\paragraph{Démonstration} \begin{enumerate}
\item Cela découle de l'associativié de l'addition dans $\R$.
\item Cela découle de la propriété de l'élément $0$ dans $\R$.
\item Cela résulte des propriétés de l'opposé d'un réel.
\end{enumerate}

%
\subsection{Produit avec un réel}
%
\paragraph{Définition} Soient $n \geq 1, m \geq 1$ deux entiers naturels, $A = (a_{ij}) \in M_{n, m}(\R)$ et $\alpha \in \R$. On  définit le produit de $A$ par $\alpha$ que l'on note $\alpha A$, par
$$\alpha A = (\alpha a_{ij}) \in M_{n, m}(\R)$$

\paragraph{Proposition} Soient $n \geq 1, m \geq 1$ des entiers naturels. On a
\begin{enumerate}
  \item $\forall \alpha \in \R, ~ A, B \in M_{n, m}(\R)$ 
    $$\alpha \cdot (A + B) = \alpha \cdot A + \alpha \cdot B$$
    
  \item $\forall \alpha, \beta \in \R, ~ A \in M_{n, m}(\R)$
    \begin{eqnarray*}
      (\alpha + \beta) \cdot A &=& \alpha \cdot A + \beta \cdot A \\
      (\alpha \cdot \beta) \cdot A &=& \alpha \cdot (\beta \cdot A)
    \end{eqnarray*}
    
  \item $\forall A \in M_{n, m}(\R)$
    $$1 \cdot A = A$$
\end{enumerate}
On peut donc dire, que $M_{n, m}(\R)$ muni de l'addition des matrices et de la multiplication par les réels est un $\R$-espace vectoriel.

%
\subsection{Produit matriciel}
%
\paragraph{Définition} Soient $n \geq 1, m \geq 1$ et $p \geq 1$ trois entiers naturels. On peut définir le produit $A \times B$ d'une matrice $A$ et d'une matrice $B$ dans cet ordre si le nombre de colonnes de $A$ est égal au nombre de lignes de $B$. \\
Si $A = (a_{ij}) \in M_{n, m}(\R)$ et $B = (b_{ij}) \in M_{m, p}(\R)$, alors on définit $C = A \times B \in M_{n, p}(\R)$
$$c_{ij} = \sum_{l=1}^{m} {a_{il} * b_{lj} = a_{i1} \cdot b_{1j} + a_{i2} \cdot b_{2j} + \ldots + a_{im} \cdot b_{mj}$$

\paragraph{Exemple} Soient $A=\begin{pmatrix}
  2 & -5 & 1} \\
 0 & 4 & 7 \\
\end{pmatrix}\in M_{2,3}(\R)$ et $B=\begin{pmatrix}
  1 & -5 & 2 & 4} \\
2 & 4 & -7 & 3\\
 0 & 1 & 0 & 1 \\
\end{pmatrix}\in M_{3,4}(\R)$. Le nombre de colonnes de $A$ est égal au nombre de ligne de $B$. Le produit est donc défini. $AB=\begin{pmatrix}
  -8 & -29 & 39 & -6} \\
8 & 23 & -28 & 19\\
\end{pmatrix}\in~ M_{2,4}(\R)$

\paragraph{Proposition} A conditions que les produits et les sommes considerés soient bien définit, on a
\begin{enumerate}
  \item $$A \times (B \times C) = (A \times B) \times C$$
  \item $$A \times (B + C) = A \times B + A \times C$$ 
    $$(B + C) \times A = B \times A + C \times A$$
  \item Pour tout entier $m \geq 1$ on appelle matrice identité de taille $n \times n$ la matrice carrée
    $$I_n = (a_{ij}) ~ ~ a_{ij} = \left\{ \begin{array}{lr} 1, & i=j \\ 0, & i \neq j \end{array}$$
\end{enumerate}

\paragraph{Attention} Deux matrices quelconques ne commutent pas nécessairement.

\paragraph{Exemple} Soient $A=\begin{pmatrix}
  1 & 3 \\
 2 & 7 \\
\end{pmatrix}$ et $B=\begin{pmatrix}
  -3 & 0 \\
 4 & 1 \\
\end{pmatrix}$. Alors $AB=\begin{pmatrix}
  9 & 3 \\
 22 & 7 \\
\end{pmatrix}$ et $BA=\begin{pmatrix}
  -3 & -9 \\
 6 & 19 \\
\end{pmatrix}$. Donc $AB\neq BA$.

\paragraph{Attention}
\begin{itemize}
  \item Si $A$ et $B$ sont deux matrices on peut avoir $A \neq 0, B \neq 0$ et $AB = 0$
  \item On peut donc avoir également $AB = AC$ et $B \neq C$ pour des matrices $A, B, C$.
\end{itemize}

\paragraph{} On a, $(M_{n, m}(\R), +, \times, \cdot)$ est une $\R$-algèbre. En particulier $(M_{n, m}(\R), +, \times)$ est un anneau.

%
%
\section{Inversion des matrices}
%
%

\paragraph{Définition} Soit $n \geq 1$ un entier naturel. Soit $A$ une matrice carrée de taille $n \times n$. On dit que $A$ est inversible s'il existe une matrice carrée $B$ de taille $n \times n$ telle que $A B = I_n = B A$. Dans ce cas on dit que $B$ est un inverse de $A$.

\paragraph{Proposition} Soit $A$ une matrice carrée de taille $n \times n$. Si $A$ est inversible alors son inverse est unique.

\paragraph{Démonstration} On suppose que $A$ est inversible. Si $B$ et $C$ sont deux inverses pour $A$, alors on a
$$B A C = (B A) C = I_n C = C$$
on a également
$$B A C = B (A C) = B I_n = B$$
d'où $B = C$.

\paragraph{Notation} Si $A$ est inversible, on note $A^{-1}$ son inverse. On a donc
$$A A^{-1} = I_n = A^{-1} A$$

\paragraph{Exemple} La matrice
\begin{pmatrix}2& 1\\7& 4\end{pmatrix}est inversible d'inverse\begin{pmatrix}4& -1\\-7& 2\end{pmatrix}.\\\begin{pmatrix}2& 1\\7& 4\end{pmatrix}\begin{pmatrix}4& -1\\-7& 2\end{pmatrix}=\begin{pmatrix}1& 0\\0& 1\end{pmatrix}=$I_2$ et \begin{pmatrix}4& -1\\-7& 2\end{pmatrix}\begin{pmatrix}2& 1\\7& 4\end{pmatrix}=\begin{pmatrix}1& 0\\0& 1\end{pmatrix}=$I_2$\\On vérifie que $A A^{-1} = I_n = A^{-1} A$.

%
\subsection{Matrice de taille $2\times 2$}
%
\paragraph{Lemme} Soient $a, b, c, d \in \R, A = \begin{pmatrix} a & b \\ c & d \end{pmatrix} \in M_{2}(\R)$ et $B = \begin{pmatrix} d & -b \\ -c & a \end{pmatrix} \in M_{2}(\R)$. Alors $A B = (a d - b c) I_n = B A$.

\paragraph{Démonstration} $$AB=\begin{pmatrix} a & b \\ c & d \end{pmatrix} \begin{pmatrix} d & -b \\ -c & a \end{pmatrix}=\begin{pmatrix}ad-bc &0\\0 &ad-bc\end{pmatrix}=(ad-bc)\begin{pmatrix}1& 0\\0& 1\end{pmatrix}=(ad-bc)I_2$$
et $$BA= \begin{pmatrix} d & -b \\ -c & a \end{pmatrix}\begin{pmatrix} a & b \\ c & d \end{pmatrix}=\begin{pmatrix}ad-bc &0\\0 &ad-bc\end{pmatrix}=(ad-bc)I_2$$

\paragraph{Proposition} Soit $A = \begin{pmatrix} a & b \\ c & d \end{pmatrix} \in M_{2}(\R)$. Alors $A$ est inversible si et seuelement si $a d - b c \neq 0$, et dans ce cas
$$A^{-1} = \frac{1}{a d - b c} \begin{pmatrix} d & -b \\ -c & a \end{pmatrix}$$

\paragraph{Démonstration} Cet énoncé est une équicalence constituée de deux implications:
\begin{enumerate}
\item si $ad-bc\neq 0$, alors $A$ est inversible et $A^{-1}=\frac{1}{ad-bc}\begin{pmatrix} d & -b \\ -c & a\end{pmatrix}$
\item si $ad-bc=0$, alors $A$ n'est pas inversible.
\end{enumerate}
\begin{enumerate}
\item On suppose que $a d - b c \neq 0$. Par suite, $a d - b c$ est inversible dans $\R$. D'après le lemme, on a
$$\begin{pmatrix} a & b \\ c & d \end{pmatrix} \begin{pmatrix} d & -b \\ -c & a \end{pmatrix} = (a d - b c) I_2 =  \begin{pmatrix} d & -b \\ -c & a \end{pmatrix}\begin{pmatrix} a & b \\ c & d \end{pmatrix}$$
d'où
$$\begin{pmatrix} a & b \\ c & d \end{pmatrix}\left( \frac{1}{a d - b c} \begin{pmatrix} d & -b \\ -c & a \end{pmatrix}\right)=I_2=\left( \frac{1}{a d - b c} \begin{pmatrix} d & -b \\ -c & a \end{pmatrix}\right)\begin{pmatrix} a & b \\ c & d \end{pmatrix}$$
Par conséquent, $A$ est inversible d'inverse $A^{-1}=\frac{1}{a d - b c}\begin{pmatrix} d & -b \\ -c & a\end{pmatrix}$.
\item On suppose que $a d - b c = 0$. On raisonne par l'absurde. Si $A$ est inversible, on obtient: D'après le lemme on a
$$\begin{pmatrix} a & b \\ c & d \end{pmatrix} \begin{pmatrix} d & -b \\ -c & a \end{pmatrix} = (a d - b c) I_2 = 0$$
d'où
\begin{eqnarray*}
  A^{-1} \left( A \begin{pmatrix} d & -b \\ -c & a \end{pmatrix} \right) &=& A^{-1} 0 = 0 \\
    &=& A^{-1} A \begin{pmatrix} d & -b \\ -c & a \end{pmatrix} \\
  0 &=& \begin{pmatrix} d & -b \\ -c & a \end{pmatrix}
\end{eqnarray*}
Par conséquent $a = b = c = d = 0$, donc $A = 0$ donc $A$ n'est pas inversible, ce qui donn lien à une contradiction. Ainsi $A$ n'est pas inversible si $a d - b c = 0$.
\end{enumerate}

\paragraph{Proposition} Soient $A$ et $B$ des matrices inversibles de taille $n \times n$. Alors
\begin{enumerate}
  \item $A^{-1}$ est inversible d'inverse $A$
  \item $A B$ est inversible et $(A B)^{-1} = B^{-1} A^{-1}$
\end{enumerate}

\paragraph{Démonstration}
\begin{enumerate}
  \item Puisuque $A$ est inversible d'inverse $A^{-1}$, on a
    $$A A^{-1} = I_n = A^{-1} A$$
    donc $A^{-1}$ est inversible d'inverse $A$.
  \item On a
    $$(A B)(B^{-1} A^{-1}) = A (B B^{-1}) A^{-1} = A I_n A^{-1} = A A^{-1} = I_n$$
    De même, on a $(B^{-1} A^{-1}) A B = I_n$ \\
    Par conséquence, $A B$ est inversible d'inverse $(B^{-1} A^{-1})$
\end{enumerate}

\paragraph{Proposition} Plus généralement, si $A_1, A_2, \ldots, A_m$ sont des matrices inversibles on a $(A_1 A_2 \ldots A_m)^{-1} = A_m^{-1} A_{m-1}^{-1} \ldots A_1^{-1}$

%
\subsection{Groupe des matrices inversibles}
%
\paragraph{Proposition} L'ensemble des matrices inversibles de taille $n \times n$ muni de la multiplication des matrices est un group. En effet
\begin{itemize}
  \item L'assertion $2.$ de la proposition précédante assure que $\times$ restreinte à l'ensemble des matrices inversibles est une loi de composition interne.
  \item $\times$ étant associative sur $M__{n}(\R)$, elle l'est en restriction à l'ensemble des matrices inversibles.
  \item $I_n$ est élément neutre.
  \item L'assertion $1.$ de la proposition assure que l'inverse d'une matrice inversible est bien inversible.
\end{itemize}
On note $(GL_{n}(\R), \times)$ le groupe des matrices inversibles de taille $n$ ($GL$ est mis pour "groupe linéaire").

\paragraph{Exemple} Soient $A=\begin{pmatrix}2&7\\1&4\end{pmatrix}$ et B=\begin{pmatrix}5&3\\3&2\end{pmatrix}.\\
$A$ est inversible car $2\cdot 4-1\cdot 7=1\neq 0$ , et
$$A^{-1}= \frac{1}{1} \begin{pmatrix}2&-7\\-1&4\end{pmatrix}= \begin{pmatrix}2&-7\\-1&4\end{pmatrix}$$
$B$ est inversible car $5\cdot2-3\cdot3=1\neq 0$, et 
$$B^{-1}=\frac{1}{1}\begin{pmatrix}2&-3\\-3&5\end{pmatrix}= \begin{pmatrix}2&-3\\-3&5\end{pmatrix}$$
On a $$AB=\begin{pmatrix}2&7\\1&4\end{pmatrix}\begin{pmatrix}5&3\\3&2\end{pmatrix}=\begin{pmatrix}31&20\\17&11\end{pmatrix}$$ et
$$B^{-1}A^{-1}=\begin{pmatrix}31&20\\17&11\end{pmatrix} \begin{pmatrix}2&-7\\-1&4\end{pmatrix}= \begin{pmatrix}11&-20\\-17&31\end{pmatrix}$$
On a $(31\cdot11)-(17\cdot20)=341-340=0\neq0$. Donc $AB$ est inversible $$\frac{1}{1} \begin{pmatrix}11&-20\\-17&31\end{pmatrix}= \begin{pmatrix}11&-20\\-17&31\end{pmatrix}$$ qui est bien égale à $B^{-1}A^{-1}$.

%
%
\section{Matrices élémentaires}
%
%

%
\subsection{Matrice élémentaire}
%
\paragraph{Définition} Soient $n \geq 1$ un entier naturel et $(i, j) \in \{1; \ldots; n\} \times \{1; \ldots; n\}$. On note $F_{i, j}$ la matrice carrée de taille $n \times n$ contenant un $1$ en $i^{ème}$ ligne et $j^{ème}$ colonne et des $0$ partout ailleurs. On dit que $F_{i, j}$ est une matrice standard.

\paragraph{Exemple} $$n=3 \rightarrow F_{2,3}=\begin{pmatrix}0&0&0\\0&0&1\\0&0&0\end{pmatrix}\text{ ou  } n=4\rightarrow F_{4,1}\begin{pmatrix}0&0&0&0\\0&0&0&0\\0&0&0&0\\1&0&0&0\\\end{pmatrix}$$

\paragraph{Proposition} Toute matrice carrée de taille $\times n$ peut s'exprimer à l'aide des matrices standard de taille $n\times n$
$$A = \sum_{(i, j) \in \{1; \ldots; n\}\times\{1; \ldots; n\}} a_{ij} F_{ij}$$

\paragraph{Proposition} Soit $A$ une matrice de taille $n\times m$.
\begin{enumerate}[a)]
  \item Soit $F_{ij}$ une matrice standard de taille $n\times n$. Alors $F_{ij} A$ est la matrice de taille $n \times m$ dont la $i^{ème}$ ligne est la $j^{ème}$ ligne de $A$ et dont les autres sont nulles.
  \item Soit $F_{ij}$ une matrice standard de taille $m\times m$. Alors $A F_{ij}$ est la matrice de taille $n\times m$ dont la $j^{ème}$ colonne est la $i^{ème}$ de $A$ et dont les autres colonnes sont nulles.
\end{enumerate}

\paragraph{Démonstration}
\begin{enumerate}[a)]
  \item Soit $k\in\{1,\cdots,n\}$ et $l\in\{1,\cdots,n\}$. On a
$$(F_{ij}A)_{k l}=\sum\limits_{r}^n{(F_{i j})_{k r}A_{r l}}$$
si $k\neq i$, alors la $k$-ième ligne de $F_{i j}$ est nulle, donc
$$(F_{i j})_{k l}=0 \text{ pour tout } l \text{.}$$
On obtient alors
$$(F_{i j})_{k l}=0 \text{ si } k\neq i$$
si $k=i$, alors 
$$(F_{i j}A)_{i l}= \sum\limits_{r=1}^n{(F_{i j})_{i r}A_{r l}}$$
si $\neq j$, alors la $r$-ième colonne de $F_{i j}$ est nulle, donc
$$(F_{ij})_{i r}=0$$
il suit donc
$$(F_{i j}A)_{i l}= (F_{i j})_{i j}A_{j l}=A_{j l}$$
D'après ce qui procède, si $k\neq i$, alors la $k$-ième ligne de $F_{i j}A$ est nulle. Si $k=i$, alors al $i$-ième ligne de $F_{i j}A$ est $(A_{j 1}A_{j 2}\cdots A_{j m})$ qui est la $j$-ième ligne de $A$.
  \item Ceci se montre de façon similaire.
\end{enumerate}

\paragraph{Exemple} Soient A=\begin{pmatrix}1&-7&5\\0&3&4\end{pmatrix}\in M_{2\times3}(\R)$ et $F_{1 2}=\begin{pmatrix}0&1\\0&0\end{pmatrix}\in M_{2\times 2}(\R)$. Alors
$$F_{1 2}A=\begin{pmatrix}0&1\\0&0\end{pmatrix}\begin{pmatrix}1&-7&5\\0&3&4\end{pmatrix}=\begin{pmatrix}0&3&3\\0&0&0\end{pmatrix}$$
Soient $A=\begin{pmatrix}1&5\\0&4\\-7&2\end{pmatrix}\in  M_{3\times2}(\R)$  et $F_{3 1}=\begin{pmatrix}0&0&0\\0&0&0\\1&0&0\end{pmatrix}\in M_{3\times3}(\R)$. Alors
$$F_{3 1}A=\begin{pmatrix}0&0&0\\0&0&0\\1&0&0\end{pmatrix}\begin{pmatrix}1&5\\0&4\\-7&2\end{pmatrix}= \begin{pmatrix}0&0\\0&0\\1&5\end{pmatrix}$$
Soient $A=\begin{pmatrix}1&5\\0&4\\-7&2\end{pmatrix}\in  M_{3\times2}(\R)$  et $F_{1 2}=\begin{pmatrix}0&1\\0&0\end{pmatrix}\in M_{2\times2}(\R)$. Alors
$$AF_{1 2}=\begin{pmatrix}1&5\\0&4\\-7&2\end{pmatrix}\begin{pmatrix}0&1\\0&0\end{pmatrix}=\begin{pmatrix}0&1\\0&0\\0&-7\end{pmatrix}$$

%
\subsection{Symbole de Kronecker}
%
\paragraph{Définition} Soit $n \geq 1$ un entier naturel. On note
$$\mS: \{1; \ldots; n\}\times\{1; \ldots; n \} \rightarrow {0; 1}$$
L'application est défini par
$$\delta_{ij} = \left\{\begin{array}{lr} 1 & \text{si i=j} \\ 0 & \text{sinon} \end{array} \right.$$ 

\paragraph{Corollaire} Soient $n \geq 1$ un entier naturel et
$$(i, j), (k, l) \in \{1; \ldots; n\}\times\{1; \ldots; n\}$$
alors
$$F_{ij}F_{kl} = \delta_{ik} F_{il} = 
  \left\{\begin{array}{lr} F_{il} & \text{si i=j} \\ 0 & \text{sinon} \end{array} \right.$$

\paragraph{Démonstration} D'après la proposition, précédente, $F_{ij} F_{kl}$ est la matrice don la $i^{ème}$ ligne est la $j^{ème}$ ligne de $F_{kl}$ et dont les autres lignes sont nulles. Il vient
\begin{itemize}
  \item si $j\neq k$, la $j^{ème}$ ligne de $F_{kl}$ est nulle
  \item si $j = k$ la $j^{ème} = k^{ème}$ ligne de $F_{kl}$ est $(0_{1} \ldots 0_{l-1} ~ 1_{l} ~ 0_{l+1} \ldots 0_{n})$
\end{itemize}
Par conséquent, si $j\neq k$, toutes les lignes de $F_{ij} F_{kl}$ sont nulles donc $F_{ij} F_{kl} = 0$; si $j=k$, alors 
$$F_{ij} F_{kl} = \begin{pmatrix} 
  & & & 0 & \\
  & & & \vdots & \\
  & & & 0 & \\
  0 & \ldots & 0 & 1 & 0 \\
  & & & 0 & \\
\end{pmatrix} = F_{il}$$
Ainsi $F_{ij} F{kl} = \delta_{ik} F_{il}$.

%
\subsection{Les opérations élémentaires}
%
\paragraph{Rappel} On rappelle que l'on distingue trois types d'opérations éLémentaires que l'on peut effectuer sur les lignes d'une matrice
\begin{itemize}
  \item multiplier une ligne par un réel
  \item ajouter à une ligne un multiple quelconque d'une autre ligne
  \item permuter deux lignes
\end{itemize}
On introdiut les matrices suivantes ($n \geq 1$ entier naturel)
\begin{itemize}
  \item $c \in \R \backslash \{0 \}, i \in \{1; \ldots; n\}$
    $$E_i(c) = \left( \sum_{l=1, l\neq i}^n F_{ll} \right) +  c F_{ii} = 
      \begin{pmatrix}
        1 & & \\
        & 1 & \\
        & & \ddots & \\
        & & & c & \\
        & & & & 1
      \end{pmatrix}$$
      
  \item $c \in \R, (i, j) \in \{1;\ldots;n\}\times\{1;\ldots;n\}, i\neq j$
    $$E_{ij}(c) = I_n + c F_{ij} =
      \begin{pmatrix}
        1 & & & \\
        & 1 & & c \\
        & & \ddots & \\
        & & & 1
      \end{pmatrix}$$
      
  \item $(i, j) \in \{1; \ldots; n\}\times\{1;\ldots; n\}, i\neq j$
    $$E_{ij} = \left( \sum_{l=1, l \notin {i; j}}^n F_{ll} \right) + F_{ij} + F_{ji} = 
      \begin{pmatrix}
        1 & & & & & & 0 \\
        & 0 & & & & 1 & \\
        & & 1 & & 0 & & \\
        & & & \ddots & & & \\
        & & 0 & & 1 & & \\
        & 1 & & & & 0 & \\
        0 & & & & & & 1
      \end{pmatrix}$$
\end{itemize} 
Les opérations élémentaires que l'un peut effectuer sur les lignes d'une matrice peuvent s'interpréter en termes de multiplications de cette par les matrices précédantes.

\paragraph{Exemple} Pour $n=2$ on a
$$E_1(-3)=\begin{pmatrix}-3&0\\0&1\end{pmatrix} \text{  } E_{1 2}=\begin{pmatrix}0&1\\1&0\end{pmatrix} \text{  } E_{2 1}(5)=\begin{pmatrix}1&5\\0&1\end{pmatrix}$$
 Pour $n=3$ on a
$$E_2(-4)=\begin{pmatrix}1&0&0\\0&-4&0\\0&0&1\end{pmatrix} \text{  } E_{2 1}=\begin{pmatrix}0&1&0\\1&0&0\\0&0&1\end{pmatrix} \text{  } E_{2 3}(7)=\begin{pmatrix}1&0&0\\0&1&7\\0&0&1\end{pmatrix}$$

\paragraph{Théorème} Soit $A$ une matrice de taille $n\times m$
\begin{enumerate}[a)]
  \item Multiplier la $i^{ème}$ ligne de $A$ par un réel $c$ non nul est équvalent à multiplier $A$ à gauche par la matrice $E_i(c)$ de taille $n \times n$
  
  \item Ajouter à la $i^{ème}$ ligne de $A$ $c$-fois la ligne $j$ de $A$ où $i\neq j$ et $c\in \R$ est équivalent à multiplier $A$ à gauche par la matrice $E_{ij}(c)$.
  
  \item Permuter la $i^{ème}$ ligne et la $ĵ^{ème}$ ligne de $A$ où $i\neq j$ est équivalent à multiplier $A$ à gauche par la matrice $E_{ij}$.
\end{enumerate}

\paragraph{Démonstration}
\begin{enumerate}[a)]
  \item $$E_i(c) A = \left( \sum_{l=1, l\neq i}^n F_{ll} \right) A +  c F_{ii} A$$
    Si $l\neq i$ on a
    $$F_{ll} A =
      \begin{pmatrix}
        \\
        l^{ème} \text{ ligne de } A \\
        \\
      \end{pmatrix} \begin{array}{c} 1, \ldots, l-1 \\ l \\ l+1, \ldots, n \end{array}$$
    sinon, on a
    $$c F_{ii} A =
      \begin{pmatrix}
        \\
        c\text{-fois } i^{ème} \text{ ligne de } A \\
        \\
      \end{pmatrix} \begin{array}{c} 1, \ldots, i-1 \\ i \\ i+1, \ldots, n \end{array}$$
    On voit alors que $\left( \sum_{l=1, l\neq i}^n F_{ll} \right) A +  c F_{ii} A$ est la matrice cherchée.
    
  \item $$E_{ij}(c) A = (I_n + c F_{ij}) A = E_{ij}(c) = A + c F_{ij} A$$
    Où on a
      $$c F_{ij} A =
    \begin{pmatrix}
      \\
      c\text{-fois } j^{ème} \text{ ligne de } A \\
      \\
    \end{pmatrix} \begin{array}{c} 1, \ldots, i-1 \\ i \\ i+1, \ldots, n \end{array}$$
    Donc $A + c F_{ij} A$ est bien la matrice cherchée.
    
  \item $$E_{ij} A = \left( \sum_{l=1, l \notin {i; j}}^n F_{ll} A \right) + F_{ij} A + F_{ji} A$$
    Si $l\neq i$ et $l\neq j$ 
    $$F_{ll} A =
      \begin{pmatrix}
        \\
        l^{ème} \text{ ligne de } A \\
        \\
      \end{pmatrix} \begin{array}{c} 1, \ldots, l-1 \\ l \\ l+1, \ldots, n \end{array}$$
    sinon
      $$F_{ij} A =
      \begin{pmatrix}
        \\
        j^{ème} \text{ ligne de } A \\
        \\
      \end{pmatrix} \begin{array}{c} 1, \ldots, i-1 \\ i \\ i+1, \ldots, n \end{array}$$
      $$F_{ji} A =
      \begin{pmatrix}
        \\
        i^{ème} \text{ ligne de } A \\
        \\
      \end{pmatrix} \begin{array}{c} 1, \ldots, j-1 \\ j \\ j+1, \ldots, n \end{array}$$
    Ainsi $\left( \sum_{l=1, l \notin {i; j}}^n F_{ll} A \right) + F_{ij} A + F_{ji} A$ est la matrice cherchée.
\end{enumerate}

\paragraph{Exemple} Soit $A=\begin{pmatrix}5&-3&0\\4&2&7\end{pmatrix}\in M_{2\times3}(\R)$. On veut multiplier la première ligne par $-7$.
$$A\xrightarrow{L_1\leftarrow L_1(-7)}\begin{pmatrix}-35&21&0\\4&2&7\end{pmatrix}$$
On calcule
$$E_1(-7)A=\begin{pmatrix}-7&0\\0&1\end{pmatrix}\begin{pmatrix}5&-3&0\\4&2&7\end{pmatrix}=\begin{pmatrix}-35&21&0\\4&2&7\end{pmatrix}$$
Ajouter à la première ligne de $A$ $3$ fois la deuxième ligne.
$$A\xrightarrow{L_1\leftarrow L_1+3L_2}\begin{pmatrix}17&3&21\\4&2&7\end{pmatrix}$$
On calcule
$$E_{1 2}(3)A=\begin{pmatrix}1&3\\0&1\end{pmatrix}\begin{pmatrix}5&-3&0\\4&2&7\end{pmatrix}=\begin{pmatrix}17&3&21\\4&2&7\end{pmatrix}$$

\paragraph{Théorème} Soit $n\in\N$ et $n\ge1$. Alors les matrices élémentaires de taille $n\times n$ sont inversibles, et l'on a:
$$E_i(C)^{-1}=E_i(\frac{1}{C}) \text{ avec }C\in\R\backslash\{0\}$$
$$E_{i j}(C)^{-1}=E_{i j}(-C) \text{ avec }C\in\R$$
$$E_{i j}^{-1}=E_{j i}$$

\paragraph{Démonstration}
$$E_i(C)E_i(\frac{1}{C})= \begin{pmatrix}
        1&0&\dots&&0\\
        0& \ddots&&&\vdots\\
        \vdots& & c\\
        & & & \ddots&0\\
       0 &\dots & & 0& 1
      \end{pmatrix} \begin{pmatrix}
        1&0&\dots&&0\\
        0& \ddots&&&\vdots\\
        \vdots& & \frac{1}{c}\\
        & & & \ddots&0\\
       0 &\dots & & 0& 1
      \end{pmatrix}= \begin{pmatrix}
        1&0&\dots&&0\\
        0& \ddots&&&\vdots\\
        \vdots& & 1\\
        & & & \ddots&0\\
       0 &\dots & & 0& 1
      \end{pmatrix}}=I_n$$
On vérifie de même que $}E_i(\frac{1}{C})E_i(C)$.
$$E_{i j}(C)E_{i j}(-C)=(I_n+CF_{i j})(I_n-CF_{i j})=I_n+CF_{i j}-CF_{i j}-C^2F_{i j}F_{i j}$$
puisque $i\neq j$, on a $F_{i j}F_{i j}=0$. Donc
$$E_{i j}(C)E_{i j}(-C)=I_n$$
On vérifie de même que $E_{i j}(-C)E_{i j}(C)=I_n$.
\begin{eqnarray*}
E_{i j}E_{j i}&=&\left( \sum\limits_{r=1, r\neq i, r\neq j}^n{\left( F_{r r}+F_{i j}+F_{j i}\right) }\right) \left( \sum\limits_{s=1, s\neq i, s\neq j}^n{(F_{s s}+F_{i j}+F_{j i}) }\right)\\
&=&(I_n-F_{i i}-F_{j j}+F_{i j}+F_{j i})(I_n-F_{i i}-F_{j j}+F_{i j}+F_{j i})\\
&=&I_n-F_{i i}-F_{j j}+F_{i j}+F_{j i}-F_{j j}+F_{i i}F_{i i}+F_{j j}F_{i i}-F_{i j}F_{i i}-F_{j i}F_{i i}\\&&-F_{j j}+F_{i i}F_{j j}+F_{j j}F_{j j}-F_{i j}F_{j j}-F_{j i}F_{j j}+F_{i j}-F_{i i}F_{i j}-F_{j j}F_{i j}\\&&+F_{i j}F_{i j}+F_{j i}F_{i j}+F_{j i}-F_{i i}F_{i i}-F_{j j}F_{j i}+F_{i j}F_{j i}+F_{j i}F_{j i}\\
&=&I_n-F_{i i}-F_{j j}+F_{i j}+F_{j i}-F_{i i}+F_{i i}-F_{j i}-F_{j j}\\
&&+F_{j j}-F_{i j}+F_{i j}-F_{i j}+F_{j j}+F_{j i}-F_{j i}+F_{i i}\\
&=&I_n
\end{eqnarray*}
Le fait que les matrices élémentaires soient inversibles confirme le fait que les opérations élémentaires soient réversibles.

\subsection{Écriture matricielle des systèmes linéaires}

%
\subsection{Calculer l'inverse d'une matrice}
%
\paragraph{} Soit $A$ matrice carrée de taille $n\times n$. \\
On prends la matrice $(A \vert I_m)$. En utilisant les opérations élémentaires on met la matrice $E$ en forme échlonné réduite et on éffectue parallelement les mêmes opérations éLémentaires sur les lignes de $I_n$. Si $A$ sous la forme échelonnée réduite est àgal à $I_n$ $A$ est inversible. La matrice inverse de $A$ est alors la partie $I_n$ où on a éffectue les opérations élémentaires. \\
$E_1, E_2, \ldots, E_l$ sont des matrices élémentaires.
\begin{eqnarray*}
  A \cdot E_1 \cdot E_2 \cdot &\ldots& \cdot E_l = I_n \\
  &\Downarrow& \\
  I_n \cdot E_1 \cdot E_2 \cdot &\ldots& \cdot E_l = A^{-1} \\
  &\Downarrow& \\
  E_1 \cdot E_2 \cdot &\ldots& \cdot E_l = A^{-1}
\end{eqnarray*}
De la même manière on peut calculer $A^{-1} B$.
$$A^{-1} B = E_1 \cdot E_2 \cdot \ldots \cdot E_l \cdot B$$
où $E_1, E_2, \ldots, E_l$ sont les opérations élémentaires dont on à besoin pour mettre $A$ sous forme échelonnée réduite égal à $I_n$.

%
%
\section{Matrices triangulaires, matrice diagonales}
%
%
\paragraph{} Soient $n \geq 1$ un entier naturel et $A_ij$ une matrice carrée de taille $n\times n$.

%
\subsection{Matrice triangulaire supérieure}
%
\paragraph{Définition} On dit que $A$ est triangulaire supérieure si les coefficients de $A$ en-dessous de la diagonale sont nuls, autrement dit si $i > j \Rightarrow a_{ij} = 0$

%
\subsection{Matrice triangulaire inférieure}
%
\paragraph{Définition} On dit que $A$ est triangulaire inférieure si tous les ocefficients de $A$ au-dessus de la diagonale sont nuls, autrement dit si $i < j \Rightarrow a_{ij} = 0$

%
\subsection{Matrice diagonale}
%
\paragraph{Définition} On dit que $A$ est diagonale si les coefficients de $A$ en dehors de la diagonale sont nuls, autrement dit si $i\neq j \Rightarrow a_{ij} = 0$

\paragraph{Théorème} Soit $A$ une matrice triangulaire supérieure de taille $n\times n$. Alors $A$ est inversible si et seulement si tous ses coefficients diagonaux sont nun nuls. Si $A$ est inversible, alors $A^{-1}$ est encore une matrice triangulaire supérieure.
\\\\
On applique le même pour les matrices triangulaires inférieures et les matrices diagonaux.

\paragraph{Démonstration} Comme $A$ est triangulaire supérieure, elle est échelonnée. Alors $A$ est inversible si et seuelement si sa form échelonnée réduit est $I_n$, si et seulement si chaque coefficient diagonal de $A$ donne un coefficicent pivot, autrement di si et seuelement si chacun de ces coefficients diagonaux est non nul. \\
Montrons par récurence sur $n \geq 1$ que l'inverse d'une matrice triangulaire supérieure inversible de taille $n\times n$ est encore triangulaire supérieure.
\begin{enumerate}[(1)]
  \item Si $n=1$, alors $A$ est de forme $A = ( a )$, pour un certain $a \in \R$. Puisque $A$ est inversible, $a \neq 0$ donc $a$ est inversible dans $\R$. On a alors 
    $$(a) (a^{-1}) = I_n = (a^{-1})(a)$$
    et donc $A^{-1} = (a^{-1})$ est triangulaire supérieure.
    
  \item On suppose que l'inverse de toute matrice triangulaire supérieure inversible de taille $(n-1) \times (n-1)$ est encore triangulaire supérieure.
  
  \item Soit $A$ une matrice triangulaire supérieure inversible de taille $n\times n$. Présentons $A$ par blocs de la façon suivante
    $$A = \begin{pmatrix}
      a &   & L &   \\
        & * & * & * \\
      0 & 0 & T & * \\
        & 0 & 0 & * \\
    \end{pmatrix}$$
    où $a \in \R$, $L$ est une matrice ligne de taille $1 \times (n-1)$ et $T$ est une matrice triangulaire supérieure de taille $(n-1)\times(n-1)$. Du fait que $A$ est inversible, tous ses coefficients diagonaux sont non nuls. Cela implique que $a \neq 0$ et que les coefficients diagonaux de $T$ sont non nuls, donc $T$ est inversible. Par hypothèse de récurence, l'inverse $T^{-1}$ de $T$ est une matrice triangulaire supérieure. \\
    Calculons de façon géNérale le produi de deux matrices triangulaires supérieure décombosées pa blocs:
    $$\begin{pmatrix}
        a' &   & L' & ~ \\
          &   &   &  ~  \\
        0 &   & T' & ~  \\
          &   &   &  ~ \\
      \end{pmatrix}
      \begin{pmatrix}
        a &   & L &  ~ \\
          &   &   &  ~ \\
        0 &   & T &  ~ \\
          &   &   &  ~ \\
      \end{pmatrix}
      =\begin{pmatrix}
        a' a &   & a' L + L' T & ~ \\
          &   &   & ~ \\
        0 &   & T' T & ~ \\
          &   &   & ~ \\
      \end{pmatrix}$$
    On veut que $T' T = I_{n-1}$ et que $a' L + L' T = 0$; posons $a' = a^{-1}$, $T' = T^-{1}$ et $L' = - a^{-1} L T^{-1}$. Il vient
    \begin{eqnarray*}
      \begin{pmatrix}
        a^{-1} &   & -a^{-1} L T^{-1} & ~ \\
          &   &   &  ~  \\
        0 &   & T^{-1} & ~  \\
          &   &   &  ~ \\
      \end{pmatrix} &\cdot&
      \begin{pmatrix}
        a &   & L &  ~ \\
          &   &   &  ~ \\
        0 &   & T &  ~ \\
          &   &   &  ~ \\
      \end{pmatrix} \\
      &=&\begin{pmatrix}
        a^{-1} a &   & a^{-1} L - a^{-1} L T^{-1} T & ~ \\
          &   &   & ~ \\
        0 &   & T^{-1} T & ~ \\
          &   &   & ~ \\
      \end{pmatrix}  \\
      &=&\begin{pmatrix}
        1 &   & 0 & ~ \\
          &   &   & ~ \\
        0 &   & I_{n-1} & ~ \\
          &   &   & ~ \\
      \end{pmatrix}
      = I_n
    \end{eqnarray*}
    On a également
    \begin{eqnarray*}
      \begin{pmatrix}
        a &   & L &  ~ \\
          &   &   &  ~ \\
        0 &   & T &  ~ \\
          &   &   &  ~ \\
      \end{pmatrix} &\cdot&
      \begin{pmatrix}
        a^{-1} &   & a^{-1} L T^{-1} & ~ \\
          &   &   &  ~  \\
        0 &   & T^{-1} & ~  \\
          &   &   &  ~ \\
      \end{pmatrix}\\
      &=&\begin{pmatrix}
        a a^{-1} &   & - a^{-1} L T^{-1} T + a^{-1} L & ~ \\
          &   &   & ~ \\
        0 &   & T T^{-1} & ~ \\
          &   &   & ~ \\
      \end{pmatrix}  \\ 
      &=&\begin{pmatrix}
        1 &   & 0 & ~ \\
          &   &   & ~ \\
        0 &   & I_{n-1} & ~ \\
          &   &   & ~ \\
      \end{pmatrix}
      = I_n
    \end{eqnarray*}
    donc
    $$A^{-1} = 
      \begin{pmatrix}
        a^{-1} &   & a^{-1} L T^{-1} & ~ \\
          &   &   &  ~  \\
        0 &   & T^{-1} & ~  \\
          &   &   &  ~ \\
      \end{pmatrix}$$
    est triangulaire supérieure. Par principe de récurence, la propriété est vrai pour tout $n \geq 1$
\end{enumerate} 
    

%
%
\section{Transposition}
%
%
%
\subsection{Matrice transposée}
%
\paragraph{Définition} Soit $A$ une matrice de taille $n\times m$. On appelle transposée de $A$ la matrice de taille $m\times n$ notée $A^T$, définie par
\begin{itemize}
  \item la colonne $i$ de $A$ deviens la ligne $i$ de $A^T$
  \item la ligne $i$ de $A$ deivens la colonne $i$ de $A^T$
\end{itemize}

\paragraph*{Proposition}
\begin{enumerate}[1)]
  \item $\forall A, B \in M_{n\times m}(\R)$ on a
    $$(A + B)^{T} = A^{T} + B^{T}$$
  \item $\forall A \in M_{n\times m}(\R), \lambda \in \R$ on a
    $$(\lambda A)^{T} = \lambda A^{T}$$
  \item $\forall A \in M_{n\times m}(\R), B \in M_{m\times n}(\R)$ on a
    $$(A \cdot B)^{T} = B^{T} \codt A^{T}$$
  \item $\forall A \in M_{n\times m}(\R)$ on a
    $$(A^{T})^{T} = A$$
  \item $\forall A \in M_{n}(\R)$ si $A$ est inversible, alors $A^{T}$ est inversible et 
    $$(A^{-1})^{T} = (A^{T})^{-1}$$
\end{enumerate}

%
%
\section{Matrices symétriques, matrices antisymétriques}
%
%
%
\subsection{Matrice symétrique}
%
\paragraph{Définition} Soit $A$ une matrice carrée de taille $n\times n$. On dit que $A$ est symétrique si $A^{T} = A$.

%
\subsection{Matrice antisymétrique}
%
\paragraph{Définition} Soit $A$ une matrice carrée de taille $n\times n$. On dit que $A$ est antisymétrique si $A^{T} = -A$
\paragraph{Remarque} Si $A$ est antisymétrique $a_{ij}$ pour $i=j$ sont nuls.

\paragraph{Théorème} Soit $A$ une matrice carrée. Alors $A$ peut s'écrire comme la somme d'une matrice symétrique et d'une matrice antisymétrique.

\paragraph{Démonstration}  On pose $C = \frac{1}{2} ( A + A^{T})$ et $D = \frac{1}{2} ( A - A^{T})$ et alors on a
$$ C + D = \frac{1}{2} ( A + A^{T}) + \frac{1}{2} ( A - A^{T}) = A$$
On a 
\begin{eqnarray*} 
  C^{T} = (\frac{1}{2} ( A + A^{T}))^{T} &=& \frac{1}{2} ( A^{T} + A) \\
    &=& \frac{1}{2} ( A + A^{T}) = C
\end{eqnarray*}
donc $C$ est symétrique. On a également
\begin{eqnarray*}
  D^{T} = (\frac{1}{2} ( A - A^{T}))^{T} &=& \frac{1}{2} ( A^{T} - A) \\
    &=& -\frac{1}{2} ( A - A^{T}) = -C
\end{eqnarray*}
donc $D$ est antisymétrique. Ainsi $A = C + D$ est une composition de la forme cherché.

%
%
\section{La trace}
%
%
%
\subsection{Trace d'une matrice}
%
\paragraph{Définition} Soit $A$ une matrice carrée de taille $n\times n$. On appelle trace de $A$ le nombre réel noté $tr(a)$ défini par
$$tr(a) = a_{11} + a_{22} + \ldots + a_{nn}$$
La trace de $A$ est donc la somme des coefficients diagonaux de $A$.

\paragraph{Proposition}
\begin{enumerate}[1)]
  \item Soient $A, B \in M_{n}(\R)$. Alors
    $$tr(A + B) = tr(A) + tr(B)$$
  \item Soient $A \in M_{n}(\R)$ et $\lambda \in \R$. Alors
    $$tr(\lambda A) = \lambda \cdot tr(A)$$
  \item Soient $A, B \in M_{n}(\R)$. Alors
    $$tr(A \cdot B) = tr(B \cdot A)$$
  \item Soient $A \in M_{n}(\R)$ . Alors
    $$tr(A^{T}) = tr(A)$$
\end{enumerate}

\paragraph{Démonstration}
\begin{enumerate}[1)]
  \item 
    \begin{eqnarray*}
      tr(A + B) = \sum_{k=1}^{n}(A + B)_{kk} &=& \sum_{k=1}^{n}(A)_{kk} + (B)_{kk} \\
        &=& \sum_{k=1}^{n}(A)_{kk} + \sum_{k=1}^{n}(B)_{kk} \\
        &=& tr(A) + tr(B)
    \end{eqnarray*}
    
  \item
    \begin{eqnarray*}
      tr(\lambda A) = \sum_{k=1}^{n}(\lambda A)_{kk} &=& \sum_{k=1}^{n} \lambda \cdot (A)_{kk} \\
        &=& \lambda \sum_{k=1}^{n}(A)_{kk} \\
        &=& \lambda \cdot tr(A)
    \end{eqnarray*}
  
  \item 
    \begin{eqnarray*}
      tr(A \cdot B) = \sum_{k=1}^{n} (A B)_{kk} &=& \sum_{k=1}^{n} \left( \sum_{l=1}^{n} A_{kl} B_{lk} \right) \\
        &=& \sum_{l=1}^{n} \left( \sum_{k=1}^{n} A_{kl} B_{lk} \right) \\
        &=& \sum_{l=1}^{n} \left( \sum_{k=1}^{n} B_{lk} A_{kl} \right) \\
        &=& \sum_{l=1}^{n} (B A)_{ll} = tr(B \cdot A)
    \end{eqnarray*}
    
  \item
    $$tr(A^{T}) = \sum_{k=1}^{n} (A^{T})_{kk} = \sum_{k=1}^{n} (A)_{kk} = tr(A)$$
\end{enumerate}
