\chapter{Calcul matriciel}

\section{Matrices}
\paragraph{Définition} Soient $n \geq 1$ et $m \geq 1$ deux entiers naturels. On appelle matrice de taille $n\times m$ à coefficients dans $\R$ un tableau à $n$ lignes et $m$ colonnes de la forme
$$\begin{pmatrix}
  a_{11} & a_{12} & \ldots & a_{1m} \\
  a_{21} & a_{22} & \ldots & a_{2m} \\
  \vdots & \vdots & \ddots & \vdots \\
  a_{n1} & a_{n2} & \ldots & a_{nm}
\end{pmatrix} \text{ avec } a_{ij} \in \R$$
On peut noter la matrice $A = (a_{ij})_{1 \leq i \leq n, 1 \leq j \leq m}$ ou simplement $A = (a_{ij})$.

\paragraph{Définition} On note $M_{n, m}(\R)$ l'ensemble des matrices de taille $n\times m$ à coefficients dans $\R$.

\paragraph{Définition} Soit $n \geq 1$ un entier naturel et $A$ une matrice de taille $n\times n$. On dit alors que $A$ est une matrice carrée. \\ 
On note $M_{n}(\R)$ l'ensemble des matrices carrées de taille $n\times m$ à coefficients dans $\R$. Les coefficients $a_{ii}$ pour $1 \leq i \leq n$ sont appellés coefficients diagonaux de $A$.

%
%
\section{Opérations sur les matrices}
\paragraph{Définition} Soient $m\geq 1$ et $n \geq 1$ deux entiers naturels. On peut définir la somme de deux matrices de même taille. Si $A = (a_{ij}), B = (b_{ij}) \in M_{n, m}(\R)$ on définit leur somme $S = A + B \in M_{n, m}(\R)$ par
$$s_{ij} = a_{ij} + b_{ij} ~ \forall ~ 1 \leq i \leq n, 1 \leq j \leq m$$

\paragraph{Définition} Soient $n \geq 1, m \geq 1$ deux entiers naturels, $A = (a_{ij}) \in M_{n, m}(\R)$ et $\alpha \in \R$. On  définit le produit de $A$ par $\alpha$ que l'on note $\alpha A$, par
$$\alpha A = (\alpha a_{ij}) \in M_{n, m}(\R)$$

\paragraph{Définition} Soient $n \geq 1, m \geq 1$ et $p \geq 1$ trois entiers naturels. On peut définir le produit $A \times B$ d'une matrice $A$ et d'une matrice $B$ dans cet ordre si le nombre de colonnes de $A$ est égal au nombre de lignes de $B$. \\
Si $A = (a_{ij}) \in M_{n, m}(\R)$ et $B = (b_{ij}) \in M_{m, p}(\R)$, alors on définit $C = A \times B \in M_{n, p}(\R)$
$$c_{ij} = \sum_{l=1}^{m} {a_{il} * b_{lj} = a_{i1} \cdot b_{1j} + a_{i2} \cdot b_{2j} + \ldots + a_{im} \cdot b_{mj}$$

\paragraph{Proposition} Soient $m \geq 1$ et $n\geq 1$ des entiers naturels.
\begin{enumerate}
  \item Soient $a, B, C \in M_{n, m}(\R)$. Alors
    $$(A + B) + C = A + (B + C)$$
  \item On note $0$ la matrice nulle dans $M_{n, m}(\R)$, c'est-à-dire la matrice dont tous les coefficients sont nuls. Alors pour tout $A \in M_{n, m}(\R)$ on a
    $$A + 0 = A = 0 + A$$
  \item Soit $A \in M_{n, m}(\R)$. On note $-A$ la matrice $(-a_{ij})$ dans l'ensemble $M_{n, m}(\R)$. Alors on a
    $$(-A) + A = 0 = A + (-A)$$
\end{enumerate}
Ainsi, l'addition $+$ dans $M_{n, m}(\R)$ est associative, unifère et tout $A \in M_{n, m}(\R)$ admet une matrice opposée. Par conséquence $(M_{n, m}(\R), +)$ est un groupe.\\
De plus, comme $+$ dans $\R$ est commutative, l'addition dans $M_{n,m}(\R)$ est commutative:
$$\forall A, B \in M_{n, m}(\R), ~ A + B = B + A$$
En conclusion, $(M_{n, m}(\R)$ est un groupe commutatif.

\paragraph{Proposition} Soient $n \geq 1, m \geq 1$ des entiers naturels. On a
\begin{enumerate}
  \item $\forall \alpha \in \R, ~ A, B \in M_{n, m}(\R)$ 
    $$\alpha \cdot (A + B) = \alpha \cdot A + \alpha \cdot B$$
    
  \item $\forall \alpha, \beta \in \R, ~ A \in M_{n, m}(\R)$
    \begin{eqnarray*}
      (\alpha + \beta) \cdot A &=& \alpha \cdot A + \beta \cdot A \\
      (\alpha \cdot \beta) \cdot A &=& \alpha \cdot (\beta \cdot A)
    \end{eqnarray*}
    
  \item $\forall A \in M_{n, m}(\R)$
    $$1 \cdot A = A$$
\end{enumerate}
On peut donc dire, que $M_{n, m}(\R)$ muni de l'addition des matrices et de la multiplication par les réels est un $\R$-espace vectoriel.

\paragraph{Proposition} A conditions que les produits et les sommes considerés soient bien définit, on a
\begin{enumerate}
  \item $$A \times (B \times C) = (A \times B) \times C$$
  \item $$A \times (B + C) = A \times B + A \times C$$ 
    $$(B + C) \times A = B \times A + C \times A$$
  \item Pour tout entier $m \geq 1$ on appelle matrice identité de taille $n \times n$ la matrice carrée
    $$I_n = (a_{ij}) ~ ~ a_{ij} = \left\{ \begin{array}{lr} 1, & i=j \\ 0, & i \neq j \end{array}$$
\end{enumerate}

\paragraph{Attention} Deux matrices quelconques ne commutent pas nécessairement.

\paragraph{Attention}
\begin{itemize}
  \item Si $A$ et $B$ sont deux matrices on peut avoir $A \neq 0, B \neq 0$ et $AB = 0$
  \item On peut donc avoir également $AB = AC$ et $B \neq C$ pour des matrices $A, B, C$.
\end{itemize}

\paragraph{} On a, $(M_{n, m}(\R), +, \times, \cdot)$ est une $\R$-algèbre. En particulier $(M_{n, m}(\R), +, \times)$ est un anneau.

%
%
\section{Inversion des matrices}
\paragraph{Définition} Soit $n \geq 1$ un entier naturel. Soit $A$ une matrice carrée de taille $n \times n$. On dit que $A$ est inversible s'il existe une matrice carrée $B$ de taille $n \times n$ telle que $A B = I_n = B A$. Dans ce cas on dit que $B$ est un inverse de $A$.

\paragraph{Proposition} Soit $A$ une matrice carrée de taille $n \times n$. Si $A$ est inversible alors son inverse est unique.

\paragraph{Démonstration} On suppose que $A$ est inversible. Si $B$ et $C$ sont deux inverses pour $A$, alors on a
$$B A C = (B A) C = I_n C = C$$
on a également
$$B A C = B (A C) = B I_n = B$$
d'où $B = C$.

\paragraph{Notation} Si $A$ est inversible, on note $A^{-1}$ son inverse. On a donc
$$A A^{-1} = I_n = A^{-1} A$$

\paragraph{Lemme} Soient $a, b, c, d \in \R, A = \begin{pmatrix} a & b \\ c & d \end{pmatrix} \in M_{n, m}(\R)$ et $B = \begin{pmatrix} d & -b \\ -c & a \end{pmatrix} \in M_{2}(\R)$. Alors $A B = (a d - b c) I_n = B A$.

\paragraph{Proposition} Soit $A = \begin{pmatrix} a & b \\ c & d \end{pmatrix} \in M_{2}(\R)$. Alors $A$ est inversible si et seuelement si $a d - b c \neq 0$, et dans ce cas
$$A^{-1} = \frac{1}{a d - b c} \begin{pmatrix} d & -b \\ -c & a \end{pmatrix}$$
On suppose que $a d - b c = 0$. On raisonne par l'absurde. Si $A$ est inversible, on obtient: D'après le lemme on a
$$\begin{pmatrix} a & b \\ c & d \end{pmatrix} \begin{pmatrix} d & -b \\ -c & a \end{pmatrix} = (a d - b c) I_2 = 0$$
d'où
\begin{eqnarray*}
  A^{-1} \left( A \begin{pmatrix} d & -b \\ -c & a \end{pmatrix} \right) &=& A^{-1} 0 = 0 \\
    &=& A^{-1} A \begin{pmatrix} d & -b \\ -c & a \end{pmatrix} \\
  0 &=& \begin{pmatrix} d & -b \\ -c & a \end{pmatrix}
\end{eqnarray*}
Par conséquent $a = b = c = d = 0$, donc $A = 0$ donc $A$ n'est pas inversible, ce qui donn lien à une contradiction. Ainsi $A$ n'est pas inversible si $a d - b c = 0$.

\paragraph{Proposition} Soient $A$ et $B$ des matrices inversibles de taille $n \times n$. Alors
\begin{enumerate}
  \item $A^{-1}$ est inversible d'inverse $A$
  \item $A B$ est inversible et $(A B)^{-1} = B^{-1} A^{-1}$
\end{enumerate}

\paragraph{Démonstration}
\begin{enumerate}
  \item Puisuque $A$ est inversible d'inverse $A^{-1}$, on a
    $$A A^{-1} = I_n = A^{-1} A$$
    donc $A^{-1}$ est inversible d'inverse $A$.
  \item On a
    $$(A B)(B^{-1} A^{-1}) = A (B B^{-1}) A^{-1} = A I_n A^{-1} = A A^{-1} = I_n$$
    De même, on a $(B^{-1} A^{-1}) A B = I_n$ \\
    Par conséquence, $A B$ est inversible d'inverse $(B^{-1} A^{-1})$
\end{enumerate}

\paragraph{Proposition} L'ensemble des matrices inversibles de taille $n \times n$ muni de la multiplication des matrices est un group. En effet
\begin{itemize}
  \item L'assertion $2.$ de la proposition précédante assure que $\times$ restreinte à l'ensemble des matrices inversibles est une loi de composition interne.
  \item $\times$ étant associative sur $M__{n}(\R)$, elle l'est en restriction à l'ensemble des matrices inversibles.
  \item $I_n$ est élément neutre.
  \item L'assertion $1.$ de la proposition assure que l'inverse d'une matrice inversible est bien inversible.
\end{itemize}
On note $(GL_{n}(\R), \times)$ le groupe des matrices inversibles de taille $n$ ($GL$ est mis pour "groupe linéaire").

\paragraph{Proposition} Plus généralement, si $A_1, A_2, \ldots, A_m$ sont des matrices inversibles on a $(A_1 A_2 \ldots A_m)^{-1} = A_m^{-1} A_{m-1}^{-1} \ldots A_1^{-1}$

%
%
\section{Matrices élémentaires}
\paragraph{Définition} Soient $n \geq 1$ un entier naturel et $(i, j) \in \{1; \ldots; n\} \times \{1; \ldots; n\}$. On note $F_{i, j}$ la matrice carrée de taille $n \times n$ contenant un $1$ en $i^{ème}$ ligne et $j^{ème}$ colonne et des $0$ partout ailleurs. On dit que $F_{i, j}$ est une matrice standard.

\paragraph{Proposition} Toute matrice carrée de taille $\times n$ peut s'exprimer à l'aide des matrices standard de taille $n\times n$
$$A = \sum_{(i, j) \in \{1; \ldots; n\}\times\{1; \ldots; n\}} a_{ij} F_{ij}$$

\paragraph{Proposition} Soit $A$ une matrice de taille $n\times m$.
\begin{enumerate}[a)]
  \item Soit $F_{ij}$ une matrice standard de taille $n\times n$. Alors $F_{ij} A$ est la matrice de taille $n \times m$ dont la $i^{ème}$ ligne est la $j^{ème}$ ligne de $A$ et dont les autres sont nulles.
  \item Soit $F_{ij}$ une matrice standard de taille $m\times m$. Alors $A F_{ij}$ est la matrice de taille $n\times m$ dont la $j^{ème}$ colonne est la $i^{ème}$ de $A$ et dont les autres colonnes sont nulles.
\end{enumerate}

\paragraph{Symbole de Kronecker} Soit $n \geq 1$ un entier naturel. On note
$$\mS: \{1; \ldots; n\}\times\{1; \ldots; n \} \rightarrow {0; 1}$$
L'application est défini par
$$\delta_{ij} = \left\{\begin{array}{lr} 1 & \text{si i=j} \\ 0 & \text{sinon} \end{array} \right.$$ 

\paragraph{Corollaire} Soient $n \geq 1$ un entier naturel et
$$(i, j), (k, l) \in \{1; \ldots; n\}\times\{1; \ldots; n\}$$
alors
$$F_{ij}F_{kl} = \delta_{ik} F_{il} = 
  \left\{\begin{array}{lr} F_{il} & \text{si i=j} \\ 0 & \text{sinon} \end{array} \right.$$

\paragraph{Démonstration} D'après la proposition, précédente, $F_{ij} F_{kl}$ est la matrice don la $i^{ème}$ ligne est la $j^{ème}$ ligne de $F_{kl}$ et dont les autres lignes sont nulles. Il vient
\begin{itemize}
  \item si $j\neq k$, la $j^{ème}$ ligne de $F_{kl}$ est nulle
  \item si $j = k$ la $j^{ème} = k^{ème}$ ligne de $F_{kl}$ est $(0_{1} \ldots 0_{l-1} ~ 1_{l} ~ 0_{l+1} \ldots 0_{n})$
\end{itemize}
Par conséquent, si $j\neq k$, toutes les lignes de $F_{ij} F_{kl}$ sont nulles donc $F_{ij} F_{kl} = 0$; si $j=k$, alors 
$$F_{ij} F_{kl} = \begin{pmatrix} 
  & & & 0 & & & \\
  & & & \vdots & & & \\
  & & & 0 & & & \\
  0 & \ldots & 0 & 1 & 0 & \ldots & 0 \\
  & & & 0 & & & \\
  & & & \vdots & & & \\
  & & & 0 & & & 
\end{pmatrix} = F_{il}$$
Ainsi $F_{ij} F{kl} = \delta_{ik} F_{il}$.
