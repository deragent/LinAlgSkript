\documentclass[10pt,a4paper]{scrreprt}
\usepackage{fullpage}

\usepackage[utf8]{inputenc}
\usepackage[frenchb]{babel}
\usepackage{enumerate}
\usepackage{amsmath}
\usepackage{amssymb}

\usepackage{hyperref}


\renewcommand{\familydefault}{\sfdefault}

\newcommand{\R}{\mathbb{R}} 
\newcommand{\N}{\mathbb{N}} 
\newcommand{\Q}{\mathbb{Q}} 
\newcommand{\Z}{\mathbb{Z}} 
\newcommand{\C}{\mathbb{C}} 

\newcommand{\B}{\mathcal{B}} 
\newcommand{\mS}{\mathcal{S}} 
\newcommand{\T}{\mathcal{T}} 
\newcommand{\mP}{\mathcal{P}}

\newcommand{\cB}{\mathcal{B}} 
\newcommand{\cC}{\mathcal{C}}
\newcommand{\cE}{\mathcal{E}}
\newcommand{\cF}{\mathcal{F}}
\newcommand{\cG}{\mathcal{G}}
\newcommand{\cL}{\mathcal{L}}

\title{Algèbre Linéaire I}
\subtitle{selon le cours de Prof. Lequeue (Septembre - Décembre 2011) \\ Version BETA}
\author{Johannes Wüthrich}
\date{\today}


\begin{document}

\maketitle

\setcounter{chapter}{-1}

\chapter{Préface}
%
\section{Remerciement}
%
\paragraph{Nicola Gerber} Je remercie beaucoup Nicola Gerber pour son aide à corriger les nombreuses fautes que j'ai fait à recopier tout à l'ordinateur. Il a aussi pris beaucoup de temps pour recopier des divers exemples que je n'ai pas eu dans mes notés. Sans lui ce manuel ne serra pas près dès maintenant. Merci beaucoup!

%
\section{Notes importantes}
%
\paragraph{Version BETA} Bien qu'on a investi beaucoup de temps à corriger toutes les erreurs qu'on a trouvé je suis sûr qu'il y a encore des nombreuses fautes dans le texte. Sachant ça je mets à disposition ce document seulement en version BETA. Ça veut dire que je ne prends pas de responsabilité pour quelque-chose qui se passe à cause d'une erreur dans ce document. Mais ça ne veut pas dire qu'il y a des milles d'erreurs. Justement sois attentif en lisant ce manuel.

\paragraph{Fautes et remarques} Si tu trouve une erreur où si tu a l'impression que quelque-chose manque, n'hésite pas à me envoyer un mail sous l'adresse \href{mailto:johannes.wuethrich@epfl.ch}{johannes.wuethrich@epfl.ch}. Je vais essayer de prendre en compte tous votres remarques dans la prochaine version.

\paragraph{Version finale} Je vais sûrement faire une deuxième version BETA s'il y beaucoup des fautes qui m'ont été raconté. Je pense qu'il y a aussi la possibilité que Prof. Lequeue fais une controlle de tout le texte. Mais je ne suis pas encore sûr quand et si ça se passera. 

\tableofcontents


\chapter{Systèmes d'equations linéaires}

\paragraph{Introduction} L'Object de ce chapitre est de présenter une méthode de résolution des systèmes d'équations dites linéaires. Les équations se rencontrent dans de nombreuses branches des sciences appliquées.
%
%
\section{$\R$ (Le corps $\R$)}
%
%
%
\subsection{Le groupe $(\R, +)$}
%
\paragraph{Définition} On rappelle que l'ensemble $\R$ des nombres réels est muni des deux lois de composition internes
\begin{itemize}
  \item une addition 
    \begin{eqnarray*}
      +: \R \times \R &\rightarrow& \R \\
      (x, y) &\mapsto& x+y
    \end{eqnarray*}
    qui possède les propriétes suivantes
    \begin{itemize}
      \item pour tous $x, y, z \in \R$
        $$x + (y + z) = (x + y) + z \text{ (associativité) }$$
      \item il existe $0 \in \R$ tel que pour tous $x \in \R$
        $$x + 0 = x = 0 + x \text{ (neutralité) }$$
      \item pour tous $x \in \R$, il existe $-x \in \R$ tel que
        $$x + (-x) = 0 = (-x) + x \text{ (symétrie) }$$
    \end{itemize}
\end{itemize}
On dit alors que $(\R, +)$ est un groupe. De plus on a
\begin{itemize}
  \item Pour tous $x, y \in \R$
    $$x + y = y + x \text{ (commutivité) }$$
\end{itemize}
On dit alors que $(\R, +)$ est une groupe commutatif.

%
\subsection{L'anneau $(\R, +, \cdot)$ }
%
\paragraph{Définition} On sait que $(\R, +)$ est une groupe commutatif. De plus on a
\begin{itemize}
  \item une multiplication
    \begin{eqnarray*}
      \cdot: \R \times \R &\rightarrow& \R \\
      (x, y) &\mapsto& x \cdot y = x y
    \end{eqnarray*}
    qui possède les propriétés suivantes
    \begin{itemize}
      \item pour tous $x, y, z \in \R$
        $$(x \cdot y) \cdot z = x \cdot (y \cdot z) \text{ (associativité) }$$
      \item il existe $1 \in \R$ tel que pour tout $x \in \R$
        $$1 \cdot x = x = x \cdot 1 \text{ (neutralité) }$$
      \item pour tous $x, y, z \in \R$
        $$(x + y) \cdot z = x \cdot z + y \cdot z \text{ (distributivité à droite\footnote{De $\cdot$ par rapport à $+$}) }$$
        $$x \cdot (y + z) = x \cdot y + x \cdot z \text{ (distributivité à gauche\footnote{De $\cdot$ par rapport à $+$}) }$$
        Distributivité en générale.
    \end{itemize}
\end{itemize}
On dit alors que $(\R, +, \cdot)$ est un anneau. De plus on a
\begin{itemize}
  \item pour tous $x, y \in \R$
    $$x \cdot y = y \cdot x \text{ (commutativité) }$$
\end{itemize}
On dit alors que $(\R, + , \cdot)$ est un anneau commutatif.

%
\subsection{Le corps $(\R, +, \cdot)$}
%
\paragraph{Définition} On sait que $(\R, + , \cdot)$ est un anneau commutatif. De plus on a
\begin{itemize}
  \item Pour tous $x \in \R \backslash \{0\}$ il existe $x^{-1} \in \R$ tel que 
    $$x \cdot (x^{-1}) = 1 = (x^{-1}) \cdot x$$
\end{itemize}
On dit alors que $(\R, +, \cdot)$ est un corps commutatif.

%
\subsection{L'ensemble des $n$-tuplets $\R^n$}
%
\paragraph{Définition} On note $\R^2$ le produit cartésien $\R \times \R$, c'est-à-dire l'ensemble des couples $(x, y)$ a coordonnés $x, y \in \R$. De même, on note $\R^3$ le produit cartésienne $\R \times \R \times \R$, c'est-à-dire des triplets $(x, y, z)$ à coordonnés $x, y, z \in \R$. \\
Plus généralement, si $n \geq 1$ est un entier naturel, $\R^n$ designe l'ensemble $\R \times \R \times \ldots \times \R$ des $n$-couples $(x_1, x_2, \ldots, x_n)$ à coordonnées $x_1, x_2, \ldots, x_n \in \R$.
        
%
%
\section{Systèmes d'equations lineaires}
%
%

%
\subsection{Équation linéaire}
%
\paragraph{Définition} Soit $n \geq 1$ un entier naturel. On appelle équation linéaire à coefficients dans $\R$ tout relation de la forme 
$$a_1 x_1 + a_2 x_2 + \ldots + a_n x_n = b, ~ a_1, a_2, \ldots, a_n, b \in \R$$
Les $x_i$ sont \underline{les inconnues} de l'équation et les $a_i$ sont \underline{les coefficients}.

\paragraph{Exemple} $3 x + 2 y + 5 z + t = 5$ est une équation linéaire à $4$ inconnues\footnote{Les inconnues sont $x, y, z$ et $t$.} et à coefficients dans $\R$.

%
\subsection{Système d'équations linéaires}
%
\paragraph{Définition} Soient $n \geq 1$ et $m \leq 1$ deux entiers naturels; un système d'équations linéaires à $n$ équations et $m$ inconnues est de la forme:
$$n \left\{ \begin{array}.
  a_{11}x_{1} + a_{12}x_{2} + \ldots + a_{1m}x_{m} = b_{1} \\
  a_{21}x_{1} + a_{22}x_{2} + \ldots + a_{2m}x_{m} = b_{2} \\
  \vdots \\
  a_{n1}x_{1} + a_{n2}x_{2} + \ldots + a_{nm}x_{m} = b_{n}
\end{array} \right. (a_i, b_l \in \R)$$

\paragraph{Exemple}
$$\left\vert \begin{array}{rcl}
  7 y + z + 8 t & = & 1 \\
  3 x + 2 y + 5 z + t & = & 5 \\
  2 x - y + 3 z - 2 t & = & 3
\end{array} \right.$$
est un système à $3$ équations et $4$ inconnues à coefficients dans $\R$.

\paragraph{Remarque} Par la suite nous ne considerons que des systèmes à coefficients dans $\R$.

%
\subsection{Solution d'un système d'équations linéaires}
%
\paragraph{Définition} Soit $S$ un système d'équations linéaires à $n$ équations et $m$ inconnues. Une solution du système dans $\R^m$ est un $m$-uplet $(t_1, t_1, \ldots, t_m) \in \R^m$ tel que les $n$ égalités soient vérifiés, en prenant $x_i = t_i$. Résoudre dans $\R^m$ le système $S$ consiste à déterminer l'ensemble des solutions de $S$ dans $\R^m$.

\paragraph{Exemple} On considère le système de l'exemple précédent. Alors $(2, -1, 0, 1) \in \R^n$ est une solutions (dans $\R^n$) de ce système. En effet on a
$$\left\vert \begin{array}{rcl}
  7 \cdot (-1) + 0 + 8 \cdot 1 & = & 1 \\
  3 \cdot 2 + 2 \cdot (-1) + 5 \cdot 0 + 1 & = & 5 \\
  2 \cdot 2 - (-1) + 3 \cdot 0 - 2 \cdot 1 & = & 3
\end{array} \right.$$

\paragraph{Définition} On dit que deux systèmes d'équations linéaires sont équivalents s'ils ont le même ensemble de solutions.

\paragraph{} Pour résoudre un système d'équations linéaires, on va effectuer des opérations élémentaires sur les équations du système à fin de se ramener à un système plus simple. Ces opérations sont de 3 types:
\begin{enumerate}[ a)]
  \item multiplier une équation par un réel non nul
  \item ajuter à une équation un multiple d'une autre équation
  \item permuter deux équations
\end{enumerate}

Chaque de ces opérations admet une opération inverse. Par suite, effectuer une opération élémentaire donne lieu à un système équivalent. Au terme du processus, le système obtenue aura donc le même ensemble de solutions que le système initial.

%[TODO]: xleftrightarrow ?!?

\paragraph{Exemple}
\begin{eqnarray*}
  \begin{array}{lrcl}
    L_1 & 7 y + z + 8 t & = & 1 \\
    L_2 & 3 x + 2 y + 5 z + t & = & 5 \\
    L_3 & 2 x - y + 3 z - 2 t & = & 3
  \end{array}
  \xleftrightarrow{L_2 \leftarrow L_2 - L_3}
  \left\vert \begin{array}{rcl}
    7 y + z + 8 t & = & 1 \\
    x + 3 y + 2 z + 3 t & = & 2 \\
    2 x - y + 3 z - 2 t & = & 3
  \end{array} \right. \\
  
  \xleftrightarrow{L_1 \leftrightarrow L_2}
  \left\vert \begin{array}{rcl}
    x + 3 y + 2 z + 3 t & = & 2 \\
    7 y + z + 8 t & = & 1 \\
    2 x - y + 3 z - 2 t & = & 3
  \end{array} \right.
  \xleftrightarrow{L_3 \leftarrow L_3 - 2 L_1}
  \left\vert \begin{array}{rcl}
    x + 3 y + 2 z + 3 t & = & 2 \\
    7 y + z + 8 t & = & 1 \\
    -7 y - z - 8 t & = & -1
  \end{array} \right. \\
  
  \xleftrightarrow{L_2 \leftarrow \frac{1}{7} L_2}
  \left\vert \begin{array}{rcl}
    x + 3 y + 2 z + 3 t & = & 2 \\
    y + \frac{1}{7} z + \frac{8}{7} t & = & \frac{1}{7} \\
    -7 y - z - 8 t & = & -1
  \end{array} \right. 
  \xleftrightarrow{L_3 \leftarrow L_3 + 7 L_2}
  \left\vert \begin{array}{rcl}
    x + 3 y + 2 z + 3 t & = & 2 \\
    y + \frac{1}{7} z + \frac{8}{7} t & = & \frac{1}{7} \\
    0 y - 0 z - 0 t & = & 0
  \end{array} \right. \\
  
  \xleftrightarrow{L_1 \leftarrow L_1 - 3 L_2}
  \left\vert \begin{array}{rcl}
    x + \frac{12}{7} z - \frac{3}{7} t & = & \frac{12}{7} \\
    y + \frac{1}{7} z + \frac{8}{7} t & = & \frac{1}{7} \\
    0 & = & 0
  \end{array} \right.
\end{eqnarray*} \\
Comme $x$ et $y$ sont des coefficients pivot, ils représentent les inconnues principales. De l'autre côté, $z$ et $t$ sont les inconnues sécondaires. On a alors
$$\left\{ \begin{array}{rcl}
  x & = & -\frac{11}{7} z + \frac{3}{7} t + \frac{11}{7} \\
  y & = & -\frac{1}{7} z - \frac{8}{7} t + \frac{1}{7}
\end{array} \right.$$
On pose $z=u$ et $t=v$. L'ensemble des solutions du système est alors
$$\cL = \left\{ \left(-\frac{11}{7} u + \frac{3}{7} v + \frac{11}{7}, -\frac{1}{7} u - \frac{8}{7} v + \frac{1}{7}, u, v \right) \in \R^4 \vert u, v \in \R \right\}$$
%
%
\section{L'algorithme d'elimination de Gauss}
%
%
Il s'agit d'une méthode permettant de résoudre un système linéaire quelconque en le mettant sous une forme plus simple, dite échelonnée réduite, à l'aide d'opérations élémentaires. Le système
$$\left\{\begin{array}{rcl}
  7 y + z + 8 t & = & 1 \\
  3 x + 2 y + 5 z + t & = & 5 \\
  2 x - y + 3 z - 2 t & = & 3
\end{array} \right.$$
a été résolu en utilisant cet algorithme. \\
On observe que les opérations élémentaires effectuées sur les équations du système portant sur les coefficients et rien sur les inconnues. Il serve donc au même d'effectuer ces mêmes opérations sur les lignes du tableau des coéfficiants du système.

%
\subsection{Matrice augmentée d'un SEL}
%
\paragraph{Définition} Soit S un système linéaire à $n$ équations et $m$ inconnues. La matrice associée au système est le tableau
$$\begin{pmatrix}
  a_{11} & a_{12} & \cdots & a_{1m} \\
  a_{21} & a_{22} & \cdots & a_{2m} \\
  \vdots  & \vdots  & \ddots & \vdots  \\
  a_{n1} & a_{n2} & \cdots & a_{nm}
 \end{pmatrix}$$
La matric augmentée associée au système est le tableau
$$\begin{pmatrix}
  a_{11} & a_{12} & \cdots & a_{1m} & \vert & b_{1} \\
  a_{21} & a_{22} & \cdots & a_{2m} & \vert & b_{2} \\
  \vdots  & \vdots  & \ddots & \vdots& \vert & \vdots  \\
  a_{n1} & a_{n2} & \cdots & a_{nm} & \vert & b_{n}
 \end{pmatrix}$$
 
\paragraph{Définition} Une matrice est dit échelonnée si elle se présente sous la forme suivante:
\begin{itemize}
  \item Le premier coefficient non nul d'une ligne non nulle vaut 1, On l'appelle coefficient pivot.
  \item Les lignes nulles sont regroupées en bas de la matrice.
  \item Pour deux lignes non nulles consécutives, le pivot de la ligne inférieure se trouve en bas à droit du pivot de la ligne supérieure.
  \item Une matrice échelonnée est dite \underline{échelonnée réduite} si de plus les colonnes contenant un coefficient pivot ont des zéros partout ailleurs.
\end{itemize}

\paragraph{Exemple}
\begin{itemize}
  \item $$\begin{pmatrix}
      0 & 7 & 1 & 8 \\
      3 & 2 & 5 & 1 \\
      2 & -1 & 3 & -2 
    \end{pmatrix}$$
    n'est pas échelonné. Le premier coefficient non nut de la première ligne vaut $7$ et pas $1$.
    
  \item $$\begin{pmatrix}
      1 & 3 & 2 & 8 \\
      0 & 1 & \frac{1}{7} & \frac{8}{7} \\
      0 & 0 & 0 & 0 \\
    \end{pmatrix}$$
    est échelonné, mais pas échelonné réduite, car la deuxième colonne contient un coefficient non nul $3$ en dehors du pivot.
    
  \item Enfin
    $$\begin{pmatrix}
      1 & 0 & \frac{11}{7} & \frac{3}{2} \\
      0 & 1 & \frac{1}{7} & \frac{8}{7} \\
      0 & 0 & 0 & 0 
    \end{pmatrix}$$
    est écehelonnée réduite.
\end{itemize}

%
\subsection{Algorithme de Gauss}
%
\paragraph{} On peut mettre une matrice quelconque sous forme échelonnée ou échelonnée réduite en effectuant des opérations élémentaires. C'est l'algorithme de Gauss. \\
On procède comme suit:
\begin{enumerate}
  \item On repère la première colonne non nulle.
  \item On cherche à faire apparaître dans cette colonne un coefficient notant 1, soit en multipliant une ligne par un réel non nul, soit en ajoutant à une ligne un multiple d'une autre.
  \item En permutant éventuellement deux lignes, on place ce coefficient 1 en première ligne. On l'appelle coefficient pivot.
  \item En ajoutant aux lignes concernées un multiple de la première, on fait apparaître des zéros au dessous du pivot. 
  \item On masque la première ligne contenant le pivot déjà placé, et on recommence avec la sous-matrice obtenue.
\end{enumerate}
La première partie de l'algorithme est faite. Pour mettre la matrice sous forme échelonnée réduite, on commence par le dernier pivot placé. On cherche les lignes audessus du pivot qui ne sont pas nulles, et on ajoute un multiple de la ligne avec le pivot choisi. Quand tous les coefficients audessus du dernier pivot sont nuls, on fait le même avec tout les autres pivot en direction de droit à gauche. On est fini si on a une matrice échelonnée réduite selon la définition audessus.

%BSP 23.9.11 & 29.9.11
%
\section{Résolution d'un système linéaire}
%
%
\paragraph{} Pour résoudre un système linéaire on commence par mettre la matrice augmentée du système sous forme échelonnée réduite, en limitant la recherche des pivots aux colonnes de la matrice du système\footnote{Colonnes correspendantes aux inconnues.}.
Si la dernière ligne de la matrice totale est entièrement nulle: Le système est compatible; il admit au moins une solution.
\paragraph{}On procède alors aux trois étapes suivantes:
\begin{enumerate}
  \item Les inconnues se trouvant dans le colonnes des coefficients pivot s'appellent les inconnues principales. Les autres s'appellent inconnues secondaires; ce sont elles qui vont paramétrer les solutions du système.
  \item Exprimer les inconnues principales en fonction des inconnues secondaires.
  \item Écrire l'ensemble des solutions du système.
\end{enumerate}


\chapter{Structures}
\section{Lois de composition internes}
\paragraph{Rappels} On rapelle que si $E$ est un ensemble, une loi de composition interne $\star$ sur $E$ est une application:
\begin{eqnarray*}
\star{}: & E \times E \rightarrow E \\
& (x, y) \mapsto x \star y
\end{eqnarray*}

%
\subsection{Associativité}
%
\paragraph{Définition} Soit E un ensemble muni d'une loi de composition interne $\star$. On dit que $\star$ est associative si
$$ \forall x, y, z \in E, (x \star y) \star z = x \star (y \star z)$$
Si $\star$ est associative, alors la composition d'un membre fini $m \leq 1$ quelconque d'éléments $x_1, x_2, \ldots , x_n$ de $E$ dans cet ordre est bien défini indépendamment de l'ordre dans lequel on effectue la composition des éléments 2 à 2. \\
On peut alors écrire $x_1 \star x_2 \star \ldots \star x_n$

%
\subsection{Commutativité}
%
\paragraph{Définition} On dit que $\star$ est commutative si
$$ \forall x, y \in E, x \star y = y \star x$$

%
\subsection{Uniferité}
%
\paragraph{Définition} On dit que $\star$ est unifère si $E$ posséde un élément neutre pour $\star$, c'est-à-dire un élément tel que
$$ \forall x \in E, e \star x = x = x \star e$$
Cet élément neutre, s'il existe, est unique.
%Vielleicht Beweis einfügen

%
\subsection{Symétrie}
%
\paragraph{Définition} On suppose que $\star$ est unifère d'élément neutre e. Soit $x \in E$. On dit qu'un élément $y \in E$ est un symétrique de $x$ pour $\star$ si
$$ x \star y = e = y \star x$$
Le symétrique de $x$, s'il existe, est unique. Pour un $x \in E$. Si $\star$ est l'addition ($+$) le symétrique s'appelle l'opposé de $x$ et s'écrit $-x$. Si $\star$ est la multiplication ($\times$) le symétrique s'appelle l'inverse de x et s'écrit $x^{-1}$.
 
\paragraph{Exemple} La soustraction ($-$) sur $\Z$ n'est ni associative ni commutative ni unifère. ($0$ est élément neutre à droit mais pas à gauche $\rightarrow$ $0 - x = x \neq x - 0$.)

%
%
\section{Groupes, Anneaux, Corps}
%
%
%
\subsection{Groupe}
%
\paragraph{Définition} On appelle groupe un ensemble $E$ muni d'une loi de composition interne $\star$ associative et unifère et telle que tout élément de $E$ admette un symétrique pour $\star$. \\
Si de plus $\star$ est commutative, on dit que le groupe est commutatif.

\paragraph{Exemple}
$(\Z, +)$ est un groupe commutatif. Mais $(\N, +)$ n'est pas un groupe (parce qu'il n'y a pas d'élément symétrique pour $+$).

\paragraph{Exemple} Soit $n \geq 1$ un entier naturel. On appelle permutation de $\{ 1; 2; \ldots ; n\}$ une bijection de $\{ 1; 2; \ldots ; n\}$ dans lui-même. Alors l'ensemble $S_n$ des permutations de $\{ 1; 2; \ldots ; n\}$ muni de la composition des permutations est un groupe non commutatif pour $n \leq 3$.

%
\subsection{Anneau}
%
\paragraph{Définition} On appelle anneau un triplet $(A, +, \times)$ où $A$ est un ensemble et $+$ et $\times$ sont des lois de composition internes sur A appellées respectivement addition et multiplication telles que:
\begin{itemize}
  \item $A$ muni de $+$ est un groupe d'élément neutre noté $0$.
  \item $\times$ est associative est unifère d'élément neutre noté $1$.
  \item $\times$ est un biadditive, c'est-à-dire, pour toute $x, x', y, y' \in A$
    \begin{itemize}
      \item additivité à gauche: $(x + x') \times y = (x \times y)+(x' \times y)$
      \item additivité à droite: $x \times (y + y') = (x \times y)+(x \times y')$
    \end{itemize}
\end{itemize}
Si de plus $\times$ est commutative, on dit que $(A, +, \times)$ est un anneaux commutatif.

\paragraph{Exemple}  $(\Z, +, \times), (\R, +, \times)$ sont des anneaux commutatifs.

\paragraph{Définition} Soient $(A, +, \times)$ un anneau et $x \in A$. On dit que $x$ est un élément inversible de $A$, si $A$ admet un inverse pour $x$. On note $A^{x}$ l'ensemble des éléments inversibles de $A$. L'ensemble $A^{x}$ muni de $\times$ est un groupe.

%
\subsection{Corps}
%
\paragraph{Définition} Un corps est un anneau dans lequel tout élément non nul est inversible.

\paragraph{Exemple}
\begin{itemize}
  \item $(\Q, +, \times)$ cest le corps des nombres rationels.
  \item $(\R, +, \times)$ cest le corps des nombres réels.
  \item $(\C, +, \times)$ cest le corps des nombres complexes.
\end{itemize}

%
%
\section{Espaces vectoriels, algèbres}
%
%

%
\subsection{Espace vectoriel}
%
\paragraph{Définition} Soit $K$ un corps commutatif. On appelle espace vectoriel sur $K$ (ou $K$-espace vectoriel) un ensemble $V$ muni d'une loi de composition $+$ et d'une loi de composition externe
\begin{eqnarray*}
  \times: K \times V &\rightarrow& V \\
  (\alpha, \vec{v}) &\mapsto& \alpha\cdot\vec{v} = \alpha\vec{v}
\end{eqnarray*}
telles que:
\begin{itemize}
  \item $(V, +)$ est un groupe commutatif d'élément neutre noté $\vec{0}$.
  \item \forall $\vec{v}, \vec{u} \in V, \alpha \in K$: $$\alpha \cdot (\vec{v}+\vec{u}) = \alpha \cdot \vec{v} + \alpha \cdot \vec{u}$$
  \item \forall $\vec{v} \in V, \alpha, \beta \in K$:
    \begin{eqnarray*}
      (\alpha \beta) \cdot \vec{v} &=& \alpha \cdot (\beta \cdot \vec{v}) \\
      (\alpha + \beta) \cdot \vec{v} &=& \alpha \cdot \vec{v} + \beta \cdot \vec{v} \\
      1 \cdot \vec{v} &=& \vec{v}
    \end{eqnarray*}
\end{itemize}
Les éléments de $K$ sont appellées scalaires et ceux de $V$ vecteurs.

\paragraph{Important} Ne pas confondre l'addition ($+$) de $K$ et de $V$. Ces sont deux lois de composition différentes.

\paragraph{Exemples}
\begin{enumerate}
  \item Pour tous $x \in \R$ et $a\in \R$, on pose $a \times x = a x \in \R$. Alors l'ensemble $\R$ muni de l'addition usuelle $+$ et de $\times$ est un $\R$-espace vectoriel. En effet, puisque $\R$ est un anneau, $(\R, +)$ est un groupe commutatif. Le reste des axiomes de la structure d'espace vectoriel sont assurés par les propriétés de la multiplication dans $\R$. \\
    Plus généralement, si $n \geq 1$ est un entier naturel, en possent
    \begin{eqnarray*}
      (x_1, x_2, \ldots, x_n) + (y_1, y_2, \ldots, y_n) &=& ((x_1+y_1, x_2+y_2, \ldots, x_n+y_n) \\
      \alpha \times (x_1, x_2, \ldots, x_n) &=& (\alpha x_1, \alpha x_2, \ldots, \alpha x_n)
    \end{eqnarray*}
    pour tous $(x_1, x_2, \ldots, x_n), (y_1, y_2, \ldots, y_n) \in \R^n, \alpha \in \R$ on définit une structure de $\R$-espace vectoriel sur l'ensemble $\R^n$.
    
  \item Un polynôme à une indéterminée $X$ à coefficients dans $\R$ est une somme formelle 
    $$P(X) = a_0 + a_1 X + a_2 X^2 + \ldots + a_n X^{n}$$ 
    avec $n \leq 0$ et $a_0, a_1, a_2, \ldots, a_n \in \R$. Les $a_i$ sont les coefficients de $P$. \\
    Si 
    $$P(X) = a_0 + a_1 X + a_2 X^2 + \ldots + a_n X^{n}, a_n \neq 0$$ 
    et 
    $$Q(X) = b_0 + b_1 X + b_2 X^2 + \ldots + b_n X^{m}, b_m \neq 0$$ 
    sont deux polynômes, alors $P=Q$ si et seulement si $n=m$ et $a_i = b_i$  $\forall i, 0 \leq i \leq n = m$. \\
    
    On note $\R[X]$ l'ensemble des polynômes à une indéterminée $X$ à coefficients dans $\R$. En posant $$(P + Q)(X) = \sum_{n \leq 0} (a_n + b_n)X^n$$ pour 
    \begin{eqnarray*}
      P(X) &=& a_0 + a_1 X + a_2 X^2 + \ldots + a_n X^{n} \\
      Q(X) &=& b_0 + b_1 X + b_2 X^2 + \ldots + b_n X^{m} \\
      a_i = 0 &\forall& i > n \\
      b_i = 0 &\forall& i > m
    \end{eqnarray*}
    et 
    $$(\alpha P)(X) = \alpha a_0 + \alpha a_1 X + \alpha a_2 X^2 + \ldots + \alpha a_n X^{n} \in \R[X]$$
    pour $a \in \R$ et $P(X) = a_0 + a_1 X + a_2 X^2 + \ldots + a_n X^{n} \in \R[X]$ on définit une structure de \R-espace vectoriel sur $\R[X]$.
  \item Soit $X$ un ensemble. On note $\mathcal{A}(X, \R)$ l'ensemble des applications de $X$ dans $\R$. En posant
    $$(f + g)(x) = f(x) + g(x)$$
    pour tous $x \in X$ et $f, g \in \mathcal{A}(X, \R)$,
    $$(\alpha \cdot f)(x) = \alpha f(x)$$
    pour tous $x \in X$ et $f \in \mathcal{A}(X, \R)$ on définit une structure de $\R$-espace vectoriel sur $\mathcal{A}(X, \R)$. \\
    
    En effet, la structure de groupe commutatif de $(\R, +)$ donne la structure du groupe commutatif de $(\mathcal{A}(X, \R), +)$. Ainsi, l'élément $0$ application nulle est l'application de $X$ dans $\R$ qui associé $0$ à tout $x \in X$. \\
    Si $f \in \mathcal{A}(X, \R)$, l'opposée de $f$ est l'application de $X$ dans $\R$ qui à tout $x \in X$ associe à $-f(x)$. Le reste des axiômes de la structure d'espace vectoriel sont assurés par les propriétés de la multiplication dans $\R$.
\end{enumerate}

%
\subsection{Morphisme}
%
\paragraph{Définition} Soient $(U, +, \cdot)$ et $(V, +, \cdot)$ deux $\R$-espace vectoriels. Un morphisme de $\R$-espace vectoriels de $U$ dans $V$ est une application linéaire
  $$f: U \rightarrow V$$
  telle que
  \begin{eqnarray*}
    f(\vec{u} + \ve{v}) = f(\vec{u}) + f(\vec{v}) ~ &\forall&\vec{u}, \vec{v} \in U \mbox{~ (additivité)}\\
    f(\alpha \vec{u} ) = \alpha f(\vec{u}) ~ &\forall&\vec{u} \in U, \alpha \in \R \mbox{~ (homogénéité)}
  \end{eqnarray*}

\paragraph{Remarque} Les deux conditions peuvent se résumer en une seule
  \begin{eqnarray*}
    f(\alpha \vec{u} + \ve{v}) = \alpha f(\vec{u}) + f(\vec{v}) ~ &\forall&\vec{u}, \vec{v} \in U, \alpha \in \R \mbox{~ (linéarité)}
  \end{eqnarray*}

%
\subsection{Isomorphisme}
%
\paragraph{Définition} Soient $U$ et $V$ deux $\R$-espace vectoriels. Un isomorphisme de $\R$-espace vectoriels de $U$ dans $V$ est une application $\R$-linéaire bijective de $U$ dans $V$. %TODO

\paragraph{Proposition} Soit $(V, +, \cdot)$ un $\R$-espace vectoriel. Alors on a:
\begin{enumerate}[a)]
  \item $0\cdot \vec{v} = \vec{0} ~\forall~ \vec{v} \in V$
  \item $\alpha \cdot \vec{0} = \vec{0} ~\forall~ \alpha \in \R$
  \item $\alpha \cdot \vec{v} = \vec{0}$ seulement si $\alpha = \vec{0}$ ou $\vec{v} = \vec{0}$
  \item $(-\alpha)\cdot \vec{v} = \alpha \cdot (-\vec{v}) ~\forall~ \alpha \in \R, \vec{v} \in V$
\end{enumerate}

\paragraph{Propostition}
\begin{enumerate} 
  \item Soient $(U, +, \cdot)$ et $(V, +, \cdot)$ deux $\R$-espace vectoriels et $f: U \rightarrow V$ une application $\R$-linéaire. Alors on a:
    \begin{enumerate}[a)] 
      \item $f(\vec{0}_U) = \vec{0}_V$
      \item $f(-\vec{u}) = -f(\vec{u}) ~\forall~ \vec{u} \in U$
    \end{enumerate}
  \item 
    \begin{enumerate}[a)]
      \item Soit $(V, +, \cdot)$ un $\R$-espace vectoriel. Alors
        \begin{eqnarray*}
          id_V: V &\rightarrow& V \\
          \vec{v} &\mapsto& id_{V}(\vec{v}) = \vec{v}
        \end{eqnarray*}
        est une application $\R$-linéaire.
      \item Soient $(U, +, \cdot)$, $(V, +, \cdot)$ et $(W, +, \cdot)$  trois $\R$-espaces vectoriels, $f: U \rightarrow V$ et $g: V \rightarrow W$ deux applications $\R$-linéaires. Alors l'application composé
        \begin{eqnarray*}
          g \circ f: U &\rightarrow& W \\
          \vec{u} &\mapsto& (g \circ f)(\vec{u}) = g(f(\vec{u}))
        \end{eqnarray*}
        est aussi $\R$-linéaire.
    \end{enumerate}
\end{enumerate}

\paragraph{Démonstration}
\begin{enumerate}
  \item
    \begin{enumerate}[a)]
      \item $$f(\vec{0}_U) = f(\vec{0}_U + \vec{0}_U) = f(\vec{0}_U) + f(\vec{0}_U)$$
        Ajoutons $-f(\vec{0}_V)$ (l'opposé de $f(\vec{0}_V)$.
        \begin{eqnarray*}
          \vec{0}_V = f(\vev{0}_U) + (-f(\vev{0}_U)) &=& (f(\vec{0}_U) + f(\vec{0}_U)) + (-f(\vev{0}_U)) \\
            &=& f(\vec{0}_U) + (f(\vec{0}_U) + (-f(\vev{0}_U))) \\
            &=& f(\vec{0}_U) + \vec{0}_V = f(\vec{0}_U)
        \end{eqnarray*}
      \item Soit $\vec{u} \in U$. On a
        \begin{eqnarray*}
          f(-\vec{u}) + f(\vec{u}) &=& f((-\vev{u}) + \vec{u}) = f(\vec{0}_U) = \vec{0}_V \\
          f(-\vec{u}) + f(\vec{u}) &=& f(\vec{u}) + f(-\vec{u}) = \vec{0}_V ~ \text{(+ est commutative)}
        \end{eqnarray*}
        Par conéquence, $f(-\vec{u}) = f(\vec{u})$.
    \end{enumerate}
  \item
    \begin{enumerate}[a)]
      \item Soient $\vec{u}, \vec{v} \in V, \alpha \in \R$.
        \begin{eqnarray*}
          id_V(\alpha \vec{u} + \vec{v}) &=& \alpha \vec{u} + \vec{v} \\
            &=& \alpha f(\vec{u}) + f(\vec{v})
        \end{eqnarray*}
      \item Soient $\vec{u}, \vec{v} \in V, \alpha \in \R$.
        \begin{eqnarray*}
          (g \circ f)(\alpha \vec{u} +\vec{v}) &=& g(f(\alpha \vec{u} +\vec{v})) \\
            &=& g(\alpha \vec{u} + \vec{v}) \\
            &=& \alpha g(f(\vec{u})) + g(f(\vec{v})) \\
            &=& \alpha (g \circ f)(\vec{u}) + (g \circ f)(\vec{v})
        \end{eqnarray*}
    \end{enumerate}
\end{enumerate}

%
\subsection{Algèbre}
%
\paragraph{Définition} Soit $K$ un corps commutatif. On appelle algèbre sur $K$ (ou $K$-algèbre) un quadrouplet $(A, +, \times, \cdot)$ ou $A$ est un ensemble, $+$ et $\times$ sont des lois de composition internes sur $A$ et $\cdot: K \cdot A \rightarrow A$ une loi de composition externe tels que:
\begin{itemize}
  \item $(A, +, \cdot)$ est un $K$-espace vectoriel.
  \item $\times$ est associative et possède un élément neutre noté $1$.
  \item 
    \begin{eqnarray*}
      \times: A \times A &\rightarrow& A \\
      (x, y) &\mapsto& x \times y
    \end{eqnarray*}
    est bilinéaire. C'est-à-dire pour tout $y \in A$, l'application
      \begin{eqnarray*}
        A &\rightarrow& A \\
        x &\mapsto& x \times y
      \end{eqnarray*}
      est $\R$-linéaire (linéarité à gauche) et pour tout $x \in A$, l'application
      \begin{eqnarray*}
        A &\rightarrow& A \\
        y &\mapsto& x \times y
      \end{eqnarray*}
      est aussi $\R$-linéaire (linéarité à droite).
      \begin{eqnarray*}
        \forall~ y \in A ~\forall~ x, x' \in A ~\text{et}~ \alpha \in \R \\
          (\alpha \cdot x + x') \times y &=& \alpha \cdot (x \times y) + (x' \times y) \\
        \forall~ x \in A ~\forall~ y, y' \in A ~\tex{et}~ \alpha \in \R \\
          x \times (\alpha \cdot y + y') &=& \alpha \cdot (x \times y) + (x \times y') \\
      \end{eqnarray*}
\end{itemize}

\paragraph{Exemples}
\begin{enumerate}[1)]
  \item Soit $\R[X]$ le $\R$-espace vectoriel des polynômes à une indéterminée à  coefficients dans $\R$. On définit une loi de composition interne $\times$ sur $\R[X]$ comme suit:
    \begin{eqnarray*}
      P(X) &=& a_0 + a_1 x^1 + a_2 x^2 + \ldots + a_n x^n = \sum_{i=0}^{n} a_i x^i \\
      Q(X) &=& b_0 + b_1 x^1 + b_2 x^2 + \ldots + b_m x^m = \sum_{j=0}^{m} b_j x^j \\
      (P \times Q)(X) &=& (a_0 b_0) + (a_0 b_1 + a_1 b_0) x^1 + (a_0 b_2 + a_1 b_1 + a_2 b_0) x^2 + \ldots + (a_n b_m) x^{n+m} \\
        &=& \sum_{l=0}^{n+m} \left ( \sum_{(i, j), i+j=l} a_i b_j \right ) x^l
    \end{eqnarray*}
    On vérifie que $\times$ est associative, unifère et bilinéaire. Alors $(\R[X], + , \times, \cdot)$ est un $\R$-algèbre.
    
  \item Soit $(V, +, \cdot)$ und $\R$-espace vectoriel. On note $End_{\R}(V)$ l'ensemble des applications $\R$-linéaires de $V$ dans $V$; ce sont les endomorphismes du $\R$-espace vectoriel $V$. On définit sur $End_{\R}(V)$:
    \begin{itemize}
      \item une addition $+$ par
        $$f+g \in End_{\R}(V), ~ (f+g)(\vec{v}) = f(\vec{v}) + g(\vec{v}) ~ \forall ~ \vec{v} \in V$$
      \item une multiplication externe $\cdot$ par
        $$f \in End_{\R}(V), ~ \alpha \in \R, ~ (\alpha \cdot f)(\vec{v}) = \alpha \cdot f(\vec{v}) ~ \forall ~ \vec{v} \in V$$
      \item une multiplication interne $\circ$ par
        $$f, g \in End_{\R}(V), ~ (g \circ f)(\vec{v}) = g(f(\vec{v})) ~ \forall ~ \vec{v} \in V$$
    \end{itemize} 
    On vérifie que $(End_{\R}(V), +, \circ, \cdot)$ est un $\R$-algèbre.
\end{enumerate}


\chapter{Calcul matriciel}

%
%
\section{Matrices}
%
%

%
\subsection{Matrice}
%
\paragraph{Définition} Soient $n \geq 1$ et $m \geq 1$ deux entiers naturels. On appelle matrice de taille $n\times m$ à coefficients dans $\R$ un tableau à $n$ lignes et $m$ colonnes de la forme
$$\begin{pmatrix}
  a_{11} & a_{12} & \ldots & a_{1m} \\
  a_{21} & a_{22} & \ldots & a_{2m} \\
  \vdots & \vdots & \ddots & \vdots \\
  a_{n1} & a_{n2} & \ldots & a_{nm}
\end{pmatrix} \text{ avec } a_{ij} \in \R$$

%
\subsection{Ensemble des matrices}
%
\paragraph{Définition} On note $M_{n, m}(\R)$ l'ensemble des matrices de taille $n\times m$ à coefficients dans $\R$.

\paragraph{Exemple}
$$\begin{pmatrix}
  2 & -5 & \sqrt{2} \\
 3 &0 & 1 \\
\end{pmatrix} \text{ est une matrice de taille $2\times 3$ à coefficients dans $\R$. Le coefficient $a_{12}$ est -5.} $$

%
\subsection{Matrice carrée}
%
\paragraph{Définition} Soit $n \geq 1$ un entier naturel et $A$ une matrice de taille $n\times n$. On dit alors que $A$ est une matrice carrée. \\ 
On note $M_{n}(\R)$ l'ensemble des matrices carrées de taille $n\times n$ à coefficients dans $\R$. Les coefficients $a_{ii}$ pour $1 \leq i \leq n$ sont appellés coefficients diagonaux de $A$.

%
%
\section{Opérations sur les matrices}
%
%

%
\subsection{Somme}
%
\paragraph{Définition} Soient $m\geq 1$ et $n \geq 1$ deux entiers naturels. On peut définir la somme de deux matrices de même taille. Si $A = (a_{ij}), B = (b_{ij}) \in M_{n, m}(\R)$ on définit leur somme $S = A + B \in M_{n, m}(\R)$ par
$$s_{ij} = a_{ij} + b_{ij} ~ \forall ~ 1 \leq i \leq n, 1 \leq j \leq m$$

\paragraph{Proposition} Soient $m \geq 1$ et $n\geq 1$ des entiers naturels.
\begin{enumerate}
  \item Soient $A, B, C \in M_{n, m}(\R)$. Alors
    $$(A + B) + C = A + (B + C)$$
  \item On note $0$ la matrice nulle dans $M_{n, m}(\R)$, c'est-à-dire la matrice dont tous les coefficients sont nuls. Alors pour tout $A \in M_{n, m}(\R)$ on a
    $$A + 0 = A = 0 + A$$
  \item Soit $A \in M_{n, m}(\R)$. On note $-A$ la matrice $(-a_{ij})$ dans l'ensemble $M_{n, m}(\R)$. Alors on a
    $$(-A) + A = 0 = A + (-A)$$
\end{enumerate}
Ainsi, l'addition $+$ dans $M_{n, m}(\R)$ est associative, unifère et tout $A \in M_{n, m}(\R)$ admet une matrice opposée. Par conséquence $(M_{n, m}(\R), +)$ est un groupe.\\
De plus, comme $+$ dans $\R$ est commutative, l'addition dans $M_{n,m}(\R)$ est commutative:
$$\forall A, B \in M_{n, m}(\R), ~ A + B = B + A$$
En conclusion, $(M_{n, m}(\R)$ est un groupe commutatif.

\paragraph{Démonstration}
\begin{enumerate}
  \item Cela découle de l'associativié de l'addition dans $\R$.
  \item Cela découle de la propriété de l'élément $0$ dans $\R$.
  \item Cela résulte des propriétés de l'opposé d'un réel.
\end{enumerate}

%
\subsection{Produit avec un réel}
%
\paragraph{Définition} Soient $n \geq 1, m \geq 1$ deux entiers naturels, $A = (a_{ij}) \in M_{n, m}(\R)$ et $\alpha \in \R$. On  définit le produit de $A$ par $\alpha$ que l'on note $\alpha A$, par
$$\alpha A = (\alpha a_{ij}) \in M_{n, m}(\R)$$

\paragraph{Proposition} Soient $n \geq 1, m \geq 1$ des entiers naturels. On a
\begin{enumerate}
  \item $\forall \alpha \in \R, ~ A, B \in M_{n, m}(\R)$ 
    $$\alpha \cdot (A + B) = \alpha \cdot A + \alpha \cdot B$$
    
  \item $\forall \alpha, \beta \in \R, ~ A \in M_{n, m}(\R)$
    \begin{eqnarray*}
      (\alpha + \beta) \cdot A &=& \alpha \cdot A + \beta \cdot A \\
      (\alpha \cdot \beta) \cdot A &=& \alpha \cdot (\beta \cdot A)
    \end{eqnarray*}
    
  \item $\forall A \in M_{n, m}(\R)$
    $$1 \cdot A = A$$
\end{enumerate}
On peut donc dire, que $M_{n, m}(\R)$ muni de l'addition des matrices et de la multiplication par les réels est un $\R$-espace vectoriel.

%
\subsection{Produit matriciel}
%
\paragraph{Définition} Soient $n \geq 1, m \geq 1$ et $p \geq 1$ trois entiers naturels. On peut définir le produit $A \times B$ d'une matrice $A$ et d'une matrice $B$ dans cet ordre si le nombre de colonnes de $A$ est égal au nombre de lignes de $B$. \\
Si $A = (a_{ij}) \in M_{n, m}(\R)$ et $B = (b_{ij}) \in M_{m, p}(\R)$, alors on définit $C = A \times B \in M_{n, p}(\R)$
$$c_{ij} = \sum_{l=1}^{m} {a_{il} * b_{lj} = a_{i1} \cdot b_{1j} + a_{i2} \cdot b_{2j} + \ldots + a_{im} \cdot b_{mj}$$

\paragraph{Exemple} Soient $A = 
\begin{pmatrix}
  2 & -5 & 1} \\
  0 & 4 & 7 \\
\end{pmatrix} \in M_{2,3}(\R)$ et $B = 
\begin{pmatrix}
  1 & -5 & 2 & 4 \\
  2 & 4 & -7 & 3\\
  0 & 1 & 0 & 1 \\
\end{pmatrix} \in M_{3,4}(\R)$. Le nombre de colonnes de $A$ est égal au nombre de lignes de $B$. Le produit est donc défini. 
$$AB = 
\begin{pmatrix}
  -8 & -29 & 39 & -6 \\
  8 & 23 & -28 & 19 \\
\end{pmatrix} \in M_{2,4}(\R)$$

\paragraph{Proposition} A conditions que les produits et les sommes considerés soient bien définit, on a
\begin{enumerate}
  \item $$A \times (B \times C) = (A \times B) \times C$$
  \item $$A \times (B + C) = A \times B + A \times C$$ 
    $$(B + C) \times A = B \times A + C \times A$$
  \item Pour tout entier $m \geq 1$ on appelle matrice identité de taille $n \times n$ la matrice carrée
    $$I_n = (a_{ij}) ~ ~ a_{ij} = \left\{ \begin{array}{lr} 1, & i=j \\ 0, & i \neq j \end{array}$$
\end{enumerate}

\paragraph{Attention} Deux matrices quelconques ne commutent pas nécessairement.
\paragraph{Exemple} Soient $A = \begin{pmatrix} 1 & 3 \\ 2 & 7 \end{pmatrix}$ et $B = \begin{pmatrix} -3 & 0 \\ 4 & 1 \end{pmatrix}$. Alors on a
$$A B = \begin{pmatrix} 9 & 3 \\ 22 & 7 \end{pmatrix}$$
$$B A = \begin{pmatrix} -3 & -9 \\ 6 & 19 \end{pmatrix}$$
On voit donc que $A B \neq B A$.

\paragraph{Exemple} Soient $A=\begin{pmatrix} 1 & 3 \\ 2 & 7 \end{pmatrix}$ et $B=\begin{pmatrix} -3 & 0 \\ 4 & 1 \end{pmatrix}$. Alors 
$$A B = 
\begin{pmatrix}
  9 & 3 \\
  22 & 7
\end{pmatrix} \text{ et } B A = 
\begin{pmatrix}
  -3 & -9 \\
  6 & 19
\end{pmatrix}$$ 
Donc $A B\neq B A$.

\paragraph{Attention}
\begin{itemize}
  \item Si $A$ et $B$ sont deux matrices on peut avoir $A \neq 0, B \neq 0$ et $AB = 0$
  \item On peut donc avoir également $AB = AC$ et $B \neq C$ pour des matrices $A, B, C$.
\end{itemize}

\paragraph{} On a, $(M_{n, m}(\R), +, \times, \cdot)$ est une $\R$-algèbre. En particulier $(M_{n, m}(\R), +, \times)$ est un anneau.

%
%
\section{Inversion des matrices}
%
%

%
\subsection{Matrice inversible}
%
\paragraph{Définition} Soit $n \geq 1$ un entier naturel. Soit $A$ une matrice carrée de taille $n \times n$. On dit que $A$ est inversible s'il existe une matrice carrée $B$ de taille $n \times n$ telle que $A B = I_n = B A$. Dans ce cas on dit que $B$ est un inverse de $A$.

\paragraph{Proposition} Soit $A$ une matrice carrée de taille $n \times n$. Si $A$ est inversible, alors son inverse est unique.

\paragraph{Démonstration} On suppose que $A$ est inversible. Si $B$ et $C$ sont deux inverses pour $A$, alors on a
$$B A C = (B A) C = I_n C = C$$
on a également
$$B A C = B (A C) = B I_n = B$$
d'où $B = C$.

\paragraph{Notation} Si $A$ est inversible, on note $A^{-1}$ son inverse. On a donc
$$A A^{-1} = I_n = A^{-1} A$$

\paragraph{Exemple} La matrice $\begin{pmatrix} 2 & 1 \\ 7 & 4 \end{pmatrix}$ est inversible d'inverse $\begin{pmatrix} 4 & -1 \\ -7 & 2 \end{pmatrix}$.
$$\begin{pmatrix}2 & 1 \\ 7 & 4 \end{pmatrix} 
  \begin{pmatrix} 4 & -1 \\ -7 & 2 \end{pmatrix} 
  = \begin{pmatrix} 1 & 0 \\ 0 & 1 \end{pmatrix} = I_2$$
$$\begin{pmatrix} 4 & -1 \\ -7 & 2 \end{pmatrix} 
  \begin{pmatrix}2 & 1 \\ 7 & 4 \end{pmatrix}  
  = \begin{pmatrix} 1 & 0 \\ 0 & 1 \end{pmatrix} = I_2$$
On vérifie que $A A^{-1} = I_n = A^{-1} A$.

%
\subsection{Matrice de taille $2\times 2$}
%
\paragraph{Lemme} Soient $a, b, c, d \in \R, A = \begin{pmatrix} a & b \\ c & d \end{pmatrix} \in M_{2}(\R)$ et $B = \begin{pmatrix} d & -b \\ -c & a \end{pmatrix} \in M_{2}(\R)$. Alors $A B = (a d - b c) I_n = B A$.

\paragraph{Démonstration} 
$$A B 
  = \begin{pmatrix} a & b \\ c & d \end{pmatrix} \begin{pmatrix} d & -b \\ -c & a \end{pmatrix} 
  = \begin{pmatrix} ad-bc & 0 \\ 0 & ad-bc \end{pmatrix} 
  = (ad-bc) \begin{pmatrix} 1 & 0 \\ 0 & 1 \end{pmatrix} 
  = (ad-bc) I_2$$
et 
$$B A 
  = \begin{pmatrix} d & -b \\ -c & a \end{pmatrix} \begin{pmatrix} a & b \\ c & d \end{pmatrix} 
  = \begin{pmatrix} ad-bc & 0 \\ 0 & ad-bc \end{pmatrix} 
  = (ad-bc) \begin{pmatrix} 1 & 0 \\ 0 & 1 \end{pmatrix} 
  = (ad-bc) I_2$$

\paragraph{Proposition} Soit $A = \begin{pmatrix} a & b \\ c & d \end{pmatrix} \in M_{2}(\R)$. Alors $A$ est inversible si et seuelement si $a d - b c \neq 0$, et dans ce cas
$$A^{-1} = \frac{1}{a d - b c} \begin{pmatrix} d & -b \\ -c & a \end{pmatrix}$$

\paragraph{Démonstration} Cet énoncé est une équicalence constituée de deux implications:
\begin{enumerate}
  \item si $a d - b c \neq 0$, alors $A$ est inversible et $A^{-1} = \frac{1}{ad-bc} \begin{pmatrix} d & -b \\ -c & a \end{pmatrix}$
  
  \item si $a d - b c = 0$, alors $A$ n'est pas inversible.
\end{enumerate}
Démontrons les des implications:
\begin{enumerate}
  \item On suppose que $a d - b c \neq 0$. Par suite, $a d - b c$ est inversible dans $\R$. D'après le lemme, on a
    $$\begin{pmatrix} a & b \\ c & d \end{pmatrix} \begin{pmatrix} d & -b \\ -c & a \end{pmatrix} 
      = (a d - b c) I_2 
      =  \begin{pmatrix} d & -b \\ -c & a \end{pmatrix} \begin{pmatrix} a & b \\ c & d \end{pmatrix}$$
    d'où
    $$\begin{pmatrix} a & b \\ c & d \end{pmatrix} \left( \frac{1}{a d - b c} \begin{pmatrix} d & -b \\ -c & a \end{pmatrix} \right) 
      = I_2 
      = \left( \frac{1}{a d - b c} \begin{pmatrix} d & -b \\ -c & a \end{pmatrix} \right) \begin{pmatrix} a & b \\ c & d \end{pmatrix}$$
    Par conséquent, $A$ est inversible d'inverse $A^{-1}=\frac{1}{a d - b c}\begin{pmatrix} d & -b \\ -c & a\end{pmatrix}$.
    
  \item On suppose que $a d - b c = 0$. On raisonne par l'absurde. Si $A$ est inversible, on obtient: D'après le lemme on a
    $$\begin{pmatrix} a & b \\ c & d \end{pmatrix} \begin{pmatrix} d & -b \\ -c & a \end{pmatrix} = (a d - b c) I_2 = 0$$
    d'où
    \begin{eqnarray*}
      A^{-1} \left( A \begin{pmatrix} d & -b \\ -c & a \end{pmatrix} \right) &=& A^{-1} 0 = 0 \\
        &=& A^{-1} A \begin{pmatrix} d & -b \\ -c & a \end{pmatrix} \\
      0 &=& \begin{pmatrix} d & -b \\ -c & a \end{pmatrix}
    \end{eqnarray*}
    Par conséquent $a = b = c = d = 0$, donc $A = 0$ donc $A$ n'est pas inversible, ce qui donn lien à une contradiction. Ainsi $A$ n'est pas inversible si $a d - b c = 0$.
    
\end{enumerate}

\paragraph{Proposition} Soient $A$ et $B$ des matrices inversibles de taille $n \times n$. Alors
\begin{enumerate}
  \item $A^{-1}$ est inversible d'inverse $A$
  \item $A B$ est inversible et $(A B)^{-1} = B^{-1} A^{-1}$
\end{enumerate}

\paragraph{Démonstration}
\begin{enumerate}
  \item Puisuque $A$ est inversible d'inverse $A^{-1}$, on a
    $$A A^{-1} = I_n = A^{-1} A$$
    donc $A^{-1}$ est inversible d'inverse $A$.
  \item On a
    $$(A B)(B^{-1} A^{-1}) = A (B B^{-1}) A^{-1} = A I_n A^{-1} = A A^{-1} = I_n$$
    De même, on a $(B^{-1} A^{-1}) A B = I_n$ \\
    Par conséquence, $A B$ est inversible d'inverse $(B^{-1} A^{-1})$
\end{enumerate}

\paragraph{Proposition} Plus généralement, si $A_1, A_2, \ldots, A_m$ sont des matrices inversibles on a $(A_1 A_2 \ldots A_m)^{-1} = A_m^{-1} A_{m-1}^{-1} \ldots A_1^{-1}$

%
\subsection{Groupe des matrices inversibles}
%
\paragraph{Proposition} L'ensemble des matrices inversibles de taille $n \times n$ muni de la multiplication des matrices est un group. En effet
\begin{itemize}
  \item L'assertion $2.$ de la proposition précédante assure que $\times$ restreinte à l'ensemble des matrices inversibles est une loi de composition interne.
  \item $\times$ étant associative sur $M_{n}(\R)$, elle l'est en restriction à l'ensemble des matrices inversibles.
  \item $I_n$ est élément neutre.
  \item L'assertion $1.$ de la proposition assure que l'inverse d'une matrice inversible est bien inversible.
\end{itemize}
On note $(GL_{n}(\R), \times)$ le groupe des matrices inversibles de taille $n$ ($GL$ est mis pour "groupe linéaire").

\paragraph{Exemple} Soient $A = \begin{pmatrix} 2 & 7 \\ 1 & 4 \end{pmatrix}$ et $B = \begin{pmatrix} 5 & 3 \\ 3 & 2 \end{pmatrix}$.
$A$ est inversible car $2 \cdot 4 - 1 \cdot 7 = 1 \neq 0$, et
$$A^{-1} = \frac{1}{1} \begin{pmatrix} 2 & -7 \\ -1 & 4 \end{pmatrix} = \begin{pmatrix} 2 & -7 \\ -1 & 4 \end{pmatrix}$$
$B$ est inversible car $5 \cdot 2 - 3 \cdot 3 = 1 \neq 0$, et 
$$B^{-1} = \frac{1}{1} \begin{pmatrix} 2 & -3 \\ -3 & 5 \end{pmatrix} = \begin{pmatrix} 2 & -3\\ -3 & 5 \end{pmatrix}$$
On a 
$$A B = \begin{pmatrix} 2 & 7 \\ 1 & 4 \end{pmatrix} 
  \begin{pmatrix}5&3\\3&2\end{pmatrix}
  = \begin{pmatrix} 31 & 20 \\ 17 & 11 \end{pmatrix}$$
et
$$B^{-1} A^{-1} = \begin{pmatrix} 31 & 20 \\ 17 & 11 \end{pmatrix} 
  \begin{pmatrix} 2 & -7 \\ -1 & 4 \end{pmatrix} 
  = \begin{pmatrix} 11 & -20 \\ -17 & 31 \end{pmatrix}$$
On a $(31 \cdot 11) - (17 \cdot 20) = 341 - 340 = 1 \neq 0$. Donc $A B$ est inversible 
$$\frac{1}{1} \begin{pmatrix} 11 & -20 \\ -17 & 31 \end{pmatrix} 
  = \begin{pmatrix} 11 & -20 \\ -17 & 31 \end{pmatrix}$$
qui est bien égale à $B^{-1} A^{-1}$.

%
%
\section{Matrices élémentaires}
%
%

%
\subsection{Matrice élémentaire}
%
\paragraph{Définition} Soient $n \geq 1$ un entier naturel et $(i, j) \in \{1; \ldots; n\} \times \{1; \ldots; n\}$. On note $F_{i, j}$ la matrice carrée de taille $n \times n$ contenant un $1$ en $i^{ème}$ ligne et $j^{ème}$ colonne et des $0$ partout ailleurs. On dit que $F_{i, j}$ est une matrice standard.

\paragraph{Exemple} 
$$n = 3 \rightarrow F_{2,3} = 
\begin{pmatrix}
  0 & 0 & 0 \\
  0 & 0 & 1 \\
  0 & 0 & 0
\end{pmatrix}$$
$$n = 4 \rightarrow F_{4,1} = 
\begin{pmatrix}
  0 & 0 & 0 & 0 \\
  0 & 0 & 0 & 0 \\
  0 & 0 & 0 & 0 \\
  1 & 0 & 0 & 0
\end{pmatrix}$$

\paragraph{Proposition} Toute matrice carrée de taille $\times n$ peut s'exprimer à l'aide des matrices standard de taille $n\times n$
$$A = \sum_{(i, j) \in \{1; \ldots; n\}\times\{1; \ldots; n\}} a_{ij} F_{ij}$$

\paragraph{Proposition} Soit $A$ une matrice de taille $n\times m$.
\begin{enumerate}[a)]
  \item Soit $F_{ij}$ une matrice standard de taille $n\times n$. Alors $F_{ij} A$ est la matrice de taille $n \times m$ dont la $i^{ème}$ ligne est la $j^{ème}$ ligne de $A$ et dont les autres sont nulles.
  \item Soit $F_{ij}$ une matrice standard de taille $m\times m$. Alors $A F_{ij}$ est la matrice de taille $n\times m$ dont la $j^{ème}$ colonne est la $i^{ème}$ de $A$ et dont les autres colonnes sont nulles.
\end{enumerate}

\paragraph{Démonstration}
\begin{enumerate}[a)]
  \item Soit $k \in \{1, \ldots, n \}$ et $l \in \{1, \ldots, n \}$. On a
    $$(F_{ij} A)_{k l} = \sum_{r = 1}^{n} (F_{i j})_{k r} A_{r l}$$
    si $k\neq i$, alors la $k$-ième ligne de $F_{i j}$ est nulle, donc
    $$(F_{i j})_{k l}=0 \text{ pour tout } l$$
    On obtient alors
    $$(F_{i j})_{k l}=0 \text{ si } k\neq i$$
    Si $k=i$, alors 
    $$(F_{i j} A)_{i l} = \sum_{r = 1}^{n} (F_{i j})_{i r}A_{r l}$$
    si $r \neq j$, alors la $r$-ième colonne de $F_{i j}$ est nulle, donc
    $$(F_{ij})_{i r} = 0$$
    il suit donc
    $$(F_{i j} A)_{i l} = (F_{i j})_{i j} A_{j l} =A_{j l}$$
    D'après ce qui procède, si $k \neq i$, alors la $k$-ième ligne de $F_{i j}A$ est nulle. Si $k = i$, alors la $i$-ième ligne de $F_{i j} A$ est $(A_{j 1} ~ A_{j 2} ~ \ldots ~ A_{j m})$ qui est la $j$-ième ligne de $A$.

  \item Ceci se montre de façon similaire.
\end{enumerate}

\paragraph{Exemple}
\begin{itemize}
  \item Soient $A = \begin{pmatrix} 1 & -7 & 5 \\ 0 & 3 & 4 \end{pmatrix} \in M_{2\times 3}(\R)$ et $F_{1 2} = \begin{pmatrix} 0 & 1 \\ 0 & 0 \end{pmatrix} \in M_{2 \times 2}(\R)$. Alors
    $$F_{1 2} A 
      = \begin{pmatrix} 0 & 1 \\ 0 & 0 \end{pmatrix} \begin{pmatrix} 1 & -7 & 5 \\ 0 & 3 & 4 \end{pmatrix} 
      = \begin{pmatrix} 0 & 3 & 3 \\ 0 & 0 & 0 \end{pmatrix}$$

  \item Soient $A = \begin{pmatrix} 1 & 5 \\ 0 & 4 \\ -7 & 2 \end{pmatrix} \in  M_{3 \times 2}(\R)$ et $F_{3 1} = \begin{pmatrix} 0 & 0 & 0 \\ 0 & 0 & 0 \\ 1 & 0 & 0 \end{pmatrix} \in M_{3 \times 3}(\R)$. Alors
    $$F_{3 1} A 
      = \begin{pmatrix} 0 & 0 & 0 \\ 0 & 0 & 0 \\ 1 & 0 & 0 \end{pmatrix} \begin{pmatrix} 1 & 5 \\ 0 & 4 \\ -7 & 2 \end{pmatrix} 
      = \begin{pmatrix} 0 & 0 \\ 0 & 0 \\ 1 & 5 \end{pmatrix}$$
      
  \item Soient $A = \begin{pmatrix} 1 & 5 \\ 0 & 4 \\ -7 & 2 \end{pmatrix} \in  M_{3 \times 2}(\R)$  et $F_{1 2} = \begin{pmatrix} 0 & 1 \\ 0 & 0 \end{pmatrix} \in M_{2 \times 2}(\R)$. Alors
    $$A F_{1 2} 
      = \begin{pmatrix} 1 & 5 \\ 0 & 4 \\ -7 & 2 \end{pmatrix} \begin{pmatrix} 0 & 1 \\ 0 & 0 \end{pmatrix} 
      =\begin{pmatrix} 0 & 1 \\ 0 & 0 \\ 0 & -7 \end{pmatrix}$$
\end{itemize}

%
\subsection{Symbole de Kronecker}
%
\paragraph{Définition} Soit $n \geq 1$ un entier naturel. On note
$$\mS: \{1; \ldots; n\}\times\{1; \ldots; n \} \rightarrow {0; 1}$$
L'application est défini par
$$\delta_{ij} = \left\{\begin{array}{lr} 1 & \text{si i=j} \\ 0 & \text{sinon} \end{array} \right.$$ 

\paragraph{Corollaire} Soient $n \geq 1$ un entier naturel et
$$(i, j), (k, l) \in \{1; \ldots; n\}\times\{1; \ldots; n\}$$
alors
$$F_{ij}F_{kl} = \delta_{ik} F_{il} = 
  \left\{\begin{array}{lr} F_{il} & \text{si i=j} \\ 0 & \text{sinon} \end{array} \right.$$

\paragraph{Démonstration} D'après la proposition, précédente, $F_{ij} F_{kl}$ est la matrice don la $i^{ème}$ ligne est la $j^{ème}$ ligne de $F_{kl}$ et dont les autres lignes sont nulles. Il vient
\begin{itemize}
  \item si $j\neq k$, la $j^{ème}$ ligne de $F_{kl}$ est nulle
  \item si $j = k$ la $j^{ème} = k^{ème}$ ligne de $F_{kl}$ est $(0_{1} \ldots 0_{l-1} ~ 1_{l} ~ 0_{l+1} \ldots 0_{n})$
\end{itemize}
Par conséquent, si $j\neq k$, toutes les lignes de $F_{ij} F_{kl}$ sont nulles donc $F_{ij} F_{kl} = 0$; si $j=k$, alors 
$$F_{ij} F_{kl} = \begin{pmatrix} 
  & & & 0 & \\
  & & & \vdots & \\
  & & & 0 & \\
  0 & \ldots & 0 & 1 & 0 \\
  & & & 0 & \\
\end{pmatrix} = F_{il}$$
Ainsi $F_{ij} F{kl} = \delta_{ik} F_{il}$.

%
\subsection{Les opérations élémentaires}
%
\paragraph{Rappel} On rappelle que l'on distingue trois types d'opérations éLémentaires que l'on peut effectuer sur les lignes d'une matrice
\begin{itemize}
  \item multiplier une ligne par un réel
  \item ajouter à une ligne un multiple quelconque d'une autre ligne
  \item permuter deux lignes
\end{itemize}
On introdiut les matrices suivantes ($n \geq 1$ entier naturel)
\begin{itemize}
  \item $c \in \R \backslash \{0 \}, i \in \{1; \ldots; n\}$
    $$E_i(c) = \left( \sum_{l=1, l\neq i}^n F_{ll} \right) +  c F_{ii} = 
      \begin{pmatrix}
        1 & & \\
        & 1 & \\
        & & \ddots & \\
        & & & c & \\
        & & & & 1
      \end{pmatrix}$$
      
  \item $c \in \R, (i, j) \in \{1;\ldots;n\}\times\{1;\ldots;n\}, i\neq j$
    $$E_{ij}(c) = I_n + c F_{ij} =
      \begin{pmatrix}
        1 & & & \\
        & 1 & & c \\
        & & \ddots & \\
        & & & 1
      \end{pmatrix}$$
      
  \item $(i, j) \in \{1; \ldots; n\}\times\{1;\ldots; n\}, i\neq j$
    $$E_{ij} = \left( \sum_{l=1, l \notin {i; j}}^n F_{ll} \right) + F_{ij} + F_{ji} = 
      \begin{pmatrix}
        1 & & & & & & 0 \\
        & 0 & & & & 1 & \\
        & & 1 & & 0 & & \\
        & & & \ddots & & & \\
        & & 0 & & 1 & & \\
        & 1 & & & & 0 & \\
        0 & & & & & & 1
      \end{pmatrix}$$
\end{itemize} 
Les opérations élémentaires que l'un peut effectuer sur les lignes d'une matrice peuvent s'interpréter en termes de multiplications de cette par les matrices précédantes.

\paragraph{Exemple} Pour $n = 2$ on a
$$E_1(-3) = \begin{pmatrix} -3 & 0 \\ 0 & 1 \end{pmatrix} ~~ 
  E_{1 2} = \begin{pmatrix} 0 & 1 \\ 1 &  0 \end{pmatrix} ~~ 
  E_{2 1}(5) = \begin{pmatrix} 1 & 5 \\ 0 & 1 \end{pmatrix}$$
Pour $n = 3$ on a
$$E_2(-4) = \begin{pmatrix} 1 & 0 & 0 \\ 0 & -4 & 0 \\ 0 & 0 & 1 \end{pmatrix} ~~
  E_{2 1} = \begin{pmatrix} 0 & 1 & 0 \\ 1 & 0 & 0 \\ 0 & 0 & 1 \end{pmatrix} ~~
  E_{2 3}(7) = \begin{pmatrix} 1 & 0 & 0 \\ 0 & 1 & 7 \\ 0 & 0 & 1 \end{pmatrix}$$

\paragraph{Théorème} Soit $A$ une matrice de taille $n\times m$
\begin{enumerate}[a)]
  \item Multiplier la $i^{ème}$ ligne de $A$ par un réel $c$ non nul est équvalent à multiplier $A$ à gauche par la matrice $E_i(c)$ de taille $n \times n$
  
  \item Ajouter à la $i^{ème}$ ligne de $A$ $c$-fois la ligne $j$ de $A$ où $i\neq j$ et $c\in \R$ est équivalent à multiplier $A$ à gauche par la matrice $E_{ij}(c)$.
  
  \item Permuter la $i^{ème}$ ligne et la $ĵ^{ème}$ ligne de $A$ où $i\neq j$ est équivalent à multiplier $A$ à gauche par la matrice $E_{ij}$.
\end{enumerate}

\paragraph{Démonstration}
\begin{enumerate}[a)]
  \item $$E_i(c) A = \left( \sum_{l=1, l\neq i}^n F_{ll} \right) A +  c F_{ii} A$$
    Si $l\neq i$ on a
    $$F_{ll} A =
      \begin{pmatrix} 
        \\ l^{ème} \text{ ligne de } A \\ \\
      \end{pmatrix} 
      \begin{array}{c} 
        1, \ldots, l-1 \\ l \\ l+1, \ldots, n 
      \end{array}$$
    sinon, on a
    $$c F_{ii} A =
      \begin{pmatrix}
        \\ c\text{-fois } i^{ème} \text{ ligne de } A \\ \\
      \end{pmatrix} 
      \begin{array}{c} 
        1, \ldots, i-1 \\ i \\ i+1, \ldots, n 
      \end{array}$$
    On voit alors que $\left( \sum_{l=1, l\neq i}^n F_{ll} \right) A +  c F_{ii} A$ est la matrice cherchée.
    
  \item $$E_{ij}(c) A = (I_n + c F_{ij}) A = E_{ij}(c) = A + c F_{ij} A$$
    Où on a
      $$c F_{ij} A =
        \begin{pmatrix}
          \\ c\text{-fois } j^{ème} \text{ ligne de } A \\ \\
        \end{pmatrix} 
        \begin{array}{c} 
          1, \ldots, i-1 \\ i \\ i+1, \ldots, n 
        \end{array}$$
    Donc $A + c F_{ij} A$ est bien la matrice cherchée.
    
  \item $$E_{ij} A = \left( \sum_{l=1, l \notin {i; j}}^n F_{ll} A \right) + F_{ij} A + F_{ji} A$$
    Si $l\neq i$ et $l\neq j$ 
    $$F_{ll} A =
      \begin{pmatrix}
        \\ l^{ème} \text{ ligne de } A \\ \\
      \end{pmatrix} 
      \begin{array}{c} 
        1, \ldots, l-1 \\ l \\ l+1, \ldots, n 
      \end{array}$$
    sinon
      $$F_{ij} A =
      \begin{pmatrix}
        \\ j^{ème} \text{ ligne de } A \\ \\
      \end{pmatrix} 
      \begin{array}{c} 
        1, \ldots, i-1 \\ i \\ i+1, \ldots, n 
      \end{array}$$
      $$F_{ji} A =
      \begin{pmatrix}
        \\ i^{ème} \text{ ligne de } A \\ \\
      \end{pmatrix} 
      \begin{array}{c} 
        1, \ldots, j-1 \\ j \\ j+1, \ldots, n 
      \end{array}$$
    Ainsi $\left( \sum_{l=1, l \notin {i; j}}^{n} F_{ll} A \right) + F_{ij} A + F_{ji} A$ est la matrice cherchée.
\end{enumerate}

\paragraph{Exemple} Soit $A = \begin{pmatrix} 5 & -3 & 0 \\ 4 & 2 & 7 \end{pmatrix} \in M_{2 \times 3}(\R)$. 
\begin{itemize}
  \item On veut multiplier la première ligne par $-7$.
    $$A \xrightarrow {L_1 \leftarrow L_1(-7)} 
    \begin{pmatrix} -35 & 21 & 0 \\ 4 & 2 & 7 \end{pmatrix}$$
    On calcule
    $$E_1(-7) A 
      = \begin{pmatrix} -7 & 0 \\ 0 & 1 \end{pmatrix} \begin{pmatrix} 5 & -3 & 0 \\ 4 & 2 & 7 \end{pmatrix} 
      = \begin{pmatrix} -35 & 21 & 0 \\ 4 & 2 & 7 \end{pmatrix}$$

  \item Ajouter à la première ligne de $A$ $3$ fois la deuxième ligne.
    $$A \xrightarrow{L_1\leftarrow L_1+3L_2} 
      \begin{pmatrix} 17 & 3 & 21 \\ 4 & 2 & 7 \end{pmatrix}$$
    On calcule
    $$E_{1 2}(3) A 
      = \begin{pmatrix} 1 & 3 \\ 0 & 1 \end{pmatrix} \begin{pmatrix} 5 & -3 & 0 \\ 4 & 2 & 7 \end{pmatrix} 
      = \begin{pmatrix} 17 & 3 & 21 \\ 4 & 2 & 7 \end{pmatrix}$$
\end{itemize}

\paragraph{Théorème} Soit $n$ un entier naturel et $n \geq 1$. Alors les matrices élémentaires de taille $n \times n$ sont inversibles, et l'on a:
\begin{enumerate}
  \item $$E_i(C)^{-1} = \frac{1}{C} E_i \text{ avec } C \in \R \backslash \{ 0 \}$$
  \item $$E_{i j}(C)^{-1} = E_{i j}(-C) \text{ avec } C \in \R$$
  \item $$E_{i j}^{-1} = E_{j i}$$
\end{enumerate}

\paragraph{Démonstration}
\begin{enumerate}
  \item On a
    $$E_i(C)E_i(\frac{1}{C}) = 
      \begin{pmatrix}
        1 & 0 & \dots & & 0 \\
        0 & \ddots & & & \vdots \\
        \vdots & & c \\
        & & & \ddots & 0 \\
        0 & \dots & & 0 & 1
      \end{pmatrix} 
      \begin{pmatrix}
        1 & 0 & \dots & & 0 \\
        0 & \ddots & & & \vdots \\
        \vdots & & \frac{1}{c} \\
        & & & \ddot s & 0 \\
        0 & \dots & & 0 & 1
      \end{pmatrix} = 
      \begin{pmatrix}
        1 & 0 & \dots & & 0 \\
        0 & \ddots & & & \vdots \\
        \vdots & & 1 \\
        & & & \ddots & 0 \\
       0 &\dots & & 0 & 1
      \end{pmatrix} = I_n$$
    On vérifie de même que $E_i(\frac{1}{C}) E_i(C) = I_n$.
  
  \item On a
    $$E_{i j}(C) E_{i j}(-C) = (I_n + C F_{i j}) (I_n - C F_{i j}) = I_n + C F_{i j} - C F_{i j} - C^2 F_{i j} F_{i j}$$
    puisque $i\neq j$, on a $F_{i j} F_{i j} = 0$. Donc
    $$E_{i j}(C) E_{i j}(-C) = I_n$$
    On vérifie de même que $E_{i j}(-C) E_{i j}(C) = I_n$.
  
  \item On a
    \begin{eqnarray*}
      E_{i j}E_{j i} 
        &=& \left( \sum_{r = 1, r \notin \{i, j\}}^{n} F_{r r} + F_{i j} + F_{j i}\right) \left( \sum_{s = 1, s \notin \{i, j\}}^{n} F_{s s}+F_{i j}+F_{j i} \right) \\
        &=& (I_n - F_{i i} - F_{j j} + F_{i j} + F_{j i}) (I_n - F_{i i} - F_{j j} + F_{i j} + F_{j i}) \\
        &=& I_n - F_{i i} - F_{j j} + F_{i j} + F_{j i} - F_{j j} + F_{i i} F_{i i} + F_{j j} F_{i i} - F_{i j} F_{i i} - F_{j i} F_{i i} \\
          && - F_{j j} + F_{i i} F_{j j} + F_{j j} F_{j j} - F_{i j} F_{j j} - F_{j i} F_{j j} + F_{i j} - F_{i i} F_{i j} - F_{j j} F_{i j} \\
          && + F_{i j} F_{i j} + F_{j i} F_{i j} + F_{j i} - F_{i i} F_{i i} - F_{j j} F_{j i} + F_{i j} F_{j i} + F_{j i} F_{j i}\\
        &=& I_n - F_{i i} - F_{j j} + F_{i j} + F_{j i} - F_{i i} + F_{i i} - F_{j i} - F_{j j} \\
          && + F_{j j} - F_{i j} + F_{i j} - F_{i j} + F_{j j} + F_{j i} - F_{j i} + F_{i i} \\
        &=& I_n
    \end{eqnarray*}
\end{enumerate}
Le fait que les matrices élémentaires soient inversibles confirme le fait que les opérations élémentaires soient réversibles.

%
%
\section{Écriture matricielle des systèmes linéaires}
%
%
\paragraph{Écriture} Considérons le système linéaire à $n$ équations et $m$ inconnues suivant:
$$\left\{ \begin{array}{rcl}
  a_{1 1} x_{1} + a_{1 2} x_2 + \ldots + a_{1 m} x_m & = & b_1 \\
  a_{2 1} x_{1} + a_{2 2} x_2 + \ldots + a_{2 m} x_m & = & b_2 \\
  \vdots & & \vdots \\
  a_{n 1} x_{1} + a_{n 2} x_2 + \ldots + a_{n m} x_m & = & b_n
\end{array}$$
où les $a_{i j}$ et les $b_{i}$ sont des réels. Soit
$$A = \begin{pmatrix}
  a_{1 1} & a_{1 2} & \dots & a_{1 m} \\
  a_{2 1} & a_{2 2} & \dots & a_{2 m} \\
  \vdots & \vdots & \vdots & \vdots \\
  a_{n 1} & a_{n 2} & \dots & a_{n m}
\end{pmatrix}$$
la matrice du système et soit
$$B = \begin{pmatrix} b_1 \\ b_2 \\ \vdots \\ b_m \end{pmatrix} \text{ et } 
  X = \begin{pmatrix} x_1 \\ x_2 \\ \vdots \\ x_m \end{pmatrix}$$
Alors le système peut s'écrire matriciellement sous la forme
$$A X = B$$

%
\subsection{Nombres de solutions possibles pour un système linéaire}
%
\paragraph{Théorème} Soit $A X = B$ un système linéaire. Alors, on se trouve dans l'un des cas suivants:
\begin{itemize}
  \item le système a aucune solution (système incompatible) 
  \item le système a une solution unique
  \item le système a une infinité de solutions
\end{itemize}

\paragraph{Démonstrations}
\begin{itemize}
  \item Si le système est incomaptible, il n'a pas de solutions.
  \item S'il est compatible, il faut voir que dans le cas où il admet au moins deux solutions détérminés, il admet une infinité de solutions. Supposons que $t = (t_1, t_2, \ldots, t_m)$ et $t' = (t'_1, t'_2, \ldots, t'_m)$ soient deux solutions distinctes de ce système. On a donc
    $$A \begin{pmatrix} t_1 \\ \vdots \\ t_m \end{pmatrix} = B = A \begin{pmatrix} t'_1 \\ \vdots \\ t'_m \end{pmatrix}$$
    Il vient
    $$A \left( \begin{pmatrix} t_1 \\ \vdots \\ t_m \end{pmatrix} - \begin{pmatrix} t'_1 \\ \vdots \\ t'_m \end{pmatrix} \right) = 0$$
    puisque $t \neq t'$, $t - t' \neq 0$. On obtient alors pour tout $\alpha \in \R$
    $$A \left( \begin{pmatrix} t_1 \\ \vdots \\ t_m \end{pmatrix} + \alpha \left( \begin{pmatrix} t_1 \\ \vdots \\ t_m \end{pmatrix} - \begin{pmatrix} t'_1 \\ \vdots \\ t'_m \end{pmatrix} \right) \right) 
      = A \begin{pmatrix} t_1 \\ \vdots \\ t_m \end{pmatrix} + \alpha A \left( \begin{pmatrix} t_1 \\ \vdots \\ t_m \end{pmatrix} - \begin{pmatrix} t'_1 \\ \vdots \\ t'_m \end{pmatrix} \right) 
      = B + 0 = B$$
    donc $(t_1, t_2, \ldots, t_m) + \alpha ((t_1, t_2, \ldots, t_m) - (t'_1, t'_2, \ldots, t'_m))$ est une solution du système. Lorsque $\alpha$ parcourt $\R$ tout entier, on obtien une infinité de solutions.
\end{itemize}

%
%
\section{Calcul de $A^{-1}$ et $A^{-1} B$}
%
%

\paragraph{Théorème} Soit $A$ une matrice carrée de taille $n \times n$. Alors les conditions suivantes sont équivalentes
\begin{enumerate} 
  \item $A$ est inversible,
  \item Pour tout $B \in M_{n, 1}(\R)$, le système linéaire $A X = B$ admet une solutions unique.
  \item $A$ admet $I_n$ pour forme échelonné réduite.
  \item $A$ est produit de matrices élémentaires.
\end{enumerate}

\paragraph{Démonstration}
\begin{itemize}
  \item[$1. \Rightarrow 2.$] On suppose que $A$ est inversible. Soit $B \in M_{n, 1}(\R)$. On a
    $$A (A^{-1} B) = (A A^{-1}) B = I_n B = B$$
    donc $A^{-1} B \in M_{n, 1}(\R)$ est une solution du système $A X = B$. Maintenant, si $T \in M_{n, 1}(\R)$ est une solutions de ce système, on a $A T = B$, d'où $A^{-1}(A T) = A^{-1} B$\footnote{avec $T = (A^{-1} A) T$} ce qui montre que $T$ est unique. Par conséquent, le système $A X = B$ a une unique solution.
    
  \item[$2. \Rightarrow 3.$] Supposons que pour tout $B \in M_{n, 1}(\R)$, le système $A X = B$ admet une unique solutions. Puisque ce système admet une unique solution, toutes les inconnues du système sont principale. En effet, si l'une\footnote{où plusieurs} des inconnues était sécondaire, on se trouverait dans l'un des cas suivants
    \begin{itemize}
      \item le système est incompatible et n'a pas de solution
      \item le système est compatible et a une infinité de solutions
    \end{itemize}
    Comme $A$ est carrée, cela implque que la forme réduite de $A$ est $I_n$.
    
  \item[$3. \Rightarrow 4.$] Supposons que la forme échelonnée réduite de $A$ soit $I_n$. Il existe donc des matrices élémentaires $E_1, E_2, \ldots , E_r$, telles que
    $$I_n = E_r \cdot \ldots \cdot E_2 \cdot E_1 \cdot A$$
    Il vient alors $E_1^{-1} \cdot E_2^{-1} \cdot \ldots \cdot E_r^{-1} \cdot I_n = A$ d'où
    $$A = E_1^{-1} \cdot E_2^{-1} \cdot \ldots \cdot E_r^{-1}$$
    Or les $E_i^{-1}$ sont également des matrices élémentaires. Donc $A$ est bien produit de matrices élémentaires.
    
  \item[$4. \Rightarrow 1.$] On suppose que $A$ est produit de matrices élémentaires $A = E_1^{-1} \cdot E_2^{-1} \cdot \ldots \cdot E_r^{-1}$. Comme les $E_i$ sont inversibles, il en est de même de leur produit, donc de $A$.
\end{itemize}
Ce qui précède fournit une méthode pour calcule l'inverse d'une matrice, si cet inverse existe.

%
\subsection{Calculer l'inverse d'une matrice}
%
\paragraph{Methode de calcul} Soit $A$ matrice carrée de taille $n\times n$. \\
On forme la matrice $(A ~ \vert ~ I_n)$. En utilisant les opérations élémentaires on met la matrice $E$ en forme échelonnée réduite et on éffectue parallelement les mêmes opérations élémentaires sur les lignes de $I_n$. Si $A$ sous la forme échelonnée réduite est égale à $I_n$ $A$ est inversible. La matrice inverse de $A$ est alors la partie $I_n$ où on a éffectue les opérations élémentaires. \\
$E_1, E_2, \ldots, E_l$ sont des matrices élémentaires.
\begin{eqnarray*}
  A \cdot E_1 \cdot E_2 \cdot &\ldots& \cdot E_l = I_n \\
  &\Downarrow& \\
  I_n \cdot E_1 \cdot E_2 \cdot &\ldots& \cdot E_l = A^{-1} \\
  &\Downarrow& \\
  E_1 \cdot E_2 \cdot &\ldots& \cdot E_l = A^{-1}
\end{eqnarray*}
De la même manière on peut calculer $A^{-1} B$.
$$A^{-1} B = E_1 \cdot E_2 \cdot \ldots \cdot E_l \cdot B$$
où $E_1, E_2, \ldots, E_l$ sont les opérations élémentaires dont on à besoin pour mettre $A$ sous forme échelonnée réduite égal à $I_n$.

\paragraph{Exemple} Soient $A = \begin{pmatrix} 1 & 2 & 1 \\ 4 & 0 & -1 \\ -1 & 2 & 2 \end{pmatrix}$ et $B = \begin{pmatrix} 1 & 2 & 0 & 4 \\ -3 & 1 & 2 & 0 \\ 5 & 1 & -1 & 3 \end{pmatrix}$.
On a
$$A^{-1} B 
  = \frac{1}{4} \begin{pmatrix} -2 & 2 & 2 \\ 7 & -3 & -5 \\ -8 & 4 & 8 \end{pmatrix} 
    \begin{pmatrix} 1 & 2 & 0 & 4 \\ -3 & 1 & 2 & 0 \\ 5 & 1 & -1 & 3 \end{pmatrix} 
  = \frac{1}{4} \begin{pmatrix} 2 & 0 & 2 & -2 \\ -9 & 6 & -1 & 13 \\ 20 & -4 & 0 & -8 \end{pmatrix}$$
En applicant la méthode décrite, on obtient:
\begin{eqnarray*}
  \left( \begin{array}{ccc|cccc}
    1 & 2 & 1 &    1 & 2 & 0 & 4 \\ 
    4 & 0 & -1 &   -9 & 6 & -1 & 13 \\
    -8 & 4 & 8 &   20 & -4 & 0 & -8
  \end{array}\right)
  &\xrightarrow[L_3 \leftarrow L_3 + L_1]{L_2 \leftarrow L_2 - 4 L_1}&
  \left( \begin{array}{ccc|cccc}
    1 & 2 & 1 &     1 & 2 & 0 & 4 \\ 
    0 & -8 & -5 &   -7 & -7 & 2 & -16 \\
    0 & 4 & 3 &     6 & 3 & -1 & 7
  \end{array}\right) \\
%
  \xrightarrow{L_3 \leftarrow \frac{1}{4} L_3}
  \left( \begin{array}{ccc|cccc}
    1 & 2 & 1 &             1 & 2 & 0 & 4 \\ 
    0 & -8 & -5 &           -7 & -7 & 2 & -16 \\
    0 & 1 & \frac{3}{4} &   \frac{3}{2} & \frac{3}{4} & \frac{-1}{4} & \frac{7}{4}
  \end{array}\right)
  &\xrightarrow[L_3 \leftarrow L_3 + 8 L_2]{L_2 \leftrightarrow L_3}&
  \left( \begin{array}{ccc|cccc}
    1 & 2 & 1 &             1 & 2 & 0 & 4 \\ 
    0 & 1 & \frac{3}{4} &   \frac{3}{2} & \frac{3}{4} & \frac{-1}{4} & \frac{7}{4} \\
    0 & 0 & 1 &             5 & -1 & 0 & -2
  \end{array}\right) \\
%
  \xrightarrow[L_1 \leftarrow L_1 - L_3]{L_2 \leftarrow L_2 - \frac{3}{4} L_3}
  \left( \begin{array}{ccc|cccc}
    1 & 2 & 0 &   -4 & 3 & 0 & 6 \\ 
    0 & 1 & 0 &   \frac{-9}{4} & \frac{3}{2} & \frac{-1}{4} & \frac{13}{4} \\
    0 & 0 & 1 &   5 & -1 & 0  & -2
  \end{array}\right)
  &\xrightarrow{L_1 \leftarrow L_1 - 2L_2}&
  \left( \begin{array}{ccc|cccc}
    1 & 0 & 0 &  	\frac{1}{2} & 0 & \frac{1}{2} & \frac{-1}{2} \\ 
    0 & 1 & 0 &   \frac{-9}{4} & \frac{3}{2} & \frac{-1}{4} & \frac{13}{4} \\
    0 & 0 & 1 &   5 & -1 & 0 & -2
  \end{array}\right)
\end{eqnarray*}
On a donc
$$A^{-1} B = \frac{1}{4} \begin{pmatrix} 
  2 & 0 & 2 & -2 \\ 
  -9 & 6 & -1 & 13 \\ 
  20 & -4 & 0 & -8 
\end{pmatrix}$$


%
%
\section{Matrices triangulaires, matrice diagonales}
%
%
\paragraph{} Soient $n \geq 1$ un entier naturel et $A_{i j}$ une matrice carrée de taille $n\times n$.

%
\subsection{Matrice triangulaire supérieure}
%
\paragraph{Définition} On dit que $A$ est triangulaire supérieure si les coefficients de $A$ en-dessous de la diagonale sont nuls, autrement dit si $i > j \Rightarrow a_{ij} = 0$

%
\subsection{Matrice triangulaire inférieure}
%
\paragraph{Définition} On dit que $A$ est triangulaire inférieure si tous les coefficients de $A$ au-dessus de la diagonale sont nuls, autrement dit si $i < j \Rightarrow a_{ij} = 0$

%
\subsection{Matrice diagonale}
%
\paragraph{Définition} On dit que $A$ est diagonale si les coefficients de $A$ en dehors de la diagonale sont nuls, autrement dit si $i\neq j \Rightarrow a_{ij} = 0$

\paragraph{Théorème} Soit $A$ une matrice triangulaire supérieure de taille $n\times n$. Alors $A$ est inversible si et seulement si tous ses coefficients diagonaux sont nun nuls. Si $A$ est inversible, alors $A^{-1}$ est encore une matrice triangulaire supérieure.
\\\\
On applique le même pour les matrices triangulaires inférieures et les matrices diagonaux.

\paragraph{Démonstration} Comme $A$ est triangulaire supérieure, elle est échelonnée. Alors $A$ est inversible si et seuelement si sa form échelonnée réduit est $I_n$, si et seulement si chaque coefficient diagonal de $A$ donne un coefficicent pivot, autrement di si et seuelement si chacun de ces coefficients diagonaux est non nul. \\
Montrons par récurence sur $n \geq 1$ que l'inverse d'une matrice triangulaire supérieure inversible de taille $n\times n$ est encore triangulaire supérieure.
\begin{enumerate}[(1)]
  \item Si $n=1$, alors $A$ est de forme $A = ( a )$, pour un certain $a \in \R$. Puisque $A$ est inversible, $a \neq 0$ donc $a$ est inversible dans $\R$. On a alors 
    $$(a) (a^{-1}) = I_n = (a^{-1})(a)$$
    et donc $A^{-1} = (a^{-1})$ est triangulaire supérieure.
    
  \item On suppose que l'inverse de toute matrice triangulaire supérieure inversible de taille $(n-1) \times (n-1)$ est encore triangulaire supérieure.
  
  \item Soit $A$ une matrice triangulaire supérieure inversible de taille $n\times n$. Présentons $A$ par blocs de la façon suivante
    $$A = \begin{pmatrix}
      a &   & L &   \\
        & * & * & * \\
      0 & 0 & T & * \\
        & 0 & 0 & * \\
    \end{pmatrix}$$
    où $a \in \R$, $L$ est une matrice ligne de taille $1 \times (n-1)$ et $T$ est une matrice triangulaire supérieure de taille $(n-1)\times(n-1)$. Du fait que $A$ est inversible, tous ses coefficients diagonaux sont non nuls. Cela implique que $a \neq 0$ et que les coefficients diagonaux de $T$ sont non nuls, donc $T$ est inversible. Par l'hypothèse de récurence, l'inverse $T^{-1}$ de $T$ est une matrice triangulaire supérieure. \\
    Calculons de façon générale le produit de deux matrices triangulaires supérieure décomposées par blocs:
    $$\begin{pmatrix}
        a' &   & L' & ~ \\
          &   &   &  ~  \\
        0 &   & T' & ~  \\
          &   &   &  ~ \\
      \end{pmatrix}
      \begin{pmatrix}
        a &   & L &  ~ \\
          &   &   &  ~ \\
        0 &   & T &  ~ \\
          &   &   &  ~ \\
      \end{pmatrix}
      =\begin{pmatrix}
        a' a &   & a' L + L' T & ~ \\
          &   &   & ~ \\
        0 &   & T' T & ~ \\
          &   &   & ~ \\
      \end{pmatrix}$$
    On veut que $T' T = I_{n-1}$ et que $a' L + L' T = 0$; posons $a' = a^{-1}$, $T' = T^-{1}$ et $L' = - a^{-1} L T^{-1}$. Il vient
    \begin{eqnarray*}
      \begin{pmatrix}
        a^{-1} &   & -a^{-1} L T^{-1} & ~ \\
          &   &   &  ~  \\
        0 &   & T^{-1} & ~  \\
          &   &   &  ~ \\
      \end{pmatrix} &\cdot&
      \begin{pmatrix}
        a &   & L &  ~ \\
          &   &   &  ~ \\
        0 &   & T &  ~ \\
          &   &   &  ~ \\
      \end{pmatrix} \\
      &=&\begin{pmatrix}
        a^{-1} a &   & a^{-1} L - a^{-1} L T^{-1} T & ~ \\
          &   &   & ~ \\
        0 &   & T^{-1} T & ~ \\
          &   &   & ~ \\
      \end{pmatrix}  \\
      &=&\begin{pmatrix}
        1 &   & 0 & ~ \\
          &   &   & ~ \\
        0 &   & I_{n-1} & ~ \\
          &   &   & ~ \\
      \end{pmatrix}
      = I_n
    \end{eqnarray*}
    On a également
    \begin{eqnarray*}
      \begin{pmatrix}
        a &   & L &  ~ \\
          &   &   &  ~ \\
        0 &   & T &  ~ \\
          &   &   &  ~ \\
      \end{pmatrix} &\cdot&
      \begin{pmatrix}
        a^{-1} &   & a^{-1} L T^{-1} & ~ \\
          &   &   &  ~  \\
        0 &   & T^{-1} & ~  \\
          &   &   &  ~ \\
      \end{pmatrix}\\
      &=&\begin{pmatrix}
        a a^{-1} &   & - a^{-1} L T^{-1} T + a^{-1} L & ~ \\
          &   &   & ~ \\
        0 &   & T T^{-1} & ~ \\
          &   &   & ~ \\
      \end{pmatrix}  \\ 
      &=&\begin{pmatrix}
        1 &   & 0 & ~ \\
          &   &   & ~ \\
        0 &   & I_{n-1} & ~ \\
          &   &   & ~ \\
      \end{pmatrix}
      = I_n
    \end{eqnarray*}
    donc
    $$A^{-1} = 
      \begin{pmatrix}
        a^{-1} &   & a^{-1} L T^{-1} & ~ \\
          &   &   &  ~  \\
        0 &   & T^{-1} & ~  \\
          &   &   &  ~ \\
      \end{pmatrix}$$
    est triangulaire supérieure. Par principe de récurence, la propriété est vrai pour tout $n \geq 1$
\end{enumerate} 
    

%
%
\section{Transposition}
%
%
%
\subsection{Matrice transposée}
%
\paragraph{Définition} Soit $A$ une matrice de taille $n\times m$. On appelle transposée de $A$ la matrice de taille $m\times n$ notée $A^T$, définie par
\begin{itemize}
  \item la colonne $i$ de $A$ devient la ligne $i$ de $A^T$
  \item la ligne $i$ de $A$ deivent la colonne $i$ de $A^T$
\end{itemize}

\paragraph{Exemple}
$$\text{Si } A = \begin{pmatrix} 1 & 3 & 0 \\ 2 & -7 & 4 \end{pmatrix} 
  \text{, alors } A^{-1} = \begin{pmatrix} 1 & 2 \\ 3 & -7 \\ 0 & 4 \end{pmatrix}$$
$$\text{Si } B = \begin{pmatrix} 4 & 1 & 7 \\ 0 & -5 & 3 \\ 1 & 2 & 6 \end{pmatrix} 
  \text{, alors } B^{-1} = \begin{pmatrix} 4 & 0 & 1 \\ 1 & -5 & 2 \\ 7 & 3 & 6 \end{pmatrix}$$

\paragraph*{Proposition}
\begin{enumerate}[1)]
  \item $\forall A, B \in M_{n\times m}(\R)$ on a
    $$(A + B)^{T} = A^{T} + B^{T}$$
  \item $\forall A \in M_{n\times m}(\R), \lambda \in \R$ on a
    $$(\lambda A)^{T} = \lambda A^{T}$$
  \item $\forall A \in M_{n\times m}(\R), B \in M_{m\times n}(\R)$ on a
    $$(A \cdot B)^{T} = B^{T} \codt A^{T}$$
  \item $\forall A \in M_{n\times m}(\R)$ on a
    $$(A^{T})^{T} = A$$
  \item $\forall A \in M_{n}(\R)$ si $A$ est inversible, alors $A^{T}$ est inversible et 
    $$(A^{-1})^{T} = (A^{T})^{-1}$$
\end{enumerate}

%
%
\section{Matrices symétriques, matrices antisymétriques}
%
%
%
\subsection{Matrice symétrique}
%
\paragraph{Définition} Soit $A$ une matrice carrée de taille $n\times n$. On dit que $A$ est symétrique si $A^{T} = A$.

\paragraph{Exemple}$$\begin{pmatrix}1&-7\\-7&1\end{pmatrix} \text{ et }\begin{pmatrix}0&3&5\\3&1&6\\5&6&2\end{pmatrix}\text{ sont des matrices symétriques.}$$

%
\subsection{Matrice antisymétrique}
%
\paragraph{Définition} Soit $A$ une matrice carrée de taille $n\times n$. On dit que $A$ est antisymétrique si $A^{T} = -A$
\paragraph{Remarque} Si $A$ est antisymétrique $a_{ij}$ pour $i=j$ sont nuls.

\paragraph{Exemple}$$\begin{pmatrix}0&1\\-1&0\end{pmatrix} \text{ et }\begin{pmatrix}0&-7&-3\\7&0&2\\3&-2&0\end{pmatrix}\text{ sont des matrices antisymétriques.}$$

\paragraph{Théorème} Soit $A$ une matrice carrée. Alors $A$ peut s'écrire comme la somme d'une matrice symétrique et d'une matrice antisymétrique.

\paragraph{Démonstration}  On pose $C = \frac{1}{2} ( A + A^{T})$ et $D = \frac{1}{2} ( A - A^{T})$ et alors on a
$$ C + D = \frac{1}{2} ( A + A^{T}) + \frac{1}{2} ( A - A^{T}) = A$$
On a 
\begin{eqnarray*} 
  C^{T} = (\frac{1}{2} ( A + A^{T}))^{T} &=& \frac{1}{2} ( A^{T} + A) \\
    &=& \frac{1}{2} ( A + A^{T}) = C
\end{eqnarray*}
donc $C$ est symétrique. On a également
\begin{eqnarray*}
  D^{T} = (\frac{1}{2} ( A - A^{T}))^{T} &=& \frac{1}{2} ( A^{T} - A) \\
    &=& -\frac{1}{2} ( A - A^{T}) = -C
\end{eqnarray*}
donc $D$ est antisymétrique. Ainsi $A = C + D$ est une composition de la forme cherché.

%
%
\section{La trace}
%
%
%
\subsection{Trace d'une matrice}
%
\paragraph{Définition} Soit $A$ une matrice carrée de taille $n\times n$. On appelle trace de $A$ le nombre réel noté $tr(a)$ défini par
$$tr(a) = a_{11} + a_{22} + \ldots + a_{nn}$$
La trace de $A$ est donc la somme des coefficients diagonaux de $A$.

\paragraph{Exemple} 
$$A = \begin{pmatrix} 1 & 4 \\ 3 & -2 \end{pmatrix} \rightarrow tr(A) = -2 + 1 = -1$$
$$B = \begin{pmatrix} 7 & 3 & 4 \\ 2 & 0 & -6 \\ -3 & 1 & -2 \end{pmatrix} \rightarrow tr(B) = 7 + 0 - 2 = 5$$

\paragraph{Proposition}
\begin{enumerate}[1)]
  \item Soient $A, B \in M_{n}(\R)$. Alors
    $$tr(A + B) = tr(A) + tr(B)$$
  \item Soient $A \in M_{n}(\R)$ et $\lambda \in \R$. Alors
    $$tr(\lambda A) = \lambda \cdot tr(A)$$
  \item Soient $A, B \in M_{n}(\R)$. Alors
    $$tr(A \cdot B) = tr(B \cdot A)$$
  \item Soient $A \in M_{n}(\R)$ . Alors
    $$tr(A^{T}) = tr(A)$$
\end{enumerate}

\paragraph{Démonstration}
\begin{enumerate}[1)]
  \item 
    \begin{eqnarray*}
      tr(A + B) = \sum_{k=1}^{n}(A + B)_{kk} &=& \sum_{k=1}^{n}(A)_{kk} + (B)_{kk} \\
        &=& \sum_{k=1}^{n}(A)_{kk} + \sum_{k=1}^{n}(B)_{kk} \\
        &=& tr(A) + tr(B)
    \end{eqnarray*}
    
  \item
    \begin{eqnarray*}
      tr(\lambda A) = \sum_{k=1}^{n}(\lambda A)_{kk} &=& \sum_{k=1}^{n} \lambda \cdot (A)_{kk} \\
        &=& \lambda \sum_{k=1}^{n}(A)_{kk} \\
        &=& \lambda \cdot tr(A)
    \end{eqnarray*}
  
  \item 
    \begin{eqnarray*}
      tr(A \cdot B) = \sum_{k=1}^{n} (A B)_{kk} &=& \sum_{k=1}^{n} \left( \sum_{l=1}^{n} A_{kl} B_{lk} \right) \\
        &=& \sum_{l=1}^{n} \left( \sum_{k=1}^{n} A_{kl} B_{lk} \right) \\
        &=& \sum_{l=1}^{n} \left( \sum_{k=1}^{n} B_{lk} A_{kl} \right) \\
        &=& \sum_{l=1}^{n} (B A)_{ll} = tr(B \cdot A)
    \end{eqnarray*}
    
  \item
    $$tr(A^{T}) = \sum_{k=1}^{n} (A^{T})_{kk} = \sum_{k=1}^{n} (A)_{kk} = tr(A)$$
\end{enumerate}


\chapter{Le déterminant}

%
%
\section{Groupes symétriques}
%
%
%
\subsection{Permutation}
%
\paragraph{Définition} Soit $E$ un ensemble. On appelle permutation de $E$ un bijection  de $E$ dans $D$. On note $Sym(e)$ l'ensemble de permutations de $E$

\paragraph{Définition} On munit $Sym(E)$ de la composition
\begin{eqnarray*}
  \circ: Sym(E) \times Sym(E) &\rightarrow& Sym(E) \\
  (\sigma , \tau) &\mapsto& \sigma \circ \tau
\end{eqnarray*}
Par définition, pour tout $x\in E$
$$(\sigma \circ \tau)(x) = \sigma(\tau(x))$$
La composée de deux biijections étant encore une bijection, la composée de deux permutations de $E$ est encore une permutation de $E$.

%
\subsection{Groupe symétrique}
%
Alors $(Sym(E), \circ)$ est un groupe. En effet
\begin{itemize}
  \item la composition des permutations est associative
    $$\alpha \circ (\beta \circ \gamma) = (\alpha \circ \beta) \circ \gamma ~ \forall \alpha, \beta, \gamma \in Sym(E)$$
  
  \item l'application
    \begin{eqnarray*}
      id_E: E &\rightarrow& E \\
      x &\mapsto& id_E(x) = x 
    \end{eqnarray*}
    est une permutation de $E$ qui est élément neuter pour $circ$
    
  \item si $\sigma \in Sym(E)$, alors la permutation inverse de $\sigma$ est la bijection réciproque $\sigma^{-1}: E \rightarrow E$ de $\sigma$
\end{itemize}
On appelle ce groupe le groupe symétrique de $E$.

\paragraph{Remarque} Dans la suite nous nous concentrons sur le cas où $E = \{1; 2; \ldots; n\}$ pour un certain entier naturel $n\geq 1$. On note alors
$$S_n = Sym(\{1; 2; \ldots; n\})$$
Si $\sigma \in S_n$, on pourra écrire
$$\sigma = \begin{pmatrix} 1 & 2 & \ldots & n \\ \sigma(1) & \sigma(2) & \ldots & \sigma(n) \end{pmatrix}$$

%
\subsection{Support}
%
\paragraph{Définition} Pour tout $\sigma \in S_n$, on appelle support de $\sigma$ et on note $supp(\sigma)$ l'ensemble
$$supp(\sigma) = \{i \in \{1; 2; \ldots; n\} \vert \sigma(i) \neq i\}$$

%
\subsection{Cycle}
%
\paragraph{Définition} Soit $n \geq 1$ un entier naturel et soit $k \in \{1; 2; \ldots; n\}$. On dit qu'une permutation $\sigma \in S_n$ est un cycle de longeur $k$ ou $k$-cycle, s'il existe $i_1, i_2, \ldots, i_k \in  \{1; 2; \ldots; n\}$ tels que
$$\sigma(i_1) = i_2 \rightarrow \sigma(i_2) = i_3 \rightarrow \ldots \rightarrow \sigma(i_{k-1}) = i_k \rightarrow \sigma(i_k) = i_1$$
et $\sigma(i) = i$ si $i \notin \{i_1; i_2; \ldots; i_k\}$

\paragraph{Remarque} L'ensemble $\{i_1; i_2; \ldots; i_k\}$ est le support du $k$-cycle.

\paragraph{Notation} Soit $\sigma$ un $k$-cycle, on note $\sigma = (i_1 ~ i_2 ~ \ldots ~ i_k)$


%
\subsection{Transposition}
%
\paragraph{Définition} Les 2-cycles de $S_n$ s'appellent les transpositions de $S_n$

\paragraph{Proposition} Soit $m \geq 2$ un entier naturel. Alors toute permutation de $S_n$ peut sécrire comme composée de transpositions.

\paragraph{Démonstration}  On raisonne par récurence sur le cardinal\footnote{Le cardinal d'un ensemble fini est le nombre de ses éléments.} de $supp(\sigma)$.
\subparagraph{Annonce de la récurence} Soit $\sigma \in S_n$ telle que $supp(\sigma)$ soit de cardinal égal à $0$. Donc $supp(\sigma)$ est vide (on note $supp(\sigma) = \emptyset$). Autrement dit, $\sigma$ fixe tous les éléments de $\{1; 2; \ldots; n\}$. Par conséquent, $\sigma = id$. \\
On peut écrire $\sigma = id = (1 ~ 2)\circ(1 ~ 2)$, ce qui est une écriture de $\sigma$ comme composée de transpositions.

\subparagraph{Hypothése de récurence} Soit $k \in \{1; 2; \ldots; n\}$. On suppose que pour tout $\tau \in S_n$, telle que $card(supp(\tau)) \leq k-1$, $\tau$ s'écrit comme composée de transposition.

\subparagraph{Pas de récurence} Soit $\sigma \in S_n$ telle que le cardinal de $supp(\sigma)$ soit égal à $k$.
$$supp(\sigma) = \{i_1; i_2; \ldots; i_k\} \text{ avec } 1 \leq i_1 \leq i_2 \leq \ldots \leq i_k \leq n$$
On forme la permutation $\tau = (i_k ~ \sigma(i_k)) \circ \sigma$. La permutation $\tau$ a un support de cardinal $< k$. En effet, on verifie que $supp(\tau)$ est contenue dans $supp(\sigma)$. Or on a
\paragraph{[TO CHECK]}
$$\tau(i_k) = (i_k ~ \sigma(i_k)) \circ \sigma) (i_k) = (i_k ~ \sigma(i_k))(\sigma(i_k)) = i_k$$
Donc $supp(\tau)$ est contenue dans $\{i_1; i_2; \ldots; i_k-1\}$, donc est de cardinal $\leq k-1 < k$. Par hypothèse de récurence, $\tau$ s'écrit comme composée de transpositions
$$\tau = \tau_1 \circ \tau_2 \circ \ldots \circ \tau_n$$
On en déduit que 
\begin{eqnarray*}
  \sigma &=& (i_k ~ \sigma(i_k))^{-1} \circ \tau \\
    &=& (i_k ~ \sigma(i_k)) \circ \tau_1 \circ \tau_2 \circ \ldots \circ \tau_n
\end{eqnarray*}
\\
Par principe de récurence, la propriété est vraie pour toute permutation de $S_n$.

\paragraph{Proposition} Soit $n \geq 1$ un entier naturel. Alors $S_n$ est de cardinal $n! = 1 \cdot 2 \codt \ldots \cdot n$.

\paragraph{Démonstration} Pour construire une permutation $\sigma$ de $S_n$,
\begin{itemize}
  \item on choisit l'image $\sigma(1)$ de $1$ pour $\sigma$, il y a $n$ choix possibles.
  \item puis on choisit $\sigma(2)$ qui est l'image de $2$ par $\sigma$, il y a $n-1$ choix possile dans $\{1; 2; \ldots; n\} \backslash \{\sigma(1)\}$.
  \item puis on choisit $\sigma(3)$; il y a $n-1$ choix possibles dans $\{1; 2; \ldots; n\} \backslash \{\sigma(1); \sigma(2)\}$.
  \item ainsi de suite
\end{itemize}
Par conséquent, le nombre de permutations que l'on peut construire est 
$$n \cdot (n-1) \cdot \ldots \cdot 2 \cdot 1 = n!$$

\paragraph{Exemple} Décrivons $S_2$ et $S_3$
\begin{itemize}
  \item $S_2$ est de cardinal $2! = 2$. On a 
    $$S_2 = \{id; (1 ~ 2)\}$$
  \item $S_3$ est de cardinal $3! = 6$. On a
    $$S_3 = \{id; (1 ~ 2); (1 ~ 3); (2 ~ 3); (1 ~ 2 ~ 3); (1 ~ 3 ~ 2)\}$$
\end{itemize}

%
\subsection{Inversion de paire}
%
\paragraph{Définition} Soient $i, j \in \{1; 2; \ldots; n\}$, $i < j$, et $\sigma \in S_n$. On dit que $\sigma$ présente une inversion en la paire $(i, j)$ si 
$$\sigma(i) > \sigma(j)$$

%
\subsection{Permutation paire}
%
\paragraph{Définition} Soit $\sigma \in S_n$. On dit que $\sigma$ est une permutation paire si elle présente un nombre paire d'inversions, et une permutation impaire sinon.

%
\subsection{Signature}
%
\paragraph{Définition} On appelle signature de $\sigma$, et on note $\epsilon(\sigma)$ le nombre
$$\epsilon(\sigma) \left\{ \begin{array}{lr} 1 & \text{si } \sigma \text{ est une permutation paire} \\ -1 & \text{si } \sigma \text{ est une permutation impaire} \end{array}$$
Autrement dit, si $n_{\sigma}$ est le nombre d'inversions de $\sigma$ on a
$$\epsilon(\sigma) = (-1)^{n_{\sigma}} \in \{-1; 1\}$$

\paragraph{Exemple}
$$\sigma = \begin{pmatrix} 1 & 2 & 3 & 4 & 5 \\ 3 & 1 & 5 & 2 & 4 \end{pmatrix} \in S_5$$
Paires d'inversions
$$(1, 2), (1, 4), (3, 4), (3, 5)$$
$\sigma$ presente $4$ inversion, $\sigma$ est donc paire et $\epsilon(\sigma) = 1$

\paragraph{Théorème} Soient $\sigma, \tau \in S_n$. Alors
$$\epsilon(\sigma \circ \tau) = \epsilon(\sigma) \cdot \epsilon(\tau)$$

%
%
\section{Déterminant}
%
%

%
\subsection{Produit élémentaire}
%
\paragraph{Définition} Soit $A = (a_{ij})$ une matrice carrée de taille $n\times n$. Soit $\sigma in S_n$. On appelle produit élémentaire associé à $\sigma$ le nombre
$$a_{\sigma(1) 1} \cdot a_{\sigma(2) 2} \cdot \ldots \cdot a_{\sigma(n) n} = \prod_{i=1}^{n} a_{\sigma(i) i}$$
%
\subsection{Produit signé}
%
\paragraph{Définition} On appelle produit signé associé à $\sigma$ le nombre 
$$p_{\sigma}(A) = \epsilon(\sigma) \cdot a_{\sigma(1) 1} \cdot a_{\sigma(2) 2} \cdot \ldots \cdot a_{\sigma(n) n} = \epsilon(\sigma) \prod_{i=1}^{n} a_{\sigma(i) i}$$

%
\subsection{Déterminant}
%
\paragraph{Définition} Soit $A = (a_{ij})_{1 \leq i, j \leq n}$ une matrice carrée de taille $n \times n$. On appelle déterminant de $A$ le nombre
$$det(A) = \sum_{\sigma \in S_n} p_{\sigma}(A) = \sum_{\sigma \in S_n} \epsilon(\sigma) \prod_{i=1}^{n} a_{\sigma(i) i}$$

\paragraph{Exemple} Soit $A = \begin{pmatrix} 1 & 2 & 3 \\ 0 & 5 & 1 \\ -7 & 4 & -2 \end{pmatrix} \in M_{3}(\R)$.
$$S_3 = \{id; (1 ~ 2); (1 ~ 3); (2 ~ 3); (1 ~ 2 ~ 3); (1 ~ 3 ~ 2)\}$$
On a 
\begin{eqnarray*}
  \sigma = id           &\Rightarrow& p_{\sigma}(A) = 1 \cdot A_{11} \cdot A_{22} \cdot A_{33} \\
  \sigma = (1 ~ 2)      &\Rightarrow& p_{\sigma}(A) = 1 \cdot A_{21} \cdot A_{21} \cdot A_{33} \\
  \sigma = (1 ~ 3)      &\Rightarrow& p_{\sigma}(A) = 1 \cdot A_{31} \cdot A_{22} \cdot A_{13} \\
  \sigma = (2 ~ 3)      &\Rightarrow& p_{\sigma}(A) = 1 \cdot A_{11} \cdot A_{32} \cdot A_{23} \\
  \sigma = (1 ~ 2 ~ 3)  &\Rightarrow& p_{\sigma}(A) = 1 \cdot A_{21} \cdot A_{32} \cdot A_{13} \\
  \sigma = (1 ~ 3 ~ 2)  &\Rightarrow& p_{\sigma}(A) = 1 \cdot A_{31} \cdot A_{12} \cdot A_{23}
\end{eqnarray*}

\paragraph{Notation} Si $A = (a_{ij})_{1 \leq i, j\leq n}$ est une matrice carrée de taille $n\times n$ on note
$$\vert a_{ij} \vert _{1 \leq i, j \leq n} = det(A)$$

%
\subsection{Déterminant d'un système des matrices colonnes}
%
\paragraph{Définition} On peut aussi parler du déterminant d'un système de $n$ matrices colonnes $m$-lignes
$$C_1, C_2, \ldots, C_n  \in M_{n\times 1}(\R)$$
Suivant
$$C_{j} = \begin{pmatrix} c_{1j} \\ c_{2j} \\ \vdots \\ c_{nj} \end{pmatrix} \text{ pour tout } j \in \{1; \ldots ; n\}$$
on définit le déterminant du système $(C_1, C_2, \ldots, C_n)$ par
$$det(C_1, C_2, \ldots, C_n) = \sum_{\sigma \in S_n} \epsilon(\sigma) \cdot c_{\sigma(1) 1} \cdot \ldots \cdot c_{\sigma(n) n} \in \R$$
Si $C$ est la matrice de taille $n \times n$ dont les colonnes sont les $C_j$, $C = (C_1, C_2, \ldots, C_n)$ alors on a
$$det(C_1, C_2, \ldots, C_n) = det(C)$$
\paragraph{Remarque} Le point de vue adopté, matrice ou système de matrices colonnes sera clair d'après la notation et le contexte.

\paragraph{Théorème} Soit $n\geq 1$ un entier naturel. Alors l'application $n$ fois
\begin{eqnarray*}
  det: M_{n\times 1}(\R) \times M_{n\times 1}(\R) \times \ldots \times M_{n\times 1}(\R) &\rightarrow& \R \\
  (C_1, C_2, \ldots, C_n) &\mapsto& det(C_1, C_2, \ldots, C_n)
\end{eqnarray*}
posséde les propriétés suivantes
\begin{enumerate}
  \item Pour tout $j \in \{1; \ldots; n \}$, pour tous $C_1, \ldots, C_j, C_j', \ldots, C_n \in M_{n\times 1}(\R)$
    \begin{eqnarray*}
      det(C_1, \ldots, C_j + C_j', \ldots, C_n) =& det(C_1, \ldots, C_j, \ldots, C_n) \\
        &+ det(C_1, \ldots, C_j', \ldots, C_n)
    \end{eqnarray*}
    
  \item Pour tout $j \in \{1; \ldots; n\}$, pour tous $C_1, C_2, \ldots, C_n \in M_{n\times 1}(\R)$, pour tout $\lambda \in \R$
    $$det(C_1, \ldots, \lambda C_j, \ldots, C_n) = \lambda \cdot det(C_1, \ldots, C_j, \ldots, C_n)$$
    
  \item Pour tous $C_1, C_2, \ldots, C_n \in M_{n\times 1}(\R)$, pour tous $i, j \in \{1; \ldots; n\}, i \neq j$, si $C_i = C_j$, alors
    $$det(C_1, \ldots, C_i, \ldots, C_j(=C_i), \ldots, C_n) = 0$$
\end{enumerate}

\paragraph{Démonstration} 
\begin{enumerate}
  \item On a
    $$det(C_1, \ldots, C_j + C_j', \ldots, C_n) = \sum_{\sigma \in S_n} p_\sigma(C_1, \ldots, C_j + C_j' , \ldots, C_n)$$
    où
    $$ p_{\sigma}(C_1, \ldots, C_j + C_j', \ldots, C_n) = \epsilon(\sigma) \cdot c_{\sigma(1) 1} \cdot \ldots \cdot (c_{\sigma(j) j} + c_{\sigma(j) j}') \cdot \ldots \cdot c_{\sigma(n) n}$$
    $\Rightarrow$ distributivité de $\circ$ par rapport à $+$
   
  \item On a
    $$det(C_1, \ldots, \lambda C_j', \ldots, C_n) = \sum_{\sigma \in S_n} p_\sigma(C_1, \ldots, \lambda C_j' , \ldots, C_n)$$
    où
    $$ p_{\sigma}(C_1, \ldots, \lambda C_j', \ldots, C_n) = \epsilon(\sigma) \cdot c_{\sigma(1) 1} \cdot \ldots \cdot \lambda \cdot c_{\sigma(j) j}  \cdot \ldots \cdot c_{\sigma(n) n}$$

  \item Soit $\tau = (i ~ j) \in S_n$. On regroupe les permutations de $S_n$ par paires comme suit: Chaque $\sigma \in S_n$ avec $\sigma \circ \tau \in S_n$ \\
    On a: si $\sigma = \sigma \circ \tau$ on aurait
    \begin{eqnarray*}
      id = \sigma^{-1} \circ \sigma &=& \sigma^{-1} \circ (\sigma \circ \tau) \\
        &=& (\sigma^{-1} \circ \sigma) \circ \tau \\
        &=& id \circ \tau = \tau \rightarrow\text{ impossibru}
    \end{eqnarray*}
    donc $\sigma \neq \sigma \circ \tau$. On a
    \begin{eqnarray*}
      (\sigma \circ \tau) \circ \tau &=& \sigma \circ (\tau \circ \tau) \\
        &=& \sigma \circ (id) \\
        &=& \sigma 
    \end{eqnarray*}
    %
    On obtien aisni $\frac{n!}{2}$ paires de la forme $\{\sigma; \sigma  \circ \tau\}$. Dans chaque de ces paires on sélectionne une permutation $\sigma_n$ tout, on a donc sélectioné $\frac{n!}{2}$ permutations $\sigma_1, \sigma_2, \ldots, \sigma_{\frac{n!}{2}}$. On a donc
    $$S_n = \{ \sigma_1; \sigma_1 \circ \tau; \sigma_2; \sigma_2 \circ \tau; \ldots; \sigma_{\frac{n!}{2}}; \sigma_{\frac{n!}{2}} \circ \tau \}$$
    On obtient 
    \begin{eqnarray*}
      det(C_1, C_2, \ldots, C_n) &=& \sum_{\sigma \in S_n} p_{\sigma}(C_1, C_2, \ldots, C_n) \\
       &=& \sum_{i=1}^{\frac{n!}{2}} p_{\sigma_i}(C_1, C_2, \ldots, C_n) + p_{\sigma_i \circ \tau}(C_1, C_2, \ldots, C_n)
    \end{eqnarray*}
    %
    Montrons que pour tout $\sigma \in S_n$, on a
    $$p_{\sigma \circ \tau}(C_1, C_2, \ldots, C_n) = -p_{\sigma}(C_1, C_2, \ldots, C_n)$$
    Soit $\sigma \in S_n$. On a 
    $$p_{\sigma \circ \tau}(C_1, C_2, \ldots, C_n) = \epsilon(\sigma \circ \tau) \cdot a_{(\sigma \circ\tau)(1) 1} \cdot a_{(\sigma \circ\tau)(2) 2} \cdot \ldots \cdot a_{(\sigma \circ\tau)(n) n}$$
    Soit $k \in \{1; \ldots; n\}$
    \begin{itemize}
      \item si $k\notin \{i; j\}$, alors $\tau(k) = k$. Par suite 
        $$a_{(\sigma \circ \tau)(k) k} = a_{\sigma(k) k}$$
      
      \item si $k = i$, alors $\tau(k) = j$. Par suite $a_{(\sigma \circ \tau)(k) k} = a_{\sigma(j) i}$. Puisque $C_i = C_j$ on a $a_{\sigma(j) i} = a_{\sigma(j) j}$. On a donc
        $$a_{(\sigma \circ \tau)(k) k} = a_{(\sigma \circ \tau)(i) i} = a_{\sigma(j) j}$$
        
      \item si $k = j$, alors $\tau(k) = i$. Par suite $a_{(\sigma \circ \tau)(k) k} = a_{\sigma(i) j}$. Puisque $C_j = C_i$ on a $a_{\sigma(i) j} = a_{\sigma(i) i}$. On a donc
        $$a_{(\sigma \circ \tau)(k) k} = a_{(\sigma \circ \tau)(j) j} = a_{\sigma(i) i}$$
    \end{itemize}
    On a donc
    \begin{eqnarray*}
      a_{(\sigma \circ \tau)(1) 1} \cdot \ldots \cdot a_{(\sigma \circ \tau)(i) i} \cdot \ldots \cdot a_{(\sigma \circ \tau)(j) j} \cdot \ldots \cdot a_{(\sigma \circ \tau)(n) n} \\
        = a_{\sigma(1) 1} \cdot \ldots \cdot a_{\sigma(j) j} \cdot \ldots \cdot a_{\sigma(i) i} \cdot \ldots \cdot a_{\sigma(n) n} \\
        = a_{\sigma(1) 1} \cdot \ldots \cdot a_{\sigma(i) i} \cdot \ldots \cdot a_{\sigma(j) j} \cdot \ldots \cdot a_{\sigma(n) n} \\
    \end{eqnarray*}
    %
    De plus $\epsilon(\sigma \circ \tau) = \epsilon(\sigma) \cdot \epsilon(\tau)$ où on a $\epsilon(\tau) = -1$, donc
    $$\epsilon(\sigma \circ \tau) = -\epsilon(\sigma)$$
    Par conéquent 
    $$p_{\sigma \circ \tau}(C_1, C_2, \ldots, C_n) = -p_{\sigma}(C_1, C_2, \ldots, C_n)$$
    Ainsi pour tout $i\in \{1; \ldots; \frac{n!}{2}\}$
    \begin{eqnarray*}
      p_{\sigma_i \circ \tau}(C_1, C_2, \ldots, C_n) &=& -p_{\sigma_i}(C_1, C_2, \ldots, C_n) \\
      p_{\sigma_i \circ \tau}(C_1, C_2, \ldots, C_n) &+& p_{\sigma_i}(C_1, C_2, \ldots, C_n) ~=~ 0
    \end{eqnarray*}
    Il s'ensuit que $det(C_1, \ldots, C_i, \ldots, C_j(=C_i), \ldots, C_n) = 0$
\end{enumerate}

\paragraph{Remarque} Soient $A, B \in M_{n \times n}(\R)$, $\lambda \in \R$. En général,
\begin{eqnarray*}
  det(A + B) &\neq& det(A) + det(B) \\
  det(\lambda A) &\neq& \lambda det(A)
\end{eqnarray*}
En fait, on a $det(\lambda A) = \lambda^{n} det(A)$.

\paragraph{Corollaire} Soient $i, j \in \{1; \ldots; n\}$, $i\neq j$ et $C_1, C_2, \ldots, C_n$ $n$ matrices colonnes £ $n$ lignes. Alors
$$det(C_1, \ldots, C_i, \ldots, C_j, \ldots, C_n) = -det(C_1, \ldots, C_j, \ldots, C_i, \ldots, C_n)$$

\paragraph{Démonstration} On a
\begin{eqnarray*}
  0 =& det(C_1, \ldots, C_i+C_j, \ldots, C_i+C_j, \ldots, C_n) \\
    =& det(C_1, \ldots, C_i, \ldots, C_i+C_j, \ldots, C_n) \\
    &+ det(C_1, \ldots, C_j, \ldots, C_i+C_j, \ldots, C_n) \\
    =& det(C_1, \ldots, C_i, \ldots, C_i, \ldots, C_n) \\
    &+ det(C_1, \ldots, C_i, \ldots, C_j, \ldots, C_n) \\
    &+ det(C_1, \ldots, C_j, \ldots, C_i, \ldots, C_n) \\
    &+ det(C_1, \ldots, C_j, \ldots, C_j, \ldots, C_n) \\
    =& 0 \\
    &+ det(C_1, \ldots, C_i, \ldots, C_j, \ldots, C_n) \\
    &+ det(C_1, \ldots, C_j, \ldots, C_i, \ldots, C_n) \\
    &+ 0 \\
    =& det(C_1, \ldots, C_i, \ldots, C_j, \ldots, C_n) + det(C_1, \ldots, C_j, \ldots, C_i, \ldots, C_n) = 0
\end{eqnarray*}
On dit alors que l'application 
$$det: M_{n \times 1}(\R) \times \ldots \times M_{n\times 1}(\R) \rightarrow \R$$
est antisymétrique.

\paragraph{Théorème} Soit $A =(a_{ij})$ une matrice carrée de taille $n\times n$. Alors
$$det(A^{T}) = det(A)$$

\paragraph{Démonstration} On observe que l'application
\begin{eqnarray*}
  S_n &\rightarrow& S_n \\
  \sigma &\mapsto& \sigma^{-1}
\end{eqnarray*}
est bijective (elle est sa propre bijection réciproque). Il vient
\begin{eqnarray*}
  det(A^{T}) &=& \sum_{\sigma \in S_n} p_{\sigma}(A^{T}) \\
    &=& \sum_{\sigma^{-1} \in S_n} p_{\sigma^{-1}}(A^{T})
\end{eqnarray*}
Soit $\sigma \in S_n$
$$p_{\sigma^{-1}}(A^{T}) = \epsilon(\sigma^{-1}) \cdot (A^{T})_{\sigma^{-1}(1) 1} \cdot \ldots \cdot (A^{T})_{\sigma^{-1}(n) n}$$
On a $\epsilon(\sigma) \epsilon(\sigma^{-1}) = \epsilon(\sigma \circ \sigma^{-1}) = \epsilon(id) = 1$. Puisque $\epsilon(\sigma) \pm 1$ et $\epsilon(\sigma^{-1}) \pm 1$, on a 
$$\epsilon(\sigma) = \epsilon(\sigma^{-1})$$
Maintenant
$$(A^{T})_{\sigma^{-1}(1) 1} \cdot \ldots \cdot (A^{T})_{\sigma^{-1}(n) n} = A_{1 \sigma^{-1}(1)} \cdot \ldots \cdot A_{n \sigma^{-1}(n)}$$
Comme $\sigma^{-1}: \{1; \ldots; n\} \rightarrow \{1; \ldots; n\}$ est bijective on a
\begin{eqnarray*}
  A_{1 \sigma^{-1}(1)} \cdot \ldots \cdot A_{n \sigma^{-1}(n)} &=& \prod_{j=1}^n A_{j \sigma^{-1}(j)} \\
    &=& \prod_{k=1}^n A_{\sigma(k) k} ~~ (\text{on a posé } k = \sigma^{-1}(j))
\end{eqnarray*}
Donc $p_{\sigma^{-1}}(A^{T}) = p_{\sigma}(A)$. Par suite
$$det(A^{T}) = \sum_{\sigma \in S_n} p_{\sigma}(A) = det(A)$$

\paragraph{Lemme} Soit $\sigma \in S_n$ telle que pour tout $i \in \{1; \ldots; n\}$, $\sigma(i) \leq i$. Alors $\sigma  = id$. En effet on a
\begin{itemize}
  \item $1 \leq \sigma(1) \leq 1$ (par hypothèse), donc $\sigma(1) = 1$.
  \item on a $\sigma(2) \neq \sigma(1) = 1$, donc $\sigma(2) \geq 2$, et par hypothèse $\sigma(2) \leq 2$. Par suite $\sigma(2) = 2$.
  \item Ainsi de suite.
\end{itemize}
Il est équivalent de dire que si une permutations $\sigma \in S_n$ est différendte de $id$, alors il existe $i \in \{1; \ldots; n\}$ tel que $\sigma(i) > i$.

\paragraph{Théorème} Le déterminant d'une matrice carré triangulaire supérieure (resp. triangulaire inférieure) est àgal au produit des ses coefficients diagonaux.

\paragraph{Démonstration}  Soit $A = (a_ij)$ une matrice carrée de taille $n \times n$ triangulaire supérieure. Il vient 
$$det(A) = p_{id}(A) + \sum_{\sigma \in S_n, \sigma \neq id} p_{\sigma}(A)$$
Si $\sigma \neq id$, alors, d'après le lemme, il existe $j \in \{1; \ldots; n \}$ tel que $\sigma(j) > j$. Puisque $A$ est triangulaire supérieure, $a_{\sigma(j) j} = 0$. Par suite $p_{\sigma}(A) = 0$ et donc 
$$\sum_{\sigma \in S_n, \sigma \neq id} p_{\sigma}(A) = 0$$
Il reste
$$det(A) = p_{id}(A) = a_{11} \cdot a_{22} \cdot \ldots \cdot a_{nn}$$
Le cas "triangulaire inférieure" se déduit du cas "triangulaire supérieure" par transposition.

%
\subsection{Le déterminant sous l'effet des opérations élémentaires}
%
Les résultats précédents indiquent comment se comporte le déterminant sous l'effet des opérations élémentaires sur les lignes d'une matrice:
\begin{itemize}
  \item Le déterminant ne change pa si on ajout à une ligne un multiple d'une autre ligne
    $$\begin{vmatrix} L_1 \\ L_2 + 4 L_3 \\ L_3 \end{vmatrix} 
      = \begin{vmatrix} L_1 \\ L_2 \\ L_3 \end{vmatrix} + 4 \cdot \begin{vmatrix} L_1 \\ L_3 \\ L_3 \end{vmatrix}
      = \begin{vmatrix} L_1 \\ L_2 \\ L_3 \end{vmatrix} + 0
      = \begin{vmatrix} L_1 \\ L_2 \\ L_3 \end{vmatrix}$$
      
  \item Lorsque on multiplie une ligne par un réel, le déterminant est multiplié par ce réel
    $$\begin{vmatrix} 5 L_1 \\ L_2 \\ L_3 \end{vmatrix} = 5 \cdot \begin{vmatrix} L_1 \\ L_2 \\ L_3 \end{vmatrix}$$
    
  \item Lorsque on permute deux lignes, le déterminant est multiplié par $-1$
    $$\begin{vmatrix} L_1 \\ L_3 \\ L_2 \end{vmatrix} = -1 \cdot \begin{vmatrix} L_1 \\ L_2 \\ L_3 \end{vmatrix}$$
\end{itemize}

\paragraph{Remarque} Le deux premières propriétés découlent de la multilinéarité du détérminant.

%
\subsection{Calculer un déterminant avec l'algorithme de Gauss}
%
\paragraph{Méthode de calcul} On peut donc calculer un déterminant un utilisant l'algorithme d'élimination de Gauss. Par des opérations élémentaires sur les lignes ou sur les colonnes, on met la matrice sous la forme triangulaire. On applique à chaque étape les régles du calcul du déterminant. Le déterminant est alors égal au produit des coefficients diagonaux de cette matrice triangulaire.

%
%
\section{Inversibilité des matrices et déterminant}
%
%
\paragraph{Théorème} Soient $A$ et $B$ deux matrices carrées de même taille. Alors 
$$det(A B) = det(A) \cdot det(B)$$

\paragraph{Démonstration} Nous nous contentans ici de montros cette formule dasn le cas de matrices de taille $3 \times 3$, le cas général se traite de façon analogue. \\
Soient $A$ et $B$ deux matrices de taille $3 \times 3$. On note $C_1, C_2, C_3$ les colonnes de $A$
$$A = (C_1, C_2, C_3) \text{ et } B=(b_{ij})_{1 \leq i, j \leq n}$$
On observe que
$$A B = \left(
  A \begin{pmatrix} b_{11} \\ b_{21} \\ b_{31} \end{pmatrix} ~ 
  A \begin{pmatrix} b_{12} \\ b_{22} \\ b_{32} \end{pmatrix} ~ 
  A \begin{pmatrix} b_{13} \\ b_{23} \\ b_{33} \end{pmatrix}
\right)$$
Pour $i \in \{1; 2; 3\}$, on a également
\begin{eqnarray*}
  A \begin{pmatrix} b_{1i} \\ b_{2i} \\ b_{3i} \end{pmatrix}
  &=& A \left( b_{1i} \begin{pmatrix} 1 \\ 0 \\ 0 \end{pmatrix} + b_{2i} \begin{pmatrix} 0 \\ 1 \\ 0 \end{pmatrix} + b_{3i} \begin{pmatrix} 0 \\ 0 \\ 1 \end{pmatrix} \right) \\
    &=& b_{1i} A \begin{pmatrix} 1 \\ 0 \\ 0 \end{pmatrix} + b_{2i} A \begin{pmatrix} 0 \\ 1 \\ 0 \end{pmatrix} + b_{3i} A \begin{pmatrix} 0 \\ 0 \\ 1 \end{pmatrix} \\
    &=& b_{1i} C_1 + b_{2i} C_2 + b_{3i} C_3
\end{eqnarray*}
Il vient
$$det(A B) = \begin{vmatrix} \\ b_{11} C_1 + b_{21} C_2 + b_{31} C_3 + b_{12} C_1 + b_{22} C_2 + b_{32} C_3 + b_{13} C_1 + b_{23} C_2 + b_{33} C_3 \\ \\ \end{vmatrix}$$
On développe ce déterminant par multilinéairté. On obtient $3^3 = 27$ termes. Deux sortes de termes:
\begin{itemize}
  \item Ceux qui correspondent au choix d'une permutations $\sigma \in S_3$ des colonnes $C_1, C_2, C_3$:
    $$b_{\sigma(1) 1} \cdot b_{\sigma(2) 2} \cdot b_{\sigma(3) 3} \cdot \begin{vmatrix} C_{\sigma(1)} & C_{\sigma(2)} & C_{\sigma(3)} \end{vmatrix}$$
    
  \item Les autres pour lesquels au moins deux retenues sur les trois sont égales. Alors le déterminant correspondant est nul, et donc le terme lui-même est nul.
\end{itemize}
On obtient donc
$$det(A B) = \sum_{\sigma \in S_3} b_{\sigma(1) 1} \cdot b_{\sigma(2) 2} \cdot b_{\sigma(3) 3} \cdot \begin{vmatrix} C_{\sigma(1)} & C_{\sigma(2)} & C_{\sigma(3)} \end{vmatrix}$$
où pour tout $\sigma \in S_3$,
\begin{eqnarray*}
  \begin{vmatrix} C_{\sigma(1)} & C_{\sigma(2)} & C_{\sigma(3)} \end{vmatrix}
  &=& \epsilon(\sigma) \begin{vmatrix} C_1 & C_2 & C_ 3 \end{vmatrix} \\
    &=& \epsilon(\sigma) \cdot det(A)
\end{eqnarray*}
D'où
\begin{eqnarray*}
  det(A B) &=& \sum_{\sigma \in S_3} \epsilon(\sigma) \cdot b_{\sigma(1) 1} \cdot b_{\sigma(2) 2} \cdot b_{\sigma(3) 3} \cdot det(A) \\
    &=& det(A) \left( \sum_{\sigma \in S_3} \epsilon(\sigma) \cdot b_{\sigma(1) 1} \cdot b_{\sigma(2) 2} \cdot b_{\sigma(3) 3} \right) \\
    &=& det(A) \cdot det(B)
\end{eqnarray*}

\paragraph{Théorème} Soit $A$ une matrice carrée de taille $n \times n$, alors 
$$A \text{ est inversible } \Leftrightarrow det(A) \neq 0$$

\paragraph{Démonstration} 
\begin{itemize}
  \item [$\Rightarrow$] Supposons que $A$ est inversible. On a donc
    $$A A^{-1} = I_n$$
    Il vient
    $$det(A) \cdot det(A^{-1}) = det(AA^{-1}) = det(I_n) = 1 \neq 0$$
    Cela implqiue que $det(A) \neq 0$
  
  \item [$\Leftarrow$] Réciproquement, montrons que si $A$ n'est pas inversible, alors $det(A) = 0$. Supposons docn que $A$ ne soit pas inversible. Considérons le système 
    $$A \cdot X = 0$$
    Puisque $A$ n'est pas inversible et que $0$ est solution de cesystème, ce système admet une infinité de solutions. Il existe donc des réels $\alpha_1, \ldots, \alpha_n$ non tous nuls tels que
    $$A \begin{pmatrix} \alpha_1 \\ \vdots \\ \alpha_n \end{pmatrix} = 0$$
    Soient $C_1, \ldots, C_n$ les colonnes de $A$. On a 
    $$0 = A \begin{pmatrix} \alpha_1 \\ \vdots \\ \alpha_n \end{pmatrix} = \alpha_1 \cdot C_1 + \ldots + \alpha_n \cdot C_n$$
    Soit $i_0 \in \{1; \ldots; n \}$ tel que $\alpha_{i_0} \neq 0$. On obtient 
    $$C_{i_0} = - \sum_{i=1, i\neq i_0}^n \frac{\alpha_i}{\alpha_{i_0}} C_i$$
    Cela entraîne que
    \begin{eqnarray*}
      det(A) &=& \begin{vmatrix} \\
        C_1 & \ldots & C_{i_0} & \ldots & C_i & \ldots & C_n 
      \\ \\ \end{vmatrix} \\
        &=& \begin{vmatrix} \\
          C_1 & \ldots & - \sum_{i=1, i\neq i_0}^n \frac{\alpha_i}{\alpha_{i_0}} C_i & \ldots & C_i & \ldots & C_n 
        \\ \\\end{vmatrix} \\
        &=& - \sum_{i=1, i\neq i_0}^n \frac{\alpha_i}{\alpha_{i_0}} \begin{vmatrix} \\
          C_1 & \ldots & C_{i} & \ldots & C_i & \ldots & C_n 
        \\ \\ \end{vmatrix}
    \end{eqnarray*}
    Dans le $i^{ème}$ terme de cette somme la colonne $C_i$ apparaît en position $i$ et $i_0$ donc
    $$\begin{vmatrix} \\ C_1 & \ldots & C_{i_0} & \ldots & C_i & \ldots & C_n \\ \\ \end{vmatrix} = 0$$
    Par suite
    $$det(A) = 0$$
\end{itemize}

%
\subsection{Cofacteur d'une matrice}
%
\paragraph{Définition} Soit $A = (a_{ij})$ une matrice carrée de taille $n\times n$. Soit $(i ~ j) \in \{1; \ldots; n\} \times \{1; \ldots; n \}$. On appelle cofacteur de $A$ associé à $(i ~ j)$ le nombre réel défini par
$$\Delta_{ij} = (-1)^{i+j} \begin{vmatrix}
  a_{11} & \ldots & a_{1 j-1} & a_{1 j+1} & \ldots & a_{1n} \\
  \vdots &  & \vdots & \vdots &  & \vdots \\
  a_{i-1 1} & \ldots & a_{i-1 j-1} & a_{i-1 j+1} & \ldots & a_{i-1 n} \\
  a_{i+1 1} & \ldots & a_{i+1 j-1} & a_{i+1 j+1} & \ldots & a_{i+1 n} \\
  \vdots &  & \vdots & \vdots &  & \vdots \\
  a_{n 1} & \ldots & a_{n j-1} & a_{n j+1} & \ldots & a_{nn}
\end{vmatrix}$$
Ce déterminant (de matrice de taille $(n-1) \times (n-1)$) est obtenu à partir de celui de $A$ en supprimant le $i^{ème}$ ligne et la $j^{ème}$ colonne.

\paragraph{Exemple} Soit
$$A = \begin{pmatrix}
  1 & -5 & -1 & 0 \\
  2 & 1 & 2 & 3 \\
  3 & 7 & 3 & 1 \\
  0 & 1 & 0 & 8
\end{pmatrix}$$
Alors on a
\begin{eqnarray*}
  \Delta_{23} &=& (-1)^{2+3} \begin{vmatrix} 1 & -5 & 0 \\ 3 & 7 & 1 \\ 0 & 1 & 8 \end{vmatrix}
    = - \begin{vmatrix} 1 & -5 & 0 \\ 3 & 7 & 1 \\ 0 & 1 & 8 \end{vmatrix} \\
  \Delta_{42} &=& (-1)^{4 + 2} \begin{vmatrix} 1 & -1 & 0 \\ 2 & 2 & 3 \\ 3 & 3 & 1 \end{vmatrix}
    = - \begin{vmatrix} 1 & -1 & 0 \\ 2 & 2 & 3 \\ 3 & 3 & 1 \end{vmatrix}
\end{eqnarray*}

%
\subsection{Calculer un déterminant par développement}
%
\paragraph{Théorème} Soit $A$ une matrice carrée de taille $n\times n$. Soit $j \in \{1; \ldots; n\}$. Alors
$$det(A) = \sum_{i=1}^{n} a_{ij} \Delta_{ij}$$
On dit que l'on développe $det(A)$ par rapport à la $j^{ème}$ colonne.

\paragraph{Exemple} Développons $\begin{vmatrix} 1 & -2 & 0 \\ 7 & 1 & 2 \\ 3 & 5 & 3 \end{vmatrix}$ par rapport à la $3^{ème}$ ligne
$$\begin{vmatrix} 1 & -2 & 0 \\ 7 & 1 & 2 \\ 3 & 5 & 3 \end{vmatrix} = 
  0 \cdot \begin{vmatrix} 7 & 1 \\ 3 & 5 \end{vmatrix} 
  - 2 \cdot \begin{vmatrix} 1 & -2 \\ 3 & 5 \end{vmatrix} 
  + 3 \cdot \begin{vmatrix} 1 & -2 \\ 7 & 1 \end{vmatrix} 
= 23$$

\paragraph{Théorème} Soit $A$ une matrice carrée de taille $n\times n$. Soit $i \in \{1; \ldots; n\}$. Alors
$$det(A) = \sum_{j=1}^{n} a_{ij} \Delta_{ij}$$
On dit que l'on développe $det(A)$ par rapport à la $i^{ème}$ ligne.

\paragraph{Démonstration} On démontre la formule de développememt du déterminant suivant une colonne, la formule de développement suivant une ligne s'en déduisant par transposition. \\
Soit $A = (a_{j})$ une matrice de taille $n \times n$.
$$det(A) = \sum_{\tau \in S_n} p_{\tau}(A)$$
Soit $j \in \{1; \ldots; n \}$, pour tout couple $(k, l) \in  \{1; \ldots; n \}\times\{1; \ldots; n \}$ on note
$$T_{kl} = \{\tau \in S_n \vert k = \tau(l) \}$$
On a $S_n = T_{1j} \cup T_{2j} \cup \ldots \cup T_{nj}$ (union disjointe). Il vient
$$det(A) = \sum_{\tau \in T_{1j}} p_{\tau}(A) + \sum_{\tau \in T_{2j}} p_{\tau}(A) + \ldots + \sum_{\tau \in T_{nj}} p_{\tau}(A)$$
Soit $i \in \{1; \ldots; n\}$. Montrons que $\sum_{\tau \in T_{ij}} p_{\tau}(A) = a_{ij} \Delta_{ij}$. On considère la matrice $B$ construite à partir de $A$ en permutant la $j^{ème}$ colonne de $A$ avec les colonnes uivantes de façon à l'amener à la place de la $n^{ème}$ colonne et en fasant une démarche analoque avec la $i^{ème}$ ligne. On fait donc subir aux colonnes de $A$ la permutation
$$d^{-1} = (n-1 ~~ n) \circ \ldots \circ (j+ 1 ~~ j + 2) \circ (j ~~ j+1) = \begin{pmatrix}j & n & n-1 & \ldots & j + 2 & j + 1\end{pmatrix}$$
et aux lignes de $A$ la permutation
$$c^{-1} = (n-1 ~~ n) \circ \ldots \circ (i+ 1 ~~ i + 2) \circ (i ~~ i+1) = \begin{pmatrix}i & n & n-1 & \ldots & i + 2 & i + 1\end{pmatrix}$$
On a pour tout $(k, l) \in \{1; \ldots; n \}\times\{1; \ldots; n \}$
$$b_{kl} = a_{c(k)d(l)}$$
On observe que l'application
\begin{eqnarray*}
  \phi: T_{n n} &\rightarrow& T_{ij} \\
  \tau &\mapsto& c \circ \tau \circ d^{-1}
\end{eqnarray*}
est bijective. \footnote{Sa bijections réciproque est
\begin{eqnarray*}
  \psi: T_{ij} &\rightarrow& T_{n n} \\
  \tau &\mapsto& c^{-1} \circ \tau \circ d
\end{eqnarray*}
On a
$$\phi \circ \psi = id_{T_{ij}} \text{ et } \psi \circ \phi = id_{T_{n n}}$$}
Il vient
\begin{eqnarray*}
  \sum_{\tau \in T_{ij}} p_{\tau}(A) &=& \sum_{\tau \in T_{ij}} \epsilon(\tau) \prod_{l=1}^{n} a_{\tau(l) l} \\
    &=& \sum_{\tau \in T_{ij}} \epsilon(\phi(\tau)) \prod_{l=1}^{n} a_{\phi(\tau(l)) l} \\
    &=& \sum_{\tau \in T_{n n}} \epsilon(c \circ \tau \circ d^{-1}) \prod_{l=1}^{n} a_{(c \circ \tau \circ d^{-1})(l) l} \\
    &=& \sum_{\tau \in T_{n n}} \epsilon(c) \epsilon(d^{-1}) \epsilon(\tau) \prod_{l=1}^{n} a_{c(\tau(l')) ~ d(l')} ~ (\text{on pose } l' = d^{-1}(l))\\
\end{eqnarray*}
On a $\epsilon(c) = \epsilon(c^{-1}) = (-1)^{n-i}$ car $c^{-1}$ est le produit de $n-i$ transpositions et $\epsilon(d^{-1}) = (-1)^{n-j}$ car $d^{-1}$ est le produit de $n-j$ transpositions, d'où $\epsilon(c) \epsilon(d^{-1}) = (-1)^{2n - (i+j)}$. Comme $(2n - (i+j)) + (i+j) = 2n$, on a $2n - (i+j)$ et $i+j$ ont mêm parité donc $(-1)^{2n - (i+j)} = (-1)^{i+j}$. Par suite
$$\epsilon(c) \epsilon(d^{-1}) = (-1)^{i+j}$$
Pour tout $l' \in \{1; \ldots; \n\}$, $a_{c(\tau(l')) ~ d(l')} = b_{\tau(l') ~ l'}$. On obtient donc
\begin{eqnarray*}
  \sum_{\tau \in T_{ij}} p_{\tau}(A) &=& (-1)^{i+j} \sum_{\tau \in T_{nn}} \epsilon(\tau) \prod_{l'=1}^{n} b_{\tau(l') ~ l'} \\
    &=& (-1)^{i+j} b_{nn} \sum_{\tau \in T_{nn}} \epsilon(\tau) \prod_{l' = 1}^{n-1} b_{\tau(l') ~ l'} \\
    &=& a_{ij} \Delta_{ij}
\end{eqnarray*}

%
\subsection{Comatrice}
%
\paragraph{Définition} Soit $A$ une matrice carrée de taille $n \times n$. On appelle comatrice de $A$ et on note $com(A)$ la matrice carrée de taille $n \times n$ dont les coefficients sont les cofacteurs de $A$, c'est-à-dire pour tous $1 \leq i, j \leq n$
$$(com(A))_{ij} = \Delta_{ij}$$

\paragraph{Exemple} Soit $A = \begin{pmatrix} 1 & -4 & 2 \\ 0 & 3 & -1 \\ 2 & 5 & 0 \end{pmatrix}$. On a
$$com(a) = \begin{pmatrix}
  \begin{vmatrix} 3 & -1 \\ 5 & 0 \end{vmatrix} & -\begin{vmatrix} 0 & -1 \\ 2 & 0 \end{vmatrix} & \begin{vmatrix} 0 & 3 \\ 2 & 5 \end{vmatrix} \\
  -\begin{vmatrix} -4 & 2 \\ 5 & 0 \end{vmatrix} & \begin{vmatrix} 1 & 2 \\ 2 & 0 \end{vmatrix} & -\begin{vmatrix} 1 & -4 \\ 2 & 5 \end{vmatrix} \\
  \begin{vmatrix} -4 & 2 \\ 3 & -1 \end{vmatrix} & -\begin{vmatrix} 1 & 2 \\ 0 & -1 \end{vmatrix} & \begin{vmatrix} 1 & -4 \\ 0 & 3 \end{vmatrix}
\end{pmatrix} = \begin{pmatrix}
  5 & -2 & -6 \\
  10 & -4 & -13 \\
  -2 & 1 & 3
\end{pmatrix}$$

\paragraph{Théorème} Soit $A$ une matrice carrée de taille $n \times n$. On a
$$A \cdot com(A)^{T} = com(A)^{T} \cdot A = det(A) \cdot I_n$$

\paragraph{Théorème} Si $A$ est inversible, alors $det(A) \neq 0$ et 
$$A^{-1} = \frac{1}{det(a)} com(A)^{T}$$

\paragraph{Remarque} Cette formule généralise celle qui a été vue dans le cas des matrices de taille $2 \times 2$.

%
%
\section{Systèmes de cramer}
%
%
\paragraph{Définition} On dit qu'un système linéaire $A \cdot X = B$ à $n$ équations et $n$ inconnues, $A$ matrice carrée de taille $n \times n$, $B$ matrice colonée de taille $n\times 1$ est un système de cramer si la matrice $A$ du système est inversible. \\
Dans ce cas le système admet une solution unique $(a_1, \ldots, a_n) \in \R^n$ et les $a_i \in \R$ sont donnés par les formules suivantes. Si on désigne par $C_1, \ldots, C_n$ les colonnes de $A$\footnote{$A = (C_1, \ldots, C_n)$}, on a
$$a_i = \frac{\begin{vmatrix} C_1 & \ldots & B_{(i)} & \ldots & C_n \end{vmatrix}}{det(A)} ~ (\text{on remplace } C_i \text{ avec } B)$$
Comme $A$ est inversible, $det(A) \neq 0$.

\paragraph{Démonstration} On a $B = A \begin{pmatrix} a_1 \\ \vdots \\ a_n \end{pmatrix} = a_1 C_1 + \ldots + a_n C_n$. Il vient
\begin{eqnarray*}
  \begin{vmatrix} C_1 & \ldots & B_{(i)} & \ldots & C_n \end{vmatrix} &=& \begin{vmatrix} C_1 & \ldots &  a_1 C_1 + \ldots + a_n C_n & \ldots & C_n \end{vmatrix} \\
    &=& a_1 \begin{vmatrix} C_1 & \ldots & C_1 & \ldots & C_n \end{vmatrix} \\
    &&+ \ldots \\
    &&+ a_i \begin{vmatrix} C_1 & \ldots & C_i & \ldots & C_n \end{vmatrix} \\
    &&+ \ldots \\
    &&+ a_n \begin{vmatrix} C_1 & \ldots & C_n & \ldots & C_n \end{vmatrix}
\end{eqnarray*}
On a 
$$a_i \begin{vmatrix} C_1 & \ldots & C_i & \ldots & C_n \end{vmatrix} = a_i det(A)$$
Pour $j \in \{1; \ldots; n\}$, $j\neq i$
$$a_i \begin{vmatrix} C_1 & \ldots & C_j & \ldots & C_n \end{vmatrix} = 0$$
car la matrice colonne $C_j$ apparaît deux fois , en positions $i$ et $j$. On obtient donc 
$$\begin{vmatrix} C_1 & \ldots & B_{(i)} & \ldots & C_n \end{vmatrix} = a_i det(A)$$
d'où
$$a_i = \frac{\begin{vmatrix} C_1 & \ldots & B_{(i)} & \ldots & C_n \end{vmatrix}}{det(A)}$$

%
%
\section{Récapitulatif}
%
%

%
\subsection{Méthodes de calcul du déterminant}
%
Pour calculer le déterminant d'une matrice, on peut:
\begin{itemize}
  \item utilise la définition du déterminant comme somme des produits élémentaires singés convient pour les matrices de taille $1 \times 1$ ou $2 \times 2$
    $$\begin{array}{lmr} det\big((a)\big) = a & & det\left(\begin{pmatrix} a & b \\ c & d \end{pmatrix}\right) = ad - bc \end{array}$$
  \item développer suivant une ligne ou une colonne convient pour les matrice de taille $3 \times 3$, éventuellement aussi pour une taille de $4 \times 4$
  \item utiliser l'algorithme d'élimination de Gauss convient pour les matrices de taille $2 \times 2$ et plus.
\end{itemize}

%
\subsection{Méthodes de calcul de l'inverse d'une matrice}
Pour calculer  l'inverse d'une matrice\footnote{si l'inverse existe}, on peut
\begin{itemize}
  \item calculer la comatrice de cette matruce convient pour des matrices de taille $2 \times 2$ et $3 \times 3$.
  \item utiliser l'algorithme d'élimination de Gauss convient pour des matrice de taille $3 \times 3$ et plus.


\chapter{Espaces vectoriels et applications linéaires}

%
%
\section{Rappels}
%
%
\paragraph{Définition} Un $\R$-espace vectoriel est une ensemble non vide $V$ muni d'une addition
\begin{eqnarray*}
  +: V\timesV &\rightarrow& V \\
  (\vec{u}, \vec{v}) &\mapsto& \vec{u}+\vec{v}
\end{eqnarray*}
et multiplication extérne
\begin{eqnarray*}
  \cdot: V\timesV &\rightarrow& V \\
  (\vec{u}, \vec{v}) &\mapsto& \vec{u}\cdot\vec{v}
\end{eqnarray*}
satisfaisant conditions:
\begin{itemize}
  \item $(V, +)$ est un groupe obélion (associatif et commutatif).
  \item $\forall \alpha, \beta \in \R \text{ et } \vec{u}, \vec{v} \in V$:
    \begin{eqnarray*}
      \alpha \cdot (\vec{u} + \vec{v}) &=& \alpha \cdot \vec{u} + \alpha \cdot \vec{v} \\
      (\alpha + \beta) \cdot \vec{u} &=& \alpha \cdot \vec{u} + \beta \cdot \vec{u} \\
      (\alpha \cdot \beta)\cdot \vec{u} &=& \alpha \cdot (\beta \cdot \vec{u}) \\
      1 \cdot \vec{u} &=& \vec{u}
    \end{eqnarray*}
\end{itemize}

\paragraph{Définition} Si $U$ et $V$ sont deux $\R$-espaces vectoriels, une applicaiton linéaire de $U$ dans $V$ est une application $b: U \rightarrow V$ telle que $\forall \vec{u}, \vec{v} \in U \text{et} \alpha \in \R$
$$b(\alpha \cdot \vec{u} +\vec{v}) = \alpha \cdot b(\vec{u}) + b(\vec{v})$$

%
%
\section{Sous-espaces vectoriels}
%
%
\paragraph{Définition} Soient $V$ un $\R$-espace vectoriel et $\vec{v_1}, \ldots, \vec{v_r} ~ (r\leq 1)$ des vecteurs de $V$. On dit qu'un vecteur $\vec{w}$ de $V$ est combinaison linéaire des $\vec{v_i}$ s'il existe des réels $\alpha_1, \ldots, \alpha_r$ tel que
$$\vec{w} = \alpha_1 \cdot \vec{v_1} + \ldots + \alpha_r \cdot \vec{v_r}$$

\paragraph{Définition} Soient $V$ un $\R$-espace vectoriel et $W$ un sous-ensemble non vide de $V$. On dit que $W$ est un sous-espace vectoriel de $V$ si $W$ est stable par $+$ et $\cdot$, ce que signifie:
\begin{eqnarray*}
  \forall \vec{u}, \vec{v} \in W, \vec{u}, ~ \vec{u} + \vec{v} \in W \\
  \forall \vec{u} \in W, \alpha \in \R, ~ \alpha \cdot \vec{u} \in W \\
\end{eqnarray*}
et que $W$ (muni de ces deux lois) soit un $\R$-espace vectoriel.

\paragraph{Théorème} Soient $V$ un $\R$-espace vectoriel et $W$ un sous-ensemble non vide de $V$. Alors $W$ est un sous-espace vectoriel de $V$ si $\forall \vec{u}, \vec{v} \in W, \alpha, \beta \in R$:
$$\alpha \cdot \vec{u} + \beta \cdot \vec{v} \in W$$
autrement dit, si $W$ est stable par combinaison linéaire.

\paragraph{Proposition} Soient V un $\R$-espace vectoriel et $(W_i)_{i \in I}$ une famille de sous-espaces vectoriels de $V$. Alors $\cap_{i \in I} W_i = \{ \vec{v} \in V | \forall i \in I, \vec{v} \in W_i\}$ est encore un sous-espace vectoriel de V.

\paragraph{Définition} Soient $V$ un $\R$-espace vectoriel et $\vec{v_1}, \ldots, \vec{v_r} \in V$. On appelle sous-espace vectoriel de V engerdré par $\{\vec{v_1}, \ldots, \vec{v_r}\}$ le plus petit sous-espace vectoriel de $V$ (au sens de l'inclusion) contenant $\vec{v_1}, \ldots, \vec{v_r}$. C'est l'intérsection de tous les sous-espaces vectoriels de $V$ contenant $\vec{v_1}, \ldots, \vec{v_r}$. On le noté $Vect\{\vec{v_1}, \ldots, \vec{v_r}\}$.

\paragraph{Théorème} On a $Vect\{\vec{v_1}, \ldots, \vec{v_r}\} = \{\alpha_1 \cdot \vec{v_1} + \ldots + \alpha_r \cdot \vec{v_r} | \alpha_1, \ldots, \alpha_r \in \R \}$; c'est-à-dire: le sous-espace vectoriel de $V$ engerdré par $\vec{v_1}, \ldots, \vec{v_r}$ est constitué des combinaisons linéaires de $\vec{v_1}, \ldots, \vec{v_r}$.
\paragraph{Démonstration} On procède par double inclusion.
\begin{itemize}
  \item Montrons que $\{\alpha_1 \cdot \vec{v_1} + \ldots + \alpha_r \cdot \vec{v_r} | \alpha_1, \ldots, \alpha_r \in \R \} \subset Vect\{\vec{v_1}, \ldots, \vec{v_r}\}$. $Vect\{\vec{v_1}, \ldots, \vec{v_r}\}$ est en particulier un sous-espace vectoriel de $V$, donc $Vect\{\vec{v_1}, \ldots, \vec{v_r}\}$ est stable par combinaison linéaire. Puisque $\vec{v_1}, \ldots, \vec{v_r} \in Vect\{\vec{v_1}, \ldots, \vec{v_r}\}$, alors tout combinaison linéaire $\alpha_1 \cdot \vec{v_1} + \ldots + \alpha_r \cdot \vec{v_r}$ appartient à $Vect\{\vec{v_1}, \ldots, \vec{v_r}\}$.
  
  \item Montrons que $Vect\{\vec{v_1}, \ldots, \vec{v_r}\} \subset \{\alpha_1 \cdot \vec{v_1} + \ldots + \alpha_r \cdot \vec{v_r} | \alpha_1, \ldots, \alpha_r \in \R \}$ est un sous-espace vectoriel de $V$ $\{\alpha_1 \cdot \vec{v_1} + \ldots + \alpha_r \cdot \vec{v_r} | \alpha_1, \ldots, \alpha_r \in \R \} \neq \emptyset$ (par example $\vec{v_1} = 1 \cdot \vec{v_1} + 0 \cdot \vec{v_2} + \ldots + 0 \cdot \vec{v_r} \in ensemble$) et stable par combinaison linéaire. En effet, si $\vec{u}$ et $\vec{w}$ sont des vecteurs dans cet ensemble et $\alpha$, $\beta$ deux réels, alors ils existent $\alpha_1, \ldots, \alpha_r, \beta_1, \ldots, \beta_r \in \R$ tels que
    \begin{eqnarray*}
      \vec{u} = \alpha_1 \cdot \vec{v_1} + \ldots + \alpha_r \cdot \vec{v_r} \\
      \vec{w} = \beta_1 \cdot \vec{v_1} + \ldots + \beta_r \cdot \vec{v_r}
    \end{eqnarray*}
    D'où $\alpha \cdot \vec{u} + \beta \cdot \vec{w} = (\alpha \cdot \alpha_1 + \beta \cdot \beta_1)\cdot \vec{v_1} + \ldots + (\alpha \cdot \alpha_r + \beta \cdot \beta_r)\cdot \vec{v_1}$ est encore une combinaison linéaire de $\vec{v_1}, \ldots, \vec{v_r}$. \\
    Puisque $\{\alpha_1 \cdot \vec{v_1} + \ldots + \alpha_r \cdot \vec{v_r} | \alpha_1, \ldots, \alpha_r \in \R \}$ est un sous-espace vectoriel de $V$ qui contient $\vec{v_1}, \ldots, \vec{v_r}$ par définition de $Vect(\vec{v_1}, \ldots, \vec{v_r})$, on a $Vect\{\vec{v_1}, \ldots, \vec{v_r}\} \subset \{\alpha_1 \cdot \vec{v_1} + \ldots + \alpha_r \cdot \vec{v_r} | \alpha_1, \ldots, \alpha_r \in \R \}$
\end{itemize}

\paragraph{Théorème} Soient $V$ un espace-vectoriel et $S = (\vec{v_1}, \ldots, \vec{v_r})$ et $S' = (\vec{v_{1'}}, \ldots, \vec{v_{r'}})$ deux systèmes de vectuers de $V$. Alors $Vect(\vec{v_1}, \ldots, \vec{v_r}) = Vect(\vec{v_{1'}}, \ldots, \vec{v_{r'}})$ si tout vecteurs de $S$ est combinaison linéaire des vecteurs de $S'$ et vice-versa.

%
%
\section{Systèmes générateurs, systèmes libres}
%
%
\paragraph{Définition} Soient $V$ un espace vectoriel et $\vec{v_1}, \ldots, \vec{v_r}$ des vecteurs de $V$. On dit que les $\vec{v}_i$ forment un système générateur de $V$ si $V = Vect\{\vec{v}_1, \ldots, \vec{v}_r\}$, autrement di si tout vecteur de $V$ est combinaison linéaire des $\vec{v}_i$.

\paragraph{Remarque} Lorsqu'un vetuer est combinaison linéaire d'un système de vecteurs donnés, la question de savoir si son écriture comme combinaison linéaire est unique conduit à la notion de systèm libre.

\paragraph{Définition} Soit $V$ un espace vectoriel et $\vec{v}_1, \ldots , \vec{v}_r$ des vectuers de $V$. On dit que les $\vec{v}_i$ forment un system libre (ou linéaire indépendant) si:
$$\forall \alpha_i \in \R, \alpha_1 \cdot \vec{v}_1 + \ldots + \alpha_r \cdot \vec{v}_r = \vec{0} ~ \Rightarrow ~ \alpha_i = 0$$
\paragraph{Remarque} tout vecteur $\neq 0$ forme à lui seul un système libre ($\alpha \cdot \vec{v} = \vec{0} \Leftrightarrow \alpha = 0$).

\paragraph{Définition} Soit $V$ un espace vectoriel. On dit qu'un système de vecteurs de $V$ est liè (ou encore linéaire dépentant) s'il n'est pas libre, c'est-à-dire le système $(\vec{v}_1, \ldots, \vec{v}_r)$ est lié s'il existe des réels $\alpha_1, \ldots, \alpha_r$ non tous nuls tel que $\alpha_1 \cdot \vec{v}_1 + \ldots + \alpha_r \cdot \vec{v}_r = 0$.
\paragraph{Ramarque} Tout système de vecteurs contenant $\vec{0}$ est lié.

\paragraph{Théorème} Soient $V$ un espace vectoriel et $\vec{v}_1, \ldots, \vec{v}_r$ des vecteurs de $V$. Alors $\vec{v}_1, \ldots, \vec{v}_r$ forment un système lié si et seulement si l'un des $\vec{v}_i$ est combinaison linéaire des autres.
\paragraph{Démonstration} 
\begin{itemize}
  \item On suppose que $\vec{v}_1, \ldots, \vec{v}_r$ forment un système liè. Alors il existent $\alpha_i \in \R$ non tous nul tel que $\alpha_1 \cdot \vec{v}_1 + \ldots + \alpha_r \cdot \vec{v}_r = \vec{0}$. Soit $i_0$ entre $1$ et $r$ et $\alpha_{i_0} \neq 0$. Il vient
    \begin{eqnarray*}
      \sum_{j=1}^{r} \alpha_j \cdot \vec{v}_j = \vec{0} \\
      \alpha_{i_0} \cdot \vec{v}_{i_0} + \sum_{j=1, j \neq i_0}^{r} \alpha_j \cdot \vec{v}_j = \vec{0}
    \end{eqnarray*}
    D'où
    $$\vec{v}_{i_0} = -\frac{1}{\alpha_{i_0}} \cdot \sum_{j=1, j \neq i_0}^{r} \alpha_j \cdot \vec{v}_j$$
    Donc $\vec{v}_{i_0}$ est combinaison linéaire des autres vecteurs.
  
  \item Supposons que l'un des vecteurs soit combinaison linéaire des autres. Il existe donc $i$ entre $1$ et $r$ tel que 
    $$\vec{v}_i = \sum_{j=1, j \neq i}^{r} \alpha_j \cdot \vec{v}_j$$
    Il vient 
    $$\alpha_1 \cdot \vec{v}_1 + \ldots + \alpha_{i-1} \cdot \vec{v}_{i-1} - \vec{v}_i (\neq 0) + \alpha_{i+1} \cdot \vec{v}_{i+1} + \ldots + \alpha_r \cdot \vec{v}_r = \vec{0}$$
    Par conséquent $\vec{v}_1, \ldots, \vec{v}_r$ forment un système lié.
\end{itemize}

\paragraph*{Théorème} Soient $V$ un espace vectoriel et $(\vec{v}_1, \ldots, \vec{v}_R)$ un système libre de vectuers de $V$. Alors si un vecteur peut s'écrire comme combinaison linéaire des $\vec{v}_i$, alors son écriture comme tel est unique.

\paragraph*{Démonstration} Soiv $\vec{v} \in V$. Si 
\begin{eqnarray*}
  \vec{v} &=& \alpha_1 \cdot \vec{v}_1 + \ldots + \alpha_r \cdot \vec{v}_r, \alpha_i \in \R \\
  \vec{v} &=& \beta_1 \cdot \vec{v}_1 + \ldots + \beta_r \cdot \vec{v}_r, \beta_i \in \R 
\end{eqnarray*}
sont duex écritures de $\vec{v}$ comme combinaison liNöaire des $\vec{v}_i$, il vient
\begin{eqnarray*}
  \alpha_1 \cdot \vec{v}_1 + \ldots + \alpha_r \cdot \vec{v}_r &=& \beta_1 \cdot \vec{v}_1 + \ldots + \beta_r \cdot \vec{v}_r \\
  (\alpha_1 - \beta_1) \cdot \vec{v}_1 + \ldots + (\alpha_r - \beta_r) \cdot \vec{v}_r &=& \vec{0}
\end{eqnarray*}
comme les $\vec{v}_i$ forment un système libre, on a pour tout $i, \alpha_i - \beta_i = 0 ~\Leftrightarrow~ \alpha_i = \beta_i$.

\subsection{Interprétation géométrique de la dépendance linéaire}
Dans le plan $\R^2$, deux vectuers sont liés s'ils sont colinéaires. c'est-à-dire si l'un est égal à un multiple de l'autre. \\
Dans l'espace $\R^3$, trois vecteurs sont liés s'ils sont coplanaires, c'est-à-dire si le sous-espace vectoriel qu'ils engendrent est contenue dans un plan.

%
%
\section{Bases et dimension d'un espace vectoriel}
%
%
\paragraph{Définition} Soit $V$ un espace vectoriel. On appelle base de $V$ tout système de vecteurs de $V$ libre et générateur.

\paragraph{Example}
\begin{itemize}
  \item Dans $\R^3$, les vecteur $(1, 0, 0), (0, 1, 0) \text{ et } (0, 0, 1)$ forment une base.  De façon plus générale, dans $\R^n$, les n vecteurs:
    \begin{eqnarray*}
      (1, 0, \ldots, 0) \\
      (0, 1, \ldots, 0) \\
      \vdots \\
      (0, 0, \ldots, 1) 
    \end{eqnarray*}
    forment une base appelée base canonique de $\R^n$.
  \item Dans $M_{2\times 2}(\R)$ (espace vectoriel des matrices de taille $2\times 2$ à coefficients dans $\R$) les matrices 
    \begin{eqnarray*}
      F_{11} = \begin{pmatrix} 1 & 0 \\ 0 & 0 \end{pmatrix} \\
      F_{12} = \begin{pmatrix} 0 & 1 \\ 0 & 0 \end{pmatrix} \\
      F_{21} = \begin{pmatrix} 0 & 0 \\ 1 & 0 \end{pmatrix} \\
      F_{22} = \begin{pmatrix} 0 & 0 \\ 0 & 1 \end{pmatrix} \\
    \end{eqnarray*}
    forment une base appelée également base canonique de $M_{2 \times 2}(\R)$.
\end{itemize}

\paragraph{Example} Montrons que le système de vectuers de $\R^3$ 
$$\left( (1, 2, 1), (2, 7, 0), (3, 1, -1) \right)$$
est une base de $\R^3$. \\
Voir que ce système est génératuer consiste à voire que tout $(a, b, c) \in \R^3$ est combinaison linéaire de ces 3 vecteurs, autrement dit que pour tout $(a, b, c) \in \R^3$ il existe $\lambda_1, \lambda_2, \lambda_3 \in \R$ tel que 
$$(a, b, c) = \lambda_1 \cdot (1, 2, 1) +\lambda_2 \cdot (2, 7, 0) + \lambda_3 \cdot (3, 1, -1)$$
autrement dit, que le système linéaire suivant
$$\begin{pmatrix}
  1 & 2 & 3 \\
  2 & 7 & 1 \\
  1 & 0 & -1
\end{pmatrix}
\begin{pmatrix}
  x \\
  y \\
  z 
\end{pmatrix}
=
\begin{pmatrix}
  a \\
  b \\
  c
\end{pmatrix}$$
admet au moins une solution. Voir que le système est libreconsiste à voir que, pour tous $\lambda_1, \lambda_2, \lambda_3 \in \R$, 
$$\lambda_1 \cdot (1, 2, 1) +\lambda_2 \cdot (2, 7, 0) + \lambda_3 \cdot (3, 1, -1) = (0, 0, 0)$$
alors $\lambda_1=0, \lambda_2=0, \lambda_3=0$ autrement dit uqe le système linéaire
$$\begin{pmatrix}
  1 & 2 & 3 \\
  2 & 7 & 1 \\
  1 & 0 & -1
\end{pmatrix}
\begin{pmatrix}
  x \\
  y \\
  z 
\end{pmatrix}
=
\begin{pmatrix}
  0 \\
  0 \\
  0
\end{pmatrix}$$
admet une unique solution, à savoir $(0, 0, 0$. \\
Ainsi, voire que ce système est une base est équivalent à voir que la matrice de ce système est inversible. Pour cela, calculons le déterminant de cette matrice:
$$\begin{vmatrix}
  1 & 2 & 3 \\
  2 & 7 & 1 \\
  1 & 0 & -1
\end{vmatrix}
=
1 \cdot 
\begin{vmatrix}
  2 & 3 \\
  7 & 1
\end{vmatrix}
-1 \cdot
\begin{vmatrix}
  1 & 2 \\
  2 & 7
\end{vmatrix}
= -19 -3 = -22 \neq 0$$
Cette matrice est donc inversible, par conséquent ce système est une base de $\R^3$. \\
Plus généralement, un système de $n$ vectuers de $\R^n$ est une base d $\R^n$ si et seulement si la matrice dont les colonnes sont les composantes de ces vecteurs est inversible.

\paragraph{Théorème} Soit $V$ espace vectoriel pssédant une base $(\vec{v}_1, \ldots, \vec{v}_n)$. Alors tout vectuer de $V$ s'écrit de façon unique comme combinaison linéaire des $\vec{v}_i$
$$\vec{v} \in V, \vec{v}=\alpha_i \cdot \vec{v}_1 + \ldots + \alpha_r \cdot \vec{v}_n, \alpha_i \in \R$$
Les réels $\alpha_i$ s'appellent les coordonnées de $\vec{v}$ dans la base $(\vec{v}_1, \ldots, \vec{v}_n)$.
\paragraph{Démonstration} Comme le système $(\vec{v}_1, \ldots, \vec{v}_n)$ est générateur il existe  $\alpha_1, \ldots, \alpha_n \in \R$ tels que $\vec{v}=\alpha_i \cdot \vec{v}_1 + \ldots + \alpha_r \cdot \vec{v}_n$. Comme ce système est libre, les $\alpha_i$ sont uniques.

\paragraph{Définition} Soit $V$ un espace vectoriel. Si $S$ est un système libre (resp. système générateur, base) de $V$, on appelle cardinal de $S$ et on note $card(S)$ le nombre de vecteurs de $S$.

\paragraph{Remarque} Dans la suite, nous ne considérerons que des systèmes de cardinal fini.

\paragraph*{Définition} On dit qu'un espace vectoriel $V$ est de type fini s'il admet un système générateur de cardinal fini.

\paragraph{Théorème} Soit $V$ un espace vectoriel de type fini. Alors
\begin{enumerate} 
  \item De tout sysème génératuer de $V$, on peut extraire une base.
  \item On peut compléter un système libre par des vectuers d'un système générateur pour obtenir une base de $V$.
  \item Si $S$ est un système libre de vecteurs de $V$ et $T$ un système générateur, alors on a $card(S) \leq card(T)$.
\end{enumerate}

\paragraph{Démonstration}
\begin{enumerate}
  \item Soit $(\vec{v}_1, \ldots, \vec{v}_n)$ un système génératuer de vecteurs de $V$- On extrait de S une base de $V$ en procédant comme suit:
    \begin{itemize}
      \item Soit $i_1$ l'indice du premier vecteur non nul de $S$.
      \item Si $\vec{v}_{i_1 + 1}$ est multiple de $\vec{v}_{i_1}$ on ne retient pas $\vec{v}_{i_1 + 1}$. Sinon, on le retient et on le note $\vec{v}_{i_2}$.
      \item Si $\vec{v}_{i_1 + 2}$ est combinaison linéaire du ou des vectuers sélectionnés auparavant, on ne le retien pas, sinson on le retien, et on le note $\vec{v}_{i_2}$ si auparavant on a sélectionné seulement $\vec{v}_{i_1}$, ou $\vec{v}_{i_3}$ si on a sélectionné $\vec{v}_{i_1}$ et $\vec{v}_{i_2}$.
      \item Ainsi de suite ...
    \end{itemize}
    Au terme du processus, on obient un système $T = (\vec{v}_{i_1}, \vec{v}_{i_2}, \ldots, \vec{v}_{i_r})$ de vectuers extrait de $S$. Ce système est libre et générateur (un vectuer de $S$ appartient à $T$ ou bien est combinaison linéaire des vecteurs de $T$). Par suite, $T$ est une base de $V$.
  
  \item Soient $S=(\vec{u}_1, \ldots, \vec{u}_r)$ un système libre de vecteurs de $V$ et $T=(\vec{v}_1, \ldots, \vec{v}_r)$ un système générateur. On effectue le processus suivant:
    \begin{itemize}
      \item Si $\vec{v}_1$ est combinaison linéaire des  vectuers de $S$, on ne le retien pas, sinon on le retient on obtient alors un nouveau système $S_1$.
      \item Si $\vec{v}_2$ est combinaison linéaire des vecteurs de $S_1$, on ne le retient pas, sinon on le retient. On obtient un nouveau système $S_2$.
    \end{itemize}
    Au terme de ce processus, on obtient un système $S_r$ de vecteurs de $V$ obtenu en completant $S$ par des vectuers de $T$. $S_r$ est libre et générateur, donc est un base de $V$.
    
    \item On note $S = (\vec{u}_1, \ldots, \vec{u}_r)$ et $T = (\vec{v}_1, \ldots, \vec{v}_s)$ quitte à extraire une base de T, on peut supposer que $T$ est une base. Pout tout $j \in \{1; \ldots ; r\}$ on a
      $$\vec{u}_j = \sum_{i=1}^{j} {a_{ij} \vec{v}_i}$$
      Raisonnons par l'absurde, et supposons que $s < r$. Pour tous $\lambda_1, \ldots, \lambda_r \in \R$, on a
      \begin{eqnarray*}
        \sum_{j=1}^r { \lambda_j \vec{u}_j} &=& \sum_{j=1}^r {\lambda_j \left( \sum_{i=1}^r {a_{ij}\vec{v}_i} \right)} \\
          & =& \sum_{i=1}^s {\left( \sum_{j=1}^r {a_{ij}\lambda_j} \right) \vec{v}_i
      \end{eqnarray*}
      On a $\sum_{j=1}^r {\lambda_j \vec{u}_j } = \vec{0}$ si eut seulement si pour tout $i$, $\sum_{j=1}^r {a_{ij} \lambda_j } = 0$ car $T$ est en particulier un système libre. Puisque $s < r$, le nombre d'équations du système
      $$ \left\{ \begin{matrix}
        a_{11} \lambda_1 + \ldots + a_{1r} \lambda_r = 0 \\
        \vdots \\
        a_{s1} \lambda_1 + \ldots + a_{sr} \lambda_r = 0
      \end{matrix} $$
      est strictement inférieur au nombre d'inconnues. Puisque le système admet au moins une solution, à savoir $(0, \ldots, 0) \in \R^r$, il admet une infinité de solutions. \\
      Soit $(\lambda_1, \ldots, \lambda_r) \neq (0, \ldots, 0)$ une solution de ce système. On a alors
      $$\sum_{j=1}^r {\lambda_j \vec{u}_j } = \vec{0}$$
      donc $S$ est lié, ce qui donne lien à une contradiction.
\end{enumerate}

\paragraph{Corollaire} Tout espace vectoriel de type fini, admet une base de cardinal fini.
\paragraph{Démonstration} Si $S$ est un système générateur de cet espace de cardinal fini, on en extrait une base de cardinal fini.

\paragraph{Corollaire} Soit $V$ un epsace vectoriel de type fini. Alors toutes les bases de $V$ ont même cardinal.
\paragraph{Démonstration} Soient $\B$ et $\B'$ deux bases de $V$. En particulier $\B$ est un système libre, et $\B'$ un système générateur. On a donc $card(\B) \leq card(\B')$. \\
En échangeant les rôles de $\B$ et $\B'$, on obtient l'inégalité $card(\B') \leq card(\B)$, d'où $card(\B) = card(\B')$.

\paragraph{Définition} Soit $V$ un epsace vectoriel de type fini. On appelle dimension de $V$ et on note $dim_\R(V)$ (ou simplement $dim(V)$) le cardinal d'une base quelconque de $V$.
\paragraph{Examples}
\begin{itemize}
  \item L'espace vectoriel $\{\vec{0}\}$ réduit à $\vec{0}$ a pour dimension $0$.
  \item Pout toute entier $n \geq 1$, $\R^n$ est de dimension $n$.
  \item Pour tout entier $n \geq 1$, l'espace vectoriel $P_n$ des polynômes de degré $\leq n$ admet pour base le système $(1, X, X^2, \ldots, X^n)$. \\ 
    Par conséquent, $dim(P_n) n+1$.
  \item On a $dim(M_{2\times 2}(\R) ) =4$. \\
    Plus généralement, si $n\leq 1$ et $n\leq 1$ sont deux entires naturels, on a $dim(M_{n\times m}(\R) ) = n\cdot m$
  \item L'espace vectoriel$\R[X]$ n'est pas de dimension finie. Il admet pour base le système $(X^m)_{n\in \N$ que est de cardinal infini.
\end{itemize}

\paragraph{Théorème} Soit $V$ un espace vectoriel de dimension $n$. Soit $S$ un système de $n$ vecteurs de $V$. Alors les affirmations suivantes sont équivalentes.
\begin{enumerate}
  \item $S$ est une base de $V$;
  \item $S$ engendre $V$;
  \item $S$ est libre.
\end{enumerate}

\paragraph{Théorème} Soit $V$ un espace vectoriel de dimension $n$. Alors 
\begin{enumerate}
  \item Si $S$ est un système libre de vecteurs de V,  $card(S) \leq n$.
  \item Si $T$ est un système générateur de $V$, alors $card(T) \geq n$.
\end{enumerate}
\paragraph{Démonstration} Soit $\B une base de V$.
\begin{enumerate}
  \item Comme $\B$ est en particulier un système générateur de $V$ on a $card(s) \leq card(\B) = n$.
  \item Comme $\B$ est un particulier un système libre, on a $n = card(\B) \leq card(T)$.
\end{enumerate}

\paragraph{Démonstration du théorème précèdent}
\begin{enumerate}
  \item $\Rightarrow$ 2. est clair par définition d'une base
  \item $\Rightarrow$ 3. Raisonnons par l'absurde et supposons que $S$ est lié. Alors l'un des vectuers de $S$ est combinaison linéaire des autres. Notons $vec{v}$ un tel vecteur de $S$. \\
    Alors $S - \{ \vec{v}\}$ est encore un système générateur de $V$. Mais $card(S - \{\vec{v}\}) = n-1$, ce qui contredit le théorème précédent.
  \item $\Rightarrow$ 1. Il s'agit de voir que $S$ engendre $V$. On raisonne par l'absurde, et on suppose que $S$ n'engendre pas $V$. On écrit $S = (\vec{v}_1, \ldots, \vec{v}_n)$. Soit $\vec{v} \in V$ tel que $\vec{v}$ ne soit pas combinaison linéaire des $\vec{v}_i$. \\
    Alors le système $S \cup \{ \vec{v}\}$ est encore libre. Soient $\lambda_1, \ldots, \lambda_n, \lambda \in \R$ tels que 
    $$\lambda_1 \cdot \vec{v}_1 + \ldots + \lambda_n \cdot \vec{v}_n + \lambda \cdot \vec{v} = \vec{0}$$
    si $\lambda \neq 0$, on pourrait écrire 
    $$\vec{v} = -\frac{\lambda_1}{\lambda} \cdot \vec{v}_1 \ldots - \frac{\lambda_n}{\lambda} \cdot \vec{v}_n$$
    ce qui contredirait l'hypothèse faite sur $\vec{v}$. Donc $\lambda = 0$. Il vient alors
    $$\lambda_1 \cdot \vec{v}_1 + \ldots + \lambda_n  \cdot \vec{v}_n = \vec{0}$$
    Puisque $S$ est libre, on a $\lambda_1 = 0, \ldots, \lambda_n = 0$. Or $card(S \cup \{\vec{v}\} ) = n+1$, ce qui contredit le théorème précédent.
\end{enumerate}

\paragraph{Théorème} Soient $V$ un espace vectoriel de dimension $n$ et $W$ un sous-espace vectoriel de $V$. Alors 
$$dim(W) \leq dim(V)$$
  Si $dim(W) = dim(V)$, alors $W = V$.
  
\paragraph{Démonstration} Soit $\B$ une base de $W$. Alors $\B$ est en particulier un système libre de vecteurs de $V$. Donc 
$$dim(W) = card(\B) \leq n = dim(V)$$
Si $dim(W) = dim(V)$, alors $\B$ est un système libre de $n$ vecteurs de $V$, donc  $\B$ engendre $V$. Par suite 
$$W = Vect(\B) = V$$

%
%
\section{Systèmes linéaires homogènes}
%
%
\paragraph{Proposition} Soit $m \geq 1$ un entier naturel. Alors une application $f: \R^m \rightarrow \R$ est linéaire si et seulement s'il existe $a_1, \ldots, a_m \in \R$ tels que pour tout $(x_1, \ldots, \x_m) \in \R^m$, on a
$$f(x_1, \ldots, \x_m) = a_1 \cdot x_1 + \ldots + a_m \cdot x_m$$

\paragraph{Démonstration}
\begin{itemize}
  \item Si $f: \R^m \rightarrow \R$ est une applicaiton de la forme $f(x_1, \ldots x_m) = a_1 \cdot x_1 + \ldots + a_m \cdot x_m$ pour tout $(x_1, \ldots, x_m) \in \R^m$, avec $a_i \in \R$, alors $f$ est linéaire. En effet, pour tous $(x_1, \ldots, x_m), (y_1, \ldots, y_m) \in \R^m$ et $\alpha \in \R$ on a
    \begin{eqnarray*}
      f(\alpha(x_1, \ldots, x_m) + (y_1, \ldots, y_m)) &=& f(\alpha x_1 + y_1, \ldots, \alpha x_m + y_m) \\
        &=& a_1 \cdot (\alpha x_1 + y_1) + \ldots + a_m \cdot (\alpha x_m + y_m) \\
        &=& \alpha (a_1 x_1 + \ldots + a_m x_m) + (a_1 y_1 + \ldots + a_m y_m) \\
        &=& \alpha f(x_1, \ldots, x_m) + f(y_1, \ldots, y_m).
    \end{eqnarray*}
  \item Réciproquement, si $f: \R^m \righarrow \R$ est linéaire, pout tout i, on pose 
    $$a_i = f(0_1, \ldots, 0_{i-1}, 1_i, 0_{i+1}, \ldots, 0_m) \in \R$$
    Alors pout tout $(x_1, \ldots, x_m) \in \R^m$
    \begin{eqnarray*}
      f(x_1, \ldots, x_m) &=& f(x_1 (1, 0, \ldots, 0) + \ldots + x_m (0, \ldots, 0, 1)) \\
        &=& x_1 f(1, 0, \ldots, 0) + \ldots + x_m f(0, \ldots, 0, 1) \\
        &=& a_1 x_1 + \ldots + a_m x_m
    \end{eqnarray*}
\end{itemize}

\paragraph*{Définition} On appelle équation linéaire à $m$ inconnues une équation de la forme $f(x_1, \ldots, x_m) = a$, où $f: \R^m \rightarrow \R$ est une application linéaire et $a \in \R$. \\
On dit que ctte équation est homogène si $a=0$. \\
Cela justice la terminologie d'équation linéaire et de système utilisée jusqu'ici.

\paragraph{Théorème} L'ensemble des solutions d'un système linéaire homogène (c'est-à-dire tel que toutes les équations du système soient linéaires homogènes) est un sous-espace vectoriel de $\R^m$.
\paragraph{Démonstration} Écrivons ce système sous la forme:
$$ \left\{ \begin{matrix}
  f_1(x_1, \ldots, x_m) = 0\\
  \vdots \\
  f_n(x_1, \ldots, x_m = 0
\end{matrix}$$
pour tout $i \in \{1; \ldots; n\}$, montrons que l'ensemble $\mS_i$ des solutions est un ssous-espace vectoriel de $\R^m$. \\
Puisque $f_i$ est linéaire, $f_i(0, \ldots, 0) = 0$, donc $(0,  \ldots, 0) \in \mS_i$ et $\alpha \in \R$. Il vient 
$$f_i(\alpha \cdot x + y) = \alpha f_i(x) + f_i(y) = \alpha \cdot 0 
+ 0 = 0$$
donc $\alpha \cdot x + \beta \cdot y \in \mS_i$ \\
Maintenant, l'ensemble $\mS$ des solutions du système est 
$$\mS =  \bigcap_{i=1}^n {\mS_i}$$
Par conséquent, $\mS$ est un sous-espace vectoriel de $\R^m$.

%
%
\section{Applications linéaires}

\paragraph{Rappel} Soient $E$ et $F$ deux ensembles et $f: E \rightarrow F$ une application de $E$ dans $F$.
\begin{itemize}
 \item On dit que $f$ est injective si pour tous $x, x' \in E$, si $f(x) = f(x')$ alors $x=x'$. De façon équivalente, $f$ est injective si pour tous $x, x' \in E$, si $x \neq x'$, alors $f(x) \neq f(x')$.
  \item On dit que $f$ est surjective si pour tout $y \in F$, il existe $x \in E$ tel que $f(x) = y$. 
  \item On dit que $f$ est bijective si $f$ est injective et surjective.
  \item On montre que $f$ est bijective si et seulement s'il existe une application $g: F \rightarrow E$ telle que $g \circ f = id_E$ et $f \circ g = id_F$. Dans ce case $g$ s'appelle la bijection réciproque de $f$.
  \item De façon générale, si $x \in E$ et $y \in F$ tels gue $y = f(x)$, on dit que $x$ est un antécédent de $y$ par $f$.
\end{itemize}

\paragraph{Définition} Soient $E$ et $F$ deux espaces vectoriels et $f: E \rightarrow F$ une application linéaire de $E$ dans $F$. On appelle \underline{noyau} de $f$, et on note $Ker(f)$ le sous-ensemble de $E$ définit par
$$Ker(f) = \left\{ x\in E \vert f(x) = 0_F \right\}$$

\paragraph{Example} Si $f: \R^m \rightarrow \R$ est une application linéaire, l'ensemble des solutions de l'équation linéaire homogène
$$f(x_1, \ldots, x_m) = 0$$
est le noyau de $f$.

\paragraph{Proposition} Si $f: E \rightarrow F$ est une application linéaire. Alors $Ker(f)$ est un sous-epsace vectoriel de $E$.

\paragraph{Démonstration} Puisque $f$ est linéaire, $f(\vec{0}_E) = \vec{0}_F$ donc $\vec{0}_E \in Ker(f)$ et par conséquent $Ker(f) \neq \emptyset$. \\
Soient $\vec{x}, \vec{y} \in Ker(f)$ et $\alpha, \beta \in \R$.
$$f(\alpha \vec{x} + \beta \vec{y}) = \alpha f(\vec{x}) + \beta f(\vec{y}) = \vec{0}_F$$
donc $\alpha \vec{x} + \beta \vec{y} \in Ker(f)$.

\paragraph{Théorème} Soit $f: E \rightarrow F$ une application linéaire. Alors
$$f \text{est injective} \Leftrightarrow Ker(f) = \{\vec{0}_E\}$$
\paragraph{Démonstration} 
\begin{itemize}
  \item[$\Rightarrow$] On suppose que $f$ est injective. Puisque $f$ est linéaire, $f(\vec{0}_E) = \vec{0}_F$, donc $\vec{0}_E \in Ker(f)$. Soit $\vec{v} \in E, \vec{v}\neq \vec{0}_E$. Puisque $f$ est injective, $f(\vec{v}) \neq f(\vec{0}_E) = \vec{0}_F$. Par conséquent, $Ker(f) = \{\vec{0}_E\}$.
  \item[$\Leftarrow$] Supposons que $Ker(f) = \{\vec{0}_E\}$. Soient $\vec{u}, \vec{v} \in E$ tels que $f(\vec{u}) = f(\vec{v})$. Puisque $f$ est linéaire, on a $f(\vec{u} - \vec{v}) = f(\vec{u}) - f(\vec{v}) = \vec{0}_F$. Par suite, $\vec{u} - \vec{v} \in Ker(f)$, donc $\vec{u}-\vec{v} = \vec{0}_E$, c'est-à-dire $\vec{u} = \vec{v}$.
\end{itemize}
Donc $f$ est injective.

\paragraph{Remarque} Si $f: E \rightarrow F$ est une application linéaire, alors pour tout $y \in F$, si $x \in E$ est un antécédent de $y$ par $f$, autrement dit $y = f(x)$, l'ensemble des antecédents de $y$ par $f$ est
$$x + Ker(f) = \left\{ x+z \vert z \in Ker(f) \right\}$$
Autrement dit, si $x \in E$, alors pour tout $x' \in E, f(x') = f(x) \Leftrightarrow x' \in x + Ker(f)$.

\paragraph*{Définition} Soit $f: E \rightarrow F$ une application linéaire. On appelle image de $f$ le sous-esemble de $F$ noté $Im(f)$ défini par
$$Im(f) = \left\{ y \in F \vert \text{ il existe } x \in E \text{ tel que } y = f(x) \right\}$$

\paragraph{Proposotion} Soit $f: E \rightarrow F$ une application linéaire. Alors $Im(f)$ est un sous-espace vectoriel de $F$.
\paragraph{Démonstration} On a $\vec{0}_f = f(\vec{0}_E) \in Im(f)$, donc $Im(f) \neq \emptyset$. Soient $\vecy}, \vec{y}' \in Im(f)$ et $\alpha, \beta \in \R$. Il existe $\vec{x}, \vec{x}' \in E$ tels que $\vec{y} = f(\vec{x})$ et $\vec{y}' = f(\vec{x}')$. Il vient
$$\alpha \vec{y} + \beta \vec{y}' = \alpha f(\vec{x}) + \beta f(\vec{x}') = f(\alpha \vec{x} + \beta \vec{x}') \in Im(f)$$

\paragraph{Théorème} Soit $f: E \rightarrow F$  une application linéaire. Alors $f$ est surjective si et seulement si $Im(f) = F$.

\paragraph{} On rappelle que si $E$ et $F$ sont deux espaces vectoriels, alors un isomorphisme (d'espaces vectoriels) de $E$ dans $F$, s'il en existe, est une application de $E$ dans $F$ linéaire et bijective. \\
On dit que $F$ est \underline{isomorphe} à $E$ s'il existe un isomorphisme d'espaces vectoriels de $E$ dans $F$.

\paragraph{Théorème} Soit $f: E \rightarrow F$  une application linéaire. Alors $f$ est un isomorphisme si et seulement si $Ker(f) = \{\vec{0}_E\}$ et $Im(f) = F$.

\paragraph{Proposition} Soient $E$ et $F$ deux espaces vectoriels. On suppose que $E$ est de dimension finie $n$. Soient $\B = (\vec{u}_1, \ldots, \vec{u}_n)$ une base de E et $\vec{v}_1, \ldots, \vec{v}_n$ $n$ vecteurs de $F$. Alors il existe une unique application linéaire $f: E \rightarrow F$ telle que pour tout $i$, $f(\vec{u}_i) = \vec{v}_i$.

\paragraph{Démonstration}
\begin{itemize}
  \item[Existence:] Soit $\vec{u} \in E$. Alors \vec{u} s'écrit de façon unique sous la forme
    $$\vec{u} = \alpha_1 \vec{u}_1 + \ldots + \alpha_n \vec{u}_n, \alpha_i \in \R$$
    On pose alors $f(\vec{u}) = \alpha_1 \vec{v}_1 + \ldots + \alpha_n \vec{v}_n \in F$. On a donc défini une application $f: E \rightarrow F$ telle que pour tout $i$, $f(\vec{u}_i) = \vec{v}_i$ puisque
    $$\vec{u}_i = 0 \cdot \vec{u}_1 + \ldots + 1 \cdot \vec{u}_i + \ldots \vec{u}_n$$
    Montrons que $f$ est linéaire. Soient $\vec{u}, \vec{u}' \in E$ et $\alpha \in \R$. Écrivons 
    \begin{eqnarray*}
      \vec{u} &=& \alpha_1 \vec{u}_1 + \ldots + \alpha_n \vec{u}_n \\
      \vec{u}' &=& \alpha_1' \vec{u}_1 + \ldots + \alpha_n' \vec{u}_n \\
      \alpha \vec{u} + \vec{u}' &=& (\alpha \alpha_1 + \alpha_1') \vec{u}_1 + \ldots + (\alpha \alpha_n + \alpha_n') \vec{u}_n \\
    \end{eqnarray*}
    Cette écriture est bien la décomposition de $\alpha \vec{u} + \vec{u}'$ dans öa base $B$ par unicité de cette décomposition. On a
    \begin{eqnarray*}
      f(\alpha \vec{u} + \vec{u}') &=& (\alpha \alpha_1 + \alpha_1') \vec{v}_1 + \ldots + (\alpha \alpha_n + \alpha_n') \vec{v}_n \\
        &=& \alpha ( \alpha_1 \vec{v}_1 + \ldots + \alpha_n \vec{v}_n ) + (\alpha_1' \vec{v}_1 + \ldots + \alpha_n' \vec{v}_n') \\
        &=& \alpha f(\vec{u}) + f(\vec{u}')
    \end{eqnarray*}
    
  \item[Unicité:] Si $f: E \rightarrow F$ est une application telle que pour tout i, $f(\vec{u}_i) = \vec{v}_i$, alors on a nécessairement pour tout $\vec{u} = \alpha_1 \vec{u}_1 + \ldots + \alpha_n \vec{u}_n \in E$,
    \begin{eqnarray*}
      f(\vec{u}) &=& f(\alpha_1 \vec{u}_1 + \ldots + \alpha_n \vec{u}_n) \\
       &=& \alpha_1 f(\vec{u}_1) + \ldots + \alpha_n f(\vec{u}_n) \\
       &=& \alpha_1 \vec{v}_1 + \ldots + \alpha_n \vec{v}_n.
    \end{eqnarray*}
    Par conséquent, une telle application linéaire est unique.
\end{itemize}
Ainsi, une application linéaire $f: E \rightarrow F$ est entièrement détermineé par la donnée de ses valeurs sur une base de $E$.

\paragraph{Théorème} Soient $E$ et $F$ deux espaces vectoriels de dimension finie.et $f: E \rightarrow F$ une application linéaire. Alors $f$ es tun isomorphisme si eut seulement si, pour toute base $(\vec{u}_1, \ldots, \vec{u}_n)$ de $F$, $(f(\vec{u}_1), \ldots, f(\vec{u}_n))$ est une base de $F$.

\paragraph{!!!TODO: Une heure manque!!!}

\paragraph{Corrolaire}
\begin{enumerate}
  \item Tout espace vectoriel de dimension finie $n$ est isomorphe à $\R^n$.
  \item Duex espaces vectorielsde dimensions finies sont isomorphes si et seulement si ils ont même dimension.
\end{enumerate}

%
%
\section{Espace vectoriel des solutions d'un système linéaire homogène}
%
%
\paragraph{Rappel} On rappelle que si $S$ est un système linéaire homogène à $m$ inconnues, alors l'ensemble des solutions de $S$ est un sous-espace vectoriel de $\R^m$. On cherche à détéerminer une base et la dimension de cet espace.

\paragraph{Example} Considérons le système linéaire homogène
$$\left\{\begin{array}{cccc}
   x & + y & -3z & = 0\ \
  2x & +3y & +2z & = 0
\end{array}$$
ce système. L matrice augmentée est
$$\begin{pmatrix}
 1 & 1 & -3 \\
 2 & 3 & 2
\end{pmatrix}$$
Il vient
\begin{eqnarray*}
  \begin{pmatrix} 
    1 & 1 &-3 \\ 
    2 &3 & 2
  \end{pmatrix} 
  & \rightarrow^{L_2 \leftarrow L_2 - 2L_1} &
  \begin{pmatrix}
    1 & 1 &-3 \\ 
    0 & 1 & 8
  \end{pmatrix} 
  \\
  & \rightarrow^{L_1 \leftarrow L_1 - L_2} &
  \begin{pmatrix} 
    1 & 0 & -11 \\ 
    0 & 1 & 8
  \end{pmatrix}
\end{eqnarray*}
On a 
$$S \Leftrightarrow \left\{\begin{array}{cccc} x & & -11z & = 0 \\ & y & +8z & =0\end{array}$$
$x$ et$y$ sont les inconnues principales, $z$ est inconnue secondaire. \\
L'ensemble $\mS$ des solutions de $S$ est $\mS = {(11t, -8t, t) \in \R^3 \vert t \in \R \}$.\\
Consideérons l'application
\begin{eqnarray*}
  f: \R &\rightarrow& \mS \\
  t &\mapsto & (11t, -8t, t)
\end{eqnarray*}
On vérifie que:
\begin{itemize}
  \item $f$ est linéaire.
  \item $f$ est surjective car toute solutions de $S$ est obtenu pour une certain valeur de t.
  \item $f$ est injective car $Ker(f) = \{0\}$.
\end{itemize}
Par conséquent, $f$ est un isomorphisme de $\R$ dans $\mS$. \\
Par suite l'image par $f$ d'une base de $\R$ est une base de $\mS$. Si l'on prend la base canonique $(1)$ de $\R$, alors $(f(1)) = ((11, -8, 1))$ est une base de $\mS$. On peut faire apparaître cette base en écrivant $(11t, -8t, t) = t (11, -8, 1)$. Ici, $\mS$ est donc de dimension $1$. C'est une droite vectorielle.

\paragraph{Autre example}
$$S = \left\{ x - 3y + 2z = 0$$
$x$ est inconnue principale, et $y$ et $z$ sont inconnues seconaires. Alors l'ensemble $\mS$ des solutions de $S$ est $\mS = \{(3u - 2v, u, v) \in \R^3 \vert u, v \in \R \}$. \\
L'application 
\begin{eqnarray*}  
  g: \R^2 &\rightarrow& \mS \\
    (u, v) & \mapsto & (3u - 2v, u, v)
\end{eqnarray*}
est un isomorphisme d'espaces vectoriels. Alors 
$$(g(1, 0), g(0, 1)) = \left( (3, 1, 0), (-2, 0, 1) \right)$$ 
est une base de $\mS$ que l'on peut mettre en évidence en écrivant
$$\mS = \left\{ u(3, 1, 0) + v(-2, 0, 1) \vert u, v \in \R \right\}$$
De plus $\mS$ est de dimension $2$: c'est un plan vectoriel.
  
%
%
\section{Réprésentation matricielle d'un vecteur dans une base}
%
%
\paragraph{} Soit $V$ un esapace vectoriel de dimsension finie $n$. Soit $\B = (\vec{v}_1, \ldots, \vec{v}_n)$ une base de $V$. Si $\vec{v} \in V$, on peut écrire 
$$\vec{v} = \alpha_1 \vec{v}_1 + \ldots + \alpha_n \vec{v}_n$$
où les $\alpha_i$ sont les coordonnées de $\vec{v}$ dans la base $\B$.

\paragraph{Définition} On appelle  matrice de $\vec{v}$ dans la base $\B$ la matrice colonne 
$$[\vec{v}]_{\B} = 
\begin{pmatrix} 
  \alpha_1 \\ 
  \vdots \\ 
  \alpha_n 
\end{pmatrix} 
\in M_{n, 1}(\R)$$

\paragraph{Example} Soient $(1, -2, 3) \in \R^3$ et $\T$ la base canonique de $\R^3$. Alors
$$[\vec{v}]_{\B} = 
\begin{pmatrix} 
  1 \\ 
  -2 \\ 
  3 
\end{pmatrix}$$

\paragraph{Proposition} 
\begin{enumerate}
  \item Pour tous $\vec{u}, \vec{v} \in V$, 
    $$[\vec{u} + \vec{v}]_{\B} = [\vec{u}]_\B + [\vec{v}]_\B$$
  \item Pour tous $\vec{v} \in V$ et $\lambda  \in \R$, 
    $$[\lambda \vec{v}]_{\B} = \lambda [\vec{v}]_\B$$
  \item L'application
    \begin{eqnarray*}
      V &\rightarrow& M_{n, 1}(\R) \\
      \vec{v} &\mapsto& [\vec{v}]_\B
    \end{eqnarray*}
    est un isomorphisme d'espaces vectoriels.
\end{enumerate}
L'isomorphisme précédent fournit un lien de nature linéaire entre l'espace vectoriel $V$ "abstrait" et l'espace vectoriel $M_{n, 1}(\R)$ "concret". \\
En particulier, lorsque l'on voudra tester des propriétés de nature linéaire sur $V$ ("être un système libre", "être un système générateur"), on pourra le faire sur $M_{n, 1}(\R)$ vie le choix d'une base de $V$. \\
Plus généralement, si $S=(\vec{v}_1, \ldots, \vec{v}_m)$ est un système de vecteurs de $V$, on peut former la matrice de $S$ dans $\B$.
$$M_{\B S} = \bigg( [\vec{v}_1]_\B, \ldots, [\vec{v}_m]_\B \bigg) \text{ de taille } n \times m$$

\paragraph{Théorème} Soit $V$ un espace vectoriel de dimension $n$. Soient $S = (\vec{v}_1, \ldots, \vec{v}_n)$ un système de $n$ vecteurs de $V$, et $\B$ une base de $V$. Alors $S$ est une base de $V$ si et seulement si la matrice carrée $M_{\B S}$ est inversible.

\paragraph{Example} Soit $\mP_2$ l'espace vectoriel des polynômes de degré $\leq 2$. Soit le système
$$S = (1, 1 + X, 1 + X + X^2)$$
On a 
$$M_{\T S} = \begin{pmatrix}
  1 & 1 & 1 \\
  0 & 1 & 1 \\
  0 & 0 & 1
\end{pmatrix}$$
où $\T$ désigne la base canonique $\big(1, X, X^2 \big)$ de $\mP_2$. Comme $M_{\T S}$ est inversible, S est une base de $\mP_2$.
\\\\
Soient $V$ un espace vectoriel de dimension $n$ et $\B = (\vec{v}_1, \ldots, \vec{v}_n)$ et $\B' = (\vec{v}'_1, \ldots, \vec{v}'_n)$ deux bases de $V$.
\paragraph{Définition} On appelle \underline{matrice de passage} de $\B$ à $\B'$ la matrice carrée $P_{\B \B'}$ du système $\B'$ dans la base $\B$. Du fait que $\B'$ est une base de $V$, $P_{\B \B'}$ est inversible.

\paragraph{Théorème (formule de changement de base)} Pour tout $\vec{v} \in V$, on a 
$$[\vec{v}]_\B = P_{\B \B'} \cdot [\vec{v}]_{\B'}$$

\paragraph{Remarque} Cette formule indique que l'on obtient les coordonnées de $\vec{v}$ dans l'ancienne base $\B$ en fonction des coordonnées de $\vec{v}$ dans la nouvelle base $\B'$, alors que l'on souhaiterait plutôt l'inverse. Pour ce faire, il faut calculer $P_{\B \B'}$. En fait on a
$${P_{\B \B'}}^{-1} = P_{\B \B'}$$
En effet, on a
\begin{eqnarray*}
  P_{\B \B'} \cdot P_{\B' \B} \cdot [\vec{v}]_{\B} &=& P_{\B \B'} \cdot [\vec{v}]_{\B'} \\
   &=& [\vec{v}]_{\B}
\end{eqnarray*}
pour tout $\vec{v} \in V$. \\
Ceci implique que $P_{\B \B'} \cdot P_{\B' \B} = I_n$. On obtient de même $P_{\B' \B} \cdot P_{\B \B'} = I_n$. On aura alors
$$[\vec{v}]_{\B'} = P_{\B' \B} \cdot [\vec{v}]_{\B} = {P_{\B \B'}}^{-1} \cdot [\vec{v}]_{\B}$$
Si $\B, \B', \B''$ sont trois bases de $V$, alors
$$P_{\B \B''} = P_{\B \B'} \cdot P_{\B' \B''}$$

\paragraph{} Considérons le problème suivant: Soient $V$ un espace vectoriel de dimension $n$, $W$ un sous-espace vectoriel de $V$, $S = (\vec{v}_1, \ldots, \vec{v}_m)$ un système générateur de $V$ de sorte que $W = Vect(\vec{v}_1, \ldots, \vec{v}_m)$, et $\B$ une base de $V$. On cherche à extraire de $S$ une base de $W$. On peut procéder comme suit:
\begin{itemize}
  \item on construit une suite $S_0, S_1, \ldots, S_m$ de système de vecteurs de $W$, $S_0 = \{\vec{0}\}$
  \item si $\vec{v}_1 = \vec{0}$ on pose $S_1 = S_0$, \\
    sinon, on pose $S_1 = \{\vec{v}_1\}$
  \item si $\vec{v}_2$ est combinaison linéaire des vecteurs de $S_1$, on pose $S_2 = S_1$, \\
    sinon, on pose $S_2 = S_1 \cup \{\vec{v_2}\}$.
  \item si $\vec{v}_3$ est combinaison linéaire des vecteurs de $S_2$, on pose $S_3 = S_2$, \\
    sinon, on pose $S_3 = S_2 \cup \{\vec{v_3}\}$.
\end{itemize}
On passe ainsi en serve tous les vecteurs de $S$. Au terme du processus, on obtient un système $S_m$ extrait de $S$. Alors, $S_m$ est libre et engendre le même sous-espace vectoriel que $S$,  donc $S_m$ est une base de $W$. 
\\\\
On peut trouver $S_m$ de la façon suivante:
\begin{itemize}
  \item on forme la matrice $M$ du système $S$ dans la base $\mB$.
  \item on échelonne et on réduit $M$.
\end{itemize}
Soit $R$ la forme échelonnée réduite de $M$. Soit $P$ la matrice carrée de taille $n\times n$ produit des matrices élémentaires correspondant aux opérations élémentaires effectuées. On a 
$$R = P M$$
On a 
\begin{eqnarray*}
  R &=& P \cdot \bigg([\vec{v}_1]_\B \ldots [\vec{v}_m]_\B \bigg) \\
   &=& \bigg(P \cdot [\vec{v}_1]_\B \ldots P \cdot [\vec{v}_m]_\B \bigg) \\
\end{eqnarray*}
On repère alors dans $R$ les indices des colonnes qui contiennent les coefficients pivot. Les vecteurs de $S$ indexés par ces mêmes indices sont exactement, dans le même ordre, les vecteurs qui constituent $S_m$.

\paragraph{Exemple} Dans $\R^3$, soient $\vec{v}_1 = (1, 0, 2)$, $\vec{v}_2 = (-1, 1, 3)$ et $\vec{v}_3 = (-1, 2, 8)$, $S = (\vec{v}_1, \vec{v}_2, \vec{v}_3)$ et $W = Vect(\vec{v}_1, \vec{v}_2, \vec{v}_3)$. Formons la matrice de $S$ dans la base canonique de $\R^3$:
$$M = \begin{pmatrix}
  1 & -1 & -1 \\
  0 & 1 & 2 \\
  2 & 3 & 8
\end{pmatrix}$$
On échelonne et réduit $M$:
\begin{eqnarray*}
  \begin{pmatrix}
    1 & -1 & -1 \\
    0 & 1 & 2 \\
    2 & 3 & 8
  \end{pmatrix}
  &\rightarrow^{L_3 \leftarrow L_3 - 2 L_1}&
  \begin{pmatrix}
    1 & -1 & -1 \\
    0 & 1 & 2 \\
    2 & 5 & 10
  \end{pmatrix}
  \rightarrow^{L_3 \leftarrow \frac{1}{5} L_3}
  \begin{pmatrix}
    1 & -1 & -1 \\
    0 & 1 & 2 \\
    0 & 1 & 2
  \end{pmatrix}
  \\
  &\rightarrow^{L_3 \leftarrow L_3 - L_2}&
  \begin{pmatrix}
    1 & -1 & -1 \\
    0 & 1 & 2 \\
    0 & 0 & 0
  \end{pmatrix}
  \rightarrow^{L_1 \leftarrow L_1 + L_2}
  \begin{pmatrix}
    1 & 0 & 1 \\
    0 & 1 & 2 \\
    0 & 0 & 0
  \end{pmatrix}
\end{eqnarray*}
Alors $\R$ posséde $2$ coefficients pivot, dans les colonnes $1$ et $2$. Il s'ensuit que $(\vec{v}_1, \vec{v}_2)$ est une base de $W$ extraite de $S$. 
\\\\
Dans la relation
$$R = P M$$
$P$ peut être interprêté comme la matrice de passage de la base canonique £ une base de $V$ contenant $(\vec{v}_1, \vec{v}_2)$. Dans $\R^3$ $\vec{v}_1 = (1, 0, 2), \vec{v}_2 = (-1, 2, 3)$ et$\vec{v}_3 = (-1, 2, 8)$
$$M = \begin{pmatrix} 
  1 & -1 & -1 \\
  0 & 1§ & 2  \\
  2 & 3 & 8
\end{pmatrix}
\rightsquigarrow  
R = \begin{pmatrix}
  1 & 0 & 1 \\
  0 & 1 & 2 \\
  0 & 0 & =
\end{pmatrix}$$
\begin{eqnarray*}
  R &=& P M \\
    &=& P \left([\vec{v}_1]_{\T} [\vec{v}_2]_{\T} [\vec{v}_3]_{\T}\right)
\end{eqnarray*}
on peut interpréter $P$ comme la matrice de passage de $\T$ à une base de $\R^3$ comprenant $(\vec{v}_2, \vec{v}_2)$ ucomme premiers vecteurs. En effet, on a
$$[\vec{v}_1]_{\B} \left( \begin{array}{m} 1 \\ 0 \\ 0 \end{array} \right) = P[\vec{v}_1]_{\T} \text{ et } 
  [\vec{v}_2]_{\B} \left( \begin{array}{m} 0 \\ 1 \\ 0 \end{array} \right) = P[\vec{v}_2]_{\T}$$
Ces relations signifient que $\vec{v}_1$ est la premier vecteur de $\mB$ et $\vec{v_2}$ le deuxième. On a également
$$[\vec{v}_3}]_{\B} = \left( \begin{array}{m} 1 \\ 2 \\ 0 \end{array}\right) = P [\vec{v}_3]_{\T}$$
d'où $\vec{v}_3 = \vec{v}_1 + 2 \vec{v}_2$. \\
Le troisième vecteur de $\B$ est généré par les opération élémentaires effectuèes. Ce peut être n'importe quel vecteur de $\R^3$ qui complète $(\vec{v}_1, \vec{v}_2)$ en une base de $\R^3$. \\
La forme échelonnée réduite $R$ de $M$ est donc la matrice du système $(\vec{v}_1, \vec{v}_2, \vec{v}_3)$ dans la base $\B$ dont les premiers vecteurs sont ceux de la base qu'on peut extraire du système en appliquant lâlgorithme défini précédemment. $R$ ne fait intervenir que les vecteurs de cette base extraite. \\
Cela implique en particulier que la forme échelonnée réduité de $M$ est unique. Par contre la matrice $P$ n'est panécessairement unique.

\paragraph{Définition} Soit $E$ un espace vectoriel. Soit $S = (\vec{v}, \ldots, \vec{v}_r)$ un système de vecteurs de $E$. On appelle rang de $S$ la dimension de $Vect(\vec{v}_1, \ldots, \vec{v}_r)$.

\paragraph{Définition} Soit $M$ une matrice de taille $n\times M$ quelconque. On appelle rang de $M$ le rand dans $M_{n\times 1}(\R)$ du système des colonnes de $M$.

%
%
\section{Systèmes d'équations cartésiennes} 
%
%
Soient $R$ un espace vectoriel de dimension finie $n$. Soient $W$ un sous-espace vectoriel de $E$ et $\B$ une base de $E$.
\paragraph{Définition} On appelle système d'équations cartésiennes de $W$ dans la base $\B$ tout système d'équations à $n$ inconnues tel que, pour tout vecteur $\vec{w} \in E$ on ait $\vec{v} \in W \Leftrightarrow$ le n-uplet $(\alpha_1, \ldots, \alpha_n)$ des coordonnées de $\vec{w}$ dans la base $\B$ est solution de $S$.

\paragraph{Example} Si $D$ est la droite vectorielle dans $\R^2$ engendrée par $\vec{u} = (2, 1)$, alors une équation cartésienne de $D$ (dnas la base canonique de $\R^2$) est
$$ - \frac{1}{2} x + y = 0$$
On a, pour tout $\vec{v} = (\alpha, \beta) \in \R^2$ 
$$\vec{v} \in D \Leftrightarrow -\frac{1}{2} \alpha + \beta = 0$$

\paragraph{Question} Comment déterminer un système d'équations cartésiennes d'un sous-espace vectoriel lorsque celui-ci est donné par un système générateur?

\paragraph{Example} Dans $\R^3$, considérons le sous-espace vectorie engendré par $\vec{v}_1 = (1, 0, 0)$, $\vec{v}_2 = (1, 1, 10)$ et $\vec{v}_3 = (2, 1, 0)$. Déterminons un système d'équations cartésiennes du sous-espace $W = Vect(\vec{v}_1, \vec{v}_2, \vec{v}_3)$ de $\R^3$. Soit $\vec{v} = (a, b, c) \in \R^3$. \\
Alors $\vec{v} \in W$ 
\begin{itemize}
  \item[$\Leftrightarrow$] il existe $\lambda_1, \lambda_2, \lambda_3 \in \R$ tels que $\vec{v} = \lambda_1 \vec{v}_1 + \lambda_2, \vec{v}_2 + \lambda_3 \vec{v}_3$ 
  \item[$\Leftrightarrow$] il existe $\lambda_1, \lambda_2, \lambda_3 \in \R$ tels que 
    $\left\{ \begin{array}{rrrc}
       \lambda_1 & + \lambda_2 & + 2 \lambda_3 & = a \\
       & \lambda_2 & + \lambda_3 & = b \\
       &  & 0 & = c
    \end{array}$
  \item [$\Leftrightarrow$] le système 
    $\left\{ \begin{array}{rrrc}
      x_1 & + x_2 & + 2 x_3 & = a \\
       & x_2 & + x_3 & = b \\
       &  & 0 & = c
    \end{array}$ admet au moins une solution.
\end{itemize}
On cherche alors les conditions de compatibilité du système, c'est-à-dire les conditions sous lesquelles le système admet au moins une solution. Pour cela, on échelonne et réduit le système
\begin{eqnarray*}
  \begin{pmatrix}
    1 & 1 & 2 & a \\
    0 & 1 & 1 & b \\
    0 & 0 & 0 & c
  \end{pmatrix}
  \rightarrow^{L_1 \leftarrow L_1 - L_2} 
  \begin{pmatrix}
    1 & 0 & 1 & a \\
    0 & 1 & 1 & b \\
    0 & 0 & 0 & c
  \end{pmatrix}
\end{eqnarray*}
Pour ce système, une seule condition de compatibilité:
$$\left\{ c= 0$$
On a obtenu un système d'équations cartésiennes de $W$ (ici, on a une seule équation). Ainsi, on a $\vec{v} = (a, b, c) \in \R^3$
$$\vec{v} \in W \Leftrightarrow c = 0$$
Si maintenant $V$ est un sous-espace de $E$ donné par un système linéaire d'équations cartésiennes dans une base de $E$, pour déterminer une base et la dimension, on résout ce système. À partir de l'expression des solutions du système, on trouve une base de $V$, puis sa dimension.

%
%
\section{Représentation matricielle des applications linéaires}
%
%
Soient $E$ est $F$ deux espaces vectoriels de dimensions finies $m$ et $n$ respectivement, $\cE = \{ \vec{u}_1, \dots, \vec{u}_m\}, \cF = \{\vec{v}_1, \ldots, \vec{v}_n\}$ des bases de $E$ et $F$ respectivement, et $f: E \rightarrow F$ une application linéaire. \\
Nous avons vu précédemment que $f$ est entièrement déterminée par la donnée de $f(\vec{u}_1), \ldots, f(\vec{u}_m)$.

\paragraph{Définition} On appelle matrice de $f$ dans les bases $cE$ et $cF$ la  matrice
$$[F]_{\cF \cE} = \bigg(
  [f(\vec{u}_1]_{\cF}] & \ldots & [f(\vec{u}_1]_{\cF}]
\bigg)$$
de taille $n \times m$.

\paragraph{Théorème} Pour tout $\vec{u} \in E$, on a 
$$[f(\vec{u})]_{\cF} = [f]_{\cF \cE} [\vec{v}]_{\cE}$$


\chapter{Diagonalisation des endomorphismes et des matrices}

%
\subsection{Réduction des endomorphismes}
%
\paragraph{Rappels} Si $E$ est un espace vectoriel, on rappelle qu'un \underline{endomorphisme de $E$} est une application linéaire de $E$ dans $E$.

\paragraph{Définition} On appelle réduction des endomorphismes (d'un espavec vectoriel de dimension finie) la démarche consistant à chercher une base de l'espace dans laquelle la matrice de l'endomorphisme à une form relativement simple. La diagonalisation des endomorphisme, lorsqu'elle est possible, s'inscrit dans cette démarche.

%
%
\section{Matrice d'une application linéaire et changement de bases}
%
%

\paragraph{Théorème} Soient $E$ et $F$ deux espaces vectoriels de dimension finie, $f: E \rightarrow F$ une application linéaire, $\cE$ et $\cE'$ deux bases de $E$, et $\cF$ et $\cF'$ deux bases de $F$. On note $M = [f]_{\cF \cE}$ la matrice de $f$ dans les bases $\cE$ et $\cF$ et $M' = [f]_{\cF' \cE'}$ la matrice de $f$ dans les bases $\cE'$ et $\cF'$. On note également $P = P_{\cF \cF'}$ la matrice de passage de $\cF$ à $\cF'$ et $Q = P_{\cE \cE'}$ la matrice de passage de $\cE$ à $\cE'$. On a alors 
\begin{eqnarray*}
  M' &=& P^{-1} M Q \\
  \left[f\right]_{\cF' \cE'} &=& P_{\cF \cF'}^{-1} [f]_{\cF \cE} P_{\cE \cE'}
\end{eqnarray*}

\paragraph{Démonstration} Pour tout $\vec{v} \in E$, on a:
$$[f(\vec{v})]_{\cF} = M [\vec{v}]_{\cE}$$
On a $[\vec{v}]_{\cE} = Q [\vec{v}]_{\cE'}$ et $[f(\vec{v})]_{\cF} = P [f(\vec{v})]_{\cF'}$ Il vient $P [f(\vec{v})]_{\cF'} = M Q [\vec{v}]_{\cE'}$, d'où $[f(\vec{v})]_{\cF'} = (P^{-1} M Q ) [\vec{v}]_{\cE'}.$ il s'ensuit que $M' = P^{-1} M Q$.

\paragraph{Example} Considérons l'application linéaire
\begin{eqnarray*}
  f: \R^3 &\rightarrow& \R^2 \\
  (x, y, z) &\mapsto& f(x, y, z) = (x + y, x - y + 2z)
\end{eqnarray*}
Soient $\cC_3$ la base canonique de $\R^3$ et $\cC_2$ celle de $\R^2$. Soient $\cE = ((1, 0, 0), (1, 1, 0), (1, 1, 1))$ une base de $\R^3$ et $\cF = ((1, 1), (1, 2))$ une base de $\R^2$. \\
Calculons la matrice de $f$ dans les bases $\cE$ et $\cF$. On a
$$\left[f\right]_{\cC_2 \cC_3} = 
\begin{pmatrix}
  1 & 1  & 0 \\
  1 & -1 & 2
\end{pmatrix}$$
On a également $P = P_{\cC_2 \cF} = \begin{pmatrix} 1 & 1 \\ 1 & 2 \end{pmatrix}$ et $Q = P_{\cC_3 \cE} = \begin{pmatrix} 1 & 1 & 1 \\ 0 & 1 & 1 \\ 0 & 0 & 1 \end{pmatrix}$. Calculons $P^{-1}$. On a $det(P) = 2 - 1 = 1$, $com(P) = \begin{pmatrix} 2 & -1 \\ -1 & 1 \end{pmatrix}$ d'où
$$P^{-1} = \frac{1}{det(P)} com(P)^{T} = \begin{pmatrix} 2 & -1 \\ -1 & 1 \end{pmatrix}$$
On obtien alors
\begin{eqnarray*}
  \left[f\right]_{\cF \cE} &=& P^{-1}  [f]_{\cC_2 \cC_3} Q \\
    &=& \begin{pmatrix} 2 & -1 \\ -1 & 1 \end{pmatrix} \begin{pmatrix} 1 & 1 & 0 \\ 1 & -1 & 2 \end{pmatrix} \begin{pmatrix} 1 & 1 & 1 \\ 0 & 1 & 1 \\ 0 & 0 & 1 \end{pmatrix} \\
    &=& \begin{pmatrix} 2 & -1 \\ -1 & 1 \end{pmatrix} \begin{pmatrix} 1 & 2 & 2 \\ 1 & 0 & 2 \end{pmatrix} \\
    &=& \begin{pmatrix} 1 & 4 & 2 \\ 0 & -2 & 0 \end{pmatrix}
\end{eqnarray*}

%
%
\section{Valeurs propres, vecteurs propres d'un endomorphisme}
%
%
%
\subsection{Valeur propre}
%
\paragraph{Définition} Soient $E$ un espace vectoriel et $f: E \rightarrow E$ un endomorphisme de $E$. On dit qu'un réel $\lambda$ est une valeur propre de $f$ s'il existe $\vec{v} \in E$, $\vec{v} \neq \vec{0}$, tel que 
$$f(\vec{v}) = \lambda\vec{v}$$

%
\subsection{Vecteur propre}
%
\paragraph{Définition} Soit $\lambda$ une valeur propre de $f$. On appelle vecteur propre de $f$ pour $\lambda$ tout vecteur $\vec{v}$ de $E$ tel que 
$$f(\vec{v}) = \lambda \vec{v}$$

%
\subsection{Sous-espace propre}
%
\paragraph{Définition} L'ensemble des vecteurs de $f$ associés à $\lambda$, noté généralement $E_\lambda$, s'appelle le sous-espace propre de $f$ associé à $\lambda$. On a
\begin{eqnarray*}
  E_{\lambda} &=& \{\vec{v} \in E \vert f(\vec{v} = \lambda\vec{v} \} \\
    &=& \{\vec{v} \in E \vert (f-\lambda id_E)(\vec{v}) = \vec{0} \} \\
    &=& Ker(f-id_E)
\end{eqnarray*}
Par suite, $E_\lambda$ est un sous-espace vectoriel de $E$.

\paragraph{Examples}
\begin{enumerate}
  \item Soit 
    \begin{eqnarray*}
      f: \R^2 &\rightarrow& \R^2 \\
      (x, y) &\mapsto& f(x, y) = (2x - 4 y, x -3 y)
    \end{eqnarray*}
    Alors, on a 
    $$f(1, 1) = (-2, -2) = -2 \cdot (1, 1)$$
    Par suite, $-2$ est valeur propre de $f$, et $(1, 1)$ est un vecteur propre de $f$ associé à $-2$. Déterminons 
    \begin{eqnarray*}
      E_{-2} &=& Ker(f-(-2)\cdot id_{\R^2} ) \\
        &=& Ker(f + 2 \cdot id_{\R^2})
    \end{eqnarray*}
    On est conduit à résoudre le sytème linéaire.
    $$\left(\begin{pmatrix}
        2 & -4 \\
        1 & -3
      \end{pmatrix} + 2 \cdot
      \begin{pmatrix}
        1 & 0 \\
        0 & 1
      \end{pmatrix} \right) X = 0$$
    c'est-à-dire $\begin{pmatrix} 4 & -4 \\ 1 & -1 \end{pmatrix} X = 0$ Il vient
    $$\begin{pmatrix} 4 & -4 & \vert & 0 \\ 1 & 1 & \vert & 0 \end{pmatrix} \rightsquigarrow \begin{pmatrix} 1 & 0 & \vert & 0 \\ 0 & 0 & \vert & 0 \end{pmatrix}$$
    Donc 
    \begin{eqnarray*}
      Ker(f + 2 \cdot id_E) &=& \{(t, t) \in \R^2 \vert \R \} \\
        &=& \{t \cdot (1, 1) \in R^2 \vert t \in \R \} \\
        &=& Vect((1, 1))
    \end{eqnarray*}
    
  \item Soit 
    \begin{eqnarray*}
      f: \M_{3 \times 3}(\R) &\rightarrow& M_{3 \times 3}(\R) \\
      A &\mapsto& f(A) = A^{T}
    \end{eqnarray*}
    On sait que $f$ est un endomorphisme de $M_{3 \times 3}(\R)$. Alors $1$ est valeur propre de $f$ et
    $$E_{1} = \{A \in M_{3 \times 3}(\R) \vert A^{T} = A \}$$
    est le sous-espace vectoriel de $M_{3 \times 3}(\R)$ consittué des matrices symétriques. \\
    De même $-1$ est valeur propre de $f$, et
    $$E_{-1} = \{A \in M_{3\times 3}(\R) \vert A^{T} = -A\}$$
    est le sous-espace vectoriel de $M_{3\times 3}(\R)$ constitué des matrices antisymétriques.
    
\end{enumerate}

\paragraph{Théorème} Soit $f: E \rightarrow E$ un endomorphisme de $E$. On suppose que $E$ est de dimension finie. Soit $\lambda \in \R$. \\
Alors $\lambda$ est valeur propre de $f \Leftrightarrow f- \lambda \cdot id_E$ n'est pas bijectif.
      
\paragraph{Démonstration} $\lambda$ est valeur propre de $f$
\begin{enumerate}[$\Leftrightarrow$]
  \item il existe $\vec{v} \in E$, $\vec{v} \neq \vec{0}$, $(f - \lambda \cdot id_E)(\vec{v}) = \vec{0}$ 
  \item $Ker(f-\lambda\cdot id_E) \neq \{\vec{0}\}$
  \item $f-\lambda \cdot id_E$ n'est pas injectif 
  \item $f-\lambda \cdot id_E$ n'est pas bijectif.
\end{enumerate}

%
\subsection{Polynôme caractéristique}
%
\paragraph{Définition} Soient $f: E \rightarrow E$ un endomorphisme ($E$ est de dimension finie), \cE une base $E$, et $M$ la matrice de $f$ dans les bases $\cE$ et $\cE$\footnote{On dit simplement "dans la base $\cE$".} \\
On appelle polynôme caractéristique de $f$ le polynôme
$$P(x) = (-1)^{n} det(M - X I_n)$$

\paragraph{Remarque importante} Le polynôme $P$ est indépendant de la base choisie. \\
En effet, si $\cE'$ est une autre base de $E$, et si $P = P_{\cE \cE'}$ est la matrice de passe de $\cE$ à $\cE'$, alors:
\begin{itemize}
  \item la matrice de $f$ dans la base $\cE'$ est $M' = P^{-1} M P$
  \item 
    \begin{eqnarray*}
      (-1)^{n} det(M' - X I_n) &=& (-1)^{n} det(P^{-1} M P - X I_n) \\
        &=& (-1)^{n} det(P^{-1} (M - X I_n) P) \\
        &=& (-1)^{n} det(P)^{-1} det(M - X I_n) det(P) \\
        &=& (-1)^{n} det(M - X I_n) \\
    \end{eqnarray*}
\end{itemize}

\paragraph{Définition} $\lambda$ valeur de propre de $f$ $\Leftrightarrow$ $P(\lambda) = 0$
\paragraph{Définition} $\lambda$ f est diagonalisable s'il existe une base de $E$ formée de vecteurs propres de $f$



\end{document}
