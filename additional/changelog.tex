\chapter{Changelog}
\begin{itemize}
  \item[BETA 0.1.1 (05.01.2012)] Correction des erreurs d'orthographe et correction d'une faute de contenue (\ref{sec:espv_applin__rapp}):
    \begin{quote}
      et d'une multiplication externe
      \begin{eqnarray*}
        \cdot: V \times V &\rightarrow& V \\
        (\vec{u}, \vec{v}) &\mapsto&  \vec{u} \cdot \vec{v}
      \end{eqnarray*}
    \end{quote}
    a été changé à
    \begin{quote}
      et d'une multiplication externe
      \begin{eqnarray*}
        \cdot: \R \times V &\rightarrow& V \\
        (\alpha, \vec{v}) &\mapsto& \alpha \cdot \vec{v}
      \end{eqnarray*}
    \end{quote}

  \item[BETA 0.1.2 (17.01.2012)] Correction des erreurs d'orthographe et des fautes suivantes:
    \begin{itemize}
      \item Section (\ref{sec:isomorphisme}): Avant
        \begin{quote}
          \begin{eqnarray*}
            (g \circ f)(\alpha \vec{u} +\vec{v}) &=& g(f(\alpha \vec{u} +\vec{v})) \\
              &=& g(\alpha \vec{u} + \vec{v}) \\
              &=& \alpha \cdot g(f(\vec{u})) + g(f(\vec{v})) \\
              &=& \alpha \cdot (g \circ f)(\vec{u}) + (g \circ f)(\vec{v})
          \end{eqnarray*}
        \end{quote}
        Après
        \begin{quote}
          \begin{eqnarray*}
            (g \circ f)(\alpha \vec{u} +\vec{v}) &=& g(f(\alpha \vec{u} +\vec{v})) \\
              &=& g(\alpha \cdot f(\vec{u}) + f(\vec{v})) \\
              &=& \alpha \cdot g(f(\vec{u})) + g(f(\vec{v})) \\
              &=& \alpha \cdot (g \circ f)(\vec{u}) + (g \circ f)(\vec{v})
          \end{eqnarray*}
        \end{quote}
      
      \item Section (\ref{sec:grp_sym}), dans les exemples: Avant
        \begin{quote}
          Calculons $\tau \circ \sigma^{-1} \in S_4$. Pour tout $i \in :\{1; 2; 3; 4\}$,
          $$(\tau \circ \sigma)^{-1}(i) = \tau(\sigma^{-1}(i))$$
        \end{quote}
        Après
        \begin{quote}
          Calculons $\tau \circ \sigma^{-1} \in S_4$. Pour tout $i \in :\{1; 2; 3; 4\}$,
          $$(\tau \circ \sigma^{-1})(i) = \tau(\sigma^{-1}(i))$$
        \end{quote}
        
      \item Section (\ref{sec:se_prop}), dans les exemples: Avant:
        \begin{quote}
          $$\left( \begin{array}{cc|c} 4 & -4 & 0 \\ 1 & 1 & 0 \end{array} \right) 
            \rightsquigarrow 
            \left( \begin{array}{cc|c} 1 & 0 & 0 \\ 0 & 0 & 0 \end{array} \right) $$
          Donc 
          \begin{eqnarray*}
            Ker(f + 2 \cdot id_E) &=& \{(t, t) \in \R^2 ~ \vert ~ \R \} \\
              &=& \{t \cdot (1, 1) \in R^2 ~ \vert ~ t \in \R \} \\
              &=& Vect((1, 1))
          \end{eqnarray*}
        \end{quote}
        Après:
        \begin{quote}
          $$\left( \begin{array}{cc|c} 4 & -4 & 0 \\ 1 & -1 & 0 \end{array} \right) 
            \rightsquigarrow 
            \left( \begin{array}{cc|c} 1 & -1 & 0 \\ 0 & 0 & 0 \end{array} \right) $$
          Donc 
          \begin{eqnarray*}
            Ker(f + 2 \cdot id_E) &=& \{(t, t) \in \R^2 ~ \vert ~ \R \} \\
              &=& \{t \cdot (1, 1) \in \R^2 ~ \vert ~ t \in \R \} \\
              &=& Vect((1, 1))
          \end{eqnarray*}
        \end{quote}
        
    \end{itemize}
    
  \item[BETA 0.1.3 (18.01.2012)] Correction des fautes suivantes:
    \begin{itemize}
      \item Section (\ref{sec:symb_kroen}): Avant:
        \begin{quote}
          $$F_{ij}F_{kl} = \delta_{ik} F_{il} = 
            \left\{\begin{array}{lr} F_{il} & \text{si i=j} \\ 0 & \text{sinon} \end{array} \right.$$
        \end{quote}
        Après:
        \begin{quote}
          $$F_{ij}F_{kl} = \delta_{jk} F_{il} = 
            \left\{\begin{array}{lr} F_{il} & \text{si j=k} \\ 0 & \text{sinon} \end{array} \right.$$
        \end{quote}
            
      \item Section (\ref{sec:calc_inv_matr}): Avant:
        \begin{quote}
          \begin{eqnarray*}
            A \cdot E_1 \cdot E_2 \cdot &\ldots& \cdot E_l = I_n \\
            &\Downarrow& \\
            I_n \cdot E_1 \cdot E_2 \cdot &\ldots& \cdot E_l = A^{-1} \\
            &\Downarrow& \\
            E_1 \cdot E_2 \cdot &\ldots& \cdot E_l = A^{-1}
          \end{eqnarray*}
          De la même manière on peut calculer $A^{-1} B$.
          $$A^{-1} B = E_1 \cdot E_2 \cdot \ldots \cdot E_l \cdot B$$
        \end{quote}
        Après:
        \begin{quote}
          \begin{eqnarray*}
            I_n = E_l \cdot &\ldots& \cdot E_2 \cdot E_1 \cdot A \\
            &\Downarrow& \\
            E_1^{-1} \cdot E_2^{-1} \cdot &\ldots& \cdot E_l^{-1} \cdot I_n = A \\
            &\Downarrow& \\
            E_l \cdot &\ldots& \cdot E_2 \cdot E_1 = A^{-1}
          \end{eqnarray*}
          De la même manière on peut calculer $A^{-1} B$.
          $$A^{-1} B = E_l \cdot &\ldots& \cdot E_2 \cdot E_1 \cdot B$$
        \end{quote}
    
      \item Section (\ref{sec:se_prop}): Avant:
        \begin{quote}
          \begin{eqnarray*}
            E_{\lambda} &=& \{\vec{v} \in E ~ \vert ~ f(\vec{v}) = \lambda\vec{v} \} \\
              &=& \{\vec{v} \in E ~ \vert ~ (f-\lambda id_E)(\vec{v}) = \vec{0} \} \\
              &=& Ker(f - id_E)
          \end{eqnarray*}
        \end{quote}
        Après:
        \begin{quote}
          \begin{eqnarray*}
            E_{\lambda} &=& \{\vec{v} \in E ~ \vert ~ f(\vec{v}) = \lambda\vec{v} \} \\
              &=& \{\vec{v} \in E ~ \vert ~ (f-\lambda id_E)(\vec{v}) = \vec{0} \} \\
              &=& Ker(f - \lambda id_E)
          \end{eqnarray*}
        \end{quote}
        
    \end{itemize}
    
    \item[BETA next (DATE)] Correction des fautes suivantes:
    \begin{itemize}
      \item Section (\ref{sec:determ}), dans l'example: Avant:
        \begin{quote}
          \begin{eqnarray*}
            \sigma = id           &\Rightarrow& p_{\sigma}(A) = 1 \cdot A_{11} \cdot A_{22} \cdot A_{33} \\
            \sigma = (1 ~ 2)      &\Rightarrow& p_{\sigma}(A) = 1 \cdot A_{21} \cdot A_{21} \cdot A_{33} \\
            \sigma = (1 ~ 3)      &\Rightarrow& p_{\sigma}(A) = 1 \cdot A_{31} \cdot A_{22} \cdot A_{13} \\
            \sigma = (2 ~ 3)      &\Rightarrow& p_{\sigma}(A) = 1 \cdot A_{11} \cdot A_{32} \cdot A_{23} \\
            \sigma = (1 ~ 2 ~ 3)  &\Rightarrow& p_{\sigma}(A) = 1 \cdot A_{21} \cdot A_{32} \cdot A_{13} \\
            \sigma = (1 ~ 3 ~ 2)  &\Rightarrow& p_{\sigma}(A) = 1 \cdot A_{31} \cdot A_{12} \cdot A_{23}
          \end{eqnarray*}
        \end{quote}
        Après:
        \begin{quote}
          \begin{eqnarray*}
            \sigma = id           &\Rightarrow& p_{\sigma}(A) = 1 \cdot A_{11} \cdot A_{22} \cdot A_{33} \\
            \sigma = (1 ~ 2)      &\Rightarrow& p_{\sigma}(A) = -1 \cdot A_{21} \cdot A_{12} \cdot A_{33} \\
            \sigma = (1 ~ 3)      &\Rightarrow& p_{\sigma}(A) = -1 \cdot A_{31} \cdot A_{22} \cdot A_{13} \\
            \sigma = (2 ~ 3)      &\Rightarrow& p_{\sigma}(A) = -1 \cdot A_{11} \cdot A_{32} \cdot A_{23} \\
            \sigma = (1 ~ 2 ~ 3)  &\Rightarrow& p_{\sigma}(A) = -1 \cdot A_{21} \cdot A_{32} \cdot A_{13} \\
            \sigma = (1 ~ 3 ~ 2)  &\Rightarrow& p_{\sigma}(A) = 1 \cdot A_{31} \cdot A_{12} \cdot A_{23}
          \end{eqnarray*}
        \end{quote}
        
    \end{itemize}
\end{itemize}
